\documentclass[usenames,dvipsnames]{beamer}
\usetheme{Boadilla}

\makeatother
\setbeamertemplate{footline}
{
  \leavevmode%
  \hbox{%
  \begin{beamercolorbox}[wd=.4\paperwidth,ht=2.25ex,dp=1ex,center]{author in head/foot}%
    \usebeamerfont{author in head/foot}\insertshortauthor
  \end{beamercolorbox}%
  \begin{beamercolorbox}[wd=.6\paperwidth,ht=2.25ex,dp=1ex,center]{title in head/foot}%
    \usebeamerfont{title in head/foot}\insertshorttitle\hspace*{3em}
    \insertframenumber{} / \inserttotalframenumber\hspace*{1ex}
  \end{beamercolorbox}}%
  \vskip0pt%
}
\makeatletter
\setbeamertemplate{navigation symbols}{}

\usepackage[utf8]{inputenc}
\usepackage[usenames,dvipsnames]{xcolor}

\title{Dependent types}
\author{Ali Caglayan}
\institute{University of Bath}

\newcommand{\blu}[1]{\textcolor{blue}{#1}}
\newcommand{\gre}[1]{\textcolor{ForestGreen}{#1}}
\newcommand{\red}[1]{\textcolor{red}{#1}}
\newcommand{\yel}[1]{\textcolor{violet}{#1}}

\begin{document}
    %\frame{\titlepage}
    
    
    %% So my second marker is Guy McKusker which will mean I can brush over lambda calculus quicker
    
    % applications of which
    
    % extensions of lambda calculus
    % lambda calculus
    
    % Alessio says these things are important
    % rationale
    % state of the art
    % results
    % the future
    
    \begin{frame}{Lambda calculus}
        \begin{itemize}
            \item Lambda calculus is a formal system.
            \item We can restrict some functions from being applied to others.
            \item This is known as typing.
            \item The system we get is called simply typed lambda calculus.
            \item We only have function types, but we can have more...
        \end{itemize}
    \end{frame}
    
    \begin{frame}{Extending typed lambda calculus}
        When adding a new type we must write down rules to define how it will behave.
        Usually these are sorted into 4 kinds of rules:
        \begin{itemize}
            \item \blu{Introduction rules} (how to make the type)
            \item \gre{Constructors} (how to make terms of the type)
            \item \red{Eliminators} (how to break terms of the type)
            \item \yel{Computation rules} (how a function coming out of the type computes)
        \end{itemize}
        \textbf{Note}: Computation rules can usually be derived from the other rules, and therefore can be omitted.
    \end{frame}

    \begin{frame}{Product types}
        \begin{block}{\blu{Introduction}}
            $$\frac{\Gamma \vdash A\ \text{Type}\qquad \Gamma \vdash B\ \text{Type} }{\Gamma \vdash A \times B\ \text{Type}}$$
        \end{block}
        
        \begin{block}{\gre{Constructors}}
            $$\frac{\Gamma \vdash a : A\qquad \Gamma \vdash b : B}{\Gamma\vdash (a,b) : A\times B}$$
        \end{block}
        
        \begin{block}{\red{Eliminators}}
            $$\frac{\Gamma \vdash t : A \times B}{\Gamma\vdash\text{fst}(t) : A} \qquad\qquad
              \frac{\Gamma \vdash t : A \times B}{\Gamma\vdash\text{snd}(t) : B}$$
        \end{block}
        
        \begin{block}{\yel{Computation rules}}
            $$(\text{fst}(t), \text{snd}(t)) \equiv t$$
        \end{block}
    \end{frame}    
    
    \begin{frame}{Sum types}
        \begin{block}{\blu{Introduction}}
            $$\frac{\Gamma \vdash A\ \text{Type}\qquad \Gamma \vdash B\ \text{Type} }{\Gamma \vdash A + B\ \text{Type}}$$
        \end{block}
        
        \begin{block}{\gre{Constructors}}
            $$\frac{\Gamma \vdash a : A}{\Gamma \vdash \text{inl}(a) : A + B} \qquad\qquad
              \frac{\Gamma \vdash b : B}{\Gamma \vdash \text{inr}(b) : A + B}$$
        \end{block}
        
        \begin{block}{\red{Eliminators}}
            $$\frac
                {\Gamma \vdash f : A \to C \qquad \Gamma \vdash g : B \to C}
                {\Gamma \vdash \text{ind}_{A+B}(f, g) : A + B \to C}$$
        \end{block}
        
    \end{frame}
    
    % Probably can get rid of this
    \begin{frame}{Motivation}
        \begin{itemize}
            \item Programming languages
            \begin{itemize}
                \item Haskell
                \item ML
                \item basically every other typed functional programming language
            \end{itemize}
        \end{itemize}
    \end{frame}
    
    \begin{frame}{Mathematical motivation - Curry-Howard correspondance}
        There is a correspondance between propositional logic and type theory. Types are propositions, and terms are proofs.
        \begin{table}
            \begin{tabular}{c || c}
                \textbf{Propositional logic} & \textbf{Type theory} \\
                \hline\hline
                proposition $A$         & $A\ \text{Type}$          \\
                proof of $A$            & term of $A$               \\
                and $A \land B$         & product type $A \times B$ \\
                or $A \lor B$           & sum type $A + B$          \\
                implies $A \implies B$  & function type $A \to B$   \\
                true                    & unit type $\mathbf{1}$    \\
                false                   & empty type $\mathbf{0}$   \\
                not $¬A$                & $A \to \mathbf{0}$        \\
            \end{tabular}
        \end{table}
        
        This is the begining of using type theory to encode mathematics.
        This is how proof assistants work.
    \end{frame}
    
    % 
    \begin{frame}{What are dependent types?}
        \begin{itemize}
            \item Functions allow terms to depend on other terms
            \item Polymorphism allows types to depend on other types
            \item Terms already depend on types
            \item Dependent types allow types to depend on terms
        \end{itemize}
        What problems can dependent types solve?
        \begin{itemize}
            %\item Generalising polymorphism.
            % the type theory behind ML is clunky for example`
            \item Encoding hard to encode data types such as lists (or vectors) of fixed length.
            \item It is equivalent to first-order logic in some suitable sense.
        \end{itemize}
    \end{frame}
    
    \begin{frame}{Pi types}
        What if the target of a function type could change depending on the input?
        % intros, constructos, eliminos
    \end{frame}
    
    \begin{frame}{Sigma types}
        Some times product types are not enough. Especially when we need a family.
    \end{frame}
    
    \begin{frame}{(Dependent) Curry-Howard}
        \begin{table}
            \begin{tabular}{c || c}
                \textbf{Propositional logic} & \textbf{Type theory} \\
                \hline\hline
                $\forall a \in A, P(a)$ & pi type $\prod_{(a : A)}P(a)$     \\
                $\exists a \in A, P(a)$ & sigma type $\sum_{(a:A)}P(a)$     \\
                proposition $A$         & $A\ \text{Type}$                  \\
                proof of $A$            & term of $A$                       \\
                and $A \land B$         & product type $A \times B$         \\
                or $A \lor B$           & sum type $A + B$                  \\
                implies $A \implies B$  & function type $A \to B$           \\
                true                    & unit type $\mathbf{1}$            \\
                false                   & empty type $\mathbf{0}$           \\
                not $¬A$                & $A \to \mathbf{0}$                \\
            \end{tabular}
        \end{table}
    \end{frame}
    
    \begin{frame}{How can we model type theories?}
        \begin{block}{Answer:}
            Categorical semantics.
        \end{block}
        This allows us to use category theory to reason about the metatheory of our type theory.
        
        \begin{block}{But theres more...}
            When modelling ``type theories'' in mathematics it was found that there is really a two way correspondance.
            
            $$\text{Type theory} \rightleftharpoons \text{Category theory}$$
            
            Type theory can be used to reason about a category. Lots of people have investigated this, notably Topos theorists.
        \end{block}
        
    \end{frame}
    
    % general type theory
    % models of type theory CCCs
    
    % example sum type, product type
    % show intros, eliminators, constructors and computation when relevent
    % talk about curry-howard
    
    % what about quantities that "depend" on others?
    % example the type of vectors
    % introduce pi tpyes
    % introduce sigma types
    % show this extends curry-howard to a first-order logic without LEM
    
    % this gives a very expressive language
    % example include encoding mathematics
    % e.g. coq, agda, lean
    % these are also programming languages
    % example of a dependent data type:
    % trees, vectors, 
    
    % obviously lambda calculi + some stuff are turing complete
    % examples: haskell
    
    % but dependent types allow easier expression of dependent ideas
    % they have nice type theoretic properties, initiality, canonicity, normalising
    % how are these modelled?
    % comprehension categories, categories with families, natural model style
    % I investigate the relatively new natural model style
    % currently being used to prove an initiality theorem
    % which means the model is "freely generated" by the typing rules
    % important for meta theoretic arguments
    % categorical models alllow for the tools of category theory to be applied in type theoretic settings
    
    
\end{document}
