\section{STLC Examples}

%% Here are examples I would like to add
%
%    i) \x.x                (Identity)
%   ii) \x.\y.xy            (Application function)
%  iii) ((\x.\y.(x+y))3)5   (Obviously with N and + added)
%   iv) (\x.x)(\x.x)        (Doesn't type-check)
%    v) \x.xx               (Doesn't type-check (M combinator (Mockingbird)))
%   vi) \x.\y.((xy)(xy))    (Doesn't type-check)
%  vii) \x.\y.\z.x(yz)      (B combinator (Bluebird))
% viii) \x.\y.\z.xyzz       (W* combinator (Warbler once removed))
%   ix) \x.\y.y(xy)         (O combinator (Owl))
%    x) \x.\y.\z.x(y,z)     (Curry)


Untyped lambda calculus, as we mentioned, is in fact \emph{stronger} than the typed lambda calculus. This we will see by looking at some examples of type checking. Many of these are combinators from untyped lambda calculus in combinatory logic. \ref{} [[Need reference of Mockingbird combinator thing]]

Note we don't have very much choice on types, so it may be useful to enrich our type theory with $+$-types or even the natural numbers. But we will see soon that these both are special cases of dependent types.

\subsection{Identity function $\lambda x . x$}

\begin{example}[Identity function]
    Let's consider the following lambda term $\lambda x . x$. We wish to find a type $T$ such that given some context $\Gamma$ we have $\Gamma \vdash \lambda x . x \Leftarrow T$. Our inversion lemma will tell us exactly which rules let us get to this point. So we will essentially be performing a tree search. Firstly we need to switch modes to get $\lambda x . x \Rightarrow T$. But mode switching also lets us change our 
    \begin{prooftree}
        \AxiomC{$\Gamma \vdash \lambda x . x \Rightarrow T$}
    \end{prooftree}
\end{example}

\subsection{Function application $\lambda x . \lambda y . x y$}

%% rewrite!!
\begin{example}
    Here is another example of a term that type checks. Unfortunately we see the disadvantage with type-setting derivation trees: they are very difficult to write down, and get really wide very quickly.  We want to find a type $T$ such that $\Gamma \vdash \lambda x . \lambda y . x y \Leftarrow T$ is true. Here is a derivation tree: 
%%%
        
        \begin{proof}
        We begin with the judgement $\Gamma \vdash \lambda x . \lambda y . x y \Leftarrow T$, now the only way to arrive at this judgement is via the mode-switching rule. Whilst doing this we add type variables $A$ and $B$ which can easily be seen to form into $A \to B$ and let $T \equiv A \to B$. We can come back later and validate this judgement. The mode-switching should have given us $\Gamma \vdash \lambda x . \lambda y . x y \Rightarrow A \to B$ which we can only arrive at by applying the ($\to$-intro) rule. This gives us $\Gamma , x : A \vdash \lambda y . xy \Leftarrow B$. Which we have to mode-switch, and as before we take this chance to introduce type variables $C$ and $D$ in order to arrive at the judgement $\Gamma , x : A \vdash \lambda y . x y \Rightarrow C \to D$. This allows us to apply ($\to$-intro) giving us $\Gamma , x : A , y : C \vdash xy \Leftarrow D$. Now we apply the ($\to$-elim) rule since we have an application. For this we need $\Gamma , x : A, y : C \vdash y \Leftarrow C$, which is marked as $(\dagger)$, and observe that a simple application of mode-switching and the variable rule allows us to derive this judgement. The other hypothesis we need is $\Gamma , x : A, y : C \vdash x \Leftarrow C \to D$. Again by mode-switching and setting $C \to D \equiv A$ we get $\Gamma , x : A, y : C \vdash x \Rightarrow A$ which is clearly derivable by the variable rule.
        
        Now we have 3 type equations $(*)$, $(**)$ and $(***)$, substituting back in we get $\Gamma \vdash T \equiv (C \to D) \to C \to D$ for some types $C$ and $D$. So $\Gamma \vdash \lambda x . \labmda y . x y \Leftarrow T$ if we have types $C$ and $D$.
        \end{proof}
\end{example}

\begin{remark}
    There is a lot going on the the previous example, but crucially it should be observed that it is in fact the \emph{inversion lemmas} that allow us to make choices of which rules to use. So a type-checking algorithm would have to make choices based on what the inversion lemmas say. We also introduced equalities of types which was brushed over. In general, type equalities are only generated by reflexivity so in a way our equations were lifted to equality of syntax. This gave us a classical equality problem. Since all our syntax are trees, we can easily decide their equality. [[CAN YOU???!!]]
\end{remark}

%% Here are examples I would like to add
%
%    i) \x.x                (Identity)
%   ii) \x.\y.xy            (Application function)
%   iv) (\x.x)(\x.x)        (Doesn't type-check)
%    v) \x.xx               (Doesn't type-check (M combinator (Mockingbird)))
%   vi) \x.\y.((xy)(xy))    (Doesn't type-check)
%  vii) \x.\y.\z.x(yz)      (B combinator (Bluebird) function composition!)
%   ix) \x.\y.y(xy)         (O combinator (Owl))
%    x) \x.\y.\z.x(y,z)     (Curry!)

\subsection{Mockingbird $\lambda x . x x$} % Definitely does not type check M
\subsection{$(\lambda x . x)(\lambda x . x)$} %  MI
\subsection{$\lambda x . \lambda y . (xy)(xy)$} % Does not type check BMB
\subsection{Y-combinator $\lambda x . (\lambda y . x (y y)) (\lambda y . x (y y))$} % Does not type check :O oh no!! no recursion!
\subsection{Function composition $\lambda x . \lambda y . \lambda z . x ( y z)$} % Function comp
\subsection{Owl combinator $\lambda x . \lambda y . \lambda z . y (x y)$} % Type checks ((A->B)->A)->(A->B)->A
\subsection{Currying $\lambda x . \lambda y . \lambda z . x (y, z)$} % Curry
\subsection{Swap $\lambda t . (\snd(t), \fst(t))$}






