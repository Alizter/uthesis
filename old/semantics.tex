\section{Theories and models}

[NOTE: This is a rough outline of what the document ought to look like, not even worthy of being a draft]

[TODO: Find references for these]
\begin{defin}
    A theory asserts data and axioms.
    A model is a particular example of a theory.
\end{defin}

For example a model of "the theory of groups" in the category of sets is simply a group. A model of "the theory of groups" in the category of topological spaces is a topological group. A model of "the theory of groups" in a the category of manifolds is a Lie group.

Categorical semantics is a general procedure to go from "a theory" to the notion of an internal object in some category.

The internal objects of interest is a model of the theory in a category.

Then anything we prove formally about the theory is true for all models of the theory in any category.

For each kind of "type theory" there is a corresponding kind of "structured category" in which we consider models.

\begin{itemize}
    \item Lawvere theories $\leftrightarrow$ Category with finite products
    \item Simply typed lambda calculus $\leftrightarrow$ Cartesian closed category
    \item Dependent type theory $\leftrightarrow$ Locally CC category
\end{itemize}

A doctrine specifies:
 - A collection of type constructors
 - A categorical structure realising these constructors as operations.

Once we fix a doctrine $\mathbb{D}$, then a $\mathbb{D}$-theory specifies "generating" or "axiomatic" types and terms.
A $\mathbb{D}$-category is one processing the specified structure.
A model of a $\mathcal{D}$-theory $T$ in a $\mathcal{D}$-category $C$ realises the types and terms in $T$ as objects and morph isms of $C$.

A finite-product theory is a type theory with unit and Cartesian product as the only type constructors. Plus any number of axioms.

Example:

The theory of magmas has one axiomatic type M, and axiomatic terms $\vdash e : M$ and $x : M, y : M \vdash xy : M$. For monoids and groups we will need equality axioms.

Let $T$ be a finite-product theory, $C$ a category with finite products

A model of $T$ in $C$ assigns:

\begin{enumerate}
\item To each type $A$ in $T$, an object $\llbracket A \rrbracket$ in $C$
\item To each judgement derivable in $T$:
    $$x_1 : A_1, \dots, x_n : A_n \vdash b : B$$
    A morphism in $C$
    $$\llbracket A_1 \rrbracket \times \dots \times \llbracket A_n \rrbracket \xrightarrow{\llbracket b \rrbracket} \llbracket B \rrbracket$$
\item Such that $\llbracket A \times B \rrbracket = \llbracket A \rrbracket \times \llbracket B \rrbracket$ etc.
\end{enumerate}

To define a model of $T$ in $C$, it suffices to interpret the axioms, since they "freely generate" the model.

%% Talk about doctrines

Talk about doctrines

Talk about type theory categories adjunction via syntactic category and complete category. (Syntax-semantics adjunction) Possible to set it up to be an equivalence but may not be needed.

WHY Categorical semantics:
\begin{enumerate}
 \item Proving things in a D-theory means it is valid for models of that D-theory in all categories
 \item We can use type theory to prove things about a category by working in its complete theory (internal language)
 \item We can use category theory to prove things about a type theory by working in its syntactic category.
\end{enumerate}






