\documentclass{article}

\usepackage[english]{babel}

\usepackage[utf8]{inputenc}
\usepackage{bussproofs}
\usepackage{amsmath}
\usepackage{amssymb}
\usepackage{amsthm}

\usepackage{framed}

\usepackage{natbib}
\usepackage{url}
\usepackage{dirtytalk}
\usepackage{pst-node}
\usepackage{tikz-cd}
\usepackage{enumerate}

\usepackage{forest}
\usepackage{mwe}


\title{Introduction to dependent type theory}
\author{Ali Caglayan}

%\bibliographystyle{bath.bst}

\theoremstyle{definition}
\newtheorem{defin}{Definition}[subsection]
\newtheorem{example}[defin]{Example}
\newtheorem{theorem}[defin]{Theorem}
\newtheorem{remark}[defin]{Remark}
\newtheorem{lemma}[defin]{Lemma}

\newcommand{\N}{\mathbb{N}} %% Need to say NN has 0

\newcommand{\catname}[1]{{\normalfont\textbf{#1}}}
\newcommand{\Set}{\catname{Set}}
\newcommand{\Cat}{\catname{Cat}}

\newcommand{\el}{\mathrm{el}}
\newcommand{\op}{\mathrm{op}}
\newcommand{\Ty}{\mathbf{Ty}}

\newcommand{\of}{\mathbf{of}}

%%
%   Common variables
%%

\newcommand{\FV}{\mathrm{FV}}
\newcommand{\A}{\mathbb{A}}
\newcommand{\T}{\mathsf{T}}

\newcommand{\Var}{\mathbf{Var}}
\newcommand{\Dec}{\mathbf{Dec}}
\newcommand{\Con}{\mathbf{Con}}
\newcommand{\Typ}{\mathbf{Type}} %%
\newcommand{\Tm}{\mathbf{Term}} %%%
\newcommand{\Jud}{\mathbf{Jud}}


\newcommand{\Id}{\mathrm{Id}}

%%
%   Type theories
%%

\newcommand{\stcu}{${\boldsymbol \lambda}_\to^{\mathrm{Cu}}$} % simply typed curry 
\newcommand{\stch}{${\boldsymbol \lambda}_\to^{\mathrm{Ch}}$} % simply typed church
\newcommand{\stdb}{${\boldsymbol \lambda}_\to^{\mathrm{dB}}$} % simply typed de brujin
\newcommand{\utbe}{${\boldsymbol {\lambda \beta \eta}}$ } % untyped beta eta

%%
%   Turn styles
%%

\newcommand{\vdashstcu}{\vdash_{\boldsymbol \lambda_\to}^{\mathrm{Cu}}} % vdash for curry
\newcommand{\vdashstcu}{\vdash_{\boldsymbol \lambda_\to}^{\mathrm{Ch}}} % vdash for church
\newcommand{\vdashstcu}{\vdash_{\boldsymbol \lambda_\to}^{\mathrm{dB}}} % vdash for de Brujin

%%
%   Harper notations
%%

\newcommand{\Term}{\mathbf{Term}}
\newcommand{\App}{\mathrm{App}}

\begin{document}

\maketitle

\tableofcontents

%\section{Introduction}

%Simply typed lambda calculus (STLC) has been well documented and studied by type theorists and mathematicians, and it's features have been used by many programming languages [NEED REFERENCE].

%In \cite{BarendregtHenk2013Lcwt} it is noted that \say{Research monographs on dependent and inductive types are lacking.} This will essentially be one of the goals of this thesis, to provide a guide for mathematicians and computer scientists about the use of dependent type theory. As this document is written there is no single account of all approaches to \i{dependent} type theory.

%Awodey \cite{2014arXiv1406.3219A} made an observation that Dybjer's \cite{dybjer1996} categories with families (CwF) is a presheaf category with a representable natural transformation (it's fibers are representable). He then proceeds to show conditions needed to model a dependent type theory with $\Pi$, $\Sigma$ and $\mathrm{Id}$ types.


%This thesis will have three main goals.

%\begin{enumitem}
%	\item To present a dependent type theory
%	\item To model the semantics of such a type theory using categorical methods
%	\item To discuss the applications to mathematics and computer science (proof assistants, programming languages and foundations)
%\end{enumitem}

%Finally we may also discuss recent developments of something called "Homotopy type theory" and how that fits into the general picture.

%Roughly a \textit{type system} is a set of loosely organised rules outlining how ``atomic sentences'' called \textit{judgements} can be derived from each other in a given context. A \textit{context} can simply be thought of as a list of terms. 

%The aim of this thesis is to present to two sorts of audience, the utility of dependent type theory. The audiences that I have in mind are computer scientists, roughly individuals who wish to write good code, and mathematicians, roughly individuals who wish to write good proofs.

%These will be our main aims however we do also wish to develop the machinery formally.

%\section{Propositions as types}

%There is a rich interplay between programming and logic known as the Curry-Howard correspondance or propositions as types. 





%\section{What is type theory}

%Type theory is the study of types systems. That is a system that orginizes data manipulated by programs into types. This has been a very useful concept in computer science. It has allowed the writing of programs taht a more 

%\subsection{Lambda calculus}
%\subsection{Modelling type theory}
%\section{What is dependent type theory?}
%\subsection{What are dependent types?}
%\subsection{Motivation for computer scientists}
%\subsection{Motivation for mathematicians}
%\subsection{Category theory}
%\subsection{Categorical logic}
%\subsection{Future directions}

\begin{itemize}
\item a[Begin with history and implications of curry howard]

\item a[outline the ``what they should do'' of dependent types]

\item a[start to rigoursly model syntax and talk about how bad a job most authors do]

\item a[small section about inductive definitions]

\item a[small section on why categorical semantics]

\item a[model simply typed lambda calculus with categorical semantics]

\item a[show natural extensions of the idea and why contexts break when dependnet]

\item a[outline different approches to solving these problems]

\item a[discuss Awodey's natural models]

\item a[finally talk about future directions for type theory]

\item a[maybe some mention on applications to programming (generalising various constructs, polymorphism, GA data types)]

\item a[equality, inductive types, [[[[[maybe a tinsy bit of homotopy type theory]]]]]]
\end{itemize}

\section{Curry-Howard correspondance}

\subsection{Mathematical logic}

At the beginning of the 20th century, Whitehead and Russell pubished their \emph{Principia Mathematica} \cite{GlossarWiki:Whitehead_Russell:1910}, demonstrating to mathematicians of the time that formal logic could express much of mathematics. It served to popularise modern mathematical logic leading to many mathematicians taking a more serious look at topic such as the foundations of mathematics.

One of the most influencial mathematicians of the time was David Hilbert. Inspired by Whitehead and Russell's vision, Hilbert and his coleagues at G\"ottingen became leading researchers in formal logic. Hilbert proposed the \emph{Entscheidungsproblem} (decision problem), that is, to develop an ``effectually calculable procedure'' to determine the truth or falsehood of any logical statement. At the 1930 Mathematical Congress in K\"onigsberg, Hilbert affirmed his belief in the conjecture, concluding with his famous words ``Wir m\"ussen wissen, wir werden wissen'' (``We must know, we will know''). At the very same conference, Kurt G\"odel announced his proof that arithmetic is incomplete \cite{GlossarWiki:Goedel:1931}, not every statement in arithmetic can be proven.

This however did not deter logicians, who were still interested in understanding why the \emph{Entscheidungsproblem} was undecidable, for this a formal efinition of ``effectively calculable'' was required. So along came three proposed definitions of what it meant to be ``effectively calculable'': \emph{lambda calculus}, pusblished in 1936 by Alonzo Church \cite{church-unsolvableproblemof-1936}; \emph{recursive functions}, proposed by G\"odel in 1934 later published in 1936 by Stephen Kleene \cite{Kleene1936}; and finally \emph{Turing machines} in 1937 by Alan Turing \cite{turing1936a}.

\subsection{Lambda calculus}

(Untyped) lambda calculus was discovered by Church at princeton, originally as a way to define notations for logical formulas. It is a remarkaly compact idea, with only three constructs: variables; lambda abstraction; and function application. It was realised at the time by Church and others that ``There may, indeed, be other applications of the system than its use as a logic.'' [CITATION NEEDED]\cite{}. Church discovered a way of encoding numbers as terms of lambda calculus. From this addition and multiplication could be defined. Kleene later discovered how to define the predecessor function. [CITATION NEEDED] \cite{}. Church later rpoposed $\lambda$-definability as the definition of ``effectively calculable'', what is now known as Church's Thesis, and demonstrated that the problem of determining whether or not a given $\lambda$-term  has a normal form is not $\lambda$-definable. This is now known as the Halting Problem. 

\subsection{Recursive functions}

In 1933 G\"odel arrived in Princeton, unconvinced by Church's claim that every effectively calculable function was $\lambda$-definable. Church responded by offering that if Go\"odel would propose a different definition, then Church would ``undertake to prove it was included in $\lambda$-definability''. In a series of lectures at Princeton, G\"odel proposed what came to be known as ``general recursive functions'' as his candidate for effective calculability. Kleene later published the definition [CITATION NEEDED]\cite{}. Church later outlined a proof [CITATION NEEDED]\cite{} and Kleene later published it in detail. This however did not have the intended effect on G\"odel, whereby he then became convinced that his own definition was incorrect.

\subsection{Turing machines}

Alan Turing was at Camrbdige when he independently formulated his own idea of what it means to be "effectively calculable", now known today as Turing machines. He used it to show that the Entscheidungsproblem is undecidable, that is it cannot be proven to be true or false. Before publication, Turing's advisor Max Newman was worried since Church had published a solution, but since Turing's approach was sufficiently novel it was published anyway. Turing had added an appendix sketching the equivalence of $\lambda$-definability to Turing machines. It was Turings argument that later convinced G\"odel that this was the correct notion of ``effectively calculable''.

\subsection{Russells paradox}

[Talk about the origin of types and stuff]

\subsection{The problem with lambda calculus as a logic}

Church's lambda calculus turned out to be inconsistent. \cite{}[CITATION NEEDED]. The reason was related to russels paradox, in that a predicate was allowed to act on itself. This led to an abandoning of the use of lambda calculus as a logic for a short time. In order to solve this Church adapted a solution similar to Russell's: use types. What was discovered is now known today as \emph{simply-typed lambda calculus}. \cite{} [CITATION NEEDED, 10 ?]. What is nice about Church's STLC is that every term has a normal form, or in the language of Turing machines every computation halts. \cite{} [CITATION NEEDED] From this consistency of Church's STLC as a logic could be established.

\subsection{Types to the rescue}

[Talk in detail why typing is good for mathematicians, programmers and logicians]

\subsection{The theory of proof a la Gentzen}

[Go into the history of the theory of proof e.g. Gentzen's work; take notice of natural deduction]

\subsection{Curry and Howard}

[Curry makes an observation that Gentzens natural deduction corresponds to simply typed lambda calculus, Howard takes this further and defines it formally, eventually predicting a notion of dependent type.

\subsection{Propositions as types}

[Overview of the full nature of the observation, much deeper than a simple correspondance since logic is in some sense ``very correct'' and programming constructs corresponding to these must therefore also be ``very correct''.]

\subsection{Predicates [CHANGE] as types?}

[Talk about predicate quantifiers $\forall, \exists$ and what a ``dependent type ought to do'']


\subsection{Dependent types}

[Perhaps expand on the simply typed section]

[talk about pi and sigma types

[talk about ``dependent contexts'']




%%
% Simply typed lambda calculus
%
\section{Simply typed lambda calculus} 


First develop the features needed. Discuss the arbitrary nature of such features, then use Curry-Howard as motivation for ``the language that ought to be''. Develop STLC, discuss in detail the implications, give categorical semantics. Discuss breifly the dynamics of simply typed lambda calculus. A big disadvantage of STLC over the untyped version (which we ought to discuss since we have the tools to) is that there is no recursion. There are many ways to fix this, see G\"odel for example. In order to fix this we will introduce dependent types.

We begin by discussing the syntax of our type theory. We have a set of types $\mathbf{T}::= $
and a set of terms $\mathbf{T}::=$

\subsection{Judgements}


[[TODO: Clean up this whole paragraph(s)]]
We begin with our basic judgements. Of which there will be 5. Our STLC will have bidirectional typechecking, in that we will distinguish between the direction of type checking. There are several advantages of this and historically the two main systems called STLC are Curry's and Church's which simply differ in the direction of type checking. By having both directions and a sort of ``mode-switching rule'' we have far greater control and ease when describing type checking properties. We will also need to have a notion of \emph{judgemental equality} since we wish to do some computation. There are variations of this theme discussed in the statics chapter that allow us to have transition systems instead but we will use an equational style since transition systems can be derived from this. This also has the advantage of STLC becomming what is known as an ``equational theory''. This will be a useful feature for when we want to derrive categorical semantics. 

A context is a list of basic judgements. Our basic judgements are $x : A$. [[No it is not fix this]]

There are 5 judgements that we have:

\begin{itemize}
    \item $\Gamma \vdash A\ \mathsf{type}$ - ``$A$ is a type in context $\Gamma$''.
    \item $\Gamma \vdash T \Leftarrow A$ - ``$T$ can be checked to have type $A$ in context $\Gamma$''.
    \item $\Gamma \vdash T \Rightarrow A$ - ``$T$ synthesises the type $A$ in context $\Gamma$''.
    \item $\Gamma \vdash A \equiv B\ \mathsf{type}$ - ``$A$ and $B$ are jdugementally equal types in context $\Gamma$''.
    \item $\Gamma \vdash S \equiv T : A$ - ``$S$ and $T$ are judgementally equal terms of type $A$ in context $\Gamma$''.
\end{itemize}

\subsection{Structural rules}

Structural rules will dictate how our judgements interact with eachother, how different contexts can be formed and how substitution works. This is all roughly what a ``type theory'' ought to provide.

We begin with the \emph{variable} rule, this says that if a term $x$ appears with a type $A$ as an element in a context $\Gamma$ then $x$ synthesises a type $A$ in context $\Gamma$. Or written more succiently as:

$$
    \frac{(x:A) \in \Gamma }{\Gamma \vdash x \Rightarrow A}
$$

Other structural rules: weakening, contraction and substitution are all admissible. [[What does it mean for a rule to be admissible? We have defined this previously but we need to carefully state these facts, and prove them too!]]

\subsection{Mode-switching}

One of the features of bidirectional type checking is that we can switch the mode we are in. This is expressed as the mode switching rule:

\begin{prooftree}
    \AxiomC{$\Gamma \vdash T \Rightarrow A$}
    \AxiomC{$\Gamma \vdash A \equiv B \ \mathsf{type}$}
    \BinaryInfC{$\Gamma \vdash T \Leftarrow B$}
\end{prooftree}

This rule has been specially set up in that it will be the \emph{only way} to derive $\Gamma \vdash T \Leftarrow B$. [[TODO: talk more about this]]

In a unidirectional type system, the judgements $\Gamma \vdash T \Rightarrow A$ and $\Gamma \vdash T \Leftarrow B$ are collapsed into one: $\Gamma \vdash T : A$. And now the mode-switching rule may have a more familiar form:

\begin{prooftree}
    \AxiomC{$\Gamma \vdash T : A$}
    \AxiomC{$\Gamma \vdash A \equiv B \ \mathsf{type}$}
    \BinaryInfC{$\Gamma \vdash T : B$}
\end{prooftree}

Which shows that it is actually a rule about substituting along a judgemental equality! But this is a problem since a type checking algorithm will have to decide when to stop doing this. This is one of the big advantages that bidirectional type checking has over unidirectional type checking. The type checking algorithm will be simpler! [[TODO: Clean up and discuss type checking in more detail]]

\subsection{Equality rules}
Finally we have some structural rules for our two judgemental equality judgements. We wish for these to be an equivalence relation and that they are compatible with eachother.

First we begin with the structural rules for the judgement form $- \equiv -\ \mathbf{type}$:

We wish for our judgemental equality of types to be reflexive:
\begin{prooftree}
    \AxiomC{}
    \RightLabel{($\equiv_{\mathbf{type}}$-reflexivity)}
    \UnaryInfC{$\Gamma \vdash A \equiv A\ \mathbf{type}$}
\end{prooftree}

We want our judgemental equality of types to be symmetric:
\begin{prooftree}
    \AxiomC{$\Gamma \vdash A \equiv B \ \mathbf{type}$}
    \RightLabel{($\equiv_{\mathsf{type}}$-symmetry)}
    \UnaryInfC{$\Gamma \vdash B \equiv A \ \mathbf{type}$}
\end{prooftree}

and our judgemental equality of types to be transitive:

\begin{prooftree}
    \AxiomC{$\Gamma \vdash B \ \mathsf{type}$}
    \AxiomC{$\Gamma \vdash A \equiv B\ \mathsf{type}$}
    \AxiomC{$\Gamma \vdash B \equiv C\ \mathsf{type}$}
    \RightLabel{($\equiv_\mathsf{type}$-transitivity)}
    \TrinaryInfC{$\Gamma \vdash A \equiv C\ \mathsf{type}$}
\end{prooftree}

Notice how the previous rule also checks that $B$ is a type. This is because if we did not do this, we could insert any symbol in. This is clearly undesirable. It also demonstrates how subtly sensitive rules are.

Now we list the rules making the judgement form $- \equiv - : A$ into an equivalence relation:

We wish for our judgemental equality of terms to be reflexive:
\begin{prooftree}
    \AxiomC{}
    \RightLabel{($\equiv_{\mathsf{term}}$-reflexivity)}
    \UnaryInfC{$\Gamma \vdash T \equiv T : A$}
\end{prooftree}

We want our judgemental equality of terms to be symmetric:
\begin{prooftree}
    \AxiomC{$\Gamma \vdash S \equiv T : A$}
    \RightLabel{($\equiv_{\mathsf{term}}$-symmetry)}
    \UnaryInfC{$\Gamma \vdash T \equiv S : A$}
\end{prooftree}

and our judgemental equality of terms to be transitive:
\begin{prooftree}
    \AxiomC{$\Gamma \vdash T \Leftarrow A $}
    \AxiomC{$\Gamma \vdash S \equiv T : A$}
    \AxiomC{$\Gamma \vdash T \equiv R : A$}
    \RightLabel{($\equiv_{\mathsf{term}}$-transitivity)}
    \TrinaryInfC{$\Gamma \vdash S \equiv R : A$}
\end{prooftree}

as we stated before for transitivity judgemental equality of types we need to also check that the middle term $T$ is actually a term.

Finally we need a rule that will make  that judgemental equality of types and judgemental equality of terms interact the way we expect them to:
\begin{prooftree}
    \AxiomC{$\Gamma \vdash A \ \mathsf{type}$}
    \AxiomC{$\Gamma \vdash S \equiv T : A$}
    \AxiomC{$\Gamma \vdash A \equiv B\ \mathsf{type}$}
    \RightLabel{($\equiv_{\mathsf{term}}$-$\equiv_{\mathsf{type}}$-compat)}
    \TrinaryInfC{$\Gamma \vdash S \equiv T : B$}
\end{prooftree}


\subsection{Type formers}
What we have constructed thusfar is essentially an ``empty type theory''. What we have included which other authors typcially gloss over is a clean way of constructing a typechecking algorithm: bidirectional typechecking and an account of judgemental equality. We now study what are known as type formers, typically when we wish to add a new type to a type theory we need to think about a collection of rules. These can roughly be sorted into 5 kinds of rules:

\begin{itemize}
    \item Formation rules - How can I construct my type?
    \item Introduction rules - Which terms synthesise this type?
    \item Elimination rules - How can terms of this type be used?
    \item Computation (or equality) rules - How do terms of this type compute? (Normalise, etc.)
    \item Congruence rules - How do all the previous rules interact with judgemental equality
\end{itemize}

We make a note that although we will be providing all the rules, the congruence rules can be typically derrived from the others. Although we do not know exactly how to do this so we will provide them explicitly. We also note that not every type need computation rules.

\subsection{Inversion lemmas}
Having listed all these rules we need some lemmas detailing how different terms can \emph{only} come from a set of specified rules. This is a crucial analysis if we wish to construct a type checking algorithm.

\subsection{$\to$-types}

Building on top of our ``empty type theory'' we introduce $\to$ the function type former:

\subsubsection{Formation rules}

We start with our formation rule which simply states that in order to derive $\Gamma \vdash A \to B \ \mathsf{type}$ it is required to derive that both $\Gamma \vdash A \ \mathsf{type}$ and $\Gamma \vdash B \ \mathsf{type}$:

\begin{prooftree}
    \AxiomC{$\Gamma \vdash A\ \mathsf{type}$}
    \AxiomC{$\Gamma \vdash B\ \mathsf{type}$}
    \RightLabel{}
    \BinaryInfC{$\Gamma \vdash A \to B \ \mathsf{type}$}
\end{prooftree}

\subsubsection{Introduction rules}
\subsubsection{Elimination rules}
\subsubsection{Computation rules}
\subsubsection{Congruence rules}

\begin{comment}
%\subsection{Lambda calculus}
%We recall that there are 3 kinds of expressions in lambda calculus: variables, abstractions and applications. These are defined inductively on themselves. A variable is simply a string of characters from an alphabet. A lambda abstraction looks like $\lambda x.y$ where $x$ is some variable and $y$ is some expression. There are alternate ways of writing this, allowing us to drop the need for naming $x$, for example de Brujin indices. Finally an application is simply the concatenation $ab$ of two expressions $a$ and $b$. We will assume that  This fully describes the syntax of this type theory. We will now introduce some rules that tell us which expressions we can derive from other expressions. Firstly we have $\beta$-reduction which tells us if we have an expression of the form $(\lambda x . y)z$ this can be reduced to an expression where all occurrences of $x$ in $y$ are replaced with the expression $z$. We also have $\alpha$-conversion which I would argue isn't really a rule as naming of variables can be completely avoided in the first place using de Brujin indices or even combinators. \cite{BarendregtHenk2013Lcwt, hottbook}

%\subsection{Contexts}
%In mathematics we work with contexts implicitly. That is there is always an ambient knowledge of what has been defined. Mostly due to the nature of how we read mathematical papers. We can make this explicit using contexts. We will not however, use contexts in our discussion of type theory but we will provide a formal exposition in the appendix.

\subsection{Judgements}
Our judgements:
\begin{center}
    \begin{tabular}{c | c}
        $\Gamma\ \mathrm{ctx}$ &  $\Gamma$ is a well-formed context. \\
        $\Gamma \vdash A\ \mathrm{Type}$ & $A$ is a type in context $\Gamma$. \\
        $\Gamma \vdash x : A$ & $x$ is a term of type $A$ in context $\Gamma$. \\
%        $\Gamma \vdash x \equiv y : A$ & the terms $x$ and $y$ of type $A$ are definitionally equal in context $\Gamma$
    \end{tabular}
\end{center}


Type theory ``will be about'' deriving judgements from other judgements. Which can be concisely summarised in the form of an inference rule

$$\frac{A_1\quad A_2 \quad \cdots \quad A_n}{B}$$

which says that given the judgements $A_1,\dots,A_n$ we can derive the judgement $B$.

\subsection{Structural rules}
We now look at the rules that govern contexts and the structure of our type system.

We begin with a rule stating that the empty context (which as contexts are sets or lists is well-defined) is well-formed. Which is another way of stating that the context was grown in a specified way and is not just an arbitrary list or set of variables.

\begin{prooftree}
    \AxiomC{}
    \RightLabel{empty-ctx}
    \UnaryInfC{$\varnothing$ ctx}
    \singleLine
\end{prooftree}

We also want the concatenation of two well-formed contexts to be well-formed.

\begin{prooftree}
    \AxiomC{$\Gamma$ ctx}
    \AxiomC{$\Delta$ ctx}
    \BinaryInfC{$\Gamma,\Delta$ ctx}
\end{prooftree}

We omit rules about repeating or removing repeated elements and ordering lists (think of them as finite sets).

A variable is a statement of the form $x : A$ where $x$ is known as the term and $A$ its type.

\subsection{Function types}

We introduce a formation rule for the function type.

\begin{prooftree}
    \RightLabel{$(\to)$-form}
    \AxiomC{$\Gamma \vdash A\ \mathrm{Type}$}
    \AxiomC{$\Gamma \vdash B\ \mathrm{Type}$}
    \BinaryInfC{$\Gamma \vdash A \to B\ \mathrm{Type}$}
\end{prooftree}

We now need a rule for producing terms of this new type. We introduce the introduction rule for the function type.

\begin{prooftree}
    \RightLabel{$(\to)$-intro}
    \AxiomC{$\Gamma, x : A \vdash y : B$}
    \UnaryInfC{$\Gamma \vdash (\lambda x . y) : A \to B$}
\end{prooftree}

We will sometimes call this lambda abstraction. We next introduce a way to apply these functions to terms in their domains. We introduce our elimination rule for the function type.

\begin{prooftree}
    \RightLabel{$(\to)$-elim}
    \AxiomC{$\Gamma \vdash f : A \to B$}
    \AxiomC{$\Gamma \vdash a : A$}
    \BinaryInfC{$\Gamma \vdash f(a) : B$}
\end{prooftree}

This is essentially useless unless we have a way to compute (or reduce) this expression. This is where our computation rule comes in. The computation rule will tell us how our elimination rule and introduction rule interact.
\begin{prooftree}
    \RightLabel{$(\to)$-comp}
    \AxiomC{$(\lambda x . y) : A \to B$}
    \AxiomC{$\Gamma, a : A \vdash (\lambda x.y)a : B$}
    \AxiomC{$\Gamma, x : A, y : B, (\lambda x . y) : A \to B, a : A \vdash (\lambda x . y) (a) : B$}
    \UnaryInfC{$\Gamma \vdash y[x / a] : B$}
\end{prooftree}

%%%%%%%%%%%%%%%%%%%%

We will describe what is known as a simply typed lambda calculus. There is a lot of literature on type theory, and it doesn't seem that there are many authors in agreement of ways to present it.

In \cite{BarendregtHenk2013Lcwt} a more type theoretic approach, analysing the type theory mostly in the syntactic world. This gives us a good starting point for how we want our type theory to be presented however it may not be so easy to keep an eye on how the categorical semantics (the ways we model types in mathematics) behave. In order to do this we will use references such as \cite{CroleRoyL1993Cft, JacobsCLTT, LambekJ1986Itho}. This will be from the more categorical logic school of thought, which will study type theory that is "generated" by certain categories in interest.

We start by describing a general class of simple type theories as outlined in \cite{JacobsCLTT}. Firstly we introduce the notion of a {\it signature}. Similar accounts can be found in \cite{CroleRoyL1993Cft}. This will essentially consist of "generating" a category from some signature (which can be thought of as a stripped down type theory syntax), and then studying the functors from that category into other categories. This allows nice properties from the second category to be "pulled back" onto our type theory giving it features we desire.

\begin{defin}
	A {\bf signature} is a pair $(\Typ, \mathcal{F})$ where $\Typ$ is a finite set of {\bf basic} (or {\bf atomic}) {\bf types}. And a functor $\mathcal{F} : \Typ^\star \times \Typ \to \Set$. Where $\Typ^\star$ is the Kleene-Star operation on a set (or the free monoid over $\Typ$), defined as $X^\star := \bigcup_{n\in \N} X^n$ whose elements are finite tuples of elements of $X$ for a set $X$. We have $\mathbf{Set}$ for the category of finite sets. Note that the sets in the domain of the functor are realised as discrete categories.
\end{defin}

We will usually write a signature as $\Sigma := (\Typ, \mathcal{F})$, denote $|\Sigma|:=\Typ$ and write $F: \sigma_1,\dots,\sigma_n\to\sigma_{n+1}$ when $F \in \mathcal{F}(( \sigma_1,\dots,\sigma_n ), \sigma_{n+1})$.

\begin{defin}
    Let $\Var$ be a countable set. Elements $x\in \Var$ are called {\bf variables}.
\end{defin}

Note this style of variables is essentially de Brujin indices. But allows us to have a set of names for our variables, which allows future annoyances like $\alpha$-equivalence to be sorted out easily due to the plentiful existence of bijections from $\Var \to \Var$.

\begin{defin}
	A {\bf variable declaration} is a pair $(x, \sigma) \in \Var \times \Typ$ usually written as $x : \sigma$. This can be read as "the variable $x$ has type $\sigma$. We will define $\Dec:=\Var \times \Typ$.
\end{defin}

\begin{defin}
    A {\bf context} $\Gamma$ is an element of $\Con:=\Dec^\star$. In other words, a context is a finite list of variable declarations. We will usually write a context $\Gamma$ as $v_1 : \sigma_1, \dots ,v_n : \sigma_n$. Note that the Kleene-Star has a monoid structure with operation $","$. We can thus give $\Con$ a monoid structure and write, for contexts $\Gamma$ and $\Delta$ another context $\Gamma,\Delta$ which is the concatenation of two contexts. The notation here allows the "expanded version" to coincide, as in $\Gamma,\Delta$ can be written as $v_1 : \sigma_1, \dots ,v_n : \sigma_n, w_1 : \tau_1, \dots, w_m, \tau_m$.
\end{defin}

We also note that there is a canonical inclusion $\Dec \hookrightarrow \Con$ given that $\Dec$ freely generates the monoid $\Con$. This will allow us to write $\Gamma, x:\tau$ for $v_1 : \sigma_1, \dots ,v_n : \sigma_n, x:\tau$.

We now denote the basic statements of our language. These statements are called {\bf judgements} and we will derive

%%%%%%%%%%%%%%%%%%%%
\end{comment}









%% General type theory stuff

\section{Syntax}

\section{Rules}

\section{Symantics}

\begin{defin}
	Let $C$ be a small category. The {\bf category of elements} $\el(F)$ of a functor 
	$F : C \to \Set$ is the following pullback in \Cat:
	
	\begin{equation}
		\begin{tikzcd}
			%	\el(F) \arrow[rr, "\rho_F"] & & \Set
			\el(F) \arrow[dd, "\pi_F"'] \arrow[rr, "\rho_F"] &  & \Set_* \arrow[dd, "U"] \\
			&  &  \\
			C \arrow[rr, "F"] &  & \Set
		\end{tikzcd}
	\end{equation}

	where $U$ is the forgetful functor from the category of pointed sets 
	$\Set_*$ to the category of sets $\Set$.
\end{defin}
Thus $\el : [C, \Set] \to \Cat$ is a functor. TODO: Prove this.


\begin{defin}
	A {\bf category with families (CwF)} consists of:
	\begin{itemize}
		\item A small category $C$
		\item A terminal object $1 \in C$
		\item Two presheaves $\Tm, \Ty \in [C^\op, \Set]$
		\item A morphism of presheaves $\of : \Tm \to \Ty$
		\item An algebraic representation of the map $\of$ or in other words
		a right adjoint to $\el(\of) : \el(\Tm) \to \el(\Ty)$ (TODO: Don't link 
		this definition so much need rephrasing).
	\end{itemize}
\end{defin}




%% Category theory


\section{Category theory}

\subsection{Introduction}

%Category theory has a pervasive influence throughout mathematics. It is used as an organisational tools, allowing thoughts and ideas about structure and preservation to be expressed in clear, familiar terms. But it is much more than that. Category theory 

%We wish to model dependent types using category theory. In order to do this we will introduce some important category theory, give examples and illustrate how one might go about modelling dependent types. A model is a loose term used to describe the process of finding a mathematical structure, studying how it acts and using this to reason about your desired thing to study. This process can be made much more rigorous than described here, and discussion of the process is really an escapade of mathematical philosophy which we will gloss over for the sake of clarity.

 

We will introduce basic category theory. Good references are: \cite{category, BarrWellsCTCS, MacLaneSaunders1998Cftw,rotman2008introduction}

Category theory will allow us to model the desired behaviour of dependent types. 

% Definition of category
\begin{defin}
	A {\bf category} $\mathcal{C}$ consists of:
	\begin{itemize}
		\item A class $\text{Ob}(\mathcal{C})$ (usually simply denoted $\mathcal{C}$ without ambiguity) of {\bf objects}.
		\item For each object $A,B \in \mathcal{C}$, a set $\mathcal{C}(A,B)$ of \textbf{morphisms} or \textbf{arrows} called a \textbf{homset}. When writing $f \in \mathcal{C}(A,B)$ we usually denote this $f : A \to B$.
		\item For each object $A \in \mathcal{C}$ a morphism $1_A : A \to A$ called the {\bf identity}.
		\item For each object $A,B,C \in \mathcal{C}$, and for each $f : A \to B$ and $g : B \to C$ there is a function (written infix or sometimes simply omitted ($gf \equiv g \circ f$)
		
		$$
			- \circ - : \mathcal{C}(B,C) \times \mathcal{C}(A,B) \to \mathcal{C}(A,C)
		$$
		
		called {\bf composition}.
	\end{itemize}
	
	Such that the following hold:
	
	\begin{itemize}
		\item (Identity) For each $A,B \in \mathcal{C}$ and $f : A \to B$ we have $f \circ 1_A = f$ and $1_B \circ f = f$.
		\item (Associativity) For all $A,B,C,D \in \mathcal{C}$ and $f : A \to B$, $g : B \to C$, $h : C \to D$. We have: $h \circ (g \circ f) = (h \circ g) \circ f$.
	\end{itemize}
\end{defin}

%% remark about collections of morphisms
\begin{remark}
    There are many similar and mostly equivalent definitions of category in mathematics. The mostly fall into two main camps: how they treat their collection of morphisms. The two definitions are equivalent in the usual foundations of mathematics but each has their own advantages. In books such as \cite{riehl2017category} a collection of morphisms is used. This approach lends itself more naturally to the notion of an \textit{internal category} which will be an important concept later on. The other definition uses a family of collections of morphisms which lends itself to easily generalise to the notion of an \textit{enriched category}, the definitive reference for which is \cite{kelly1982basic}.

    The reason it cannot be swept under the rug so easily is because the issue of size is fundamental in category theory. Depending on what definition we chose, it will effect how we can talk about it. For an introduction to category theory, these ideas would mostly confuse the reader, hence we will simply point to \cite{2008arXiv0810.1279S} for a survey on how size issues are treated in category theory. From here on 
\end{remark}

We now give some examples:

% Category of sets
\begin{example}
	The \textbf{category of sets} denoted $\Set$ is the category whose objects are small\footnote{due to Russellian paradoxes we must distinguish between "all sets" and "enough sets". See appendix for details. } sets and morphisms are functions between sets. Composition is given by composition of functions. This is a very important category in category theory for reasons we shall come across later.
\end{example}

Choosing the direction in which our arrows point was arbitrary, but it does also mean that if we had chosen the other way we would also get a category. So every category we make canonically comes with a "friend".

% Opposite category
\begin{example}
	For any category $\mathcal{C}$, there is another category called the {\bf opposite category} $\mathcal{C}^\op$ whose objects are the same as $\mathcal{C}$ however the homsets are defined as follows: $\mathcal{C}^\op(x,y):=\mathcal{C}(y,x)$. Composition is defined using the composition from the original category.
\end{example}

%% Reword this
[NEEDS REWORDING]
Size is a common issue in category theory with many similar ways of dealing with it. It can however cause much confusion and hoop-jumping to be correct. For our purposes we will safely ignore these issues. A formal treatment can be found in the appendix. [TODO: Add formal treatment of size].

\begin{defin}
	We call a category {\bf small} if its class of objects is really a set.
\end{defin}

\begin{defin}
	Let $\mathcal{C},\mathcal{D}$ be categories. A {\bf functor} $F$ from $\mathcal{C}$ to $\mathcal{D}$ (written $F : C \to D$) consists of:
	
	\begin{itemize}
		\item An object $F(A)\in \mathcal{D}$, for all $A \in \mathcal{C}$ (also denoted $FA$).
		\item For each $A,B \in \mathcal{C}$, a function $F_{A,B} : \mathcal{C}}(A,B) \to \mathcal{D}(FA,FB)$ (also denoted $F$).
		\item For each $A \in \mathcal{C}$, $F(1_A) = 1_{FA}$.
		\item For each $A,B,C \in \mathcal{C}$, $f : A \to B$, $g : B \to C$, we have $$F(g \circ f) = F(g)\circ F(f)$$
	\end{itemize}
\end{defin}

\begin{remark}
    Historically in category theory, one would define covariant, as defined above, and contravariant functors, as a result this terminology has crept into uses of category in certain fields [REFERNCE pretty much any homological algebra book before 80s]. Contravariant functors mean to swap the order of composition when the functor is applied. In modern category theory texts, this is completely dropped as a contravariant functor from $\mathcal{C}$ to $\mathcal{D}$ is simply a covaraint functor from $\mathcal{C}^\op$ to $\mathcal{D}$. Henceforth, we shall not mention co(tra)variance of functors and refer to them simply a functors.
\end{remark}

\begin{remark}
    Given two functors $F : \mathcal{C} \to \mathcal{D}$ and $G : \mathcal{D} \to \mathcal{E}$ we can make a new functor $G \circ F$ called its \textbf{composite}, by first applying $F$ then applying $G$ on objects or morphisms. It is simple to check that this is indeed a functor. 
\end{remark}

Now that we have 'morphisms' between categories we can define another category:

\begin{example}
	The category of small categories $\Cat$ has objects small categories and morphisms functors. Composition is given by composition of functors.
\end{example}

%% Talk about natural transformations
\begin{defin}
    [Definition of natural transformation]
\end{defin}

%% Talk about functor categories
\begin{example}
    Given two categories $\mathcal{C}$ and $\mathcal{D}$ we can from a category $[\mathcal{C}, \mathcal{D}]$ called the functor category between $\mathcal{C}$ and $\mathcal{D}$. It's objects are functors $\mathcal{C} \to \mathcal{D}$ and morphisms are natural transformations between functors.
\end{example}

Special cases of this example include:

\begin{example}
    A functor $\mathcal{C}^\op \to \Set$ is typically called a \textbf{presheaf} in geometry and logic. They live in the functor category $[\mathcal{C}^\op, \Set]$ which we will call the \textbf{category of presheaves}. This is an interesting construction as it acts like the category $\mathcal{C}$ in some ways with some nice properties from $\Set$.
\end{example}

% Co(ntra)variant hom-functors

%% Prove yoneda lemma
[CHECK THIS] One of the first theorems that is proven in category theory is the \textbf{Yoneda lemma}. It says if an object acts like a certain object in every possible way, then it must be isomorphic to that object. Akin to how particles are discovered in particle accerlators by observing how they interact when bombarded with differnt particles.

\begin{lemma}
    Let $\mathcal{C}$ be a category. There is an embedding $\mathbf{y} : \mathcal{C} \to [\mathcal{C}^\op, \Set]$. 
    Where $\mathbf{y}(A) := \mathcal{C}( A, - )$, maps each object to its contravariant hom functor. 
    Presheaves that arise this way are called \textbf{representable presheaves}.
\end{lemma}

\begin{remark}
    ![WHAT IS A FULL AND FAITHFUL FUNCTOR?]
    An embedding is a functor that is full and faithful. We haven't actually proven that the "yoneda embedding" is an embedding however this is a corollary of the yoneda lemma which will prove now.
\end{remark}

[PICTURES]

\begin{theorem}{Yoneda lemma}
    Let $\mathcal{C}$ be a category. For all $X \in [\mathcal{C}^\op, \Set]$, there is a natural isomorphism between the following functors:
        $$[\mathcal{C}^\op, \Set](\mathbf{y}(-), X) \cong X(-)$$
\end{theorem}

\begin{remark}
    The set of natural transformations between $\mathbf{y}(A)$ and a presheaf $X$ is bijective to the sections of $X$ at $A$.
\end{remark}



%   There are 3 main corollaries of the yoneda lemma
%   1. The yoneda embedding is an embedding
%   2. Representable objects are unique
%   3. Being a representable object is a universal construction




%Let $\Sigma$ be a signature with $\Typ=|\Sigma|$ as its underlying set of types.

Let $\Var$ be a countable set, elements of which will be called (term) variables.

A variable declaration is a pair $(x, t) \in \Var \times \Typ$ usually written $x:t$.

Given that $\Var$ is countable there is a function $v : \N \to \Var$. Therefore when we write $v_n$ we are specifying the $n$th element of $\Var$. A context is then a sequence of types $\Typ^*$, whose elements $\Gamma = (t_1,\dots, t_n)$, we write like this $\Gamma = v_1:t_1, \dots, v_n:t_n$. Now if we have another context $\Delta =v_1:u_1, \dots, v_m:u_m$, we can concatenate them like this: $\Gamma,\Delta = v_1:t_1,\dots, v_n:t_n, v_{n+1}:u_1,\dots,v_{n+m}: u_m$. So a context is really just a sequence of types, but the index of the sequence also refers to the variable. The set of contexts is called $\Con$.

Judgements are the basic statements or assertions of our theory. We will have starting judgements (perhaps called axioms) whereby we derive other judgements according to rules.

One judgement in this simply typed lambda calculus can be defined as a triple $(\Gamma, t, T) \in \Con \times \Var \times \Typ$. 

Another is a well-formed context, which 

We may also add other kinds of judgements so we will accumulate those in a set called $\Jud$.

An inference rule is a function $\Jud^* \to \Jud$. We will pick these carefully as they will essentially "generate" our type theory.

For simply typed lambda calculus


% induction in set theory

\section{Induction}

We begin by preparing the tools we will use later on. One of these tools will be structural induction. We will instead prove[state?] a vastly more general theorem[principle?] of set theory called the \textit{well-founded induction principal}.

\subsection{Well-founded induction}

This is a standard theorem in set theory [cite set theory books]. For definitions of well-foundedness that have been treated carefully we follow \cite[\S 8]{2018arXiv180805204S}.

\begin{defin}\ 
    \begin{enumerate}[(i)]
        \item A \textbf{graph} is a set $X$, whose elements are called \textbf{nodes}, equipped with a binary relation $\prec$.
        \item If $x \prec y$ then we say $x$ is a \textbf{child} of $y$.
        \item We call a graph \textbf{pointed} if it has a distinguished node $\star$ called the \textbf{root}.
        \item A pointed graph is \textbf{accessible}, if for every $x \in X$, there exists a path $x = x_n \prec \cdots \prec x_0 = \star$.
        \item For any node $x$ of $X$ we write $X/x$ to denote the graph whose nodes are nodes of $X$ that admit a path to $x$. This is naturally pointed by $x$. The relation is the same, simply restricted to the subset $X/x$ of $X$.
    \end{enumerate}
\end{defin}

\begin{remark}
    Note that the definition of accessible relies on a definition of natural numbers. If one was to carry out this construction in a setting more general than sets we would need what is called a \textit{Natural Numbers Object (NNO)}.
\end{remark}

We now make sure that subsets of graphs bring the nodes' parents.

\begin{defin}
    A subset $S$ of a graph $X$ is \textbf{inductive} if for any node $x$ in $X$, when all the children of $x$ are in $S$, $x$ is also in $S$.
\end{defin}

\begin{defin}
    A graph $X$ is \textbf{well-founded} if any inductive subset of $X$ is equal to all of $X$.
\end{defin}

\begin{remark}
    This is slightly weaker than classical versions of well-foundedness in logic, in which a graph would be well-founded if every inhabited subset has a $\prec$-minimal element. In fact such a definition would imply the law of excluded middle.
\end{remark}

\begin{lemma}\label{sswfg}
    Any subset of a well-founded graph is a well-founded graph with the induced relation.
\end{lemma}

We will follow the proof outlined in \cite{winskel1993formal}. This has been the simplest proof to understand. Other proofs can be found in for example \cite{johnstone1987notes} and \cite[Chapter 7]{barwise1982handbook}. However we found these proof too technical and verbose for simple use.

\begin{theorem}[Principle of well-founded induction]
    Let $X$ be a well-founded graph and $P$ be a property of nodes of $X$.
    We have that $$\forall x \in X, P(x) \Leftrightarrow \forall x \in X,((\forall y \prec x, P(y)) \Rightarrow P(x)).$$
\end{theorem}

\begin{proof}
    The proof in the forward direction is an easy tautology. For the converse, we assume  $ \forall x \in X,((\forall y \prec x, P(y)) \Rightarrow P(x))$ and produce a contradiction by supposing $\neg P(x)$ for some $x \in X$. Hence we have a subset $\{y \in X \mid \neg P(y) \} $ which is well-founded by Lemma \ref{sswfg}. [FINISH PROOF]
\end{proof}



% harper

\section{Syntax}

\subsection{Introduction}


We will follow the structure of syntax outlined in Harper \cite{harper_2016}. There are several reasons for this. 

Firstly, for example in Barendregt et. al. \cite{BarendregtHenk2013Lcwt} we have notions of substitution left to the reader under the assumption that they can be fixed. Generally Barendregt's style is like this and even when there is much formalism, it is done in a way that we find peculiar.

In Crole's book \cite{CroleRoyL1993Cft}, syntax is derived from an \textit{algebraic signature} which comes directly from categorical semantics. We want to give an independent view of type theory. The syntax only has types as well, meaning that only terms can be posed in this syntax. Operations on types themselves would have to be handled separately. This will also make it difficult to work with \textit{bound variables}.

In Lambek and Scott's book \cite{LambekJ1986Itho}, very little attention is given to syntax and categorical semantics and deriving type theory from categories for study is in the forefront of their focus.

In Jacob's book \cite{JacobsCLTT}, we again have much reliance on categorical machinery. A variant of algebraic signature called a \mathit{many-typed signature} is given, which has its roots in mathematical logic. Here it is discussed that classically in logic the idea of a sort and a type were synonymous, and they go onto preferring to call them types. This still has the problems identified before as terms and types being treated separately, when it comes to syntax.

In Barendregt's older book \cite{barendregt1984lambda}, there are models of the syntax of (untyped) lambda calculus, using Scott topologies on complete lattices. We acknowledge that this is a working model of the lambda calculus but we believe it to be overly complex for the task at hand. It introduces a lot of mostly irrelevant mathematics for studying the lambda calculus. And we doubt very much that these models will hold up to much modification of the calculus. Typing seems impossible.

In S{\o}rensen and Urzyczyn's book \cite{Sorensen:2006:LCI:1197021} a more classical unstructured approach to syntax is taken. This is very similar to the approaches that Church, Curry and de Brujin gave early on. The difficulty with this approach is that it is very hard to prove things about the syntax. There are many exceptional cases to be weary of (for example if a variable is bound etc.). It can also mean that the syntax is vulnerable to mistakes. We acknowledge it's correctness in this case, however we prefer to use a safer approach.

We will finally look at one more point of view, that of mathematical logic. We look at Troelstra and Schwichtenberg's book \cite{troelstra_schwichtenberg_2000} which studies proof theory. This is essentially the previous style but done to a greater extent, for they use that kind of handling of syntax to argue about more general logics. As before, we do not choose this approach.

We have seen books from either end of the spectrum, on one hand Barendregt's type theoretic camp, and on the other, the more categorical logically oriented camp. We have argued that the categorical logically oriented texts do not do a good job of explaining and defining syntax, their only interest is in their categories. The type theoretic texts also seem to be on mathematically shaky ground, sometimes much is left to the reader and finer details are overlooked.

Harper's seems more sturdy and correct in our opinion. Harper doesn't concern himself with abstraction for the sake of abstraction but rather when it will benefit the way of thinking about something. The framework for working with syntax also seems ideal to work with, when it comes to adding features to a theory (be it a type theory or otherwise).

\subsection{Abstract Syntax Trees}

We begin by outlining what exactly syntax is, and how to work with it. This will be important later on if we want to prove things about our syntax as we will essentially have good data structures to work with.

%We will begin with the notion of an {\it abstract syntax tree}. Which can be what is informally known as syntax, thus formal statements about the syntax are referring to its manifestation as an abstract syntax tree.

%% Sort
\begin{defin}[Sorts]
    Let $\mathcal{S}$ be a finite set, which we will call \textbf{sorts}. An element of $\mathcal{S}$ is called a \textbf {sort}.
\end{defin}

A sort could be a term, a type, a kind or even an expression. It should be thought of an abstract notion of the kind of syntactic element we have. Examples will follow making this clear.

%% Arity
\begin{defin}[Arities]
    An \textbf{arity} is an element $((s_1,\dots,s_n),s)$ of the set of \textbf{arities} $\mathcal{Q}:=\mathcal{S}^\star \times \mathcal{S}$ where $\mathcal{S}^\star$ is the Kleene-star operation on the set $\mathcal{S}$ (a.k.a the free monoid on $\mathcal{S}$ or set of finite tuples of elements of $\mathcal{S}$). An arity is typically written as $(s_1,\dots,s_n)s$. 
\end{defin}

%% Operator
\begin{defin}[Operators]
    Let $\mathcal{O} :=\{ \mathcal{O}_\alpha \}_{\alpha \in \mathcal{Q}}$ be an $\mathcal{Q}$-indexed (arity-indexed) family of disjoint sets of \textbf{operators} for each arity. An element $o \in \mathcal{O}_\alpha$ is called an \textbf{operator} of arity $\alpha$. If $o$ is an operator of arity $(s_1,\dots,s_n)s$ then we say $o$ has \textbf{sort} $s$ and that $o$ has $n$ \textbf{arguments} of sorts $s_1,\dots,s_n$ respectively.
\end{defin}

%% Variables
\begin{defin}[Variables]
    Let $\mathcal{X}:= \{ \mathcal{X}_s\}_{s \in \mathcal{S}}$ be an $\mathcal{S}$-indexed (sort-indexed) family of disjoint (finite?) sets $\mathcal{X}_s$ of \textbf{variables} of sort $s$. An element $x\in\mathcal{X}_s$ is called a \textbf{variable} $x$ of \textbf{sort} $s$. 
\end{defin}

%% Fresh variables
\begin{defin}[Fresh variables]
    We say that $x$ is \textbf{fresh} for $\mathcal{X}$ if $x \not\in \mathcal{X}_s$ for any sort $s\in \mathcal{S}$. Given an $x$ and a sort $s\in \mathcal{S}$ we can form the family $\mathcal{X},x$ of variables by adding $x$ to $\mathcal{X}_s$. 
\end{defin}

%% Remark about notation for adding variables
\begin{remark}
    The notation $\mathcal{X},x$ is ambiguous because the sort $s$ associated to $x$ is not written. But this can be remedied by being clear from the context what the sort of $x$ should be.
\end{remark}

%% Abstract syntax trees
\begin{defin}[Abstract syntax trees]
    The family $\mathcal{A}[\mathcal{X}]=\{ \mathcal{A}[\mathcal{X}]_s \}_{s \in \mathcal{S}}$ of \textbf{abstract syntax trees} (or asts), of \textbf{sort} $s$, is the smallest family satisfying the following properties:
    
    \begin{enumerate}
        \item A variable $x$ of sort $s$ is an ast of sort $s$: if $x \in \mathcal{X}_s$, then $x \in \mathcal{A}[\mathcal{X}]_s$.
        
        \item Operators combine asts: If $o$ is an operator of arity $(s_1, \dots, s_n)s$, and if $a_1 \in \mathcal{A}[\mathcal{X}]_{s_1}, \dots, a_n \in \mathcal{A}[\mathcal{X}]_{s_n}$, then $o(a_1;\dots; a_n) \in \mathcal{A}[\mathcal{X}]_s$.
    \end{enumerate}
\end{defin}

%% Remark about inductively generated sets
\begin{remark}
    The idea of a smallest family satisfying certain properties is that of structural induction. So another way to say this would be a family of sets inductively generated by the following constructors.
\end{remark}

%% Remark about asts being trees
\begin{remark}
    An ast can be thought of as a tree whose leaf nodes are variables and branch nodes are operators. 
\end{remark}

%% Lambda calculus example
\begin{example}[Syntax of lambda calculus]
    The (untyped) lambda calculus has one sort $\Term$, so $\mathcal{S} = \{ \Term \} $. We have an operator $\App$ of application whose arity is $(\Term, \Term)\Term$ and an family of operators $\{\lambda_x \}_{x \in \Var}$ which is the lambda abstraction with bound variable $x$, so $\mathcal{O} = \{ \lambda_x \} \cup \{ \App \} $. The arity of each $\lambda_x$ for some $x \in \Var$ is simply $(\Term) \Term$.
    
    Take the term $$\lambda x . (\lambda y . x y)  z$$

    We can consider this the \textit{sugared} version of our syntax. If we were to \textit{desugar} our term to write it as an ast it would look like this:

    $$
        \lambda_x(\App(\lambda_y(\App(x ; y)); z) )
    $$

    Sugaring allows for long-winded terms to be written more succiently and clearly. Most readers would agree that the former is easier to read. We have mentioned the tree structure of asts so we will illustrate with this example:
    
        \begin{figure}[h]
        \begin{framed}
            \centering
            \begin{minipage}{0.45\textwidth}
                \centering
                \begin{forest}
                    for tree = {
                        %grow = 2,
                        inner sep = 0.1em,
                        l = 0,
                        l sep = 0.7em
                    }
                    [$\lambda_x$ 
                        [$\App$
                            [$\lambda_y$
                                [$\App$
                                    [$x$] 
                                    [$y$]
                                ]
                            ]
                            [$z$]
                        ]
                    ]
                \end{forest}
                \caption{Vertically oriented tree representing the lambda term}
            \end{minipage}
            \hfill
            \begin{minipage}{0.45\textwidth}
                \centering
                \begin{forest}
                    for tree = {
                        grow = 0,
                        inner sep = 0.1em,
                        l sep = 1em
                    }
                    [$\lambda_x$ 
                        [$\App$
                            [$\lambda_y$
                                [$\App$ [$x$] [$y$]
                                ]
                            ]
                            [$z$]
                        ]
                    ]
                \end{forest}
                \caption{Horizontally oriented tree representing the lambda term}
            \end{minipage}
        \end{framed}
    \end{figure}

    
\end{example}

\begin{remark}
    Note that later we will enrich our notion of abstract syntax tree that takes into account binding and scope of variables but for now this is purely structural.
\end{remark}

[ Examples and pictures]


%% Remark about proving things by structural induction on asts
\begin{remark}
    When we prove properties $\mathcal{P}(a)$ of an ast $a$ we can do so by structural induction on the cases above.
\end{remark}

[Some more notes on structural induction, perhaps this can be defined and discussed with trees in the section before?]

%% Plenty of examples of asts with examples of sorts, operators and variables
[add examples of sorts, operators, variables and how they fit together in asts]

%% Lemma about subsets of variables
\begin{lemma}
    If we have $\mathcal{X} \subseteq \mathcal{Y}$ then, $\mathcal{A}[\mathcal{X}] \subseteq \mathcal{A}[\mathcal{Y}]$.
\end{lemma}
\begin{proof}
    Suppose $\mathcal{X} \subseteq \mathcal{Y}$ and $a \in \mathcal{A}[\mathcal{X}]$, now by structural induction on $a$:
    
    \begin{enumerate}
        \item If $a$ is in $\mathcal{X}$ then it is obviously also in $\mathcal{Y}$.
        \item If $a := o(a_1;\dots;a_n) \in \mathcal{A}[\mathcal{X}]$ we have $a_1, \dots, a_n\in \mathcal{A}[\mathcal{X}]$ also. By induction we can assume these to be in $\mathcal{A}[\mathcal{Y}]$ hence giving us $a \in \mathcal{A}[\mathcal{Y}]$.
    \end{enumerate}
    
    Hence by induction we have shown that $\mathcal{A}[\mathcal{X}] \subseteq \mathcal{A}[\mathcal{Y}]$.
\end{proof}

%% Substitution
\begin{defin}[Substitution]
    If $a \in \mathcal{A}[\mathcal{X},x]_{s'}$, and $b \in \mathcal{A}[\mathcal{X}]_s$, then $[b/x]a \in \mathcal{A}[\mathcal{X}]_{s'}$ is the result of \textbf{substituting} $b$ for every occurrence of $x$ in $a$. The ast $a$ is called the \textbf{target}, the variable $x$ is called the \textbf{subject} of the \textbf{substitution}. We define substitution on an ast $a$ by induction:
    \begin{enumerate}
        \item $[b/x]x = b$ and $[b/x]y = y$ if $x\ne y$.
        \item $[b/x]o(a_1;\dots;a_n)=o([b/x]a_1;\dots;[b/x]a_n)$
    \end{enumerate}
\end{defin}

%% Examples of substitution
[Examples of substitution]

\begin{theorem}
    If $a \in \mathcal{A}[\mathcal{X},x]$, then for every $b \in \mathcal{A}[\mathcal{X}]$ there exists a unique $c \in \mathcal{A}[\mathcal{X}]$ such that $[b/x]a = c$.
\end{theorem}
\begin{proof}
    By structural induction on $a$, we have three cases: $a := x$, $a:=y$ where $y \ne x$ and $a := o(a_1; \dots; a_n)$. In the first we have $[b/x]x=b=c$ by definition. In the second we have $[b/x]y=y=c$ by definition. In both cases $c \in \mathcal{A}[\mathcal{X}]$ and are uniquely determined. Finally, when $a := o(a_1; \dots; a_n)$, we have by induction unique $c_1,\dots, c_n$ such that $c_i:=[b/x]a_i$ for $1 \le i \le n$. Hence we have a unique $c=o(c_1,\dots,c_n) \in \mathcal{A}[\mathcal{X}]$.

\end{proof}










%% Barendregt Simply typed lambda calculus

\section{Type theory}

Barendregt \cite{BarendregtHenk2013Lcwt} (or B for short) introduces {\it simply typed lambda calculus} by introducing three versions \stcu, \stch, \stdb.



\subsection{Untyped lambda calculus}

\begin{defin}

    Let $\Var$ (what B calls $\mathsf{V}$) be a set of variables perhaps defined as $\Var := \{x, x', x'', \dots\}$. We will use B's inductive notation and write this as $$\Var::= x \mid \Var '$$
    which is read: elements of $\Var$ are of the form $x$ or an element of $\Var$ with a $'$.

\end{defin}

We then define a set $\Tm$ (what B calls $\Lambda$) of terms (what B calls lambda terms). 

\begin{defin}

    Elements of $\Tm$ are defined as follows $$\Tm ::= \Var \mid \lambda\ \Var\ \Tm \mid \Tm\ \Tm$$
    where a {\bf term} is either a {\bf variable}, a {\bf lambda term} (usually of the form $\lambda x.t$) or an {\bf application} of two terms.

\end{defin}

B goes ahead and eases the notation slightly, which we also do. This is for readability mostly.


\begin{remark}
    We introduce the following notation:

    \begin{enumerate}[(i)]
        \item Letting $x,y,z, \dots, x_0,y_0,z_0, \dots, x_1,y_1,z_1,\dots$ denote arbitrary variables.
        \item $M,N,L,\dots$ denote arbitrary lambda terms (elements of $\Tm$).
        \item Application of terms is left-associative i.e. $(AB)C \equiv ABC$
        \item Nested lambda terms have their inner lambdas dropped i.e. $\lambda x_1.(\lambda x_2.M) \equiv \lambda x_1 x_2.M$. Although this will almost never be used.
    \end{enumerate}

\end{remark}

If we were to choose not to introduce these notational simplifications, it would be very tedious to write all the brackets and not very helpful to the reader.

We will now introduce the notion of a {\bf free variable}.

\begin{defin}
    Let $M \in \Tm$.
    
    \begin{enumerate}[(i)]
        \item The set of {\bf free variables} of $M$, written $\FV(M)$. Variables that are not free are called {\bf bound}.
        \item If $\FV(M) = \varnothing$, then $M$ is called {\bf closed} or a {\bf combinator}. The set of combinators can be written as $$\Tm^\varnothing = \{ M \in \Tm \mid \FV(M) = \varnothing\}$$
    \end{enumerate}
    We can define $\FV : \Tm \to P(\Var)$ by induction on $M$ which we can do due to the inductive definition of $\Tm$. So we have three cases:
    \[\begin{aligned}    
        &M \equiv x ,  &\FV(M) &:= \{x\} \\
        &M \equiv \lambda x . N,  &\FV(M)&:= \FV(N) - \{ x \} \\
        &M \equiv N L,  &\FV(M) &:= \FV(N) \cup \FV(L)
    \end{aligned}\]
\end{defin}

\begin{example}
Some well known combinators are $\mathbf{I} :\equiv \lambda x . x$, $\mathbf{K} :\equiv \lambda x y .y$ and $\mathbf{S}:\equiv \lambda x y z . xz(yz)$. These are well studied however we will not discuss them much here. For a comprehensive study of various combinators and their uses see \cite{smullyan2012mock}.
\end{example}

We now define (untyped) lambda calculus. B does this by defining what they call an equational theory on $\Tm$. This is where the calculus has a notion of equality. We will simply say that this equality is an equality from the metatheory (the logic used to define the calculus). For all intents and purposes our logic is first order logic with ZFC. Although it is very unlikely we will use choice anywhere.

\begin{defin}
    The symbol $\equiv$ denotes the equality in the metatheory. This will have all the usual properties of an equivalence relation and also play nicely with our terms. For example $M \equiv N \implies \lambda x . M \equiv \lambda x . N$.
\end{defin}

This means that we will not have to define properties like reflexivity and transitivity as they essentially come for free from our metatheory. This also has the advantage that we can comfotably add equalities (forcing two things to be equal) without having to define it in our calculus.

\begin{defin}(Term substitution)
    We define term substituion by induction on $\Term$. Let $a,b \in \Var$ and $M,N,K \in \Tm$
    \[\begin{aligned}
        a[a := M] &:\equiv M \\
        b[a := M] &:\equiv b, \qquad b \not\equiv a \\
        (K N)[a := M] &:\equiv (K[a:=M])(N[a:=M]) \\
        (\lambda x . N)[a := M] &:\equiv \lambda (x[a := M]) . (N[a := M])
    \end{aligned}\]
    
    Note in the last case if $M$ is not a variable and replaces $x$ we may get a nonsense term. We will explicitly disallow this but it doesn't matter too much.
\end{defin}

We go onto define \utbe  as the terms $\Tm$ modulo the equivalence relation of the equality in our metatheory. To which we will add the following equalities:

\[(\lambda x . M) N \equiv M[x := N]\tag{$\boldsymbol \beta$-rule}\]
\[\lambda x . M x \equiv M, \quad x \not\in \FV(M)\tag{$\boldsymbol \eta$-rule}\]

Note that when we write terms from now on we are really talking about the representative of the equivalence class of terms in the set of terms modulo our definitional equality. B talks about reductive theories where we have essentially inference rules giving 

\begin{prooftree}
    \RightLabel{($\boldsymbol \beta$)}
    \AxiomC{$(\lambda x . M) N$}
    \UnaryInfC{$ M[x := N]$}
\DisplayProof
    \hskip 5em
    \RightLabel{($\boldsymbol \eta$)\quad $x \not\in\FV(M)$}
    \AxiomC{$\lambda x . M x$}
    \UnaryInfC{$ M$}
\end{prooftree}

\begin{remark}
It is here that B talks about $\alpha$-equivalence. We will go ahead and do the same by adding in equalities for $\alpha$-conversion of terms. Thus our terms modulo definitional equality will be up to $\alpha$-equivalence too.
\end{remark}

\begin{remark}
B also talks about properties of the reduction defined such as satisfaction of the Church-Rosser theorem. This is not entirely relevent here but may be important that it holds.
\end{remark}

[TODO: REMOVE]
\begin{remark} 
    We used the notation $M[x:= N]$ which means take all (free) occurances of the variable $x$ inside the term $M$ and replace it with the term $N$. Note it isn't clear what $(\lambda x . t)[x := N]$ should do when $N$ is a term, however we will try to restrict it's use for free variables only. We also note that in other literature this is written $M[x/N]$ and perhaps sometimes confusingly $M[N/x]$. We will occasionally use the former, but never the latter. It should be read: ``replace every free occurance of $x$ with the term $N$". [TODO: Perhaps make this a definition?? There are many problems with this as it stands, I would hardly think this is a remark.]
\end{remark}

\subsection{Simple types}

So far we have been working in untyped lambda calculus, which in itself has been the basis of many functional programming languages. However for our purposes we could argue it is uninteresting.

We will now try to classify our terms in such a way that we assign a type to them. Then we will restrict our lambda terms' applicability by checking the type. This may seem restrictive but it is a very useful notion that will be prevelent in the theory to come.

\begin{defin}

Let $\A$ be a set. An element of $\A$ is called an {\bf type atom}. The set of {\bf simple types} over $\A$, written $\Typ$ is defined by induction as follows:

\[
\Typ ::= \A \mid \Typ \to \Typ
\]

that is a simple type is either an atomic type or a {\bf function type} between two atomic types. We define $\to$ to be an operator in the definition. It can be said that $\Typ$ is a tree, with leaves type atoms and braches the operator $\to$ pointing to its arguments. The operator $\to$ will be right-associative, i.e. $A \to (B \to C)\equiv A \to B \to C$.
\end{defin}


\begin{example}
    We give the following examples of type atoms:
    \begin{enumerate}[(i)]
        \item Let $\A_0:=\{\T\}$. Then the set of simple types $\Typ$ over $\A_0$  consiststs of the following elements: 
        $$\Typ = \{ \T,\ \T \to \T,\ \T \to \T \to \T,\ (\T \to \T) \to \T,\ \dots \}$$
        
        \item Let $\A_\infty:=\{\T_1, \T_2, \T_3, \dots\}$ be a countable set. Then the set of simple types over $\A_\infty$ is also countable. Note for any natural number $n$ we can count trees with depth $n$. Then we take the union of all these sets of trees. Giving us a countably indexed union of countable sets, hence countable. This may be a useful fact if we are to iterate over types in the future. [TODO: Rephrase this argument]
        
        \item Let $\A_\varnothing:= \varnothing$. Clearly $\Typ = \varnothing$. This isn't a very interesting collection of types.
    \end{enumerate}
\end{example}

\begin{remark}
    We will usually reference types as upper case variables and perhaps greek letters for type atoms. And terms as lower case variables. It should be clear from the context which is which. When we write $\A$ we mean a generic set of type atoms.
\end{remark}

\begin{defin}
Let $\alhpa,\beta \in \A$ and $A,B,C \in \Typ$. The operation of type substitution is defined by induction on $\Typ$ as follows:

\[\begin{aligned}
    \alpha[\alpha:=A] &:\equiv A \\
    \beta[\alpha:=A] &:\equiv \beta,\quad \beta\not\equiv \alpha \\
    (A \to B)[\alpha:=C] &:\equiv (A[\alpha:=C]) \to (B[\alpha:=C])
\end{aligned}\]


\end{defin}

We now explore three different versions of simply typed lambda calculus, \stcu, \stch and \stdb. Note that the definitions in each section will be local as we will use the same words but subtly change definitions. After we introduce these we will choose one system to develop on.


\subsubsection{Curry style simply typed lambda calculus}

\begin{defin}
    We now define \stcu
    \begin{enumerate}[(i)]
        \item A {\bf(type assignment) statement} is of the form $a : A$ with $a \in \Tm$ and $A \in \Typ$. The type $A$ is called the {\bf predicate} and the term $a$ is called the subect of the statement.
        
        \item A {\bf(typing) decleration} is a statement with subject a term variable.
        
        \item A {\bf basis} [context/well-formed context?] is a set of declerations with {\it distinct} variables as subjects.
        
        \item A statement $a:A$ is {\bf derivable} from a basis $\Gamma$, written $$\Gamma \vdashstcu a:A $$ (usually just $\vdash$ when it is obvious) if $\Gamma \vdash a : A$ can be produced by the inference rules listed in Figure \ref{fig:curryrules}. Note $\Gamma , x:A$ denotes concatenation which will be defined later.
    \end{enumerate}
\end{defin}

\begin{figure}

    \begin{framed}
        \begin{prooftree}
            \RightLabel{\  if $(a:A) \in \Gamma$}
            \AxiomC{}
            \UnaryInfC{$\Gamma \vdash a : A$}
        \end{prooftree}

        \begin{prooftree}
            \RightLabel{($\to$-elimination)}
            \AxiomC{$\Gamma \vdash f : A \to B$}
            \AxiomC{$\Gamma \vdash x : A$}
            \BinaryInfC{$\Gamma \vdash f x : B$}
        \end{prooftree}

        \begin{prooftree}
            \RightLabel{($\to$-introduction)}
            \AxiomC{$\Gamma, x : A \vdash y : B$}
            \UnaryInfC{$\Gamma \vdash (\lambda x . y) : A \to B$}
        \end{prooftree}

        \caption{Inference rules of \stcu \label{fig:curryrules}}
    \end{framed}
\end{figure}














\bibliographystyle{plain} 
\bibliography{uthesis}


\end{document}

%
%   Plan for writing this document
%  
%   1. Complete bibliography and use
%   2. Introduce simply typed lambda calculus (similar in style to the way John wrote it for his course notes)
%   3. Lay out general outline of how new types are added
%   4. Talk about different ways to introduce dependent types
%   5. introduce universes
%   6. Prefer the type families/univereses approach
%   7. Talk about type families
%   8. Introduce Pi types
%   9. Examples of Pi types
%   10. Introduce Sigma types
%   11. Examples of Sigma types
%   12. How they interact together
%   13. Relationship with logic
%   14. Curry-Howard correspondance
%   15. Inductive types
%   16. Examples, uses, general syntax, implementation (W-types)
%   17. Identity types
%   18. Discuss the induction principle on the identity type and axiom K etc.
%   19. Consider no restriction of axiom K
%   20. Give Identity type as inductive definition
%   21. Say that what we have now is Martin-Lof intensional type theory, reference all these papers about its semantics
% ~ ~ ~ ~ ~ Experimental territory / Could be a masters? ~ ~ ~ ~ ~
%   22. Consider functional extensionality
%   23. Consider univalence with examples
%   24. Introduce higher inductive types
%   25. Examples of HITs with reference to semantics by Schulman-Lumsdaine
%   26. Discuss HoTT, what has been done (in algebraic topology etc.) its usefulness as a language for mathematics
%   27. Category theory in HoTT
%   28. Survey of synthetic homotopy theory
%   the list goes on...



