\documentclass{article}

%%
%% Package includes to provide the basic style
%%
%\usepackage{harvard}    % Uses harvard style referencing
%\usepackage{graphicx}   % Permits import of various graphics formats
%\usepackage{hyperref}   % Provides hyperlinks to sections automatically
%\usepackage{pdflscape}  % Provides landscape mode for end code listings
%\usepackage{multicol}   % Provides ability to split output into columns
%\usepackage{listings}   % Provides styled code listings


%%
%% Set some page size changes from the standard article class
%%
\usepackage[inner=4cm,outer=4cm]{geometry}
%\usepackage{calc}
%\setlength{\parskip}{6pt}
%\setlength{\parindent}{0pt}
%\addtolength{\hoffset}{-0.5cm}
%\addtolength{\textwidth}{2.5cm}


%%
%% Format definitions for the style
%%
%\bibliographystyle{agsm}  %{alpha}
%\citationstyle{dcu}
\pagestyle{headings}
\fussy


%%
%% Definitions to provide layout in the dissertation title pages
%%
\newenvironment{spaced}[1]
  {\begin{minipage}[c]{\textwidth}\vspace{#1}}
  {\end{minipage}}


\newenvironment{centrespaced}[2]
  {\begin{center}\begin{minipage}[c]{#1}\vspace{#2}}
  {\end{minipage}\end{center}}


\newcommand{\declaration}[2]{
  \thispagestyle{empty}
  \begin{spaced}{4em}
    \begin{center}
      \LARGE\textbf{#1}
    \end{center}
  \end{spaced}
  \begin{spaced}{3em}
    \begin{center}
      Submitted by: #2
    \end{center}
  \end{spaced}
  \begin{spaced}{5em}
    \section*{COPYRIGHT}

    Attention is drawn to the fact that copyright of this dissertation rests
    with its author. The Intellectual Property Rights of the products
    produced as part of the project belong to the author unless otherwise specified
    below, in accordance with the University of Bath's policy on intellectual property 
   (see http://www.bath.ac.uk/ordinances/22.pdf).

    This copy of the dissertation has been supplied on condition that anyone
    who consults it is understood to recognise that its copyright rests with its
    author and that no quotation from the dissertation and no information
    derived from it may be published without the prior written consent of
    the author.

    \section*{Declaration}
    This dissertation is submitted to the University of Bath in accordance
    with the requirements of the degree of Bachelor of Science in the
    Department of Computer Science. No portion of the work in this dissertation
    has been submitted in support of an application for any other degree
    or qualification of this or any other university or institution of learning.
    Except where specifically acknowledged, it is the work of the author.
  \end{spaced}

  \begin{spaced}{5em}
    Signed:
  \end{spaced}
  }


\newcommand{\consultation}[1]{%
\thispagestyle{empty}
\begin{centrespaced}{0.8\textwidth}{0.4\textheight}
\ifnum #1 = 0
This dissertation may be made available for consultation within the
University Library and may be photocopied or lent to other libraries
for the purposes of consultation.
\else
This dissertation may not be consulted, photocopied or lent to other
libraries without the permission of the author for #1 
\ifnum #1 = 1
year
\else
years
\fi
from the date of submission of the dissertation.
\fi
\vspace{4em}

Signed:
\end{centrespaced}
}

%%
%% END OF DEFINITIONS
%%

    %% These are the includes required for the doc 

\usepackage[english]{babel}
\usepackage[utf8]{inputenc}
\usepackage{bussproofs}
\usepackage{framed}

% Here we squish our bibliography up
\usepackage{natbib}
\setlength{\bibsep}{1pt}
\renewcommand{\bibfont}{\small} 

% Todo purge useless packages
\usepackage{url}
\usepackage{dirtytalk}
\usepackage{pst-node}
\usepackage{tikz-cd}

\usepackage[shortlabels]{enumitem}
%\usepackage{enumerate}

\usepackage{forest}
\usepackage{mwe}
\usepackage{verbatim}
\usepackage{amsmath}
\usepackage{amssymb}
\usepackage{amsthm}
\usepackage{stmaryrd}
\usepackage[hidelinks]{hyperref}
\usepackage{pdflscape}

\usepackage[framemethod=tikz]{mdframed}
%\usepackage{tcolorbox}

\usepackage{silence}
\WarningFilter{mdframed}{You got a bad break}
\makeatletter
\mdf@PackageWarning{You got a bad break\MessageBreak
  because the last split box is empty\MessageBreak
  You have to change the settings}
\makeatother


\usepackage[toc,page]{appendix}
\usepackage[nottoc,numbib]{tocbibind}
% Ommiting this because it looks terrible
%\bibliographystyle{bath.bst}



% Theorems, definitions etc.
\theoremstyle{definition}
\newtheorem{defi}{Definition}[subsection]
\newtheorem{example}[defi]{Example}
\newtheorem{theorem}[defi]{Theorem}
\newtheorem{remark}[defi]{Remark}
\newtheorem{lemma}[defi]{Lemma}
\newtheorem{cor}[defi]{Corollary}

\newenvironment{defin}{\begin{mdframed}\begin{defi}}{\end{defi}\end{mdframed}}

\newcommand{\N}{\mathbb{N}} %% Need to say NN has 0


%%
%   Common variables
%%

\newcommand{\Var}{\mathbf{Var}}

\newcommand{\Id}{\text{Id}}

%%
%   Harper notations
%%

\newcommand{\Term}{\mathbf{Term}}
\newcommand{\App}{\text{App}}
\newcommand{\scope}{\triangleleft}
\newcommand{\soa}{\text{soa}}
\newcommand{\sov}{\text{sov}}

% stlc
\newcommand{\tm}{\mathrm{tm}}
\newcommand{\ty}{\mathrm{ty}}
\newcommand{\fst}{\mathrm{fst}}
\newcommand{\snd}{\mathrm{snd}}
\newcommand{\inl}{\mathrm{inl}}
\newcommand{\inr}{\mathrm{inr}}
\newcommand{\indz}{\mathbf{ind}_{\mathbf{0}}}
\newcommand{\indp}{\mathbf{ind}_{+}}
\newcommand{\indn}{\mathbf{ind}_\N}

% Type theories
\newcommand{\lc}{\lambda_{\to}}
\newcommand{\stlc}{$\lambda_{\to \times}$}
\newcommand{\stlct}{\lambda_{\to \times \N}}
\newcommand{\dtt}{\lambda\Pi_{\times}}

%% Multicol bib
\begin{comment}
\usepackage{multicol}

\makeatletter
\renewenvironment{thebibliography}[1]
     {\begin{multicols}{2}[\section*{\refname}]%
      \@mkboth{\MakeUppercase\refname}{\MakeUppercase\refname}%
      \list{\@biblabel{\@arabic\c@enumiv}}%
           {\settowidth\labelwidth{\@biblabel{#1}}%
            \leftmargin\labelwidth
            \advance\leftmargin\labelsep
            \@openbib@code
            \usecounter{enumiv}%
            \let\p@enumiv\@empty
            \renewcommand\theenumiv{\@arabic\c@enumiv}}%
      \sloppy
      \clubpenalty4000
      \@clubpenalty \clubpenalty
      \widowpenalty4000%
      \sfcode`\.\@m}
     {\def\@noitemerr
       {\@latex@warning{Empty `thebibliography' environment}}%
      \endlist\end{multicols}}
\makeatother
\end{comment}

\title{Simply Typed Lambda Calculus and the Curry-Howard Correspondence}
\author{Ali Caglayan}
\date{Bachelor of Science in Computer Science and Mathematics with Honours\\The University of Bath\\May 2019}


%\showoutput
\begin{document}

%\maketitle
%\input{title.tex}
\setcounter{page}{0}
\pagenumbering{roman}


\maketitle
\newpage


% Set this to the number of years consultation prohibition, or 0 if no limit
\consultation{0}
\newpage

\declaration{Simply Typed Lambda Calculus and the Curry-Howard Correspondence}{Ali Caglayan}
\newpage

\begin{abstract}
    The goal of this dissertation is to give an introduction to the formal study of lambda calculus and type theory. We begin by analysing the intuitive notion of \emph{syntax}, highlighting the many subtleties associated with it. We discuss possible solutions to these issues, but ultimately remark that it is very difficult to be certain of correctness. We will however give a notion of syntax which is ``correct enough'' for our purposes.

The next section is to discuss the formality of \emph{judgements}. This is a concept oft overlooked in the study of type theory. We will give a careful and detailed account of derivability and admissibility. We will also remark on inconsistencies of the treatment of certain concepts.

This will lead us into studying the \emph{simply type lambda calculus} (STLC), in some ways one of the simplest (functional) programming languages. We will give syntax, judgements and rules governing its semantics. After which, we will prove meta properties about our type theory and discuss the notion of \emph{type checking}.

We will then analyse the dynamics of the STLC. There is a long history of normalisation results we wish to briefly sketch. We will set up some machinery to prove some of these results. Finally we will discuss notions of canonicity and what these results mean for the design of programming languages.

Next there will be several examples of terms to be type checked. This will show the intricacies that go into designing a type checker. We will see that typing makes lambda calculus much weaker, in that many terms from the untyped lambda calculus cannot be typed. It is precisely these terms which gave the computational power of the untyped lambda calculus to begin with.

We will sketch some modifications to the simply typed lambda calculus that will give us certain desired features. We will show how these can be designed and discuss their normalisation results too.

Finally we will give a detailed account of the ideas that went in to, what is now known as the \emph{Curry-Howard} correspondence. This is a very deep package of ideas with far reaching consequences, of which we will try to make account of.

Our closing remarks will be about future directions in type theory, questions that need to be answered and future of programming language design.
\end{abstract}
\newpage

\tableofcontents
\newpage

\section*{Acknowledgements}
I would like to thank my advisor John Power for his guidance and support. For teaching me how to be a mathematician, how to think carefully and how to do category theory correctly. I think my parents for being very patient and supportive of me whilst writing this dissertation.

\newpage

\setcounter{page}{1}
\pagenumbering{arabic}

% Introduction and direction of thesis
\section{Introduction}

%Simply typed lambda calculus (STLC) has been well documented and studied by type theorists and mathematicians, and it's features have been used by many programming languages [NEED REFERENCE].

%In \cite{BarendregtHenk2013Lcwt} it is noted that \say{Research monographs on dependent and inductive types are lacking.} This will essentially be one of the goals of this thesis, to provide a guide for mathematicians and computer scientists about the use of dependent type theory. As this document is written there is no single account of all approaches to \i{dependent} type theory.

%Awodey \cite{2014arXiv1406.3219A} made an observation that Dybjer's \cite{dybjer1996} categories with families (CwF) is a presheaf category with a representable natural transformation (it's fibers are representable). He then proceeds to show conditions needed to model a dependent type theory with $\Pi$, $\Sigma$ and $\mathrm{Id}$ types.


%This thesis will have three main goals.

%\begin{enumitem}
%	\item To present a dependent type theory
%	\item To model the semantics of such a type theory using categorical methods
%	\item To discuss the applications to mathematics and computer science (proof assistants, programming languages and foundations)
%\end{enumitem}

%Finally we may also discuss recent developments of something called "Homotopy type theory" and how that fits into the general picture.

%Roughly a \textit{type system} is a set of loosely organised rules outlining how ``atomic sentences'' called \textit{judgements} can be derived from each other in a given context. A \textit{context} can simply be thought of as a list of terms. 

%The aim of this thesis is to present to two sorts of audience, the utility of dependent type theory. The audiences that I have in mind are computer scientists, roughly individuals who wish to write good code, and mathematicians, roughly individuals who wish to write good proofs.

%These will be our main aims however we do also wish to develop the machinery formally.

%\section{Propositions as types}

%There is a rich interplay between programming and logic known as the Curry-Howard correspondance or propositions as types. 





%\section{What is type theory}

%Type theory is the study of types systems. That is a system that orginizes data manipulated by programs into types. This has been a very useful concept in computer science. It has allowed the writing of programs taht a more 

%\subsection{Lambda calculus}
%\subsection{Modelling type theory}
%\section{What is dependent type theory?}
%\subsection{What are dependent types?}
%\subsection{Motivation for computer scientists}
%\subsection{Motivation for mathematicians}
%\subsection{Category theory}
%\subsection{Categorical logic}
%\subsection{Future directions}

\begin{itemize}
\item a[Begin with history and implications of curry howard]

\item a[outline the ``what they should do'' of dependent types]

\item a[start to rigoursly model syntax and talk about how bad a job most authors do]

\item a[small section about inductive definitions]

\item a[small section on why categorical semantics]

\item a[model simply typed lambda calculus with categorical semantics]

\item a[show natural extensions of the idea and why contexts break when dependnet]

\item a[outline different approches to solving these problems]

\item a[discuss Awodey's natural models]

\item a[finally talk about future directions for type theory]

\item a[maybe some mention on applications to programming (generalising various constructs, polymorphism, GA data types)]

\item a[equality, inductive types, [[[[[maybe a tinsy bit of homotopy type theory]]]]]]
\end{itemize}

\section{Curry-Howard correspondance}

\subsection{Mathematical logic}

At the beginning of the 20th century, Whitehead and Russell pubished their \emph{Principia Mathematica} \cite{GlossarWiki:Whitehead_Russell:1910}, demonstrating to mathematicians of the time that formal logic could express much of mathematics. It served to popularise modern mathematical logic leading to many mathematicians taking a more serious look at topic such as the foundations of mathematics.

One of the most influencial mathematicians of the time was David Hilbert. Inspired by Whitehead and Russell's vision, Hilbert and his coleagues at G\"ottingen became leading researchers in formal logic. Hilbert proposed the \emph{Entscheidungsproblem} (decision problem), that is, to develop an ``effectually calculable procedure'' to determine the truth or falsehood of any logical statement. At the 1930 Mathematical Congress in K\"onigsberg, Hilbert affirmed his belief in the conjecture, concluding with his famous words ``Wir m\"ussen wissen, wir werden wissen'' (``We must know, we will know''). At the very same conference, Kurt G\"odel announced his proof that arithmetic is incomplete \cite{GlossarWiki:Goedel:1931}, not every statement in arithmetic can be proven.

This however did not deter logicians, who were still interested in understanding why the \emph{Entscheidungsproblem} was undecidable, for this a formal efinition of ``effectively calculable'' was required. So along came three proposed definitions of what it meant to be ``effectively calculable'': \emph{lambda calculus}, pusblished in 1936 by Alonzo Church \cite{church-unsolvableproblemof-1936}; \emph{recursive functions}, proposed by G\"odel in 1934 later published in 1936 by Stephen Kleene \cite{Kleene1936}; and finally \emph{Turing machines} in 1937 by Alan Turing \cite{turing1936a}.

\subsection{Lambda calculus}

(Untyped) lambda calculus was discovered by Church at princeton, originally as a way to define notations for logical formulas. It is a remarkaly compact idea, with only three constructs: variables; lambda abstraction; and function application. It was realised at the time by Church and others that ``There may, indeed, be other applications of the system than its use as a logic.'' [CITATION NEEDED]\cite{}. Church discovered a way of encoding numbers as terms of lambda calculus. From this addition and multiplication could be defined. Kleene later discovered how to define the predecessor function. [CITATION NEEDED] \cite{}. Church later rpoposed $\lambda$-definability as the definition of ``effectively calculable'', what is now known as Church's Thesis, and demonstrated that the problem of determining whether or not a given $\lambda$-term  has a normal form is not $\lambda$-definable. This is now known as the Halting Problem. 

\subsection{Recursive functions}

In 1933 G\"odel arrived in Princeton, unconvinced by Church's claim that every effectively calculable function was $\lambda$-definable. Church responded by offering that if Go\"odel would propose a different definition, then Church would ``undertake to prove it was included in $\lambda$-definability''. In a series of lectures at Princeton, G\"odel proposed what came to be known as ``general recursive functions'' as his candidate for effective calculability. Kleene later published the definition [CITATION NEEDED]\cite{}. Church later outlined a proof [CITATION NEEDED]\cite{} and Kleene later published it in detail. This however did not have the intended effect on G\"odel, whereby he then became convinced that his own definition was incorrect.

\subsection{Turing machines}

Alan Turing was at Camrbdige when he independently formulated his own idea of what it means to be "effectively calculable", now known today as Turing machines. He used it to show that the Entscheidungsproblem is undecidable, that is it cannot be proven to be true or false. Before publication, Turing's advisor Max Newman was worried since Church had published a solution, but since Turing's approach was sufficiently novel it was published anyway. Turing had added an appendix sketching the equivalence of $\lambda$-definability to Turing machines. It was Turings argument that later convinced G\"odel that this was the correct notion of ``effectively calculable''.

\subsection{Russells paradox}

[Talk about the origin of types and stuff]

\subsection{The problem with lambda calculus as a logic}

Church's lambda calculus turned out to be inconsistent. \cite{}[CITATION NEEDED]. The reason was related to russels paradox, in that a predicate was allowed to act on itself. This led to an abandoning of the use of lambda calculus as a logic for a short time. In order to solve this Church adapted a solution similar to Russell's: use types. What was discovered is now known today as \emph{simply-typed lambda calculus}. \cite{} [CITATION NEEDED, 10 ?]. What is nice about Church's STLC is that every term has a normal form, or in the language of Turing machines every computation halts. \cite{} [CITATION NEEDED] From this consistency of Church's STLC as a logic could be established.

\subsection{Types to the rescue}

[Talk in detail why typing is good for mathematicians, programmers and logicians]

\subsection{The theory of proof a la Gentzen}

[Go into the history of the theory of proof e.g. Gentzen's work; take notice of natural deduction]

\subsection{Curry and Howard}

[Curry makes an observation that Gentzens natural deduction corresponds to simply typed lambda calculus, Howard takes this further and defines it formally, eventually predicting a notion of dependent type.

\subsection{Propositions as types}

[Overview of the full nature of the observation, much deeper than a simple correspondance since logic is in some sense ``very correct'' and programming constructs corresponding to these must therefore also be ``very correct''.]

\subsection{Predicates [CHANGE] as types?}

[Talk about predicate quantifiers $\forall, \exists$ and what a ``dependent type ought to do'']


\subsection{Dependent types}

[Perhaps expand on the simply typed section]

[talk about pi and sigma types

[talk about ``dependent contexts'']




% Rigourous treatement and analysis of syntax
\section{Syntax}

\subsection{Introduction}

\subsection{Abstract binding trees}

% Sorts
\begin{defin}[Sorts]
    Let $\mathcal{S}$ be a finite set of elements called \emph{sorts}.
\end{defin}

% Arity
\begin{defin}[Arity]
    An \emph{arity} (or \emph{signature}) consists of the following data:
    \begin{enumerate}
        \setlength{\itemsep}{0pt}
        \item A sort $s \in \mathcal{S}$.
        \item A natural number $n$ called the \emph{argument arity}.
        \item A natural number $m$ called the \emph{binding arity}.
        \item A function $\soa : [n] \to \mathcal{S}$ called the \emph{sort of argument} function.
        \item A function $\sov : [m] \to \mathcal{S}$ called the \emph{sort of variable} function.
        \item A relation $\scope \subseteq [n] \times [m]$ called \emph{scoping}.
    \end{enumerate}
    We denote the set of arities by $\mathbf{A}$.
\end{defin}

\begin{remark}
    Let $1 \le k \le n$ and $1 \le l \le m$. We say that the sort of argument $k$ is $\soa(k)$ and the sort of variable $l$ is $\sov(l)$. If $k \scope l$ then we would say that the $k$th argument is in scope of the $l$th variable.
\end{remark}

\begin{remark}
    This is a modification to the definition given in \cite{harper_2016}. In which each argument has a set of variables. For our purposes we want all arguments to use the same variables. This is achieved with a scoping relation. Details of this idea can be found in \cite{nlab:initiality_project_-_raw_syntax}.
\end{remark}

% Operators
\begin{defin}
    Let $\mathcal{O}$ be a set of elements called \emph{operators}, and let $\mathrm{ArityOf} : \mathcal{O} \to \mathbf{A}$ be the function picking the \emph{arity of an operator}. The arity of an operator $o\in \mathcal{O}$ is $\mathrm{ArityOf}(o)$.
\end{defin}

% Variables
\begin{defin}\label{variables}
    A set of \emph{variables} is simply a set $\mathcal{X}$ and a function $\mathrm{SortOf} : \mathcal{X} \to \mathcal{S}$ choosing the sort of the variable. We write $\mathcal{X}_s$ for all the variables $x \in \mathcal{X}$ with $\mathrm{SortOf}(x) = s \in \mathcal{S}$. Observe that $\mathcal{S}$ is the inverse image of $\mathrm{SortOf}$ over $s$.
\end{defin}

\begin{remark}\label{var_order}
    Typically a set of variables is endowed with some sort of order. They are also typically countable. We could say that every set of variables should necesserily be equipped with an injection into the natural numbers.
\end{remark}

% Fresh variables
\begin{defin}
    We say a set of variables $\mathcal{V}$ is \emph{fresh} for a set of variables $\mathcal{X}$ if $\mathcal{V} \cap \mathcal{X} = \varnothing$. We can then take the \emph{union} of sets of variables $\mathcal{V} \cup \mathcal{U}$ with the obvious well-defined definition of $\mathrm{SortOf}$.
\end{defin}

% Abstract binding trees
\begin{defin}
    The set of \emph{abstract binding trees} (\emph{abts}) of \emph{sort} $s\in \mathcal{S}$ on a set of variables $\mathcal{X}$, is the least set $\mathcal{B}[\mathcal{X}]_s$ satisfying the following conditions:
    \begin{enumerate}
        \item If $x \in \mathcal{X}_s$ then $x \in \mathcal{B}[\mathcal{X}]_s$.
        \item Let $\mathtt{G}$ be an operator of sort $s$, argument arity $n$, binding arity $m$. Let $\mathcal{V} := \{v_1, \dots , v_m \}$ be a finite set of $m$ variables fresh for $\mathcal{X}$. For $1 \le j \le n$, let $\mathcal{Y}_j := \{ v_k \in \mathcal{V} \mid j \scope k\}$ be the set of variables that the $j$th argument is in scope of. Now suppose for each $1 \le j \le n$, there are $M_j \in \mathcal{B}[\mathcal{X} \cup \mathcal{Y}_j]_{\soa(j)}$. Then $\mathtt{G}(\mathcal{V}; M_1, \dots, M_n) \in \mathcal{B}[\mathcal{X}]_s$.
    \end{enumerate}
\end{defin}

\begin{remark}
    Harper's notion of \emph{abstract binding tree} is a generalisation of the more common \emph{abstract syntax tree}. The difference is that abts keep track of how their variables are bound. We will later demonstrate this by showing how variable capture is avoided. The above definition may seem complicated but it is simply a tree where branches are operators and nodes are variables. All these trees do not live in the same set however since the bound and free variables are being kept track of.
\end{remark}

% Alpha equivalence
\begin{defin}[$\alpha$-equivalence]\lable{alpha}
    Let $\mathcal{X}$ and $\mathcal{X}'$ be bijective sets of variables, and let $\rho : \mathcal{X} \to \mathcal{X'}$ be a bijection. Define the following relation $\sim_\rho \subseteq \mathcal{B}[\mathcal{X}]_s \times \mathcal{B}[\mathcal{X}']_s$ by induction on both abts:
    \begin{itemize}
        \item If $x \in \mathcal{X}$ and $y \in \mathcal{X}'$ then $x \sim_\rho y $ if and only if $\rho(x) = y$.
        \item For bijective sets of variables $\mathcal{V}$ and $\mathcal{V}'$ of size $n$, free for $\mathcal{X}$ and $\mathcal{X}'$ respectively. By Remark \ref{var_order} we give them orders. Let $\xi : \mathcal{V} \to \mathcal{V}'$ be the \emph{unique} order-preserving bijection between them. For $1 \le j \le n$, let $\mathcal{Y}_j := \{ v_k \in \mathcal{V} \mid j \scope k\}$ and $\mathcal{Y}_j' := \{ v_k' \in \mathcal{V}' \mid j \scope k \}$ be the sets of variables the $j$th argument is in scope of in $\mathcal{V}$ and $\mathcal{V}'$ respectively. Observe that the restriction $\xi_j : \mathcal{Y}_j \to \mathcal{Y}_j'$ is also a bijection. Then $\mathtt{G}(\mathcal{V}; m_1, \dots, m_n) \sim_\rho \mathtt{G}(\mathcal{V}'; m_1', \dots, m_n')$ if and only if $m_j \sim_{\rho \cup \xi_j} M'_j$ for all $1 \le j \le n$.
        \item In all other cases the relation is false.
    \end{itemize}

    We say that an abt $a \in \mathcal{B}[\mathcal{X}]_s$ is \emph{$\alpha$-equivalent} to an abt $b \in \mathcal{B}[\mathcal{X}']_s$, written $a \simeq_{\alpha} b$, if there exists a bijection $\rho : \mathcal{X} \to \mathcal{X'}$ such that $a \sim_\rho b$.
\end{defin}

We quickly sketch some routine proofs showing $\alpha$-equivalence is in fact an equivalence relation.

\begin{lemma}[Reflexivity]
    $\alpha$-equivalence is reflexive.
\end{lemma}
    
\begin{proof}    
     Observe that for any $m \in \mathcal{B}[\mathcal{X}]_s$ we have $m \sim_{\mathrm{id}} m$. 
\end{proof}

\begin{lemma}[Symmetry]
    $\alpha$-equivalence is symmetric.
\end{lemma}

\begin{proof}
    Suppose $a \simeq_\alpha b$ then $a \sim_\rho b$ for some bijection $\rho$. The inverse $\rho^{-1}$ is also a bijection, and observe that $b \sim_{\rho^{-1}} a$.
\end{proof}

\begin{lemma}
    $\alpha$-equivalence is transitive.
\end{lemma}

\begin{proof}
    Suppose $a \simeq_\alpha b$ and $b \simeq_\alpha c$ then $a \sim_\rho b$ and $b \sim_{\rho'} c$ for some bijections $\rho$ and $\rho'$. Observe that the composite $\rho' \cdot \rho$ is also a bijection, and that as a result $a \sim_{\rho' \cdot \rho} b$. It can then easily be checked that $a \simeq_\alpha c$.
\end{proof}

\begin{cor}
    $\alpha$-equivalence is an equivalence relation.
\end{cor}

% Weakening, exchange and contraction
\begin{defin}\label{sub}
    Given $\sigma : \mathcal{X} \to \mathcal{Y}$ such that $\sigma$ preserves sorts, i.e. $\mathrm{SortOf} \cdot \sigma = \mathrm{SortOf}$, we define a function $\mathcal{B}[\mathcal{X}]_s \to \mathcal{B}[\mathcal{Y}]_s$ denoted $a \mapsto a[\sigma]$ by induction on $a$:
    \begin{itemize}
        \item If $a = x \in \mathcal{X}$, then $x[\sigma] = \sigma(x)$.
        \item If $a = \mathtt{G}(\mathcal{V}; m_1, \dots , m_n)$ we would like to define $a[\sigma]$ as $$\mathtt{G}(\mathcal{V}, m_1[\sigma], \dots, m_n[\sigma])$$ but this is not possible since $\mathcal{V}$ may not be disjoint from $\mathcal{Y}$. Therefore we observe that we can accomadate for this by first freshening up our variables in $\mathcal{V}$ with respect to $\mathcal{Y}$ by finding another $\mathcal{V}'$ whose elements are all fresh in $\mathcal{Y}$ and $\mathcal{V} \simeq_{\alpha}\mathcal{V}'$. We will call such an operation $\mathcal{V}^{[\mathcal{Y}]}$ and then define $\mathtt{G}(\mathcal{V}; m_1 , \dots, m_n)[\sigma] := \mathtt{G}(\mathcal{V}^{[\mathcal{Y}]}; m_1[\sigma], \dots, m_n[\sigma])$. 
    \end{itemize}
    
    If $\sigma$ is an inclusion, then the operation is \emph{weakening} (at the level of syntax).
    If $\sigma$ is a permutation (a self-bijection) then this is known as \emph{exchange} (at the level of syntax).
    If $\sigma$ is a surjection, then the operation is \emph{contraction} (at the level of syntax).
\end{defin}

\begin{remark}
    We must be weary not to get confused later on with the \emph{structural rules} with the same names. These operations are intrinsic to syntax, and are not directly related with rules we will look at later.
\end{remark}

\begin{remark}\label{op_alpha_respect}
    It can be seen that $\alpha$-equivalence between $a$ and $b$ can be stated as the existence of a bijection $\rho$ such taht $a[\rho] = b$.
\end{remark}

% Composition of functions
\begin{lemma}\label{sub_comp}
    Given functions $\sigma$ and $\sigma'$, we have $a[\sigma][\sigma'] = a[\sigma' \cdot \sigma]$.
\end{lemma}

\begin{proof}
    Expanding the definition of $a[\sigma]$ and $a[\sigma][\sigma']$ this can be observed.
\end{proof}

\begin{lemma}\label{sub_alpha}
    If $\rho, \rho'$ are bijections, then $a \sim_\rho b$ implies $a[\sigma] \sim_{\rho'} b[\rho ' \cdot \sigma \cdot \rho^{-1}]$. Hence the operation $-[\sigma]$ respects $\alpha$-equivalence.
\end{lemma}

\begin{proof}
    By Remark \ref{op_alpha_respect} and Lemma \ref{sub_comp}, we have $a \sim_\rho b \iff a[\rho] = b \iff a = b[\rho^{-1}]$. So $a[\sigma] = b[\rho^{-1}][\sigma]=b[\sigma \cdot \rho^{-1}]$ and hence  $a[\sigma][\rho']=b[\rho' \cdot \sigma \cdot \rho^{-1}]\iff a[\sigma] \sim_{\rho'} b[\rho' \cdot \sigma \cdot \rho^{-1}]$.
\end{proof}

%
\begin{defin}
    We override our definition of abstract binding tree by defining the set of all abts of sort $s$ over a set of variables $\mathcal{X}$ as $\mathcal{B}[\mathcal{X}]_s / \simeq_{\alpha}$.
\end{defin}

\begin{remark}
    Whenever we refer to an abt we typically write it as some representing element of the equivalence class.
\end{remark}

\begin{remark}
    Due to Lemma \ref{sub_alpha}, Definition \ref{sub} makes sense for equivalence classes too. Thus we do not need to change the meaning of $-[\sigma]$, by simply noting that it acts on representatives of the equivalcen class in a well-defined way.
\end{remark}

\begin{defin}
    We call the disjoint union $\sqcup_{s \in \mathcal{S}} \mathcal{B}[\mathcal{X}]_s$ of abts over $\mathcal{X}$ with sort $s$ the set of all abts over $\mathcal{X}$. We could have defined this first and then defined $\mathcal{B}[\mathcal{X}]_s$ as the inverse image of some sort choosing function over a sort $s$ like in Definition \ref{variables}. When we talk about the sort of an abt $a \in \mathcal{B}[\mathcal{X}]$ we refer to the $s$ which corresponds to the set in which $a$ lives. 
\end{defin}

\subsection{Substitution}

\begin{defin}\label{subst}
    Let $M \in \mathcal{B} [\mathal{X} \cup \{ x \}]$ and $N \in \mathcal{B}[\mathcal{X}]_{\mathrm{SortOf(x)}}$. Then $M[N/x]$, read as the substitution of $x$ for $N$ in $M$, is defined by induction on $M$ as follows:
    \begin{itemize}
        \item Suppose $M = x$ then $M[N / x] := N$.
        \item Suppose $M = y \in \mathcal{X}_{\SortOf(x)}$ then $M[N / x] := M$.
        \item Suppose $M = \mathtt{G}(V; m_1 , \dots, m_n)$ then $$M[N / x] := \mathtt{G}(V; m_1[N / x], \dots, m_n[N / x])$$.
    \end{itemize}
\end{defin}

\begin{remark}
    The reason we set up abstract binding trees is that it avoids \emph{``variable capture''}. Take for example the following: $(\lambda x . x y)[M / x]$, this statement makes sense in most formulations of syntax, therefore a complicated exceptions need to be taken into consideration for the definition of substitution. The way we have set up syntax we see that this sentence is complete nonsense. $(\lambda x . x y)$ lives in some set $\mathcal{B}[\mathcal{X}]$ which definitely doesn't have $x \in \mathcal{X}$ or else the operator $\lambda$ would not be able to introduce $x$ as a variable fresh for $\mathcal{X}$.
\end{remark}




% Judgements, inference rules and general notions of logical
\section{Judgements}

We will now develop the basic formal tools to describe how our programming languages work.  We will first describe judgements and how to specify a type system. Then our first example will be the simply typed lambda calculus. We use the ideas developed in \cite{harper_2016} though these ideas are much older. [Probably tracable back to Gentzen]. [There are many more references to be included here]

\begin{defin}
    The notion of a \emph{judgement} or \emph{assertion} is a logical statement about an abt. The property or relation itself is called a \emph{judgement form}. The judgement that an object or objects have that property or stand in relation is said to be an \emph{instance} of that judgement form. A judgment form has also historically been called a \emph{predicate} and its instances called \emph{subjects}.
\end{defin}

\begin{remark}
    Typically a judgement is denoted $\mathsf{J}$. We can write $a\ \mathsf{J}$, $\mathsf{J}\ a$ to denote the judgment asserting that the judgement form $\mathsf{J}$ holds for the abt $a$. For more abts this can also be written prefix, infix, etc. This will be done for readability. Typically for an unspecified judgement, that is an instance of some judgement form, we will write $J$.
\end{remark}

    $$\frac
        {}
        {}
    $$


\begin{defin}
    An \emph{inductive definition} of a judgement form consists of a collection of rules of the form
    
    $$\frac
        {J_1 \quad \cdots \quad J_k}
        {J}
    $$
    
    in which $J$ and $J_1, \dots , J_k$ are all judgements of the form being defined. THe judgements above the horizontal line are called the \emph{preimises} of the rules, and the judgement below the line is called its \emph{conclusion}. A rule with no premises is called an \emph{axiom}.
\end{defin}

\begin{remark}
    An inference rule is read as starting that the premises are \emph{sufficient} for the conclusion: to show $J$, it is enough to show each of $J_1, \dots J_k$. Axioms hold unconditionally. If the conclusion of a rule holds it is not necesserily the case that the premises held, in that the conclusion could have been derived by another rule.
\end{remark}

\begin{example}
    Consider the following judgement from $-\ \mathsf{nat}$, where $a\ \mathsf{nat}$ is read as ``$a$ is a natural number''. The following rules form an inductive definition of the judgement form $-\ \mathsf{nat}$:

    $$\frac
        {}
        {\texttt{zero}\ \mathsf{nat}}
      \qquad\qquad\qquad
      \frac
        {a\ \mathsf{nat}}
        {\texttt{succ}(a)\ \mathsf{nat}}
    $$

    We can see that an abt $a$ is zero or is of the form $\texttt{succ}(a)$. We see this by induction on the abt, the set of such abts has an operator $\texttt{succ}$. Taking these rules to be exhaustive, it follows that $\textt{succ}(a)$ is a natural number if and only if $a$ is.
\end{example}

\begin{remark}
    We used the word \emph{exhaustive} without really defining it. By this we mean necessary and sufficient. Which we will define now.
\end{remark}

\begin{defin}
    A collection of rules is considered to define the \emph{strongest} judgement form that \emph{closed under} (or \emph{respects}) those rules. To be closed under the rules means that the rules are \emph{sufficient} to show the validity of a judgement: $J$ holds if there is a way to obtain it using the given rules. To be the \emph{strongest} judgement form closed under the rules means that the rules are also \emph{necessary}: $J$ holds \emph{only if} there is a way to obtain it by applying the rules.
\end{defin}

Let's add some more rules to our example, to get a richer structure.

\begin{example}
    The judgement form $a = b$ expresses the equality of two abts $a$ and $b$. We define it inductively on our abts as we did for $\mathsf{nat}$.
    
    $$\frac
        {}
        {\texttt{zero} = \texttt{zero}}
      \qquad\qquad\qquad
    \frac
        {a = b}
        {\texttt{succ}(a) = \texttt{succ}(b)}
    $$
    Our first rule is an axiom declaring that \texttt{zero} is equal to itself, and our second rule shows that abts of the form $\texttt{succ}$ are equal only if their arguments are. Observe that these are exhaustive rules in that they are necessary and sufficient for the formation of $=$.
\end{example}

\subsection{Derivations}

To show that an inductively defined judgement holds, we need to exhibit a \emph{derivation} of it.

\begin{defin}
    A \emph{derivation} of a judgement is a finite composition of rules, starting with axioms and ending with the judgement. It is a tree in which each node is a rule and whose children are derivations of its premises. We sometimes say that a derivation of $J$ is evidence for the validity of an inductively defined judgement $J$.

    Suppose we have a judgement $J$ and
    $$\frac
        {J_1\quad \cdots\quad J_k}
        {J}
    $$
    is an inference rule. Suppose $\triangledown_1, \dots, \triangledown_k$ are derivations of its premises, then
    $$\frac
        {\triangledown_1\quad \cdots\quad \triangledown_k}
        {J}
    $$
    is a derivation of its conclusion. Notice that if $k=0$ then the node has no children.
\end{defin}

Writing derivations as trees can be very enlightening to how the rules compose. Going back to our example with $\mathsf{nat}$ we can give an example of a derivation.

\begin{example}
    Here is a derivation of the judgement $\texttt{succ}(\texttt{succ}(\texttt{succ}(\texttt{zero})))\ \mathsf{nat}$:
    
    \begin{prooftree}
        \AxiomC{}
        \UnaryInfC{ $\texttt{zero}\ \mathsf{nat}$ }
        \UnaryInfC{ $\texttt{succ}(\texttt{zero})\ \mathsf{nat}$ }
        \UnaryInfC{ $\texttt{succ}(\texttt{succ}(\texttt{zero}))\ \mathsf{nat}$ }
        \UnaryInfC{ $\texttt{succ}(\texttt{succ}(\texttt{succ}(\texttt{zero})))\ \mathsf{nat}$ }
    \end{prooftree}
\end{example}

\begin{remark}
    To show that a judgement is \emph{derivable} we need only give a derivation for it. There are two main methods for finding derivations:
    \begin{itemize}
        \item \emph{Forward chaining} or \emph{bottom-up construction}
        \item \emph{Backward chaining} or \emph{top-down construction}
    \end{itemize}
    
    Forward chaining starts with the axioms and works forward towards the desired conclusion. Backward chaining starts with the desired conclusion and works backwards towards the axioms.
\end{remark}

It is easy to observe the \emph{algorithmic} nature of these two processes. In fact this is an important point to think about, since it may become relevent in the future.

\begin{lemma}
    Given a derivable judgement $J$, there is an algorithm giving a derivation for $J$ by forward chaining.
\end{lemma}

\begin{proof}
    This is not a difficult algorithm to descrive. We start with a set of rules $\mathcal{R} := \varnothing $ which we initially set to be empty. Now we consider all the rules that have premises in $\mathcal{R}$, initially this will be all the axioms. We add these rules to $\mathcal{R}$ and repeat this process until $J$ appears as a conclusion of one of the rules in $\mathcal{R}$. It is not difficult to see that this will necesserily give all derivations of all derivable judgements and since $J$ is derivable, it will eventually give a derivation for $J$.
\end{proof}

\begin{remark}
    Notice how we had to specify that our judgement is derivable. Since if were not, then our process would not terminate, hence would not be an algorithm. It is also worth noting that this algorithm is very inefficient since the size of $\mathcal{R}$ will grow rapidly, especially when we have more rules available. This is sort of a brute force approach. What we will need is more clever picking of the rules we wish to add. This is nontrivial problem and is basically what a mathematician does.
\end{remark}

Forward chaining does not take into account any of the information given by the judgement $J$. The algorithm is in a sense blind. 









% Model simply typed lambda calculus using syntax machinary
%
% Simply typed lambda calculus
%
\section{Simply typed lambda calculus} 


First develop the features needed. Discuss the arbitrary nature of such features, then use Curry-Howard as motivation for ``the language that ought to be''. Develop STLC, discuss in detail the implications, give categorical semantics. Discuss breifly the dynamics of simply typed lambda calculus. A big disadvantage of STLC over the untyped version (which we ought to discuss since we have the tools to) is that there is no recursion. There are many ways to fix this, see G\"odel for example. In order to fix this we will introduce dependent types.

We begin by discussing the syntax of our type theory. We have a set of types $\mathbf{T}::= $
and a set of terms $\mathbf{T}::=$

\subsection{Judgements}


[[TODO: Clean up this whole paragraph(s)]]
We begin with our basic judgements. Of which there will be 5. Our STLC will have bidirectional typechecking, in that we will distinguish between the direction of type checking. There are several advantages of this and historically the two main systems called STLC are Curry's and Church's which simply differ in the direction of type checking. By having both directions and a sort of ``mode-switching rule'' we have far greater control and ease when describing type checking properties. We will also need to have a notion of \emph{judgemental equality} since we wish to do some computation. There are variations of this theme discussed in the statics chapter that allow us to have transition systems instead but we will use an equational style since transition systems can be derived from this. This also has the advantage of STLC becomming what is known as an ``equational theory''. This will be a useful feature for when we want to derrive categorical semantics. 

A context is a list of basic judgements. Our basic judgements are $x : A$. [[No it is not fix this]]

There are 5 judgements that we have:

\begin{itemize}
    \item $\Gamma \vdash A\ \mathsf{type}$ - ``$A$ is a type in context $\Gamma$''.
    \item $\Gamma \vdash T \Leftarrow A$ - ``$T$ can be checked to have type $A$ in context $\Gamma$''.
    \item $\Gamma \vdash T \Rightarrow A$ - ``$T$ synthesises the type $A$ in context $\Gamma$''.
    \item $\Gamma \vdash A \equiv B\ \mathsf{type}$ - ``$A$ and $B$ are jdugementally equal types in context $\Gamma$''.
    \item $\Gamma \vdash S \equiv T : A$ - ``$S$ and $T$ are judgementally equal terms of type $A$ in context $\Gamma$''.
\end{itemize}

\subsection{Structural rules}

Structural rules will dictate how our judgements interact with eachother, how different contexts can be formed and how substitution works. This is all roughly what a ``type theory'' ought to provide.

We begin with the \emph{variable} rule, this says that if a term $x$ appears with a type $A$ as an element in a context $\Gamma$ then $x$ synthesises a type $A$ in context $\Gamma$. Or written more succiently as:

$$
    \frac{(x:A) \in \Gamma }{\Gamma \vdash x \Rightarrow A}
$$

Other structural rules: weakening, contraction and substitution are all admissible. [[What does it mean for a rule to be admissible? We have defined this previously but we need to carefully state these facts, and prove them too!]]

\subsection{Mode-switching}

One of the features of bidirectional type checking is that we can switch the mode we are in. This is expressed as the mode switching rule:

\begin{prooftree}
    \AxiomC{$\Gamma \vdash T \Rightarrow A$}
    \AxiomC{$\Gamma \vdash A \equiv B \ \mathsf{type}$}
    \BinaryInfC{$\Gamma \vdash T \Leftarrow B$}
\end{prooftree}

This rule has been specially set up in that it will be the \emph{only way} to derive $\Gamma \vdash T \Leftarrow B$. [[TODO: talk more about this]]

In a unidirectional type system, the judgements $\Gamma \vdash T \Rightarrow A$ and $\Gamma \vdash T \Leftarrow B$ are collapsed into one: $\Gamma \vdash T : A$. And now the mode-switching rule may have a more familiar form:

\begin{prooftree}
    \AxiomC{$\Gamma \vdash T : A$}
    \AxiomC{$\Gamma \vdash A \equiv B \ \mathsf{type}$}
    \BinaryInfC{$\Gamma \vdash T : B$}
\end{prooftree}

Which shows that it is actually a rule about substituting along a judgemental equality! But this is a problem since a type checking algorithm will have to decide when to stop doing this. This is one of the big advantages that bidirectional type checking has over unidirectional type checking. The type checking algorithm will be simpler! [[TODO: Clean up and discuss type checking in more detail]]

\subsection{Equality rules}
Finally we have some structural rules for our two judgemental equality judgements. We wish for these to be an equivalence relation and that they are compatible with eachother.

First we begin with the structural rules for the judgement form $- \equiv -\ \mathbf{type}$:

We wish for our judgemental equality of types to be reflexive:
\begin{prooftree}
    \AxiomC{}
    \RightLabel{($\equiv_{\mathbf{type}}$-reflexivity)}
    \UnaryInfC{$\Gamma \vdash A \equiv A\ \mathbf{type}$}
\end{prooftree}

We want our judgemental equality of types to be symmetric:
\begin{prooftree}
    \AxiomC{$\Gamma \vdash A \equiv B \ \mathbf{type}$}
    \RightLabel{($\equiv_{\mathsf{type}}$-symmetry)}
    \UnaryInfC{$\Gamma \vdash B \equiv A \ \mathbf{type}$}
\end{prooftree}

and our judgemental equality of types to be transitive:

\begin{prooftree}
    \AxiomC{$\Gamma \vdash B \ \mathsf{type}$}
    \AxiomC{$\Gamma \vdash A \equiv B\ \mathsf{type}$}
    \AxiomC{$\Gamma \vdash B \equiv C\ \mathsf{type}$}
    \RightLabel{($\equiv_\mathsf{type}$-transitivity)}
    \TrinaryInfC{$\Gamma \vdash A \equiv C\ \mathsf{type}$}
\end{prooftree}

Notice how the previous rule also checks that $B$ is a type. This is because if we did not do this, we could insert any symbol in. This is clearly undesirable. It also demonstrates how subtly sensitive rules are.

Now we list the rules making the judgement form $- \equiv - : A$ into an equivalence relation:

We wish for our judgemental equality of terms to be reflexive:
\begin{prooftree}
    \AxiomC{}
    \RightLabel{($\equiv_{\mathsf{term}}$-reflexivity)}
    \UnaryInfC{$\Gamma \vdash T \equiv T : A$}
\end{prooftree}

We want our judgemental equality of terms to be symmetric:
\begin{prooftree}
    \AxiomC{$\Gamma \vdash S \equiv T : A$}
    \RightLabel{($\equiv_{\mathsf{term}}$-symmetry)}
    \UnaryInfC{$\Gamma \vdash T \equiv S : A$}
\end{prooftree}

and our judgemental equality of terms to be transitive:
\begin{prooftree}
    \AxiomC{$\Gamma \vdash T \Leftarrow A $}
    \AxiomC{$\Gamma \vdash S \equiv T : A$}
    \AxiomC{$\Gamma \vdash T \equiv R : A$}
    \RightLabel{($\equiv_{\mathsf{term}}$-transitivity)}
    \TrinaryInfC{$\Gamma \vdash S \equiv R : A$}
\end{prooftree}

as we stated before for transitivity judgemental equality of types we need to also check that the middle term $T$ is actually a term.

Finally we need a rule that will make  that judgemental equality of types and judgemental equality of terms interact the way we expect them to:
\begin{prooftree}
    \AxiomC{$\Gamma \vdash A \ \mathsf{type}$}
    \AxiomC{$\Gamma \vdash S \equiv T : A$}
    \AxiomC{$\Gamma \vdash A \equiv B\ \mathsf{type}$}
    \RightLabel{($\equiv_{\mathsf{term}}$-$\equiv_{\mathsf{type}}$-compat)}
    \TrinaryInfC{$\Gamma \vdash S \equiv T : B$}
\end{prooftree}


\subsection{Type formers}
What we have constructed thusfar is essentially an ``empty type theory''. What we have included which other authors typcially gloss over is a clean way of constructing a typechecking algorithm: bidirectional typechecking and an account of judgemental equality. We now study what are known as type formers, typically when we wish to add a new type to a type theory we need to think about a collection of rules. These can roughly be sorted into 5 kinds of rules:

\begin{itemize}
    \item Formation rules - How can I construct my type?
    \item Introduction rules - Which terms synthesise this type?
    \item Elimination rules - How can terms of this type be used?
    \item Computation (or equality) rules - How do terms of this type compute? (Normalise, etc.)
    \item Congruence rules - How do all the previous rules interact with judgemental equality
\end{itemize}

We make a note that although we will be providing all the rules, the congruence rules can be typically derrived from the others. Although we do not know exactly how to do this so we will provide them explicitly. We also note that not every type need computation rules.

\subsection{Inversion lemmas}
Having listed all these rules we need some lemmas detailing how different terms can \emph{only} come from a set of specified rules. This is a crucial analysis if we wish to construct a type checking algorithm.

\subsection{$\to$-types}

Building on top of our ``empty type theory'' we introduce $\to$ the function type former:

\subsubsection{Formation rules}

We start with our formation rule which simply states that in order to derive $\Gamma \vdash A \to B \ \mathsf{type}$ it is required to derive that both $\Gamma \vdash A \ \mathsf{type}$ and $\Gamma \vdash B \ \mathsf{type}$:

\begin{prooftree}
    \AxiomC{$\Gamma \vdash A\ \mathsf{type}$}
    \AxiomC{$\Gamma \vdash B\ \mathsf{type}$}
    \RightLabel{}
    \BinaryInfC{$\Gamma \vdash A \to B \ \mathsf{type}$}
\end{prooftree}

\subsubsection{Introduction rules}
\subsubsection{Elimination rules}
\subsubsection{Computation rules}
\subsubsection{Congruence rules}

\begin{comment}
%\subsection{Lambda calculus}
%We recall that there are 3 kinds of expressions in lambda calculus: variables, abstractions and applications. These are defined inductively on themselves. A variable is simply a string of characters from an alphabet. A lambda abstraction looks like $\lambda x.y$ where $x$ is some variable and $y$ is some expression. There are alternate ways of writing this, allowing us to drop the need for naming $x$, for example de Brujin indices. Finally an application is simply the concatenation $ab$ of two expressions $a$ and $b$. We will assume that  This fully describes the syntax of this type theory. We will now introduce some rules that tell us which expressions we can derive from other expressions. Firstly we have $\beta$-reduction which tells us if we have an expression of the form $(\lambda x . y)z$ this can be reduced to an expression where all occurrences of $x$ in $y$ are replaced with the expression $z$. We also have $\alpha$-conversion which I would argue isn't really a rule as naming of variables can be completely avoided in the first place using de Brujin indices or even combinators. \cite{BarendregtHenk2013Lcwt, hottbook}

%\subsection{Contexts}
%In mathematics we work with contexts implicitly. That is there is always an ambient knowledge of what has been defined. Mostly due to the nature of how we read mathematical papers. We can make this explicit using contexts. We will not however, use contexts in our discussion of type theory but we will provide a formal exposition in the appendix.

\subsection{Judgements}
Our judgements:
\begin{center}
    \begin{tabular}{c | c}
        $\Gamma\ \mathrm{ctx}$ &  $\Gamma$ is a well-formed context. \\
        $\Gamma \vdash A\ \mathrm{Type}$ & $A$ is a type in context $\Gamma$. \\
        $\Gamma \vdash x : A$ & $x$ is a term of type $A$ in context $\Gamma$. \\
%        $\Gamma \vdash x \equiv y : A$ & the terms $x$ and $y$ of type $A$ are definitionally equal in context $\Gamma$
    \end{tabular}
\end{center}


Type theory ``will be about'' deriving judgements from other judgements. Which can be concisely summarised in the form of an inference rule

$$\frac{A_1\quad A_2 \quad \cdots \quad A_n}{B}$$

which says that given the judgements $A_1,\dots,A_n$ we can derive the judgement $B$.

\subsection{Structural rules}
We now look at the rules that govern contexts and the structure of our type system.

We begin with a rule stating that the empty context (which as contexts are sets or lists is well-defined) is well-formed. Which is another way of stating that the context was grown in a specified way and is not just an arbitrary list or set of variables.

\begin{prooftree}
    \AxiomC{}
    \RightLabel{empty-ctx}
    \UnaryInfC{$\varnothing$ ctx}
    \singleLine
\end{prooftree}

We also want the concatenation of two well-formed contexts to be well-formed.

\begin{prooftree}
    \AxiomC{$\Gamma$ ctx}
    \AxiomC{$\Delta$ ctx}
    \BinaryInfC{$\Gamma,\Delta$ ctx}
\end{prooftree}

We omit rules about repeating or removing repeated elements and ordering lists (think of them as finite sets).

A variable is a statement of the form $x : A$ where $x$ is known as the term and $A$ its type.

\subsection{Function types}

We introduce a formation rule for the function type.

\begin{prooftree}
    \RightLabel{$(\to)$-form}
    \AxiomC{$\Gamma \vdash A\ \mathrm{Type}$}
    \AxiomC{$\Gamma \vdash B\ \mathrm{Type}$}
    \BinaryInfC{$\Gamma \vdash A \to B\ \mathrm{Type}$}
\end{prooftree}

We now need a rule for producing terms of this new type. We introduce the introduction rule for the function type.

\begin{prooftree}
    \RightLabel{$(\to)$-intro}
    \AxiomC{$\Gamma, x : A \vdash y : B$}
    \UnaryInfC{$\Gamma \vdash (\lambda x . y) : A \to B$}
\end{prooftree}

We will sometimes call this lambda abstraction. We next introduce a way to apply these functions to terms in their domains. We introduce our elimination rule for the function type.

\begin{prooftree}
    \RightLabel{$(\to)$-elim}
    \AxiomC{$\Gamma \vdash f : A \to B$}
    \AxiomC{$\Gamma \vdash a : A$}
    \BinaryInfC{$\Gamma \vdash f(a) : B$}
\end{prooftree}

This is essentially useless unless we have a way to compute (or reduce) this expression. This is where our computation rule comes in. The computation rule will tell us how our elimination rule and introduction rule interact.
\begin{prooftree}
    \RightLabel{$(\to)$-comp}
    \AxiomC{$(\lambda x . y) : A \to B$}
    \AxiomC{$\Gamma, a : A \vdash (\lambda x.y)a : B$}
    \AxiomC{$\Gamma, x : A, y : B, (\lambda x . y) : A \to B, a : A \vdash (\lambda x . y) (a) : B$}
    \UnaryInfC{$\Gamma \vdash y[x / a] : B$}
\end{prooftree}

%%%%%%%%%%%%%%%%%%%%

We will describe what is known as a simply typed lambda calculus. There is a lot of literature on type theory, and it doesn't seem that there are many authors in agreement of ways to present it.

In \cite{BarendregtHenk2013Lcwt} a more type theoretic approach, analysing the type theory mostly in the syntactic world. This gives us a good starting point for how we want our type theory to be presented however it may not be so easy to keep an eye on how the categorical semantics (the ways we model types in mathematics) behave. In order to do this we will use references such as \cite{CroleRoyL1993Cft, JacobsCLTT, LambekJ1986Itho}. This will be from the more categorical logic school of thought, which will study type theory that is "generated" by certain categories in interest.

We start by describing a general class of simple type theories as outlined in \cite{JacobsCLTT}. Firstly we introduce the notion of a {\it signature}. Similar accounts can be found in \cite{CroleRoyL1993Cft}. This will essentially consist of "generating" a category from some signature (which can be thought of as a stripped down type theory syntax), and then studying the functors from that category into other categories. This allows nice properties from the second category to be "pulled back" onto our type theory giving it features we desire.

\begin{defin}
	A {\bf signature} is a pair $(\Typ, \mathcal{F})$ where $\Typ$ is a finite set of {\bf basic} (or {\bf atomic}) {\bf types}. And a functor $\mathcal{F} : \Typ^\star \times \Typ \to \Set$. Where $\Typ^\star$ is the Kleene-Star operation on a set (or the free monoid over $\Typ$), defined as $X^\star := \bigcup_{n\in \N} X^n$ whose elements are finite tuples of elements of $X$ for a set $X$. We have $\mathbf{Set}$ for the category of finite sets. Note that the sets in the domain of the functor are realised as discrete categories.
\end{defin}

We will usually write a signature as $\Sigma := (\Typ, \mathcal{F})$, denote $|\Sigma|:=\Typ$ and write $F: \sigma_1,\dots,\sigma_n\to\sigma_{n+1}$ when $F \in \mathcal{F}(( \sigma_1,\dots,\sigma_n ), \sigma_{n+1})$.

\begin{defin}
    Let $\Var$ be a countable set. Elements $x\in \Var$ are called {\bf variables}.
\end{defin}

Note this style of variables is essentially de Brujin indices. But allows us to have a set of names for our variables, which allows future annoyances like $\alpha$-equivalence to be sorted out easily due to the plentiful existence of bijections from $\Var \to \Var$.

\begin{defin}
	A {\bf variable declaration} is a pair $(x, \sigma) \in \Var \times \Typ$ usually written as $x : \sigma$. This can be read as "the variable $x$ has type $\sigma$. We will define $\Dec:=\Var \times \Typ$.
\end{defin}

\begin{defin}
    A {\bf context} $\Gamma$ is an element of $\Con:=\Dec^\star$. In other words, a context is a finite list of variable declarations. We will usually write a context $\Gamma$ as $v_1 : \sigma_1, \dots ,v_n : \sigma_n$. Note that the Kleene-Star has a monoid structure with operation $","$. We can thus give $\Con$ a monoid structure and write, for contexts $\Gamma$ and $\Delta$ another context $\Gamma,\Delta$ which is the concatenation of two contexts. The notation here allows the "expanded version" to coincide, as in $\Gamma,\Delta$ can be written as $v_1 : \sigma_1, \dots ,v_n : \sigma_n, w_1 : \tau_1, \dots, w_m, \tau_m$.
\end{defin}

We also note that there is a canonical inclusion $\Dec \hookrightarrow \Con$ given that $\Dec$ freely generates the monoid $\Con$. This will allow us to write $\Gamma, x:\tau$ for $v_1 : \sigma_1, \dots ,v_n : \sigma_n, x:\tau$.

We now denote the basic statements of our language. These statements are called {\bf judgements} and we will derive

%%%%%%%%%%%%%%%%%%%%
\end{comment}









% Normalisation of STLC
\section{Normalisation of STLC}

\subsection{Introduction}
We now wish to analyse the computational power of our type theory. When designing the type checking algorithm we made a point not to invoke any computational rules, since this will give us a decidable type checking algorithm. We now wish to show that successive applications of mode-switching, betas and eta will always terminate and to the same term, this will be known as the \emph{normal form}. The theorem is known as the Church-Rosser theorem [[CITE]]. This is a subtle property of the type theory and is determined by the computational rules we have added. Further addition of term constructors and type formers should leave this property untouched.

Our proof will follow the proof in \cite[p. 67]{Sorensen} albeit with modifications to make it work here. [[TODO rewrite and add good citations]]

\subsection{Properties of relations}

% Compatible relation
First we define what we mean by a binary relation being \emph{compatible} with the syntax of the STLC.
\begin{defin}
    A binary relation $\succ$ on $\mathrm{Term}$ the set of all terms, is said to be \emph{compatible with the syntax of STLC} (or just simply \emph{compatible}) if the following conditions hold:
    \begin{enumerate}
        \item If $M \succ N$ then $\lambda x . M \succ \lambda x . N$.
        \item If $M \succ N$ then $M Z \succ N Z$.
        \item If $M \succ N$ then $Z M \succ Z N$.
        \item If $M \succ N$ then $(Z,M) \succ (Z,N)$.
        \item If $M \succ N$ then $(M, Z) \succ (N, Z)$.
    \end{enumerate}
\end{defin}

\begin{remark}
    The notion of compatibility allows us to make sure a relation also considers sub-terms. This is a tricky thing to get right but due to our focus on the correct structure of syntax we are fine.
\end{remark}

\begin{remark}
[[CLEAN THIS UP]]
    The reader may ask what relations have to do with normalisation, but it is a formalism that we have chosen. This is definitely not the only way to prove properties like Church-Rosser. The main reason we have chosen this method is for its simplicity. In fact earlier we discussed the dynamics of languages, this is exactly that. There are many ways to go about dynamics including transition systems and equational dynamics. Our approach corresponds to the more classical and simple transition systems approach. It can be shown that this is equivalent to equational dynamics in that a reduction step will be justified by application of rules from STLC.
\end{remark}

We will demonstrate our last remark by considering the following relation:

\begin{defin}
    Let $\sim_{\ty}$ denote the relation amond terms of having the same type. Suppose $\Gamma \vdash s \Leftarrow S$ and $\Gamma \vdash t \Leftarrow T$, then:
    $$
        s \sim_{\ty} t \iff \Gamma \vdash S \equiv T \ \mathsf{type}
    $$
\end{defin}

\begin{lemma}
    The relation $\sim_{\ty}$ is a compatible relation.
\end{lemma}

\begin{proof}
    Suppose $M \sim_{\ty} N$, then we have $\Gamma \vdash M \Leftarrow S$, $\Gamma \vdash N \Leftarrow T$ and $\Gamma \vdash S \equiv T \ \mathsf{type}$.
    \begin{enumerate}
        
    \end{enumerate}
\end{proof}

% Transitive and reflexive closure
\begin{defin}
    Given a relation $\succ$ on a set $X$, we denote by $\succ^+$ the \emph{transitive closure} of $\succ$. This is the smallest relation which coincides with $\succ$ and is transitive. We also consider the \emph{reflexive-transitive closure} $\succ^*$ of $\succ$ which is simply the relation $\Delta(X)\cup \succ^+ $ where $\Delta(X)$ is the image of the diagonal function $x \mapsto (x,x)$. (We've simply added that $x \succ^* x$)
\end{defin}

\begin{remark}
    Transitive closures correspond to chains of the relation, and reflexive-transitive closures allow for chains of length $0$. It should also be noted that we took the \emph{union} of a relation. This is a well-defined notion and can easily be seen to be a relation.
\end{remark}

Let $\to$ be a binary relation on a set $A$, $\twoheadrightarrow^+$ be its transitive closure and $\twoheadrightarrow$ be its reflexive-transitive closure.

Now we define (very generally) what it means for an element of a set to be in \emph{normal form} and \emph{normalising} with respect to some relation.

\begin{defin}
    An element $a \in A$ is said to be of \emph{normal form} if $\forall b \in A$, $a {\not \to} b$.
\end{defin}

\begin{defin}
    An element $a \in A$ is said to be \emph{normalising} (or \emph{weakly normalising}) if there is a reduction sequence $a \to a_1 \to a_2 \to \cdots \to a_n$ where $a_n$ is in normal form, for some $n$. We call $a_n$ a \emph{normal form} or \emph{reduct} of $a$.
\end{defin}

\begin{remark}
    Note that not every reduction sequence is guaranteed to be finite. We also note that if $\to$ a relation is Church-Rosser (to be defined below) then $a_n$ is \emph{the} normal form or reduct.
\end{remark}

We discuss what it means for a relation to be Church-Rosser:

% Church-Rosser
\begin{defin}
    A relation $\to$ has the \emph{Church-Rosser} (CR) property if and only if for all $a,b,c \in A$ such that $a \twoheadrightarrow b$ and $a \twoheadrightarrow c$, there exists $d \in A$ with $b \twoheadrightarrow d$ and $c \twoheadrightarrow d$.
\end{defin}

\begin{remark}
    This says no matter what path we take along a relation, there will always be elements at which the paths cross.
\end{remark}

We will also need a slightly weaker version called weak Church-Rosser, for reasons we will see later:

% weak Church-Rosser
\begin{defin}
    A relation $\to$ has the \emph{weak Church-Rosser} (WCR) property if and only if for all $a, b, c \in A$ such that $a \to b$ and $a \to c$, there exists $d \in A$ with $b \twoheadrightarrow d$ and $c \twoheadrightarrow d$.
\end{defin}

We now state the obvious:

\begin{cor}\label{cr_is_wcr}
    If $\to$ is CR then $\to$ is WCR.
\end{cor}

\begin{proof}
    [[TODO]]
\end{proof}

The converse to this is in general \emph{false} but it is true when another condition holds, namely that $\to$ is \emph{strongly normalising}.

\begin{defin}
    A binary relation $\to$ is \emph{strongly normalising} (SN) if and only if there is no infinite sequence $a_0 \to a_1 \to a_2 \to  \cdots$.
\end{defin}

\begin{remark}
    In other words, a relation $\to$ is strongly normalising if and only if \emph{every} sequence $a_0 \to a_1 \to a_2 \to  \cdots$ terminates after a finite number of steps.
\end{remark}

\begin{remark}
    We typically also say an element is strongly normalising if the condition holds for that element. This allows us to state SN in a different (and perhaps more correct) way: A relation $\to$ is strongly normalising if each element is strongly normalising with respect to $\to$. Then we can define an element to be strongly normalising if all of it's reducts are strongly normalising. The nice thing about this definition is that we have seen it before, this is precisely what it means to be a \emph{well-founded relation} from Defintion \ref{wf}. So $\to$ is strongly normalising if and only if it is well-founded. This is good because we can induct over it!
\end{remark}

\begin{cor}
    If a relation $\to$ is strongly normalising then every element is normalising.
\end{cor}

\begin{proof}
    By induction on $\to$ we see that either an element is in normal form, or it reduces to normal form. This is precisely what it means to be normalising.
\end{proof}

We now state a lemma which will be very useful. It is a sufficient condition for the converse of Corollary \ref{cr_is_wcr} to hold.

\begin{lemma}[Newman's Lemma]
    If $\to$ is strongly normalising and WCR then it is CR.
\end{lemma}

\begin{proof}
    Since $\to$ is strongly normalising, any $a \in A$ has a normal form. Call an element \emph{ambiguous} if $a$ reduces to two distinct normal forms. Clearly $\to$ is CR if there are no ambiguous elements of $A$.
    Assume, for contradiction, that there is an ambiguous $a$. We will show that there is another ambiguous $a'$ where $a \to a'$.
    Suppose we have $a \twoheadrightarrow b_1$ and $a \twoheadrightarrow b_2$ where $b_1$ and $b_2$ are two different normal forms. Both reductions must make at least one step, thus both reductions can be written as $a \to a_1 \twoheadrightarrow b_1$ and $a \to a_2 \twoheadrightarrow b_2$.
    Suppose $a_1 = a_2$ then we can choose $a' = a_1 = a_2$. Now suppose $a_1 \neq a_2$, we know by WCR that $a_1 \twoheadrightarrow b_3$ and $a_2 \twoheadrightarrow b_3$ for some $b_3$. We can assume that $b_3$ is a normal form. Since $b_1$ and $b_2$ are distinct, $b_3$ is different from $b_1$ or $b_2$ so we can choose $a' = a_1$ or $a'=a_2$.
    Since we can always choose an $a'$, we can repeat this process and get an infinite chain of ambiguous elements. It is clear that this contradicts strongly normalising, hence $A$ has no ambiguous elements.
\end{proof}

\subsection{Normalisation}

Now we define what we mean by $\beta$-reduction and $\beta$-normal form.

% Define beta-reduction
\begin{defin}
    We define \emph{$\beta$-reduction} to be the least compatible relation $\to_{\beta}$ on $\mathrm{Term}$ satisfying the following conditions:
    \begin{enumerate}
        \item $(\lambda x . y)t \to_{\beta} y [t / x]$
        \item $\fst(x,y) \to_{\beta} x$
        \item $\snd(x,y) \to_{\beta} y$
    \end{enumerate}
    A term on the left hand side of any of the above is called a \emph{$\beta$-redex} (reducible expression) and the right hand sides are said to \emph{arise by contracting the redex}.
\end{defin}

\begin{remark}[[Clear up wording]]
    Observe that these are very similar to our $\beta$ rules, in fact they are exactly those. So the question may arise: why haven't we defined $\beta$-reduction using the rules that we already have? The answer is that we could but we would have a much harder time, the rules also take into account typing information but we are explicitly not worried about that since we will show later $\beta$-reduction doesn't change a typed terms type. It is somewhat simpler and clearer to focus purely on terms. We will later justify calling this $\beta$-reduction.
\end{remark}

% Define beta normal form
\begin{defin}
    A term $M$ is said to be in \emph{$\beta$-normal form} if it is in normal form with respect to $\to_\beta$.
\end{defin}

\begin{remark}
    That is to say a term is in $\beta$-normal form if there is no $\beta$-reduction to any other term. Or better yet, $M$ does not contain a $\beta$-redex.
\end{remark}

% Define multi-step beta reductions
\begin{defin}
    Let $\twoheadrightarrow_{\beta}$ be the transitive and reflexive closure of $\to_{\beta}$ called a \emph{multi-step $\beta$-reduction}.
\end{defin}

% Non normalising terms
\begin{remark}
    Not every term is normalising. Take for example the term $\Omega=(\lambda x . x x)(\lambda x . x x)$ which cannot be typed as we will see later. There is an infinite reduction sequence:
    $$
        \Omega \to_{\beta} \Omega \to_{\beta} \Omega \to_{\beta} \Omega \to_{\beta} \cdots
    $$
    Since $\Omega$ cannot be given a type, it is deemed \emph{ill-typed}.
\end{remark}

This means we have to be careful which terms we are talking about. When talking about terms of the STLC we should add that we expect them to be well-typed (derivable). We will see later there are many syntactically valid terms that are ill-typed.

We want to now prove that every derivable term is $\beta$-normalising. In order to do this we need to keep track of available redexes and bound them. We will then show there is a reduction strategy that decreases this bound yielding our result.

This proof is usually attributed to an unpublished note of Turing [[CITE]] but it has been rediscovered by various authors. We will follow the proof in Girard's book \cite{Girard1989}.

% Degree of a type
\begin{defin}
    The \emph{degree $\partial(T)$ of a type $T$} is defined by:
    \begin{itemize}
        \item $\partial(T) := 1$ if $T$ is atomic.
        \item $\partial(U \times V), \partial(U \to V) := \max(\partial(U), \partial(V))+1$.
    \end{itemize}
\end{defin}

% Degree of a redex
\begin{defin}
    The \emph{($\beta$-)degree $\partial_{\beta}(t)$ of a redex} is defined by:
    \begin{itemize}
        \item $\partial_{\beta}(\fst(u,v)), \partial_{\beta}(\snd(u,v)) := \partial(U\times V)$ where $\Gamma \vdash (u, v) \Leftarrow U \times V$.
        \item $\partial_{\beta}((\lambda x . v) u) := \partial(U \to V)$ where $\Gamma \vdash \lambda x . v \Leftarrow U \to V$.
    \end{itemize}
\end{defin}

% Degree of a term
\begin{defin}
    The \emph{($\beta$-)degree $d_{\beta}(t)$ of a term} is the maximum of the degrees of its redexes:
    $$
        d_{\beta}(t) := \max \{\partial_{\beta} (s) \mid s \text{ is a redex in } t\}
    $$
\end{defin}

\begin{remark}
    A redex is associated to two degrees, one as a redex and another as a term. Since a redex $r$ may contain other redexes we have that $\partial (r) \le d(r)$. It should be noted we have defined degree to mean 3 different things here, but as long as we are careful we should not get confused.
\end{remark}

% partial T < partial r
\begin{lemma}\label{beta_redex_ineq}
    If $r$ is a redex of type $T$ then $\partial(T) < \partial_{\beta}(r)$. 
\end{lemma}

\begin{proof}
    Checking the cases for $r$:
    \begin{itemize}
        \item $\partial (T) < \partial_{\beta}(\fst(t, u)) = \max(\partial(T), \partial(U)) + 1$.
        \item $\partial (T) < \partial_{\beta}(\snd(u, t)) = \max(\partial(U), \partial(T)) + 1$.
        \item $\partial (T) < \partial_{\beta}((\lambda x . t)u) = \max(\partial(U), \partial(T)) + 1$.
    \end{itemize}
\end{proof}

% substitution inequality
\begin{lemma}\label{beta_sub_ineq}
    If $\Gamma , x : T \vdash t \Leftarrow U$ then $d_{\beta}(t[u/x]) \leq \max(d_{\beta}(t), d_{\beta}(u), \partial(T))$.
\end{lemma}

\begin{proof}
    Analysing the redexes of $t[u/x]$ we find that they fall into the following cases:
    \begin{itemize}
        \item They are redexes of $t$ (in which $u$ has become $x$).
        \item They are redexes of $u$, proliferating due to each occurence of $x$ in $t$.
        \item They are formed when $t$ is of the form $\fst(x)$, $\snd(x)$, or $x v$ for $u$ of the form $(u', u'')$, $(u', u'')$, or $\lambda y . u'$ respectively. These new redexes have degree $\partial(T)$.
    \end{itemize}
\end{proof}

% reduction inequality
\begin{lemma}\label{beta_reduct_ineq}
    If $t \to_{\beta} u$ then $d_{\beta}(u) \le d_{\beta}(t)$.    
\end{lemma}

\begin{proof}
    Consider the reduction where $u$ is obtained from $t$ by replacing the redex $r$ in $u$ by $c$. Now we consider all the redexes of $u$ where we find:
    \begin{itemize}
        \item redexes which were originally in $t$, but not in $r$, and have been modified by the replacement of $r$ by $c$. Observe that their degree does not change.
        \item redexes which were originally in $c$. But $c$ is obtained by reducing $r$, or in other words a substitution in $r$. Notice $(\lambda x . s)s'$ becomes $s[s'/x]$ and Lemma \ref{beta_sub_ineq} tells us that $d_{\beta}(c) \le \max(d_{\beta}(s), d_{\beta}(s'), \partial(T))$, where $T$ is the type of $x$. But by Lemma \ref{beta_redex_ineq} we have $\partial (T) \le \partial (r)$. Applying $\max$ gives us $\max(d(s), d(s'), \partial(T)) \le \max(d_{\beta}(s), d_{\beta}(s'), \partial_{\beta}(r))$ and hence $d_{\beta}(c) \le \max(d_{\beta}(s), d_{\beta}(s'), \partial(r))=d(r)$.
        \item redexes which come from replacing $r$ by $c$. These redexes have degree equal to $\partial(T)$ where $T$ is the type of $r$. By Lemma \ref{beta_redex_ineq} we have $\partial(T) \le \partial (r)$.
    \end{itemize}
\end{proof}

Next we will prove a lemma bounding the number of redexes of a certain degree.

% number of redexes inequality
\begin{lemma}\label{beta_redex_number_ineq}
    Let $r$ be a redex of maximal degree $n$ in $t$, and suppose that all redexes strictly contained in $r$ have degree less than $n$. If $u$ is obtained from $t$ by reducing $r$ to $c$. Then $u$ has strictly fewer redexes of degree $n$.
\end{lemma}

\begin{proof}
    When the reduction happens we make the following observations:
    \begin{itemize}
        \item The redexes outside $r$ in $t$ remain $u$.
        \item The redexes strictly inside $r$ are in general conserved but sometimes become more prolific. Take for example $(\lambda x . (x, x)) s \to_{\beta} (s, s)$. The number of redexes in the reduct are double that of redex on the left. However the degree of the proliferated redexes must be strictly less than $n$.
        \item The redex $r$ is destroyed and possibly replaced by redexes of strictly smaller degree.
    \end{itemize}
\end{proof}

\begin{remark}
    Although not defined, we take the meaning of \emph{a redex strictly inside} to be a redex that is not the whole redex.
\end{remark}

We now have all the machinary needed to prove that typed terms in the STLC are weakly $\beta$-normalising.

% Every term is beta normalising
\begin{theorem}
    Every derivable term $\Gamma \vdash t \Leftarrow A$ in the STLC is $\beta$-normalising.
\end{theorem}

\begin{proof}
    Consider the function $\mu : \Term \to \N \times \N$ which takes $t \mapsto (n, m)$ where $n = d_{\beta}(t)$ and $m$ is the number of redexes in $t$ of degree $n$. By Lemma \ref{beta_redex_number_ineq} it is possible to choose a redex $r$ of $t$ in such a way that, after reduction of $r$ to $c$, the reduct $t'$ satisfies $\mu(t') < \mu(t)$. Thus by double induction on $n$ and $m$ it is possible to see that $\mu(t)$ can always be decreased until $t$ is normal.
\end{proof}

\begin{remark}
    The ordering in $\mu(t') < \mu(t)$ on $\N \times \N $ is the lexicographic ordering. Meaning $(n', m') < (n, m)$ if and only if $n' < n$ or $n'=n$ and $m' < m$. (Think Alphabetical order).
\end{remark}

% Coherence lemma
\begin{lemma}
    Suppose $\Gamma \vdash M \Leftarrow T$ and $M \twoheadrightarrow_{\beta} N$, then $\Gamma \vdash M \equiv N : T$.
\end{lemma}

\begin{proof}
    [[TODO]]
\end{proof}
\begin{comment}
\begin{remark}
    Observe that by ($\equiv_{\mathsf{term}}$-tran) we can show this by induction on the definition $\twoheadrightarrow_{\beta}$. Thus it is sufficient to prove that given $\Gamma \vdash M \Leftarrow T$ and $M \to_{\beta} N$, then $\Gamma \vdash M \equiv N : T$.
\end{remark}

\begin{remark}
    This justifies our choice of rules, and shows that considering only terms strips away the typing information to no consequence.
\end{remark}
\end{comment}

% eta

%We also have eta computation rules, which we haven't included yet. But we will show that normalisation with these included will follow.

% Define eta reduction
\begin{defin}
    We define \emph{$\eta$-reduction} to be the least compatible relation $\to_{\eta}$ on $\mathrm{Term}$ satisfying the following conditions:
    \begin{enumerate}
        \item $\lambda x . f x \to_{\eta} f$
        \item $(\fst(t), \snd(t)) \to_{\eta} t$
    \end{enumerate}
    Just like for $\beta$-reduction we have the notions of \emph{$\eta$-redex} and terms that \emph{arise by contracting the redex}.
\end{defin}

% Define eta normal form
\begin{defin}
    A term is said to be in $\eta$-normal form if it is in normal form with respect to $\to_{\eta}$.
\end{defin}

% Define multi-step eta reductions
\begin{defin}
    Let $\twoheadrightarrow_{\eta}$ be the transitive and reflexive closure of $\to_{\eta}$ called a \emph{multi-step $\eta$-reduction}.
\end{defin}

We will now show that $\to_\eta$ is strongly normalising.

\begin{remark}
    Originally we had thought to modify the proof of $\beta$-normal\-isa\-tion, and make it work for $\eta$. However, this is where the difference between the two is key.
    $\beta$-normal\-isa\-tion has the power to create new $\beta$-redexes whereas $\eta$-normalisation never does. In fact $\eta$-normalisation is strongly normalising even in the untyped lambda calculus. This suggests that talking about degrees is not the correct approach and there ought to be some other metric for which can be used to bound $\eta$-reducible terms. Based off of work in \cite{Fortune1983}, the authors of \cite[Ex. 3.21]{Sorensen} define a \emph{depth} function for terms. We belive this to be the actual depth of the underlying tree of the abstract binding tree of the syntax of the term. But that is not a relevent result for now.
\end{remark}

\begin{defin}
    Given a term $t$ we define the \emph{depth $\mathsf{\delta(t)}$ of $t$} by induction on terms:
    \begin{itemize}
        \item $\delta (x):=0$ for $x$ a variable or constant.
        \item $\delta (a b) := 1+ \max(\delta(a), \delta(b))$.
        \item $\delta (\lambda x . y):= 1 + \delta(y)$.
        \item $\delta ((a, b)) := 1 + \max(\delta(a), \delta(b))$.  
    \end{itemize}
\end{defin}

% eta bounds
\begin{lemma}\label{eta_red_bound}
    If $t \to_{\eta} u$ then $\delta(u) < \delta(t)$.
\end{lemma}

\begin{proof}
    Observe that since $\to_{\eta}$ is a compatible relation, we need only prove the statement for a redex. We do this by cases:
    \begin{itemize}
        \item
        $$
            \begin{aligned}
                \delta ((\fst(s), \snd(s))) &= 1+ \max(\delta(\fst(s)), \delta(\snd(s))) \\
                &= 1 + \max(1+ \delta(s), 1+ \delta(s)) \\
                &= \delta(s)+ 2
            \end{aligned}
        $$
        \item
        $$
            \begin{aligned}
                \delta (\lambda x . s x) &= 1 + \delta(s x) \\
                &= 2 + \max(\delta(s), \delta(x)) \\
                &= \delta(s) + 2
            \end{aligned}
        $$
    \end{itemize}
    Observe that in both cases we have that the depth of a redex $s$ is $\delta(s) = \delta(r) + 2$ where $r$ is the reduct of $s$. However at the level of terms we cannot garantee equality due to the nature of depth and compatibility. 
\end{proof}

\begin{lemma}
    $\eta$-reduction is strongly normalising.
\end{lemma}

\begin{proof}
    By Lemma \ref{eta_red_bound} we have that the depth of any $\eta$-reduction sequence is strictly decreasing. Hence there may only be finitely many steps in any given $\eta$-reduction sequence.
\end{proof}

\begin{comment}
We wish to show that $\eta$-reduction enjoys similar normalisation properties to $\beta$-reduction. This should be much easier since $\eta$-redexes are much more distinct.

% eta degree of a redex
\begin{defin}
    The \emph{($\eta$-)degree $\partial_{\eta}(t)$ of a redex} is defined as the degree of it's type. If $\Gamma \vdash t \Leftarrow T$ then $\partial_{\eta} := \partial(T)$. This is well-defined due to unicity of typing. [[REFERENCE INVERSION LEMMA HERE]]
\end{defin}

\begin{remark}
    Note that a property of our type theory has made the definition for $\eta$-degree much simpler. We could have defined $\eta$-degree in a similar way to $\beta$-degree but when doing so you can observe that it is redundent. This was not entirely possible in the $\beta$-reduction type because we had some ambiguity for the $\beta$ rules. For example it is not entirely clear what type $v$ is in $\snd(u,v)$, and similarly for the other $\beta$-redexes. In fact we could go a step further and completely ignore typing information since $\eta$-reduction is strongly normalising even in the untyped lambda calculus.
\end{remark}

\begin{remark}
    It should also be noted that in most practical implementations of STLC $\eta$ rules are usually completely ignored since many computations can be run without them. The computational argument is that $\eta$ rules are tedious and difficult to implement in a compiler. It has been observed that $\eta$ rules behave particularly badly with common programming constructs such as subtyping. Later we will discuss depenent types, which is a context where $\eta$ rules cannot easily be ignored.
\end{remark}

\begin{lemma}
    If $r$ is a redex of type $T$ then $\partial (T) < \partial_{\eta}(r)$.
\end{lemma}

\begin{proof}
    By definition $\partial_{\eta}(r)= \max(\partial(T), \partial(U))+1$ for some types $U$ and $V$ such that $T = U \times V$ or $T = U \to V$. Hence 
\end{proof}


\begin{lemma}
    $\eta$-reduction is weakly normalising.
\end{lemma}

% Eta is strongly normalising
\begin{lemma}
    $\eta$-reduction is strongly normalising.
\end{lemma}
\end{comment}

% Coherence lemma
\begin{lemma}
    Suppose $\Gamma \vdash M \Leftarrow T$ and $M \twoheadrightarrow_{\eta} N$, then $\Gamma \vdash M \equiv N : T$.
\end{lemma}

\begin{proof}
    Observe that in the definition of 
\end{proof}

% Define beta eta reduction

Now we need a small technical lemma that will show the utility of being strongly normalising.

% Infinite beta-eta implies infinite beta
\begin{lemma}
    If there is an infinite $\beta \eta$-reduction sequence starting from $M$, then there is an infinite $\beta$-reduction sequence starting from $M$.
\end{lemma}

\begin{proof}
    [[TODO]]
\end{proof}

We are interested in the contrapositive form of this lemma:

\begin{cor}
    If there is no infinite $\beta$-reduction sequence starting from $M$, then there is no infinite $\beta \eta$-reduction sequence starting from $M$.
\end{cor}

\begin{remark}
    In particular this means that $\to_{\beta}$ being strongly normalising implies that $\to_{\beta \eta}$ is strongly normalising.
\end{remark}

% Every term is beta strongly normalising
\begin{theorem}
    $\beta$-reduction is strongly normalising.
\end{theorem}

\begin{proof}
    [[TODO]]
\end{proof}

% beta eta reduction is strongly normalising
\begin{cor}
    $\beta \eta$-reduction is strongly normalising.
\end{cor}

% beta eta reduction is WCR
\begin{lemma}
    $\beta \eta$-reduction is WCR.
\end{lemma}

\begin{proof}
    [[TODO]]
\end{proof}

% Church-Rosser
\begin{theorem}
    The Church-Rosser property holds for $\beta \eta$-reduction.
\end{theorem}

\begin{proof}
    [[TODO]]
\end{proof}

\begin{remark}
    So not only does every well-typed term have a normal-form, but it is in fact unique!
\end{remark}

\subsection{Canonicity}



[[These two concepts are very related, we should find some way to talk about it, including Church-Rosser]]


% Examples of STLC
\section{STLC Examples}

%% Here are examples I would like to add
%
%    i) \x.x                (Identity)
%   ii) \x.\y.xy            (Application function)
%  iii) ((\x.\y.(x+y))3)5   (Obviously with N and + added)
%   iv) (\x.x)(\x.x)        (Doesn't type-check)
%    v) \x.xx               (Doesn't type-check (M combinator (Mockingbird)))
%   vi) \x.\y.((xy)(xy))    (Doesn't type-check)
%  vii) \x.\y.\z.x(yz)      (B combinator (Bluebird))
% viii) \x.\y.\z.xyzz       (W* combinator (Warbler once removed))
%   ix) \x.\y.y(xy)         (O combinator (Owl))
%    x) \x.\y.\z.x(y,z)     (Curry)


Untyped lambda calculus, as we mentioned, is in fact \emph{stronger} than the typed lambda calculus. This we will see by looking at some examples of type checking. Many of these are combinators from untyped lambda calculus in combinatory logic. \ref{} [[Need reference of Mockingbird combinator thing]]

Note we don't have very much choice on types, so it may be useful to enrich our type theory with $+$-types or even the natural numbers. But we will see soon that these both are special cases of dependent types.

\subsection{Identity function $\lambda x . x$}

\begin{example}[Identity function]
    Let's consider the following lambda term $\lambda x . x$. We wish to find a type $T$ such that given some context $\Gamma$ we have $\Gamma \vdash \lambda x . x \Leftarrow T$. Our inversion lemma will tell us exactly which rules let us get to this point. So we will essentially be performing a tree search. Firstly we need to switch modes to get $\lambda x . x \Rightarrow T$. But mode switching also lets us change our 
    \begin{prooftree}
        \AxiomC{$\Gamma \vdash \lambda x . x \Rightarrow T$}
    \end{prooftree}
\end{example}

\subsection{Function application $\lambda x . \lambda y . x y$}

\begin{example}
    Here is another example of a term that type checks. Unfortunately we see the disadvantage with type-setting derivation trees: they are very difficult to write down, and get really wide very quickly.  We want to find a type $T$ such that $\Gamma \vdash \lambda x . \lambda y . x y \Leftarrow T$ is true. Here is a derivation tree: 
        \begin{landscape}
            \centering
            \vspace*{\fill}
            \begin{prooftree}
                %\rootAtTop
                \def\ScoreOverhang{1pt}
                %%%
                \AxiomC{$x : A \in \Gamma , x : A, y : C$}
                \LeftLabel{(var)}
                \UnaryInfC{$\Gamma , x : A, y : C \vdash x \Rightarrow A$}
                \AxiomC{}
                \RightLabel{$(***)$}
                \UnaryInfC{$\Gamma , x : A, y : C \vdash C \to D \equiv A \ \mathsf{type}$}
                    %\insertBetweenHyps{\hskip -5pt}
                \BinaryInfC{$\Gamma , x : A, y : C \vdash x \Leftarrow C \to D$}
                \AxiomC{}
                \RightLabel{$(\dagger)$}
                \UnaryInfC{$\Gamma , x : A, y : C \vdash y \Leftarrow C$}
                \LeftLabel{($\to$-elim)}                
                    \insertBetweenHyps{\hskip -20pt}
                \BinaryInfC{$\Gamma , x : A, y : C \vdash x y \Rightarrow D$}
                \AxiomC{}
                \RightLabel{($\equiv_{\mathsf{type}}$-refl)}
                \UnaryInfC{$\Gamma , x : A, y : C \vdash D \equiv D\ \mathsf{type}$}
                \LeftLabel{(switch)}
                    %\insertBetweenHyps{\hskip -10pt}
                \BinaryInfC{$\Gamma , x : A , y : C \vdash xy \Leftarrow D$}
                \LeftLabel{($\to$-intro)}
                \UnaryInfC{$\Gamma , x : A \vdash \lambda y . x y \Rightarrow C \to D$}
                \AxiomC{}
                \RightLabel{$(**)$}
                \UnaryInfC{$\Gamma , x : A \vdash B \equiv C \to D \ \mathsf{type}$}
                \LeftLabel{(switch)}
                    \insertBetweenHyps{\hskip -150pt}
                \BinaryInfC{$ \Gamma , x : A \vdash \lambda y . xy \Leftarrow B$}
                \LeftLabel{($\to$-intro)}
                \UnaryInfC{$\Gamma \vdash \lambda x . \lambda y . x y \Rightarrow A \to B$}
                \AxiomC{}
                \RightLabel{$(*)$}
                \UnaryInfC{$\Gamma \vdash T \equiv A \to B \ \mathsf{type}$}
                \LeftLabel{(switch)}
                    \insertBetweenHyps{\hskip -90pt}
                \BinaryInfC{$\Gamma \vdash \lambda x . \lambda y . x y \Leftarrow T$}
            \end{prooftree}
            \vfill
        \end{landscape}
        
        \begin{proof}
        We begin with the judgement $\Gamma \vdash \lambda x . \lambda y . x y \Leftarrow T$, now the only way to arrive at this judgement is via the mode-switching rule. Whilst doing this we add type variables $A$ and $B$ which can easily be seen to form into $A \to B$ and let $T \equiv A \to B$. We can come back later and validate this judgement. The mode-switching should have given us $\Gamma \vdash \lambda x . \lambda y . x y \Rightarrow A \to B$ which we can only arrive at by applying the ($\to$-intro) rule. This gives us $\Gamma , x : A \vdash \lambda y . xy \Leftarrow B$. Which we have to mode-switch, and as before we take this chance to introduce type variables $C$ and $D$ in order to arrive at the judgement $\Gamma , x : A \vdash \lambda y . x y \Rightarrow C \to D$. This allows us to apply ($\to$-intro) giving us $\Gamma , x : A , y : C \vdash xy \Leftarrow D$. Now we apply the ($\to$-elim) rule since we have an application. For this we need $\Gamma , x : A, y : C \vdash y \Leftarrow C$, which is marked as $(\dagger)$, and observe that a simple application of mode-switching and the variable rule allows us to derive this judgement. The other hypothesis we need is $\Gamma , x : A, y : C \vdash x \Leftarrow C \to D$. Again by mode-switching and setting $C \to D \equiv A$ we get $\Gamma , x : A, y : C \vdash x \Rightarrow A$ which is clearly derivable by the variable rule.
        
        Now we have 3 type equations $(*)$, $(**)$ and $(***)$, substituting back in we get $\Gamma \vdash T \equiv (C \to D) \to C \to D$ for some types $C$ and $D$. So $\Gamma \vdash \lambda x . \labmda y . x y \Leftarrow T$ if we have types $C$ and $D$.
        \end{proof}
\end{example}

\begin{remark}
    There is a lot going on the the previous example, but crucially it should be observed that it is in fact the \emph{inversion lemmas} that allow us to make choices of which rules to use. So a type-checking algorithm would have to make choices based on what the inversion lemmas say. We also introduced equalities of types which was brushed over. In general, type equalities are only generated by reflexivity so in a way our equations were lifted to equality of syntax. This gave us a classical equality problem. Since all our syntax are trees, we can easily decide their equality. [[CAN YOU???!!]]
\end{remark}

%% Here are examples I would like to add
%
%    i) \x.x                (Identity)
%   ii) \x.\y.xy            (Application function)
%   iv) (\x.x)(\x.x)        (Doesn't type-check)
%    v) \x.xx               (Doesn't type-check (M combinator (Mockingbird)))
%   vi) \x.\y.((xy)(xy))    (Doesn't type-check)
%  vii) \x.\y.\z.x(yz)      (B combinator (Bluebird) function composition!)
%   ix) \x.\y.y(xy)         (O combinator (Owl))
%    x) \x.\y.\z.x(y,z)     (Curry!)

\subsection{Mockingbird $\lambda x . x x$} % Definitely does not type check M
\subsection{$(\lambda x . x)(\lambda x . x)$} %  MI
\subsection{$\lambda x . \lambda y . (xy)(xy)$} % Does not type check BMB
\subsection{Y-combinator $\lambda x . (\lambda y . x (y y)) (\lambda y . x (y y))$} % Does not type check :O oh no!! no recursion!
\subsection{Function composition $\lambda x . \lambda y . \lambda z . x ( y z)$} % Function comp
\subsection{Owl combinator $\lambda x . \lambda y . \lambda z . y (x y)$} % Type checks ((A->B)->A)->(A->B)->A
\subsection{Currying $\lambda x . \lambda y . \lambda z . x (y, z)$} % Curry
\subsection{Swap $\lambda t . (\snd(t), \fst(t))$}








% Extra STLC
\section{Extensions of simply typed lambda calculus}

\subsection{Introduction}

Historically the addition of a natural numbers type with a recursion principle $\N$ was done by G\"odel in his \emph{``System T''} of Higher-Order recursion \cite{godel1958}. This is different than just having \emph{encoded} numbers in type theory. For example in \stlc we have \emph{Church numerals} \cite{Leivant:1990:DP:91556.91675, Sorensen}. It is possible to define the basic operations of arithmetic, including subtraction.

Church-Encodings are what are known as impredicative encodings, whereby the terms of the types are the same as the desired one but the eliminators are not present. This is demonstrated for Church-encodings of the natural numbers by the fact that it is \emph{impossible} to define recursion over the natural numbers in \stlc \cite{Sorensen}.

This isn't the case for \emph{untyped} lambda calculus however. It is well-known that untyped lambda calculus can have recursive definitions, see Example \ref{y_comb}. But they come at a cost, not every term in untyped lambda calculus is normalising. This corresponds to a computation which doesn't halt and is intimately related to the halting problem \cite{church1936, church-unsolvableproblemof-1936}. A natural numbers type can however be added to \stlc leading to a type theory that is ``equivalent'', in some sense, to G\"odel's system T. Though not all presentations are the same, for example comparing \cite{harper_2016} with no products and \cite{Sorensen} with products. So we are not too worried about our type theory being so different from G\"odels.

We will also look at some other types such as sums and $0$ and eventually exhibit the properties of this type theory as a propositional logic, as the Curry-Howard correspondence suggests.

\subsection{Natural numbers}

We add natural numbers. This will be our first example of an \emph{inductive type}.

\begin{defin}
    The \emph{natural numbers type $\N$} is defined by the following rules. First we begin with the formation rule.
    
    \begin{prooftree}
        \AxiomC{}
        \RightLabel{($\N$-form)}
        \UnaryInfC{$\Gamma \vdash \N \ \mathsf{type}$}
    \end{prooftree}

    We next introduce our two introduction rules. The term $0$ is of type $\N$, and given any $n : \N$, then $s(n) : \N$ also. We call $s : \N \to \N$ the \emph{sucessor function} and $s(n)$ is the \emph{sucessor} of $n$.

    \begin{center}
        \AxiomC{}
        \RightLabel{($\N$-intro${}_1$)}
        \UnaryInfC{$\Gamma \vdash 0 \Rightarrow \N$}
        \DisplayProof
            \hskip 1.5em
        \AxiomC{$\Gamma \vdash n \Leftarrow \N$}
        \RightLabel{($\N$-intro${}_2$)}
        \UnaryInfC{$\Gamma \vdash s(n) \Rightarrow \N$}
        \DisplayProof
    \end{center}

    Our elimination principle tells us how to build functions out of this type. This is our induction principle. The term $c : A$ corresponds to the value of the function at $0$. The term $f : \N \to A \to A$ takes in the previous $n : \N$, and the previous value $a$ from $A$, and gives the next value $f(n)(a): A$. 

    \begin{prooftree}
        \AxiomC{$\Gamma \vdash f \Leftarrow \N \to A \to A$}
        \AxiomC{$\Gamma \vdash c \Leftarrow A$}
        \RightLabel{($\N$-elim)}
        \BinaryInfC{$\Gamma \vdash \indn (f, c) \Rightarrow \N \to A$}
    \end{prooftree}
    
    Now we want these rules to compute how we intended them to. So we add computation rules for behviour at $0$.

    \begin{prooftree}
        \AxiomC{$\Gamma \vdash f \Leftarrow \N \to A \to A$}
        \AxiomC{$\Gamma \vdash c \Leftarrow A$}
        \RightLabel{($\N$-$\beta_1$)}
        \BinaryInfC{$\Gamma \vdash \indn (f, c) (0) \equiv c : A$}
    \end{prooftree}

    And computation rules for behaviour at a sucessor.

    \begin{prooftree}
        \AxiomC{$\Gamma \vdash f \Leftarrow \N \to A \to A$}
        \AxiomC{$\Gamma \vdash c \Leftarrow A$}
        \AxiomC{$\Gamma \vdash n \Leftarrow \N$}
        \RightLabel{($\N$-$\beta_2$)}
        \TrinaryInfC{$\Gamma \vdash \indn (f, c) (s(n)) \equiv f(n)(\indn (f, c)(n)) : A$}
    \end{prooftree}
    
    All rules can be found in Appendix \ref{nat_type}.
\end{defin}

\begin{remark}
    What we have defined is a more powerful programming language than the STLC. In fact we can define many arithmetic functions on this, and most importantly, recursive functions. We won't however provide any examples, since we find this extremely tedious to show!
\end{remark}

\subsection{Empty type}

\begin{defin}
    The \emph{empty type} is defined by the following rules:
    
    \begin{center}
        % Empty formation
        \AxiomC{}
        \RightLabel{($\mathbf{0}$-form)}
        \UnaryInfC{$\Gamma \vdash \mathbf{0} \ \mathsf{type}$}
        \DisplayProof
            \hskip 1.5em
        % Empty elimination
        \AxiomC{$\Gamma \vdash A \ \mathsf{type}$}
        \RightLabel{($\mathbf{0}$-elim)}
        \UnaryInfC{$\Gamma \vdash \indz \Rightarrow \mathbf{0} \to A$}
        \DisplayProof
    \end{center}
\end{defin}

    The first is the formation rule ($\mathbf{0}$-form), simply asserting that the empty type exists.
    Notice how there are no constructors. This is precisely because the empty type is, well, empty.
    The second is the elimination rule, supposing we had a term $t : \mathbf{0}$, then we could inhabit any type $A$. For reference the rules are in Appendix \ref{empty_type}.
    

The empty type acts as our \emph{absurdity} in type theory. If we consider types $A$ as logical propositions, then the type $A \to \mathbf{0}$ can be considered \emph{not} $A$. For if we had a proof of $A$, we would have a proof of $\mathbf{0}$ which is \emph{absurd}.

\begin{remark}
    Notice that as a consequence, any type $\mathbf{0} \to A$ is inhabitated by a single term. In fact, if we had some sort of equality of types (judgmental equality is too strict), we could show that ``$\mathbf{0} \to A = \mathbf{1}$''. If we write $\mathbf{0} \to A$ in a more suggestive notation $A^{\mathbf{0}}$ then we can see what we mean.
\end{remark}

\subsection{Sum types}

\begin{defin}
    \emhp{Sum types} are given by the following collection of rules. A \emph{type theory with sum types} is understood to include these rules. These are summarised in Appendix \ref{sum_types}. 
    We begin with the formation rule of the sum type. This is pretty typical, given two types $A$ and $B$, we have a sum type $A + B$.
    % + formation
    \begin{prooftree}
        \AxiomC{$\Gamma \vdash A \ \mathsf{type}$}
        \AxiomC{$\Gamma \vdash B \ \mathsf{type}$}
        \RightLabel{($+$-form)}
        \BinaryInfC{$\Gamma \vdash A + B \ \mathsf{type}$}
    \end{prooftree}

    Terms of the sum type are where things get interesting. We see that terms either come as labelled terms of $A$ or labelled terms of $B$. The labelling allows for the determination of where the term came from, in from the left ($\inl$) or in from the right ($\inr$).

    % + intro 1
    \begin{prooftree}
        \AxiomC{$\Gamma \vdash a \Leftarrow A$}
        \RightLabel{($+$-intro${}_1$)}
        \UnaryInfC{$\Gamma \vdash \inl(a) \Rightarrow A + B$}
    \end{prooftree}

    % + intro 2
    \begin{prooftree}
        \AxiomC{$\Gamma \vdash b \Leftarrow B$}
        \RightLabel{($+$-intro${}_2$)}
        \UnaryInfC{$\Gamma \vdash \inr(b) \Rightarrow A + B$}
    \end{prooftree}

    Next our elimination rule specifies how to define functions coming out of the sum type $A + B$. Given functions $f : A \to C$ and functions $g : B \to C$, we can construct a function $\indp(f,g) : A + B \to C$.

    % + elim
    \begin{prooftree}
        \AxiomC{$\Gamma \vdash f \Leftarrow A \to C$}
        \AxiomC{$\Gamma \vdash g \Leftarrow B \to C$}
        \RightLabel{($+$-elim)}
        \BinaryInfC{$\Gamma \vdash \indp(f,g) \Rightarrow A + B$}
    \end{prooftree}

    And finally we need to make sure that this function computes as we expect it to. If a term $t : A + B$ is of the form $\inl(a)$, then applying $\indp(f, g)$ to $\inl(a)$ is the same as applying $f$ to $a$. If a term $t : A + B$ is of the from $\inr(b)$, then applying $\indp (f, g)$ to $\inr (b)$ is the same as applying $g$ to $b$.

    % + comp 1
    \begin{prooftree}
        \AxiomC{$\Gamma \vdash a \Leftarrow A$}
        \AxiomC{$\Gamma \vdash f \Leftarrow A \to C$}
        \AxiomC{$\Gamma \vdash g \Leftarrow B \to C$}
        \RightLabel{($+$-$\beta_1$)}
        \TrinaryInfC{$\Gamma \vdash \indp(f, g)(\inl(a)) \equiv f(a) : C$}
    \end{prooftree}

    % + comp 2
    \begin{prooftree}
        \AxiomC{$\Gamma \vdash b \Leftarrow B$}
        \AxiomC{$\Gamma \vdash f \Leftarrow A \to C$}
        \AxiomC{$\Gamma \vdash g \Leftarrow B \to C$}
        \RightLabel{($+$-$\beta_2$)}
        \TrinaryInfC{$\Gamma \vdash \indp(f, g)(\inr(b)) \equiv g(b) : C$}
    \end{prooftree}
\end{defin}

Sum types earn their name from how they interact with the unit type and the empty type. Clearly $A + \mathbf{0}$ is basically a version of $A$ where all the elements are labelled with $\inl$. If we had an internal notion of equality, we could show that ``$A + \mathbf{0} = A$''. An even more rich interaction occurs with the unit type $\mathbf{1}$. Consider the type $\mathbf{1} + \mathbf{1}$. It has two elements, $\inl(*)$ and $\inr(*)$, as we can see the labels are very handy. This can be considered as the definition of a \emph{two element type}. It quickly generalises to all finite types, typically denoted $\mathbf{Fin}_n$ or $\mathbf{n}$ for some natural number $n$.
The elimination principle for such a type would essentially say: to construct $\mathbf{n} \to A$, we must pick $n$ terms of the type $A$.

Of course, there are differences such as $(\mathbf{1} + \mathbf{1}) + \mathbf{1}$ and $\mathbf{1} + (\mathbf{1} + \mathbf{1})$, however these can be removed, or more rather, shown to be irrelevant, when there is an internal notion of equality. In that sense ``$(\mathbf{1} + \mathbf{1}) + \mathbf{1} = \mathbf{1} + (\mathbf{1} + \mathbf{1})$''.

\begin{remark}
    In programming languages such as C, one could roughly say that the ``types'' of C correspond to regular types. In that way \emph{structs}, which allow programmers to pair data structures together, can be thought of as product types. Then sum types would correspond to what are known as \emph{unions}. Intuitively, these are overlapping data structures. We must be careful not to take this analogy too far however since C is missing many properties we would generally expect from a type theory.
\end{remark}

\begin{remark}
    For mathematicans, sum types are simply ``disjoint unions'' of types. If we pretended that types were sets then this would correspond exactly. The notation for sum types arises from the categorical semantics, where sums are modelled by coproducts (categorical versions of disjoint unions) \cite{LambekJ1986Itho}. 
\end{remark}

\subsection{Properties of the extended lambda calculus}

All of the normalisation properties can be extended to include the reductions defined in the extended lambda calculus. Most importantly this means System T is strongly normalising, every computation terminates after finitely many steps.
We have recursive functions which is something we could not do in the plain simply typed lambda calculus, however we are no where near as powerful as the \emph{untyped} lambda calculus. It is possible to generalise to \emph{inductive types} of all sorts, leading to well-founded tree types.

It is possible to extend languages further such that they break strong normalisation. This may not seem like an ideal thing to do but it is in fact very useful from a programming point of view. There is a theorem called the \emph{Blum Size Theorem (BST)} which states that, given any blow-up factor, say $2^{2^n}$, there is a total function (a function that can be defined in system T) such that the shortest program in T, is larger than the shortest program in PCF (the extended versiom of T that is not strongly normalising) by this factor! \cite{harper_2016}.

So we have seen that features can be added to the simply typed lambda calculus, with many of them preserving desirable properties, but sometimes in order to get practical outcomes, we might need to give up some nice properties. This is a common theme in the study of programming semantics, and is rarely considered in the mathematical considerations of simply typed lambda calculus, from what we have observed.





% History and implications of curry howard
\section{Curry-Howard correspondence}

\subsection{Introduction}

In this section we outline a detailed history of what is known as the Curry-Howard correspondence. This is an important thing to consider since there are many powerful ideas that get uncovered in this story. They will shape the future of thought on the subject, so it is worthwhile to understand what was motivating mathematicians and computer scientists at the time.

Many of these ideas were developed before the comprehension of the modern idea of a computer! So in a way it is quite remarkable that these ideas were even considered. Hence we will discuss their original motivations.

This story will develop our ideas of what needs to be added to the simply typed lambda calculus going forward. How these ideas will behave with what we have studied, and finally what the future of the subject looks like.x


%%%
\subsection{Mathematical logic}\label{logic_chapter}

At the beginning of the 20th century, Whitehead and Russell published their \emph{Principia Mathematica} \cite{GlossarWiki:Whitehead_Russell:1910}, demonstrating to mathematicians of the time that formal logic could express much of mathematics. It served to popularise modern mathematical logic leading to many mathematicians taking a more serious look at topic such as the foundations of mathematics.

One of the most influential mathematicians of the time was David Hilbert. Inspired by Whitehead and Russell's vision, Hilbert and his colleagues at G\"ottingen became leading researchers in formal logic. Hilbert proposed the \emph{Entscheidungsproblem} (decision problem), that is, to develop an ``effectually calculable procedure'' to determine the truth or falsehood of any logical statement. At the 1930 Mathematical Congress in K\"onigsberg, Hilbert affirmed his belief in the conjecture, concluding with his famous words ``Wir m\"ussen wissen, wir werden wissen'' (``We must know, we will know''). At the very same conference, Kurt G\"odel announced his proof that arithmetic is incomplete \cite{GlossarWiki:Goedel:1931}, not every statement in arithmetic can be proven.

This however did not deter logicians, who were still interested in understanding why the \emph{Entscheidungsproblem} was not attainable.
For this, a formal definition of ``effectively calculable'' was required.
So along came three candidate definitions of what it meant to be ``effectively calculable'': \emph{$\lambda$-calculus}, published in 1936 by Alonzo Church \cite{church-unsolvableproblemof-1936}; \emph{recursive functions}, proposed by G\"odel in 1934 later published in 1936 by Stephen Kleene \cite{kleene1936}; and finally \emph{Turing machines} in 1937 by Alan Turing \cite{turing1936a}.

%%%
\subsection{\texorpdfstring{$\lambda$}{}-calculus}

$\lambda$-calculus was discovered by Church at Princeton in the 1930s, originally as a way to define notations for logical formulas.
It is a very compact and simple idea, with only three constructs: variables; $\lambda$-abstraction; and function application.
Curry developed the closely related idea of combinatory logic around the same time \cite{curry1930a, curry1930b}.

Interestingly, Curry had introduced the notion of \emph{Combinators} into logic for the very same reason reason we introduced abstract binding trees: to avoid mentioning named variables \cite{Sorensen}.

It was realised at the time by Church and others that ``There may, indeed, be other applications of the system than its use as a logic.'' \cite{church1932, church1933}.
This meant that $\lambda$-calculus was worth studying as a topic of interest in it's own right.
This became explicitly apparent when Church discovered a way of encoding numbers as terms of $\lambda$-calculus, known as the \emph{Church encoding} of the natural numbers. From this addition and multiplication could also be defined.

However the problem of defining a predecessor function alluded Church and his students, in fact Church later became convinced that it was impossible.
Fortunately Kleene later discovered, at his dentist's office, how to define the predecessor function \cite{kleene1935a, kleene1935b, 4392910}.
This led to Church to later propose that $\lambda$-definability ought to be the definition of ``effectively calculable'', culminating into what is now known as Church's Thesis. Church went on to demonstrate that the problem of determining whether or not a given $\lambda$-term  has a normal form is not $\lambda$-definable.
This is now known as the Halting Problem. Put differently this says that no program written in the $\lambda$-calculus can determine whether a program written in the $\lambda$-calculus halts or not.

%%%
\subsection{Recursive functions}

In 1933 G\"odel arrived in Princeton, unconvinced by Church's claim that every effectively calculable function was $\lambda$-definable. Church responded by offering that if Go\"odel would propose a different definition, then Church would ``undertake to prove it was included in $\lambda$-definability''. In a series of lectures at Princeton, G\"odel proposed what came to be known as ``general recursive functions'' as his candidate for effective calculability. Kleene later published the definition \cite{kleene1936}. Church later outlined a proof that it was equivalent to the $\lambda$-calculus \cite{church1936} and Kleene later published it in detail \cite{kleene1936b}. This however did not have the intended effect on G\"odel, whereby he then became convinced that his own definition was incorrect!

%%%
\subsection{Turing machines}

Alan Turing was at Cambridge when he independently formulated his own idea of what it meant to be ``effectively calculable'', what is now known today as a \emph{Turing machine}. He used it to show that the Entscheidungsproblem is undecidable, meaning that it cannot be proven to be true or false. Before publication, Turing's advisor Max Newman was worried since Church had already published a solution, but since Turing's approach was sufficiently novel it was published anyway \cite{turing1936a}. Turing had added an appendix sketching the equivalence of $\lambda$-definability to Turing machines. It was Turing's argument that later convinced G\"odel that this was the correct notion of ``effectively calculable''.

Of course today the argument for Turing machines as a candidate for computation seems obvious.
We are surrounded by computers in our daily lives, all based loosely on the idea of a Turing machine.
From this it is easy to see that Turing's ideas had a \emph{huge} influence on the notions of computation.

%%%
\subsection{The problem with \texorpdfstring{$\lambda$}{}-calculus as a logic}

Church's students Kleene and Rosser quickly discovered that $\lambda$-calculus was inconsistent as a logic \cite{kleene1935c}.
A logic is deemed \emph{inconsistent} if every statement can be proven.
For example assuming $1 = 2$ can lead to many bizarre consequences, such as all logical formulas becoming true, one way or another.
In that way, arithmetic with the assumption that $1 = 2$, is \emph{inconsistent as a logic}.
Curry later published a simplified version of Kleene and Rosser's result which became known as \emph{Curry's paradox} \cite{curry1942}.
Curry's paradox was related to Russell's paradox, in that a predicate was allowed to act on itself.

Russell's paradox is typically seen as a paradox of set theory, but can usually be phrased in a much more general manner. The basic idea is this: Let $A$ be the set of all sets that do not contain themselves. The question is, does $A$ belong to itself? Clearly, if it did then it would not be an element of the set. If it did not, then it would have to be an element. Either way there is a contradiction, hence we have a \emph{logical paradox}.

The issue arises with the definition of $A$. In it we defined it as something quantifying over a lot of things, but most importantly itself. This self reference is exactly the issue that leads to such a paradox. The idea of self-reference isn't that harmful if kept under control however, particularly if a relation is \emph{well-founded}.

But allowing all predicates (formulas quantifying over other formulas), leads to silly situations as above. Much of modern set theory has been developed in order to avoid being able to write down paradoxical statements as above. We will see many of these ideas in a type theoretic form later on. A good introduction to basic set theory is \cite{johnstone1987notes}.


What is nice about Church's STLC is that every term has a normal form, or in the language of Turing machines every computation halts \cite{turing1936a}. From this consistency of Church's STLC as a logic could be established, not every logical formula is true.

%%%
\subsection{Types to the rescue}

Types were originally introduced as a method to avoid paradoxes occurring in the type-free world.
However mathematicians had naturally stratified objects into different categories, without any consideration to types before \cite{GANDY1977173, kamareddine2002}.
Russell was one of the first mathematicians to introduce a formal theory of types \cite{GlossarWiki:Whitehead_Russell:1910}, precisely to avoid the paradox baring his name.
In order to solve this problem, Church adapted a solution similar to Russell's.
The first presentation of a simple theory of types was given in Church's influential paper \cite{church1940}, where he introduced the simply typed lambda calculus.

Being typed had some immediate consequences, especially on the ideas of $\lambda$-calculus as a notion of computation. We saw in Example \ref{y_comb}, certain computational properties of the untyped lambda calculus, such as recursion are lost. What we are left with is a strictly weaker programming language. But one that is at least consistent as a system of logic.

%%%
\subsection{Types in the design of programming languages}

[[TODO]]

%%%
\subsection{Propositions as types}

We have previously discussed a condensed form of the two judgements $a \Leftarrow A$ and $a \Rightarrow A$, which we will denote $a : A$. This is in fact the judgement considered without bidirectional type checking, and the one found in most literature on the subject. We had our own reasons for choosing this set up but now we want to discuss how types and propositions are related, hence we will only be mentioning $a : A$.

The basic idea of the propositions as types is to consider types as propositional formulas, and terms as proof of those propositions. Suppose I had a type $\mathrm{TheSkyIsBlue}$. Then terms of this type would correspond to evidence or proofs that the sky is indeed blue. Type formers can then be seen as logical connectives. Suppose $A$ is the type ``Is a Cow''. Terms of $A$ are proofs that what ever we are considering, it \emph{is a cow}. Now suppose there is a type $B$ called ``Goes moo''. Terms of this type are proofs that what ever we are considering, \emph{goes moo}. Say we want to prove the statement: ``If it is a cow, then it goes moo''.
Proving such a statement would go something like this: Suppose what we are considering is a cow, through this series of logical steps we arrive at the conclusion that it goes moo.
If $A$ and $B$ were \emph{propositions}, then it would be agreed that we have proven $A \implies B$. It's as if we have taken a proof of $A$ and turned it into a proof of $B$. We have types that do exactly that: Function types!
So our proof of $A \to B$ is really just a function $\lambda x . y$ that takes in a proof $x : A$ and gives a proof $y : B$.

Of course implication isn't what logic is all about, a natural question to ask is what conjunction (the fancy word for ``and'') corresponds to. Proving that ``\emph{the sky is blue}'' and ``\emph{the grass is green}'', in a way, requires \emph{two} proofs, one corresponding to each proposition. Let $A$ be the type denoting that the sky is blue and $B$ be the type denoting that the grass is green. Then $A \times B$, the product type, is the type that both these ``propositions'' hold. How do we construct a term of the product type? Well we need to give a pair $(a, b)$, which consists of a term $a : A$ and a term $b : B$. Or in other words, to give a proof of $A \times B$ we need to give a proof of $a$ and a proof of $B$. We will later look at other logical connectives. [[TODO reference other logical connectives]].

Curiously, Curry had noticed something similar in \cite[footnote 28]{10.2307/2266302}, though his motivations were far less bovinal: \say{Note the similarity of the postulates for $F$ and those for $P$. If in any of the former postulates we change $F$ to $P$ and drop the combinator we have the corresponding postulate for $P$.}
Here $P$ is the combinator for implication and $F$ is a ``functionality'' combinator, whereby $F A B f$ essentially means $f : A \to B$.
There is evidence to suggest that this idea wasn't new to Curry. Hindley \cite{hindley_1997} points out that a remark on page 588 of \cite{10.2307/86796} indicates this. Another hint is that the properties of implication (denoted $\supset$) are named PB, PC, PW and PK, after the combinators B, C, W and K.

The correspondence was made precise (in typed combinatory logic) by Curry and Feys in \cite[Chapter 9]{curry1958combinatory}. There are two theorems proved in this chapter, under the title of ``F-P transformation'' (the notation from earlier): Theorem 1 states that inhabited types are provable; Theorem 2 states that the converse.

We note that other authors had also independently noted a link between proofs and combinators \cite{meredith1963}. 

\subsection{Gentzen's cut-elimination}




\subsection{The correspondence is dynamic}






[Overview of the full nature of the observation, much deeper than a simple correspondence since logic is in some sense ``very correct'' and programming constructs corresponding to these must therefore also be ``very correct''.]

\subsection{First-order logic}


\begin{quotation}
    Figaro shaves all men of Seville who do not shave themselves.
    But he does not shave anyone who shaves himself.
    Therefore Figaro is not a man of Seville.
\end{quotation}

%%%
\subsection{Predicates [CHANGE] as types?}

[Talk about predicate quantifiers $\forall, \exists$ and what a ``dependent type ought to do'']

%%%
\subsection{Dependent types}


[Perhaps expand on the simply typed section]

[talk about pi and sigma types

[talk about ``dependent contexts'']




% Conclusion
\section{Conclusion and Future directions}

The natural direction to consider after ariving at the Curry-Howard correspondence are dependent types. We discussed mostly propositional logic in this dissertation, but there was nothing stopping us from considering first-order logic. First-order since predicates can \emph{quantify over} propositions. These correspond to the familar $\forall$ and $\exists$ that mathematicians are acustomed to using.

Dependent types generalise their non-dependent counterparts by allowing certain types that form the overall type to \emph{depend} on the value of the terms coming from a type elsewhere. So for example one could write $\mathbf{Months} \times \mathbf{Days}$, to americanly have, a type of dates. Clearly this is complete nonsense since we can have a pair $(\mathrm{Feb}, 31)$. Ideally we would like such terms to not be well-typed. The solution is to let the type of $\mathbf{Days}$ \emph{depend} on the type of $\mathbf{Months}$. In a dependent type theory one would write $$\sum_{m : \mathbf{Months}} \mathbf{Days}(m)$$, where $\mathbf{Days}(m)$ is the type of days of the month $m$. Terms of this \emph{Sigma type} are called dependent pairs. Now the term $(\mathrm{Feb}, 28)$ type checks as before, however $(\mathrm{Feb}, 29)$ doesn't. What we have written is complete nonsense. Clearly the type checker is upset because $\mathbf{Days}(\mathrm{Feb})$ does not have a term $29$.

It turns out dependent type theories have similar normalisation properties. But as a programming language, they are still not understood as well as some of our other programming langauges \cite{Sorensen, DEBRUIJN1994141, DEBRUIJN1972381}. In the future, mainstream functional programming languages such as Haskell, will slowly gain support for dependent types. Generalising a vast array of previous programming features such as Generalized Algrbraic Data Types, Parametricity, Polymorphism and so on \cite{2016arXiv161007978E}. This will allow programmers to reason in rich ways about the correctness of their programs and allow mathematicans to write proofs in a programming language, due to the Curry-Howard correspondence. This is already done at a mass scale today with Formal verification software.

And finally back to basics, we hope that in the future, syntax and its subtleties can be sorted out for good, so that computer scientists won't need to spread white lies when discussing type theories. There is some recent work (formalised too!) in these directions \cite{Binding_Syntax_Theory}. From what we have read, this is essentially a formalised version of Harper's abts, noting that the idea is not unique to him.

Our original goal was to study the categorical semantics of dependent type theories, but along the journey we learnt that even the simple things are not so well understood yet. This means there is oppurtunity to learn, teach and grow.


\newpage
\bibliographystyle{plain} 
\bibliography{uthesis}

%Fixes toc numbering for appendix
\addtocontents{toc}{\protect\setcounter{tocdepth}{1}}
\newpage
\begin{appendices}
	\section{Simply typed lambda calculus $\lambda_{\to \times}$}\label{stlc_rule}

This is the full-presentation of the simply typed lambda calculus $\lambda_{\to \times}$. It has function types, product types and a unit type.

\subsection{Syntax}\label{stlc_syntax}

Written in BNF:

$$
    \mathrm{Term} ::= x \mid \lambda x . a \mid (a, b) \mid a b \mid c
$$

$$
    \mathrm{Type} ::= \mathbf{1} \mid A \times B \mid A \to B
$$

Or listed as operators:

\begin{center}
        \begin{tabular}{|c|c|c|c|c|c|c|}
        \hline Op & Sort & Vars & Type args & Term args & Scoping & Syntax \\
        \hline $\to$           & \ty &  --- & $A,B$ &  ---  &  ---  & $A \to B$            \\
        \hline $\times$        & \ty &  --- & $A,B$ &  ---  &  ---  & $A \times B$         \\
        \hline $(-,-)$         & \tm &  --- &  ---  & $x,y$ &  ---  & $(x,y)$              \\
        \hline $\lambda$       & \tm &  $x$ & $A,B$ &  ---  &  $M$  & $\lambda (x : A).M$  \\
        \hline $\mathrm{App}$  & \tm &  --- & $A,B$ &  ---  & $M,N$ & $M N$ \\
        \hline
    \end{tabular}
\end{center}

\subsection{Judgements}\label{stlc_judgements}

\begin{center}
    \begin{tabular}{|l|l|}
        \hline Judgement & Meaning \\
        \hline $\Gamma \vdash A\ \mathsf{type}$          & $A$ is a type in context $\Gamma$. \\
        \hline $\Gamma \vdash T \Leftarrow A$            & $T$ can be checked to have type $A$ in context $\Gamma$. \\
        \hline $\Gamma \vdash T \Rightarrow A$           & $T$ synthesises the type $A$ in context $\Gamma$. \\
        \hline $\Gamma \vdash A \equiv B\ \mathsf{type}$ & $A$ and $B$ are judgmentally equal types in context $\Gamma$. \\
        \hline $\Gamma \vdash S \equiv T : A$            & $S$ and $T$ are judgmentally equal terms of type $A$ in context $\Gamma$. \\
        \hline
    \end{tabular}
\end{center}

\subsection{Structural rules}\label{stlc_structural}

% Variable rule
\begin{center}\label{stlc_rule_var}\label{stlc_rule_switch}
    \AxiomC{$(x:A) \in \Gamma$}
    \RightLabel{(var)}
    \UnaryInfC{$\Gamma \vdash x \Rightarrow A$}
    \DisplayProof
        \hskip 1.5em
% Switch rule
    \AxiomC{$\Gamma \vdash t \Rightarrow A$}
    \AxiomC{$\Gamma \vdash A \equiv B \ \mathsf{type}$}
    \RightLabel{(switch)}
    \BinaryInfC{$\Gamma \vdash t \Leftarrow B$}
    \DisplayProof
\end{center}

% Compact switch
\begin{prooftree}\label{stlc_rule_cswitch}
    \AxiomC{$\Gamma \vdash t \Rightarrow A$}
    \RightLabel{(cswitch)}
    \UnaryInfC{$\Gamma \vdash t \Leftarrow A$}
\end{prooftree}

[[TODO: Include admissible rules?]]

\subsection{Equality rules}\label{stlc_eq}

% Reflexivity of judgemental equality of types
\begin{prooftree}\label{stlc_rule_type_refl}\label{stlc_type_symm}
    \AxiomC{$\Gamma \vdash A \ \mathsf{type}$}
    \RightLabel{($\equiv_{\mathsf{type}}$-refl)}
    \UnaryInfC{$\Gamma \vdash A \equiv A\ \mathsf{type}$}
    \DisplayProof
        \hskip 1.5em
% Symmetry of judgemental equality of types
    \AxiomC{$\Gamma \vdash A \equiv B \ \mathsf{type}$}
    \RightLabel{($\equiv_{\mathsf{type}}$-symm)}
    \UnaryInfC{$\Gamma \vdash B \equiv A \ \mathsf{type}$}
    \DisplayProof
\end{prooftree}

% Transitivity of judgemental equality of types
\begin{prooftree}\label{stlc_rule_type_tran}
    \AxiomC{$\Gamma \vdash B \ \mathsf{type}$}
    \AxiomC{$\Gamma \vdash A \equiv B\ \mathsf{type}$}
    \AxiomC{$\Gamma \vdash B \equiv C\ \mathsf{type}$}
    \RightLabel{($\equiv_\mathsf{type}$-tran)}
    \TrinaryInfC{$\Gamma \vdash A \equiv C\ \mathsf{type}$}
\end{prooftree}

% Reflexivity of judgemental equality of terms
\begin{prooftree}\label{stlc_rule_term_refl}\label{stlc_rule_term_symm}
    \AxiomC{$\Gamma \vdash t \Leftarrow A$}
    \RightLabel{($\equiv_{\mathsf{term}}$-refl)}
    \UnaryInfC{$\Gamma \vdash t \equiv t : A$}
    \DisplayProof
        \hskip 1.5em
% Symmetry of judgemental equality of terms
    \AxiomC{$\Gamma \vdash s \equiv t : A$}
    \RightLabel{($\equiv_{\mathsf{term}}$-symm)}
    \UnaryInfC{$\Gamma \vdash t \equiv s : A$}
    \DisplayProof
\end{prooftree}

% Transitivity of judgemental equality of terms
\begin{prooftree}\label{stlc_rule_term_tran}
    \AxiomC{$\Gamma \vdash t \Leftarrow A $}
    \AxiomC{$\Gamma \vdash s \equiv t : A$}
    \AxiomC{$\Gamma \vdash t \equiv r : A$}
    \RightLabel{($\equiv_{\mathsf{term}}$-tran)}
    \TrinaryInfC{$\Gamma \vdash s \equiv r : A$}
\end{prooftree}

% judgemental equality of types - judgemental equality of terms - congruence
\begin{prooftree}\label{stlc_rule_term_type_cong}
    \AxiomC{$\Gamma \vdash A \ \mathsf{type}$}
    \AxiomC{$\Gamma \vdash s \equiv t : A$}
    \AxiomC{$\Gamma \vdash A \equiv B\ \mathsf{type}$}
    \RightLabel{($\equiv_{\mathsf{term}}$-$\equiv_{\mathsf{type}}$-cong)}
    \TrinaryInfC{$\Gamma \vdash s \equiv t : B$}
\end{prooftree}

\subsection{Function type}\label{stlc_rule_arrow}

% -> formation
\begin{center}\label{stlc_rule_arrow_form}\label{stlc_rule_arrow_intro}
    \AxiomC{$\Gamma \vdash A\ \mathsf{type}$}
    \AxiomC{$\Gamma \vdash B\ \mathsf{type}$}
    \RightLabel{($\to$-form)}
    \BinaryInfC{$\Gamma \vdash A \to B \ \mathsf{type}$}
    \DisplayProof
        \hskip 1.5em
% -> introduction
    \AxiomC{$\Gamma , x : A\vdash M \Leftarrow B$}
    \RightLabel{($\to$-intro)}
    \UnaryInfC{$\Gamma \vdash \lambda x . M \Rightarrow A \to B$}
    \DisplayProof
\end{center}

% -> elimination
\begin{prooftree}\label{stlc_rule_arrow_elim}
    \AxiomC{$\Gamma \vdash M \Leftarrow A \to B$}
    \AxiomC{$\Gamma \vdash N \Leftarrow A$}
    \RightLabel{($\to$-elim)}
    \BinaryInfC{$\Gamma \vdash M N \Rightarrow B$}
\end{prooftree}

% -> beta
\begin{center}\label{stlc_rule_arrow_beta}\label{stlc_rule_arrow_eta}
    \AxiomC{$\Gamma , x : A \vdash y \Leftarrow B$}
    \AxiomC{$\Gamma \vdash t \Leftarrow A$}
    \RightLabel{($\to$-$\beta$)}
    \BinaryInfC{$\Gamma \vdash (\lambda x . y) t \equiv y[t / x] : B$}
    \DisplayProof
        \hskip 1.5em
% -> eta
    \AxiomC{$\Gamma , y : A \vdash M y \equiv M' y : B$}
    \RightLabel{($\to$-$\eta$)}
    \UnaryInfC{$\Gamma \vdash M \equiv M' : A \to B$}
    \DisplayProof
\end{center}

% -> formation congruence
\begin{prooftree}\label{stlc_rule_arrow_form_cong}
    \AxiomC{$\Gamma \vdash A \equiv A' \ \mathsf{type}$}
    \AxiomC{$\Gamma \vdash B \equiv B' \ \mathsf{type}$}
    \RightLabel{($\to$-$\equiv_{\mathsf{type}}$-cong)}
    \BinaryInfC{$\Gamma \vdash A \to B \equiv A' \to B' \ \mathsf{type}$}
\end{prooftree}

% -> introduction congruence
\begin{prooftree}\label{stlc_rule_arrow_intro_cong}
    \AxiomC{$\Gamma , x : A \vdash M \equiv M' : B$}
    \RightLabel{($\to$-$\equiv_{\mathsf{term}}$-cong)}
    \UnaryInfC{$\Gamma \vdash \lambda x . M \equiv \lambda x . M' : A \to B$}
\end{prooftree}

% -> elimination congruence
\begin{prooftree}\label{stlc_rule_arrow_elim_cong}
    \AxiomC{$\Gamma \vdash M \equiv M' : A \to B$}
    \AxiomC{$\Gamma \vdash N \equiv N' : A$}
    \RightLabel{($\to$-elim-cong)}
    \BinaryInfC{$\Gamma \vdash M N \equiv M' N' : A \to B$}
\end{prooftree}

\subsection{Product type}\label{stlc_rule_prod}

% x introduction
\begin{prooftree}\label{stlc_rule_prod_form}\label{stlc_rule_prod_intro}
    \AxiomC{$\Gamma \vdash A \ \mathsf{type}$}
    \AxiomC{$\Gamma \vdash B \ \mathsf{type}$}
    \RightLabel{($\times$-form)}
    \BinaryInfC{$\Gamma \vdash A \times B \ \mathsf{type}$}
    \DisplayProof
% x formation
    \AxiomC{$\Gamma \vdash a \Leftarrow A$}
    \AxiomC{$\Gamma \vdash b \Leftarrow B$}
    \RightLabel{($\times$-intro)}
    \BinaryInfC{$\Gamma \vdash (a, b) \Rightarrow A \times B$}
    \DisplayProof
\end{prooftree}



% x eliminators
\begin{center}\label{stlc_rule_prod_elim}
    \AxiomC{$\Gamma \vdash t \Leftarrow A \times B$}
    \RightLabel{($\times$-elim${}_1$)}
    \UnaryInfC{$\Gamma \vdash \fst(t) \Rightarrow A$}        
    \DisplayProof
        \hskip 1.5em
    \AxiomC{$\Gamma \vdash t \Leftarrow A \times B$}
    \RightLabel{($\times$-elim${}_2$)}
    \UnaryInfC{$\Gamma \vdash \snd(t) \Rightarrow B$}
    \DisplayProof
\end{center}

% x betas
\begin{center}\label{stlc_rule_prod_beta}
    \AxiomC{$\Gamma \vdash x \Leftarrow A$}
    \AxiomC{$\Gamma \vdash y \Leftarrow B$}
    \RightLabel{($\times$-$\beta_1$)}
    \BinaryInfC{$\Gamma \vdash \fst(x,y)\equiv x : A$}
    \DisplayProof
        \hskip 1.5em
    \AxiomC{$\Gamma \vdash x \Leftarrow A$}
    \AxiomC{$\Gamma \vdash y \Leftarrow B$}
    \RightLabel{($\times$-$\beta_2$)}
    \BinaryInfC{$\Gamma \vdash \snd(x,y)\equiv y : B$}
    \DisplayProof
\end{center}

% x eta
\begin{prooftree}\label{stlc_rule_prod_eta}
    \AxiomC{$\Gamma \vdash \fst(t) \equiv \fst(t') : A$}
    \AxiomC{$\Gamma \vdash \snd(t) \equiv \snd(t') : B$}
    \RightLabel{($\times$-$\eta$)}
    \BinaryInfC{$\Gamma \vdash t \equiv t' : A \times B$}
\end{prooftree}

% x form cong
\begin{prooftree}\label{stlc_rule_prod_form_cong}
    \AxiomC{$\Gamma \vdash A \equiv A' \ \mathsf{type}$}
    \AxiomC{$\Gamma \vdash B \equiv B' \ \mathsf{type}$}
    \RightLabel{($\times$-$\equiv_{\mathsf{type}}$-cong)}
    \BinaryInfC{$\Gamma \vdash A \times B \equiv A' \times B' \ \mathsf{type}$}
\end{prooftree}

% x intro cong
\begin{prooftree}\label{stlc_rule_prod_intro_cong}
    \AxiomC{$\Gamma \vdash a \equiv a' : A$}
    \AxiomC{$\Gamma \vdash b \equiv b' : B$}
    \RightLabel{($\times$-$\equiv_{\mathsf{term}}$-cong)}
    \BinaryInfC{$\Gamma \vdash (a,b) \equiv (a',b') : A \times B$}
\end{prooftree}

% x elim1 cong
\begin{prooftree}\label{stlc_rule_prod_elim1_cong}
    \AxiomC{$\Gamma \vdash t \equiv t' : A \times B$}
    \RightLabel{($\times$-elim${}_1$-cong)}
    \UnaryInfC{$\Gamma \vdash \fst(t) \equiv \fst(t') : A$}
\end{prooftree}

% x elim2 cong
\begin{prooftree}\label{stlc_rule_prod_elim2_cong}
    \AxiomC{$\Gamma \vdash t \equiv t' : A \times B$}
    \RightLabel{($\times$-elim${}_2$-cong)}
    \UnaryInfC{$\Gamma \vdash \snd(t) \equiv \snd(t') : B$}
\end{prooftree}

\subsection{Unit type}\label{stlc_rule_unit}

% Unit formation
\begin{center}\label{stlc_rule_unit_form}\label{stlc_rule_unit_intro}
    \AxiomC{}
    \RightLabel{($\mathbf{1}$-form)}
    \UnaryInfC{$\mathbf{1}\ \mathsf{type}$}
    \DisplayProof
        \hskip 1.5em
% Unit introduction
    \AxiomC{}
    \RightLabel{($\mathbf{1}$-intro)}
    \UnaryInfC{$\Gamma \vdash * \Rightarrow \mathbf{1}$}
    \DisplayProof
\end{center}

	
    \begin{landscape}
    	\section{Examples}
        \subsection{\texorpdfstring{$\eta$}{}-rule} \begin{prooftree}
        \AxiomC{$\Gamma \vdash t \Leftarrow A \times B$}
        \LeftLabel{($\times$-elim${}_1$)}
        \UnaryInfC{$\Gamma \vdash \fst(t) \Rightarrow A$}
        \LeftLabel{(cswitch)}
        \UnaryInfC{$\Gamma \vdash \fst(t) \Leftarrow A$}
        
        \AxiomC{$\Gamma \vdash t \Leftarrow A \times B$}
        \RightLabel{($\times$-elim${}_2$)}
        \UnaryInfC{$\Gamma \vdash \snd(t) \Rightarrow B$}
        \RightLabel{(cswitch)}
        \UnaryInfC{$\Gamma \vdash \snd(t) \Leftarrow B$}
        
        \LeftLabel{($\times$-$\beta_1$)}
        \BinaryInfC{$\Gamma \vdash \fst(\fst(t), \snd(t)) \equiv \fst(t) : A \times B$}
        
        \AxiomC{$\Gamma \vdash t \Leftarrow A \times B$}
        \LeftLabel{($\times$-elim${}_1$)}
        \UnaryInfC{$\Gamma \vdash \fst(t) \Rightarrow A$}
        \LeftLabel{(cswitch)}
        \UnaryInfC{$\Gamma \vdash \fst(t) \Leftarrow A$}
        
        \AxiomC{$\Gamma \vdash t \Leftarrow A \times B$}
        \RightLabel{($\times$-elim${}_2$)}
        \UnaryInfC{$\Gamma \vdash \snd(t) \Rightarrow B$}
        \RightLabel{(cswitch)}
        \UnaryInfC{$\Gamma \vdash \snd(t) \Leftarrow B$}
        
        \RightLabel{($\times$-$\beta_2$)}
        \BinaryInfC{$\Gamma \vdash \snd(\fst(t), \snd(t)) \equiv \snd(t) : A \times B$}
        
        \LeftLabel{($\times$-$\eta$)}
        \BinaryInfC{$\Gamma \vdash (\fst(t), \snd(t))\equiv t : A \times B$}
        
    \end{prooftree}
 \label{ex3}
        \subsection{Function application \texorpdfstring{$\lambda x . \lambda y . x y$}{}} \begin{prooftree}
    %\rootAtTop
    \def\ScoreOverhang{1pt}
    %%%
    \AxiomC{$x : A \in \Gamma , x : A, y : C$}
    \LeftLabel{(var)}
    \UnaryInfC{$\Gamma , x : A, y : C \vdash x \Rightarrow A$}
    \AxiomC{}
    \RightLabel{$(***)$}
    \UnaryInfC{$\Gamma , x : A, y : C \vdash C \to D \equiv A \ \mathsf{type}$}
        %\insertBetweenHyps{\hskip -5pt}
    \BinaryInfC{$\Gamma , x : A, y : C \vdash x \Leftarrow C \to D$}
    \AxiomC{}
    \RightLabel{$(\dagger)$}
    \UnaryInfC{$\Gamma , x : A, y : C \vdash y \Leftarrow C$}
    \noLine
    \UnaryInfC{$\vdots$}
    \noLine
    \UnaryInfC{$\vdots$}
        \insertBetweenHyps{\hskip -40pt}
    \LeftLabel{($\to$-elim)}                
    \BinaryInfC{$\Gamma , x : A, y : C \vdash x y \Rightarrow D$}
    \AxiomC{}
    \RightLabel{($\equiv_{\mathsf{type}}$-refl)}
    \UnaryInfC{$\Gamma , x : A, y : C \vdash D \equiv D\ \mathsf{type}$}
    \LeftLabel{(switch)}
        \insertBetweenHyps{\hskip -50pt}
    \BinaryInfC{$\Gamma , x : A , y : C \vdash xy \Leftarrow D$}
    \LeftLabel{($\to$-intro)}
    \UnaryInfC{$\Gamma , x : A \vdash \lambda y . x y \Rightarrow C \to D$}
    \AxiomC{}
    \RightLabel{$(**)$}
    \UnaryInfC{$\Gamma , x : A \vdash B \equiv C \to D \ \mathsf{type}$}
    \LeftLabel{(switch)}
        \insertBetweenHyps{\hskip -130pt}
    \BinaryInfC{$ \Gamma , x : A \vdash \lambda y . xy \Leftarrow B$}
    \LeftLabel{($\to$-intro)}
    \UnaryInfC{$\Gamma \vdash \lambda x . \lambda y . x y \Rightarrow A \to B$}
    \AxiomC{}
    \RightLabel{$(*)$}
    \UnaryInfC{$\Gamma \vdash T \equiv A \to B \ \mathsf{type}$}
    \LeftLabel{(switch)}
        \insertBetweenHyps{\hskip -90pt}
    \BinaryInfC{$\Gamma \vdash \lambda x . \lambda y . x y \Leftarrow T$}
\end{prooftree}
 \label{ex2}
        \newpage\subsection{Function composition \texorpdfstring{$\lambda x . \lambda y . \lambda z . x ( y z)$}{}} \begin{prooftree}
    \def\ScoreOverhang{1pt}
    \AxiomC{$(x :B \to C) \in \Gamma , x : B \to C, y : A \to B , z : A$}
    \LeftLabel{(var)}
    \UnaryInfC{$\Gamma , x : B \to C, y : A \to B , z : A \vdash x \Rightarrow B \to C$}
    \LeftLabel{(cswitch)}
    \UnaryInfC{$\Gamma , x : B \to C, y : A \to B , z : A \vdash x \Leftarrow B \to C$}
    
        \AxiomC{$(z : A) \in \Gamma , x : B \to C, y : A \to B , z : A$}
        \LeftLabel{(var)}
        \UnaryInfC{$\Gamma , x : B \to C, y : A \to B , z : A \vdash z \Rightarrow A$}
        \LeftLabel{(cswitch)}
        \UnaryInfC{$\Gamma , x : B \to C, y : A \to B , z : A \vdash z \Leftarrow A$}
        
        \AxiomC{$(y : A \to B) \in \Gamma , x : B \to C, y : A \to B , z : A$}
        \RightLabel{(var)}
        \UnaryInfC{$\Gamma , x : B \to C, y : A \to B , z : A \vdash y \Rightarrow A \to B$}
        \RightLabel{(cswitch)}
        \UnaryInfC{$\Gamma , x : B \to C, y : A \to B , z : A \vdash y \Leftarrow A \to B$}
    
    \insertBetweenHyps{\hskip 10pt}
    \RightLabel{($\to$-elim)}
    \BinaryInfC{$\Gamma , x : B \to C, y : A \to B , z : A \vdash yz \Rightarrow B$}
    \RightLabel{(cswitch)}
    \UnaryInfC{$\Gamma , x : B \to C, y : A \to B , z : A \vdash yz \Leftarrow B$}
    \noLine
    \UnaryInfC{$\vdots$}
    
    \insertBetweenHyps{\hskip -110pt}
    \LeftLabel{($\to$-elim)}
    \BinaryInfC{$\Gamma , x : B \to C, y : A \to B , z : A \vdash x(yz) \Rightarrow C$}
    \LeftLabel{(cswitch)}
    \UnaryInfC{$\Gamma , x : B \to C, y : A \to B , z : A \vdash x(yz) \Leftarrow C$}
    \LeftLabel{($\to$-intro)}
    \UnaryInfC{$\Gamma , x : B \to C, y : A \to B \vdash \lambda z.x(yz) \Rightarrow A \to C$}
    \LeftLabel{(cswitch)}
    \UnaryInfC{$\Gamma , x : B \to C, y : A \to B \vdash \lambda z.x(yz) \Leftarrow A \to C$}
    \LeftLabel{($\to$-intro)}
    \UnaryInfC{$\Gamma , x : B \to C \vdash\lambda y.\lambda z.x(yz) \Rightarrow (A \to B) \to (A \to C)$}
    \LeftLabel{(cswitch)}
    \UnaryInfC{$\Gamma , x : B \to C \vdash\lambda y.\lambda z.x(yz) \Leftarrow (A \to B) \to (A \to C)$}
    \LeftLabel{($\to$-intro)}
    \UnaryInfC{$\Gamma \vdash \lambda x.\lambda y.\lambda z.x(yz) \Rightarrow (B \to C) \to (A \to B) \to (A \to C)$}
    \LeftLabel{(cswitch)}
    \UnaryInfC{$\Gamma \vdash \lambda x.\lambda y.\lambda z.x(yz) \Leftarrow (B \to C) \to (A \to B) \to (A \to C)$}
\end{prooftree}
 \label{ex7}        
        \newpage\subsection{Currying \texorpdfstring{$\lambda x . \lambda y . \lambda z . x (y, z)$}{}} \begin{prooftree}
    \def\ScoreOverhang{1pt}
    \AxiomC{$(x : A \times B \to C) \in \Gamma, x : A \times B \to C , y : A, z : B$}
    \LeftLabel{(var)}
    \UnaryInfC{$\Gamma, x : A \times B \to C , y : A, z : B \vdash x \Rightarrow A \times B \to C$}
    \LeftLabel{(cswitch)}
    \UnaryInfC{$\Gamma, x : A \times B \to C , y : A, z : B \vdash x \Leftarrow A \times B \to C$}
    
    \AxiomC{$(y : A) \in \Gamma, x : A \times B \to C , y : A, z : B$}
    \LeftLabel{(var)}
    \UnaryInfC{$\Gamma , x : A \times B \to C, y : A, z : B \vdash y \Rightarrow A$}
    \LeftLabel{(cswitch)}
    \UnaryInfC{$\Gamma , x : A \times B \to C, y : A, z : B \vdash y \Leftarrow A$}
    
    \AxiomC{$(z : B) \in \Gamma , x : A \times B \to C, y : A, z : B$}
    \RightLabel{(var)}
    \UnaryInfC{$\Gamma , x : A \times B \to C, y : A, z : B \vdash z \Rightarrow B$}
    \RightLabel{(cswitch)}
    \UnaryInfC{$\Gamma , x : A \times B \to C, y : A, z : B \vdash z \Leftarrow B$}
    
    \LeftLabel{($\times$-intro)}
    \BinaryInfC{$\Gamma , x : A \times B \to C, y : A, z : B \vdash (y, z) \Rightarrow A \times B$}
    \LeftLabel{(cswitch)}
    \UnaryInfC{$\Gamma , x : A \times B \to C, y : A, z : B \vdash (y, z) \Leftarrow A \times B$}
    
    \LeftLabel{($\to$-elim)}
    \UnaryInfC{$\Gamma, x : A \times B \to C , y : A, z : B \vdash x (y , z) \Rightarrow C$}
    \LeftLabel{(cswitch)}
    \UnaryInfC{$\Gamma, x : A \times B \to C , y : A, z : B \vdash x (y , z) \Leftarrow C$}
    \LeftLabel{($\to$-intro)}
    \UnaryInfC{$\Gamma, x : A \times B \to C , y : A \vdash \lambda z . x (y , z) \Rightarrow B \to C$}
    \LeftLabel{(cswitch)}
    \UnaryInfC{$\Gamma, x : A \times B \to C , y : A \vdash \lambda z . x (y , z) \Leftarrow B \to C$}
    \LeftLabel{($\to$-intro)}
    \UnaryInfC{$\Gamma, x : A \times B \to C \vdash \lambda y . \lambda z . x (y , z) \Rightarrow A \to B \to C$}
    \LeftLabel{(cswitch)}
    \UnaryInfC{$\Gamma, x : A \times B \to C \vdash \lambda y . \lambda z . x (y , z) \Leftarrow A \to B \to C$}
    \LeftLabel{($\to$-intro)}
    \UnaryInfC{$\Gamma \vdash \lambda x . \lambda y . \lambda z . x (y , z) \Rightarrow (A \times B \to C) \to A \to B \to C$}
    \LeftLabel{(cswitch)}
    \UnaryInfC{$\Gamma \vdash \lambda x . \lambda y . \lambda z . x (y , z) \Leftarrow (A \times B \to C) \to A \to B \to C$}
    
\end{prooftree}
 \label{ex8}
        \newpage\subsection{Uncurry} \begin{prooftree}
    \def\ScoreOverhang{1pt}
    \AxiomC{$(x : A \to B \to C) \in \Gamma , x : A \to B \to C, y : A \times B$}
    \LeftLabel{(var)}
    \UnaryInfC{$\Gamma , x : A \to B \to C, y : A \times B \vdash x \Rightarrow A \to B \to C$}
    \LeftLabel{(cswitch)}
    \UnaryInfC{$\Gamma , x : A \to B \to C, y : A \times B \vdash x \Leftarrow A \to B \to C$}
    
    \AxiomC{$(y : A \times B) \in \Gamma , x : A \to B \to C, y : A \times B$}
    \RightLabel{(var)}
    \UnaryInfC{$\Gamma , x : A \to B \to C, y : A \times B \vdash y \Rightarrow A \times B$}
    \RightLabel{(cswitch)}
    \UnaryInfC{$\Gamma , x : A \to B \to C, y : A \times B \vdash y \Leftarrow A \times B$}
    \RightLabel{($\times$-elim${}_1$}
    \UnaryInfC{$\Gamma , x : A \to B \to C, y : A \times B \vdash \fst(y) \Rightarrow A$}
    \RightLabel{(cswitch)}
    \UnaryInfC{$\Gamma , x : A \to B \to C, y : A \times B \vdash \fst(y) \Leftarrow A$}
    
    \LeftLabel{($\to$-elim)}
    \BinaryInfC{$\Gamma , x : A \to B \to C, y : A \times B \vdash x(\fst(y)) \Rightarrow A$}
    \LeftLabel{(cswitch)}
    \UnaryInfC{$\Gamma , x : A \to B \to C, y : A \times B \vdash x(\fst(y)) \Leftarrow A$}
    \noLine
    \UnaryInfC{$\vdots$}
    \noLine
    \UnaryInfC{$\vdots$}
    \noLine
    \UnaryInfC{$\vdots$}
    \noLine
    \UnaryInfC{$\vdots$}
    
    \AxiomC{$(y : A \times B) \in \Gamma , x : A \to B \to C, y : A \times B$}
    \RightLabel{(var)}
    \UnaryInfC{$\Gamma , x : A \to B \to C, y : A \times B \vdash y \Rightarrow A \times B$}
    \RightLabel{(cswitch)}
    \UnaryInfC{$\Gamma , x : A \to B \to C, y : A \times B \vdash \snd(y) \Leftarrow B$}
    \RightLabel{($\times$-elim${}_2$}
    \UnaryInfC{$\Gamma , x : A \to B \to C, y : A \times B \vdash \snd(y) \Rightarrow B$}
    \RightLabel{(cswitch)}
    \UnaryInfC{$\Gamma , x : A \to B \to C, y : A \times B \vdash \snd(y) \Leftarrow B$}
    
    
    \insertBetweenHyps{\hskip -200pt}
    \LeftLabel{($\to$-elim)}
    \BinaryInfC{$\Gamma , x : A \to B \to C, y : A \times B \vdash x(\fst(y))(\snd(y)) \Rightarrow C$}
    \LeftLabel{(cswitch)}
    \UnaryInfC{$\Gamma , x : A \to B \to C, y : A \times B \vdash x(\fst(y))(\snd(y)) \Leftarrow C$}
    \LeftLabel{($\to$-intro)}
    \UnaryInfC{$\Gamma , x : A \to B \to C \vdash \lambda y . \vdash x(\fst(y))(\snd(y)) \Rightarrow A \times B \to C$}
    \LeftLabel{(cswitch)}
    \UnaryInfC{$\Gamma , x : A \to B \to C \vdash \lambda y . \vdash x(\fst(y))(\snd(y)) \Leftarrow A \times B \to C$}
    \LeftLabel{($\to$-intro)}
    \UnaryInfC{$\Gamma \vdash \lambda x . \lambda y .  \vdash x(\fst(y))(\snd(y)) \Rightarrow (A \to B \to C) \to A \times B \to C$}
    \LeftLabel{(cswitch)}
    \UnaryInfC{$\Gamma \vdash \lambda x . \lambda y .  \vdash x(\fst(y))(\snd(y)) \Leftarrow (A \to B \to C) \to A \times B \to C$}
\end{prooftree}
 \label{ex9}
        
    \end{landscape}
\end{appendices}



\end{document}
