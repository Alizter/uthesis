\documentclass[varwidth]{article}

\usepackage[english]{babel}
\usepackage[utf8]{inputenc}
\usepackage{bussproofs}
\usepackage{framed}

% Here we squish our bibliography up
\usepackage{natbib}
\setlength{\bibsep}{1pt}
\renewcommand{\bibfont}{\small} 

% Todo purge useless packages
\usepackage{url}
\usepackage{dirtytalk}
\usepackage{pst-node}
\usepackage{tikz-cd}
\usepackage{enumerate}
\usepackage{forest}
\usepackage{mwe}
\usepackage{verbatim}
\usepackage{amsmath}
\usepackage{amssymb}
\usepackage{amsthm}
\usepackage{stmaryrd}
\usepackage[hidelinks]{hyperref}
\usepackage{lscape}
% Ommiting this because it looks terrible
%\bibliographystyle{bath.bst}

% Theorems, definitions etc.
\theoremstyle{definition}
\newtheorem{defin}{Definition}[subsection]
\newtheorem{example}[defin]{Example}
\newtheorem{theorem}[defin]{Theorem}
\newtheorem{remark}[defin]{Remark}
\newtheorem{lemma}[defin]{Lemma}
\newtheorem{cor}[defin]{Corollary}

\newcommand{\N}{\mathbb{N}} %% Need to say NN has 0

% Typical names of categories
\newcommand{\catname}[1]{{\normalfont\textbf{#1}}}
\newcommand{\Set}{\catname{Set}}
\newcommand{\Cat}{\catname{Cat}}
\newcommand{\op}{\text{op}}

\newcommand{\el}{\text{el}}
\newcommand{\Ty}{\text{Ty}}

\newcommand{\of}{\mathbf{of}}

%%
%   Common variables
%%

% Commands for barendregt.tex
%\newcommand{\FV}{\mathrm{FV}}
%\newcommand{\A}{\mathbb{A}}
%\newcommand{\T}{\mathsf{T}}

\newcommand{\Var}{\mathbf{Var}}

% Simply typed
%\newcommand{\Dec}{\mathbf{Dec}}
%\newcommand{\Con}{\mathbf{Con}}

%\newcommand{\Typ}{\mathbf{Type}} %%
%\newcommand{\Tm}{\mathbf{Term}} %%%
%\newcommand{\Jud}{\mathbf{Jud}}


\newcommand{\Id}{\text{Id}}

%%
%   Type theories
%%

\newcommand{\stcu}{${\boldsymbol \lambda}_\to^{\text{Cu}}$} % simply typed curry 
\newcommand{\stch}{${\boldsymbol \lambda}_\to^{\text{Ch}}$} % simply typed church
\newcommand{\stdb}{${\boldsymbol \lambda}_\to^{\text{dB}}$} % simply typed de brujin
\newcommand{\utbe}{${\boldsymbol {\lambda \beta \eta}}$ } % untyped beta eta

%%
%   Turn styles
%%

\newcommand{\vdashstcu}{\vdash_{\boldsymbol \lambda_\to}^{\text{Cu}}} % vdash for curry
\newcommand{\vdashstcu}{\vdash_{\boldsymbol \lambda_\to}^{\text{Ch}}} % vdash for church
\newcommand{\vdashstcu}{\vdash_{\boldsymbol \lambda_\to}^{\text{dB}}} % vdash for de Brujin

%%
%   Harper notations
%%

\newcommand{\Term}{\mathbf{Term}}
\newcommand{\App}{\text{App}}
\newcommand{\scope}{\triangleleft}
\newcommand{\soa}{\text{soa}}
\newcommand{\sov}{\text{sov}}

\title{Introduction to dependent type theory}
\author{Ali Caglayan}

\begin{document}

\maketitle

% TODO:
%
%   * Rewrite syntax
%   * Insert missing references throughout judgement and curry
%   * Expand discussion of statics and dynamics
%   * Design and study STLC
%   * Discuss its drawbacks
%   * Expand discussion of curry howard
%   * DEPENDENT TYPES

\tableofcontents

% Introduction and direction of thesis
\section{Introduction}

%Simply typed lambda calculus (STLC) has been well documented and studied by type theorists and mathematicians, and it's features have been used by many programming languages [NEED REFERENCE].

%In \cite{BarendregtHenk2013Lcwt} it is noted that \say{Research monographs on dependent and inductive types are lacking.} This will essentially be one of the goals of this thesis, to provide a guide for mathematicians and computer scientists about the use of dependent type theory. As this document is written there is no single account of all approaches to \i{dependent} type theory.

%Awodey \cite{2014arXiv1406.3219A} made an observation that Dybjer's \cite{dybjer1996} categories with families (CwF) is a presheaf category with a representable natural transformation (it's fibers are representable). He then proceeds to show conditions needed to model a dependent type theory with $\Pi$, $\Sigma$ and $\mathrm{Id}$ types.


%This thesis will have three main goals.

%\begin{enumitem}
%	\item To present a dependent type theory
%	\item To model the semantics of such a type theory using categorical methods
%	\item To discuss the applications to mathematics and computer science (proof assistants, programming languages and foundations)
%\end{enumitem}

%Finally we may also discuss recent developments of something called "Homotopy type theory" and how that fits into the general picture.

%Roughly a \textit{type system} is a set of loosely organised rules outlining how ``atomic sentences'' called \textit{judgements} can be derived from each other in a given context. A \textit{context} can simply be thought of as a list of terms. 

%The aim of this thesis is to present to two sorts of audience, the utility of dependent type theory. The audiences that I have in mind are computer scientists, roughly individuals who wish to write good code, and mathematicians, roughly individuals who wish to write good proofs.

%These will be our main aims however we do also wish to develop the machinery formally.

%\section{Propositions as types}

%There is a rich interplay between programming and logic known as the Curry-Howard correspondance or propositions as types. 





%\section{What is type theory}

%Type theory is the study of types systems. That is a system that orginizes data manipulated by programs into types. This has been a very useful concept in computer science. It has allowed the writing of programs taht a more 

%\subsection{Lambda calculus}
%\subsection{Modelling type theory}
%\section{What is dependent type theory?}
%\subsection{What are dependent types?}
%\subsection{Motivation for computer scientists}
%\subsection{Motivation for mathematicians}
%\subsection{Category theory}
%\subsection{Categorical logic}
%\subsection{Future directions}

\begin{itemize}
\item a[Begin with history and implications of curry howard]

\item a[outline the ``what they should do'' of dependent types]

\item a[start to rigoursly model syntax and talk about how bad a job most authors do]

\item a[small section about inductive definitions]

\item a[small section on why categorical semantics]

\item a[model simply typed lambda calculus with categorical semantics]

\item a[show natural extensions of the idea and why contexts break when dependnet]

\item a[outline different approches to solving these problems]

\item a[discuss Awodey's natural models]

\item a[finally talk about future directions for type theory]

\item a[maybe some mention on applications to programming (generalising various constructs, polymorphism, GA data types)]

\item a[equality, inductive types, [[[[[maybe a tinsy bit of homotopy type theory]]]]]]
\end{itemize}

\section{Curry-Howard correspondance}

\subsection{Mathematical logic}

At the beginning of the 20th century, Whitehead and Russell pubished their \emph{Principia Mathematica} \cite{GlossarWiki:Whitehead_Russell:1910}, demonstrating to mathematicians of the time that formal logic could express much of mathematics. It served to popularise modern mathematical logic leading to many mathematicians taking a more serious look at topic such as the foundations of mathematics.

One of the most influencial mathematicians of the time was David Hilbert. Inspired by Whitehead and Russell's vision, Hilbert and his coleagues at G\"ottingen became leading researchers in formal logic. Hilbert proposed the \emph{Entscheidungsproblem} (decision problem), that is, to develop an ``effectually calculable procedure'' to determine the truth or falsehood of any logical statement. At the 1930 Mathematical Congress in K\"onigsberg, Hilbert affirmed his belief in the conjecture, concluding with his famous words ``Wir m\"ussen wissen, wir werden wissen'' (``We must know, we will know''). At the very same conference, Kurt G\"odel announced his proof that arithmetic is incomplete \cite{GlossarWiki:Goedel:1931}, not every statement in arithmetic can be proven.

This however did not deter logicians, who were still interested in understanding why the \emph{Entscheidungsproblem} was undecidable, for this a formal efinition of ``effectively calculable'' was required. So along came three proposed definitions of what it meant to be ``effectively calculable'': \emph{lambda calculus}, pusblished in 1936 by Alonzo Church \cite{church-unsolvableproblemof-1936}; \emph{recursive functions}, proposed by G\"odel in 1934 later published in 1936 by Stephen Kleene \cite{Kleene1936}; and finally \emph{Turing machines} in 1937 by Alan Turing \cite{turing1936a}.

\subsection{Lambda calculus}

(Untyped) lambda calculus was discovered by Church at princeton, originally as a way to define notations for logical formulas. It is a remarkaly compact idea, with only three constructs: variables; lambda abstraction; and function application. It was realised at the time by Church and others that ``There may, indeed, be other applications of the system than its use as a logic.'' [CITATION NEEDED]\cite{}. Church discovered a way of encoding numbers as terms of lambda calculus. From this addition and multiplication could be defined. Kleene later discovered how to define the predecessor function. [CITATION NEEDED] \cite{}. Church later rpoposed $\lambda$-definability as the definition of ``effectively calculable'', what is now known as Church's Thesis, and demonstrated that the problem of determining whether or not a given $\lambda$-term  has a normal form is not $\lambda$-definable. This is now known as the Halting Problem. 

\subsection{Recursive functions}

In 1933 G\"odel arrived in Princeton, unconvinced by Church's claim that every effectively calculable function was $\lambda$-definable. Church responded by offering that if Go\"odel would propose a different definition, then Church would ``undertake to prove it was included in $\lambda$-definability''. In a series of lectures at Princeton, G\"odel proposed what came to be known as ``general recursive functions'' as his candidate for effective calculability. Kleene later published the definition [CITATION NEEDED]\cite{}. Church later outlined a proof [CITATION NEEDED]\cite{} and Kleene later published it in detail. This however did not have the intended effect on G\"odel, whereby he then became convinced that his own definition was incorrect.

\subsection{Turing machines}

Alan Turing was at Camrbdige when he independently formulated his own idea of what it means to be "effectively calculable", now known today as Turing machines. He used it to show that the Entscheidungsproblem is undecidable, that is it cannot be proven to be true or false. Before publication, Turing's advisor Max Newman was worried since Church had published a solution, but since Turing's approach was sufficiently novel it was published anyway. Turing had added an appendix sketching the equivalence of $\lambda$-definability to Turing machines. It was Turings argument that later convinced G\"odel that this was the correct notion of ``effectively calculable''.

\subsection{Russells paradox}

[Talk about the origin of types and stuff]

\subsection{The problem with lambda calculus as a logic}

Church's lambda calculus turned out to be inconsistent. \cite{}[CITATION NEEDED]. The reason was related to russels paradox, in that a predicate was allowed to act on itself. This led to an abandoning of the use of lambda calculus as a logic for a short time. In order to solve this Church adapted a solution similar to Russell's: use types. What was discovered is now known today as \emph{simply-typed lambda calculus}. \cite{} [CITATION NEEDED, 10 ?]. What is nice about Church's STLC is that every term has a normal form, or in the language of Turing machines every computation halts. \cite{} [CITATION NEEDED] From this consistency of Church's STLC as a logic could be established.

\subsection{Types to the rescue}

[Talk in detail why typing is good for mathematicians, programmers and logicians]

\subsection{The theory of proof a la Gentzen}

[Go into the history of the theory of proof e.g. Gentzen's work; take notice of natural deduction]

\subsection{Curry and Howard}

[Curry makes an observation that Gentzens natural deduction corresponds to simply typed lambda calculus, Howard takes this further and defines it formally, eventually predicting a notion of dependent type.

\subsection{Propositions as types}

[Overview of the full nature of the observation, much deeper than a simple correspondance since logic is in some sense ``very correct'' and programming constructs corresponding to these must therefore also be ``very correct''.]

\subsection{Predicates [CHANGE] as types?}

[Talk about predicate quantifiers $\forall, \exists$ and what a ``dependent type ought to do'']


\subsection{Dependent types}

[Perhaps expand on the simply typed section]

[talk about pi and sigma types

[talk about ``dependent contexts'']




% Rigourous treatement and analysis of syntax
% harper

\section{Syntax}

\subsection{The difficulty with syntax}
Syntax is difficult to handle rigorously. The syntax of type theory has a long history of proposed solutions and an even longer history of incorrect solutions. The main difficulty lies with the fact that syntax must acount for the deceptively subtle notions of variable binding, capture-free substitution and even multiple derivations of judgements.

Mathematicians therefore have an ``\emph{ingenious}'' way of dealing with this: Abstract away the key properties to end up with an object with the \emph{desired semantics}. This objects typically fall under names such as \emph{structured categories}, and come up in the subject of \emph{categorical semantics}. Mathematicians can therefore reason about ``type theories'' by reasoning about these particular objects.

The questions still stands however \emph{what is a type theory?}. We will not claim to solve this problem but rather provide a partial solution. In this thesis we will describe a specific kind of \emph{dependent type theory}. Later we will derived the categorical semantics for our type theory. We cannot however do this all in one step, and so we will begin with what is known as \emph{simply typed lambda calculus}. We will then modify the rules for this to give us a basic \emph{dependent type theory}.

We will discuss in detail the need for some sort of ``initiality theorem'' for a given type theory. This will make the interpretation of the syntax useful. There have been many attempts in the past to prove some sort of intiality theorem, the most notable by Streicher, but in general there is still much debate on the usefullness of these results. Notable mathematicians such as Vladamir Voevodsky have persuasively argued that this is an unacceptably unrigorous attidue. There is no precise definition of what ``suitable type theory'' means nor which methods are applicable.

The author will note that there are currently a few attempts at answering this question, but to date, not suitable solution has been proposed.

Now suppose we have some sort of ``type theory''. It is still not a completely satisfactory situation in terms of describing the syntax of such an object. Many authors [add citations] have noted that this is the case and more worryingly many other authors have claimed that it is done and dusted. There is as a result a long history of false claims of correct syntax. [add citations] 

\subsection{Introduction}

%\begin{enumerate}
%    \item $\alpha$-conversion is typically seen as an easy thing to handled
%    \item many authors as a result sweep it under the carpet
%    \item there are many examples of this in the mathematical and computer science literature
%    \item to our knowledge there is no fully adequate 
%\end{enumerate}

%The notion of correct syntax is a subtle and deceptive one. Many authors have previously tried to ``fix'' syntax by introducing various constructions: 



We will follow the structure of syntax outlined in Harper \cite{harper_2016}. There are several reasons for this. 

Firstly, for example in Barendregt et. al. \cite{BarendregtHenk2013Lcwt} we have notions of substitution left to the reader under the assumption that they can be fixed. Generally Barendregt's style is like this and even when there is much formalism, it is done in a way that we find peculiar.

In Crole's book \cite{CroleRoyL1993Cft}, syntax is derived from an \textit{algebraic signature} which comes directly from categorical semantics. We want to give an independent view of type theory. The syntax only has types as well, meaning that only terms can be posed in this syntax. Operations on types themselves would have to be handled separately. This will also make it difficult to work with \textit{bound variables}.

In Lambek and Scott's book \cite{LambekJ1986Itho}, very little attention is given to syntax and categorical semantics and deriving type theory from categories for study is in the forefront of their focus.

In Jacob's book \cite{JacobsCLTT}, we again have much reliance on categorical machinery. A variant of algebraic signature called a \mathit{many-typed signature} is given, which has its roots in mathematical logic. Here it is discussed that classically in logic the idea of a sort and a type were synonymous, and they go onto preferring to call them types. This still has the problems identified before as terms and types being treated separately, when it comes to syntax.

In Barendregt's older book \cite{barendregt1984lambda}, there are models of the syntax of (untyped) lambda calculus, using Scott topologies on complete lattices. We acknowledge that this is a working model of the lambda calculus but we believe it to be overly complex for the task at hand. It introduces a lot of mostly irrelevant mathematics for studying the lambda calculus. And we doubt very much that these models will hold up to much modification of the calculus. Typing seems impossible.

In S{\o}rensen and Urzyczyn's book \cite{Sorensen} a more classical unstructured approach to syntax is taken. This is very similar to the approaches that Church, Curry and de Brujin gave early on. The difficulty with this approach is that it is very hard to prove things about the syntax. There are many exceptional cases to be weary of (for example if a variable is bound etc.). It can also mean that the syntax is vulnerable to mistakes. We acknowledge it's correctness in this case, however we prefer to use a safer approach.

We will finally look at one more point of view, that of mathematical logic. We look at Troelstra and Schwichtenberg's book \cite{troelstra_schwichtenberg_2000} which studies proof theory. This is essentially the previous style but done to a greater extent, for they use that kind of handling of syntax to argue about more general logics. As before, we do not choose this approach.

We have seen books from either end of the spectrum, on one hand Barendregt's type theoretic camp, and on the other, the more categorical logically oriented camp. We have argued that the categorical logically oriented texts do not do a good job of explaining and defining syntax, their only interest is in their categories. The type theoretic texts also seem to be on mathematically shaky ground, sometimes much is left to the reader and finer details are overlooked.

Harper's seems more sturdy and correct in our opinion. Harper doesn't concern himself with abstraction for the sake of abstraction but rather when it will benefit the way of thinking about something. The framework for working with syntax also seems ideal to work with, when it comes to adding features to a theory (be it a type theory or otherwise).

\subsection{Well-founded induction}

Firstly we will begin a quick recap of induction. This should be a notion familiar to computer scientists and mathematicians alike. The following will be more accessible to mathematicians but probably more useful for them too since they will be generally less familiar with the generality of induction.


The notion of well-founded induction is a standard theorem of set theory. The classical proof of which usually uses the law of excluded middle \cite[p. 62]{johnstone1987notes}, \cite[Ch. 7]{barwise1982handbook}. It's use in the formal semantics of programming languages is not much different either \cite[Ch. 3]{winskel1993formal}. There are however more constructive notions of well-foundedness \cite[\S 8]{2018arXiv180805204S} with more careful use of excluded middle. We will follow \cite{10.2307/2275781}, as this is the simplest to understand, and we won't be using this material much other than an initial justification for induction in classical mathematics.

\begin{defin}
    Let $X$ be a set and $\prec$ a binary relation on $X$. A subset $Y \subseteq X$ is called \textbf{$\prec$-inductive} if
    $$
        \forall x \in X, \quad (\forall y \prec x,\ y \in Y) \Rightarrow x \in Y.
    $$
\end{defin}

\begin{defin}\label{wf}
    The relation $\prec$ is \textbf{well-founded} if the only $\prec$-inductive subset of $X$ is $X$ itself. A set $X$ equipped with a well-founded relation is called a \textit{well-founded set}.
\end{defin}

\begin{theorem}[Well-founded induction principle]
    Let $X$ be a well-founded set and $P$ a property of the elements of $X$ (a proposition). Then
    $$
        \forall x \in X, P(x) \quad \iff \quad  \forall x \in X,\ \ (\forall y \prec x, P(y)) \Rightarrow P(x).
    $$
\end{theorem}
\begin{proof}
    The forward direction is clearly true. For the converse, assume $\forall x \in X,((\forall y \prec x, P(y)) \Rightarrow P(x))$. Note that $P(y) \Leftrightarrow x \in Y := \{ x \in X \mid P(x)\} $ which means our assumption is equivalent to $\forall x \in X,\ (\forall y \prec x,\ y \in Y) \Rightarrow x \in Y$ which means $Y$ is $\prec$-inductive by definition. Hence by \ref{wf} $Y=X$ giving us $ \forall x \in X, P(x)$.
\end{proof}

We now get onto some of the tools we will be using to model the syntax of our type theory. 

\subsection{Abstract syntax trees}

We begin by outlining what exactly syntax is, and how to work with it. This will be important later on if we want to prove things about our syntax as we will essentially have good data structures to work with.

%We will begin with the notion of an {\it abstract syntax tree}. Which can be what is informally known as syntax, thus formal statements about the syntax are referring to its manifestation as an abstract syntax tree.

%% Sort
\begin{defin}[Sorts]
    Let $\mathcal{S}$ be a finite set, which we will call \textbf{sorts}. An element of $\mathcal{S}$ is called a \textbf {sort}.
\end{defin}

A sort could be a term, a type, a kind or even an expression. It should be thought of an abstract notion of the kind of syntactic element we have. Examples will follow making this clear.

%% Arity
\begin{defin}[Arities]
    An \textbf{arity} is an element $((s_1,\dots,s_n),s)$ of the set of \textbf{arities} $\mathcal{Q}:=\mathcal{S}^\star \times \mathcal{S}$ where $\mathcal{S}^\star$ is the Kleene-star operation on the set $\mathcal{S}$ (a.k.a the free monoid on $\mathcal{S}$ or set of finite tuples of elements of $\mathcal{S}$). An arity is typically written as $(s_1,\dots,s_n)s$. 
\end{defin}

%% Operator
\begin{defin}[Operators]
    Let $\mathcal{O} :=\{ \mathcal{O}_\alpha \}_{\alpha \in \mathcal{Q}}$ be an $\mathcal{Q}$-indexed (arity-indexed) family of disjoint sets of \textbf{operators} for each arity. An element $o \in \mathcal{O}_\alpha$ is called an \textbf{operator} of arity $\alpha$. If $o$ is an operator of arity $(s_1,\dots,s_n)s$ then we say $o$ has \textbf{sort} $s$ and that $o$ has $n$ \textbf{arguments} of sorts $s_1,\dots,s_n$ respectively.
\end{defin}

%% Variables
\begin{defin}[Variables]
    Let $\mathcal{X}:= \{ \mathcal{X}_s\}_{s \in \mathcal{S}}$ be an $\mathcal{S}$-indexed (sort-indexed) family of disjoint (finite?) sets $\mathcal{X}_s$ of \textbf{variables} of sort $s$. An element $x\in\mathcal{X}_s$ is called a \textbf{variable} $x$ of \textbf{sort} $s$. 
\end{defin}

%% Fresh variables
\begin{defin}[Fresh variables]
    We say that $x$ is \textbf{fresh} for $\mathcal{X}$ if $x \not\in \mathcal{X}_s$ for any sort $s\in \mathcal{S}$. Given an $x$ and a sort $s\in \mathcal{S}$ we can form the family $\mathcal{X},x$ of variables by adding $x$ to $\mathcal{X}_s$. 
\end{defin}


[[ Wording here may be confusing]]
%% Fresh variable sets
\begin{defin}[Fresh sets of variables]
    Let $V=\{ v_1 ,\dots, v_n\}$ be a finite set of variables (which all have sorts implicitly assigned so really a family of variables $\{V_s\}_{s\in\mathcal{S}}$ indexed by sorts, where each $V_s$ is finite). We say $V$ is fresh for $\mathcal{X}$ by induction on $V$. Suppose $V=\varnothing$, then $V$ is \mathbf{fresh} for $X$. Suppose $V = \{v \} \cup W$ where $W$ is a finite set, $v$ is fresh for $W$ and $W$ is fresh for $\mathcal{X}$. Then $V$ is fresh for $\mathcal{X}$ if $v$ is fresh for $\mathcal{X}$. By induction we have defined a finite set being fresh for a set $\mathcal{X}$. Write $\mathcal{X},V$ for the union (which is disjoint) of $\mathcal{X}$ and $V$. This gives us a new set of variables with obvious indexing.
\end{defin}

%% Remark about notation for adding variables
\begin{remark}
    The notation $\mathcal{X},x$ is ambiguous because the sort $s$ associated to $x$ is not written. But this can be remedied by being clear from the context what the sort of $x$ should be.
\end{remark}

%% Abstract syntax trees
\begin{defin}[Abstract syntax trees]
    The family $\mathcal{A}[\mathcal{X}]=\{ \mathcal{A}[\mathcal{X}]_s \}_{s \in \mathcal{S}}$ of \textbf{abstract syntax trees} (or asts), of \textbf{sort} $s$, is the smallest family satisfying the following properties:
    
    \begin{enumerate}
        \item A variable $x$ of sort $s$ is an ast of sort $s$: if $x \in \mathcal{X}_s$, then $x \in \mathcal{A}[\mathcal{X}]_s$.
        
        \item Operators combine asts: If $o$ is an operator of arity $(s_1, \dots, s_n)s$, and if $a_1 \in \mathcal{A}[\mathcal{X}]_{s_1}, \dots, a_n \in \mathcal{A}[\mathcal{X}]_{s_n}$, then $o(a_1;\dots; a_n) \in \mathcal{A}[\mathcal{X}]_s$.
    \end{enumerate}
\end{defin}

%% Remark about inductively generated sets
\begin{remark}
    The idea of a smallest family satisfying certain properties is that of structural induction. So another way to say this would be a family of sets inductively generated by the following constructors.
\end{remark}

%% Remark about asts being trees
\begin{remark}
    An ast can be thought of as a tree whose leaf nodes are variables and branch nodes are operators. 
\end{remark}

%% Lambda calculus example
\begin{example}[Syntax of lambda calculus]
    The (untyped) lambda calculus has one sort $\Term$, so $\mathcal{S} = \{ \Term \} $. We have an operator $\App$ of application whose arity is $(\Term, \Term)\Term$ and an family of operators $\{\lambda_x \}_{x \in \Var}$ which is the lambda abstraction with bound variable $x$, so $\mathcal{O} = \{ \lambda_x \} \cup \{ \App \} $. The arity of each $\lambda_x$ for some $x \in \Var$ is simply $(\Term) \Term$.
    
    Consider the term $$\lambda x . (\lambda y . x y)  z$$

    We can consider this the \textit{sugared} version of our syntax. If we were to \textit{desugar} our term to write it as an ast it would look like this:

    $$
        \lambda_x(\App(\lambda_y(\App(x ; y)); z) )
    $$

    Sugaring allows for long-winded terms to be written more succinctly and clearly. Most readers would agree that the former is easier to read. We have mentioned the tree structure of asts so we will illustrate with the following equivalent examples. We present two to allow for use of both styles.
    
        \begin{figure}[h]
        \begin{framed}
            \centering
            \begin{minipage}{0.45\textwidth}
                \centering
                \begin{forest}
                    for tree = {
                        %grow = 2,
                        inner sep = 0.1em,
                        l = 0,
                        l sep = 0.7em
                    }
                    [$\lambda_x$ 
                        [$\App$
                            [$\lambda_y$
                                [$\App$
                                    [$x$] 
                                    [$y$]
                                ]
                            ]
                            [$z$]
                        ]
                    ]
                \end{forest}
                \caption{Vertically oriented tree representing the lambda term}
            \end{minipage}
            \hfill
            \begin{minipage}{0.45\textwidth}
                \centering
                \begin{forest}
                    for tree = {
                        grow = 0,
                        inner sep = 0.1em,
                        l sep = 1em
                    }
                    [$\lambda_x$ 
                        [$\App$
                            [$\lambda_y$
                                [$\App$ [$x$] [$y$]
                                ]
                            ]
                            [$z$]
                        ]
                    ]
                \end{forest}
                \caption{Horizontally oriented tree representing the lambda term}
            \end{minipage}
        \end{framed}
    \end{figure}

    
\end{example}

%% Remark about asts not caring about binding
\begin{remark}
    Note that later we will enrich our notion of abstract syntax tree that takes into account binding and scope of variables but for now this is purely structural.
\end{remark}

%% Remark about proving things by structural induction on asts
\begin{remark}
    When we prove properties $\mathcal{P}(a)$ of an ast $a$ we can do so by structural induction on the cases above. We will define structural induction as a special case of well-founded induction. But for this we will need to define a relation on asts.
\end{remark}

\begin{defin}
    Suppose $\mathcal{X} \subseteq \mathcal{Y}$. An ast $a \in \mathcal{A}[\mathcal{X}]$ is a \textbf{subtree} of an ast $b \in \mathcal{A}[\mathcal{Y}]$ [This part is giving me a headache. How can I define subtree if I can't do it by induction? To do it by induction I would have to define subtree.]
\end{defin}



[Some more notes on structural induction, perhaps this can be defined and discussed with trees in the section before?]

%% Plenty of examples of asts with examples of sorts, operators and variables
[add examples of sorts, operators, variables and how they fit together in asts]

%% Lemma about subsets of variables
\begin{lemma}
    If we have $\mathcal{X} \subseteq \mathcal{Y}$ then, $\mathcal{A}[\mathcal{X}] \subseteq \mathcal{A}[\mathcal{Y}]$.
\end{lemma}
\begin{proof}
    Suppose $\mathcal{X} \subseteq \mathcal{Y}$ and $a \in \mathcal{A}[\mathcal{X}]$, now by structural induction on $a$:
    
    \begin{enumerate}
        \item If $a$ is in $\mathcal{X}$ then it is obviously also in $\mathcal{Y}$.
        \item If $a := o(a_1;\dots;a_n) \in \mathcal{A}[\mathcal{X}]$ we have $a_1, \dots, a_n\in \mathcal{A}[\mathcal{X}]$ also. By induction we can assume these to be in $\mathcal{A}[\mathcal{Y}]$ hence giving us $a \in \mathcal{A}[\mathcal{Y}]$.
    \end{enumerate}
    
    Hence by induction we have shown that $\mathcal{A}[\mathcal{X}] \subseteq \mathcal{A}[\mathcal{Y}]$.
\end{proof}

\subsection{Substitution in asts}

%% Substitution
\begin{defin}[Substitution]\label{sub}
    If $a \in \mathcal{A}[\mathcal{X},x]_{s'}$, and $b \in \mathcal{A}[\mathcal{X}]_s$, then $[b/x]a \in \mathcal{A}[\mathcal{X}]_{s'}$ is the result of \textbf{substituting} $b$ for every occurrence of $x$ in $a$. The ast $a$ is called the \textbf{target}, the variable $x$ is called the \textbf{subject} of the \textbf{substitution}. We define substitution on an ast $a$ by induction:
    \begin{enumerate}
        \item $[b/x]x = b$ and $[b/x]y = y$ if $x\ne y$.
        \item $[b/x]o(a_1;\dots;a_n)=o([b/x]a_1;\dots;[b/x]a_n)$
    \end{enumerate}
\end{defin}

%% Examples of substitution
[Examples of substitution]

\begin{cor}\label{subcheck}
    If $a \in \mathcal{A}[\mathcal{X},x]$, then for every $b \in \mathcal{A}[\mathcal{X}]$ there exists a unique $c \in \mathcal{A}[\mathcal{X}]$ such that $[b/x]a = c$.
\end{cor}
\begin{proof}
    By structural induction on $a$, we have three cases: $a := x$, $a:=y$ where $y \ne x$ and $a := o(a_1; \dots; a_n)$. In the first we have $[b/x]x=b=c$ by definition. In the second we have $[b/x]y=y=c$ by definition. In both cases $c \in \mathcal{A}[\mathcal{X}]$ and are uniquely determined. Finally, when $a := o(a_1; \dots; a_n)$, we have by induction unique $c_1,\dots, c_n$ such that $c_i:=[b/x]a_i$ for $1 \le i \le n$. Hence we have a unique $c=o(c_1,\dots,c_n) \in \mathcal{A}[\mathcal{X}]$.
\end{proof}

\begin{remark}
    Note that \ref{subcheck} was simply about checking Definition \ref{sub}. We have written out a use of the definition here so we won't have to again in the future.
\end{remark}

Abstract syntax trees are our starting point for a well-defined notion of syntax. We will modify this notion, as the author of \cite{harper_2016} does, with slight modifications that are used in \cite{nlab:initiality_project, nlab:initiality_project_-_raw_syntax}, the Initiality Project. This is a collaborative project for showing initiality of dependent type theory (the idea that some categorical model is initial in the category of such models). It is a useful reference because it has brought many mathematicians together to discuss the intricate details of type theory. The definitions here have spawned from these discussions on the nlab and the nforum.

We want to modify the notion of abstract syntax tree to include features such as binding and scoping. This is a feature used by many type theories (and even the lambda calculus). It is usually added on later by keeping track of bound and free variables. [CITE]. We will avoid this approach as it makes inducting over syntax more difficult.

\subsection{Abstract binding trees}

%% Generalize arity
\begin{defin}[Generalized arities]
    A \textbf{generalised arity} (or signature) is a tuple consisting of the following data:
    
    \begin{enumerate}
        \itemsep -0.5 \parsep
        \item A sort $s \in \mathcal{S}$.
        
        \item A list of sorts of length $n$ called the \textbf{argument sorts}, where $n$ is called the \textbf{argument arity}.
        
        \item A list of sorts of length $m$ called the \textbf{binding sorts}, where $m$ is called the \textbf{binding arity}.
        
%        \item A natural number $n \in \N$ called the \textbf{argument arity}.
%        \item A function $\mathrm{soa} : [n] \to \mathcal{S}$ choosing the sorts of the $n$ arguments. Where $\mathrm{soa}$ means \textbf{sort of argument}.
%        \item A natural number $m \in \N$ called the \textbf{binding arity} (or number of bound variables).
%        \item A function $\mathrm{sov} : [m] \to \mathcal{S}$ choosing the sorts of the $m$ bound variables. Where $\mathrm{sov}$ means \textbf{sort of variable}.
        \item A decidable relation $\scope$ between $[n]$ and $[m]$ called \textbf{scoping}. Where $j \scope k$ means the $j$th argument is in scope of the $k$th bound variable.
    \end{enumerate}
    
    The set of generalised arities $\mathbf{GA}$ could therefore be defined as $\mathcal{S} \times \mathcal{S}^\star  \times \mathcal{S}^\star$ equipped with some appropriate relation $\scope$.
\end{defin}

\begin{remark}
    In \cite{harper_2016} there is no relation but a function. And each argument has bound variables assigned to it. But as argued in \cite{nlab:initiality_project_-_raw_syntax} this means arguments can have different variables bound even if they are really the same variable. To fix this, bound variables belong to the whole signature. Which confidently makes it simpler to understand too.
    
    This definition is more general than the definition given in \cite{nlab:initiality_project_-_raw_syntax} due to bound variables having sorts chosen for them rather than being defaulted to the sort $\mathrm{tm}$. It is mentioned there however that it can be generalised to this form (but would have little utility there).
\end{remark}

We will now redefine the notion of operator, taking note that generalised arities are a super-set of arities defined previously.

%% Operator
\begin{defin}[Operators (with generalized arity)]\label{owga}
    Let $\mathcal{O}:=\{ \mathcal{O}_\alpha\}_{\alpha \in \mathbf{GA}}$ be a $\mathbf{GA}$-indexed family of disjoint sets of \textbf{operators} for each generalised arity $\alpha$. An element $o \in \mathcal{O}_{\alpha \in \mathbf{GA}}$ is called an \mathbf{operator} of (generalised) \textbf{arity} $\alpha$. If $\alpha$ has sort $s$ then $o$ has \textbf{sort} $s$. If $\alpha$ has argument sorts $(s_1,\dots,s_n)$ then we say that $o$ has \textbf{argument arity} $n$, with the $j$th argument having \textbf{sort} $s_j$. If $\alpha$ has binding sorts $(t_1,\dots,t_m)$ then we say that $o$ has \textbf{binding arity} $m$, with the $k$th bound variable having \textbf{sort} $s_k$. If the the scoping relation of $\alpha$ has $j \scope k$ then we say that the $j$th argument of $o$ is in \textbf{scope} of the $k$th bound variable of $o$. 
\end{defin}

\begin{remark}
    We overload the definitions of arity and operator to mean generalised operator and operator with generalised arity respectively.
\end{remark}

\begin{remark}\label{opdata}
    When we wish to specify an operator we need only give the following data:
    \begin{enumerate}
        \item Name - what we wish to call the operator, for example $\to$ or $\times$.
        \item Sort - what is the sort of the operator?
        \item Variables - What are the variables of the operator?
        \item Sorted arguments - What are the arguments and what are their sorts?
        \item Scoping - Which arguments are in scope of which variables?
        \item Sugared syntax - How do we write down the operator with all the variables and arguments together. By default we have been writing $\mathcal{O}(x$ 
    \end{enumerate}
\end{remark}

Now that we can equip our operators with the datum of binding and scoping we can go ahead and define abstract binding trees.

[[ Lots of concepts for asts have been redefined for abts,
perhaps its worth making note of that back in the asts definitions ]]

%% Abstract binding trees
\begin{defin}[Abstract binding trees]\label{abt}
    The family $\mathcal{B}[\mathcal{X}] = \{ \mathcal{B}[\mathcal{X}]_s \}_{s \in \mathcal{S}}$ of \textbf{abstract binding trees} (or abts), of \textbf{sort} $s$, is the smallest family satisfying the following properties:
    
    \begin{enumerate}
        \item A variable $x$ of sort $s$ is an abt of sort $s$: if $x \in \mathcal{X}_s$, then $x \in \mathcal{B}[\mathcal{X}]_s$.
        \item Suppose $\mathtt{G}$ is an operator of sort $s$, argument arity $n$ and binding arity $m$. Suppose $V$ is some finite set of length $m$ which is fresh for $\mathcal{X}$. These will be called our \textbf{bound variables}. Label the elements of $V$ as $V= \{ v_1, \dots, v_m\}$. For $j\in [n]$, let $X_j:=\{ v_k \in V \mid j \scope k\}$ be the set of bound variables that the $j$th argument is in scope of. Now suppose for each $j \in [n]$, $M_j \in \mathcal{B}[\mathcal{X},V]_{s_j}$ where $s_j$ is the sort of the $j$th argument of $\mathtt{G}$. Then $\mathtt{G}(X;M_1,\dots, M_n) \in \mathcal{B}[\mathcal{X}]_s$.
    \end{enumerate}
\end{defin}

\begin{remark}
    There is a lot going on in the second constructor of Definition \ref{abt}. It simply allows for bound variables to be constructed in syntax in a well-defined way that avoids variable capture. This will be useful when defining notions like substitution on abts as we will have the avoidance of variable capture built-in.
\end{remark}

[[What is variable capture talk about this and reference this stuff because lots of cleverer people have thought about this too you know.]]
\subsection{Substitution in abts}



% Judgements, inference rules and general notions of logical
\section{Judgements}

We will now develop the basic formal tools to describe how our programming languages work.  We will first describe judgements and how to specify a type system. Then our first example will be the simply typed lambda calculus. We use the ideas developed in \cite{harper_2016} though these ideas are much older. [Probably tracable back to Gentzen]. [There are many more references to be included here]

\begin{defin}
    The notion of a \emph{judgement} or \emph{assertion} is a logical statement about an abt. The property or relation itself is called a \emph{judgement form}. The judgement that an object or objects have that property or stand in relation is said to be an \emph{instance} of that judgement form. A judgment form has also historically been called a \emph{predicate} and its instances called \emph{subjects}.
\end{defin}

\begin{remark}
    Typically a judgement is denoted $\mathsf{J}$. We can write $a\ \mathsf{J}$, $\mathsf{J}\ a$ to denote the judgment asserting that the judgement form $\mathsf{J}$ holds for the abt $a$. For more abts this can also be written prefix, infix, etc. This will be done for readability. Typically for an unspecified judgement, that is an instance of some judgement form, we will write $J$.
\end{remark}

    $$\frac
        {}
        {}
    $$


\begin{defin}
    An \emph{inductive definition} of a judgement form consists of a collection of rules of the form
    
    $$\frac
        {J_1 \quad \cdots \quad J_k}
        {J}
    $$
    
    in which $J$ and $J_1, \dots , J_k$ are all judgements of the form being defined. THe judgements above the horizontal line are called the \emph{preimises} of the rules, and the judgement below the line is called its \emph{conclusion}. A rule with no premises is called an \emph{axiom}.
\end{defin}

\begin{remark}
    An inference rule is read as starting that the premises are \emph{sufficient} for the conclusion: to show $J$, it is enough to show each of $J_1, \dots J_k$. Axioms hold unconditionally. If the conclusion of a rule holds it is not necesserily the case that the premises held, in that the conclusion could have been derived by another rule.
\end{remark}

\begin{example}
    Consider the following judgement from $-\ \mathsf{nat}$, where $a\ \mathsf{nat}$ is read as ``$a$ is a natural number''. The following rules form an inductive definition of the judgement form $-\ \mathsf{nat}$:

    $$\frac
        {}
        {\texttt{zero}\ \mathsf{nat}}
      \qquad\qquad\qquad
      \frac
        {a\ \mathsf{nat}}
        {\texttt{succ}(a)\ \mathsf{nat}}
    $$

    We can see that an abt $a$ is zero or is of the form $\texttt{succ}(a)$. We see this by induction on the abt, the set of such abts has an operator $\texttt{succ}$. Taking these rules to be exhaustive, it follows that $\textt{succ}(a)$ is a natural number if and only if $a$ is.
\end{example}

\begin{remark}
    We used the word \emph{exhaustive} without really defining it. By this we mean necessary and sufficient. Which we will define now.
\end{remark}

\begin{defin}
    A collection of rules is considered to define the \emph{strongest} judgement form that \emph{closed under} (or \emph{respects}) those rules. To be closed under the rules means that the rules are \emph{sufficient} to show the validity of a judgement: $J$ holds if there is a way to obtain it using the given rules. To be the \emph{strongest} judgement form closed under the rules means that the rules are also \emph{necessary}: $J$ holds \emph{only if} there is a way to obtain it by applying the rules.
\end{defin}

Let's add some more rules to our example, to get a richer structure.

\begin{example}
    The judgement form $a = b$ expresses the equality of two abts $a$ and $b$. We define it inductively on our abts as we did for $\mathsf{nat}$.
    
    $$\frac
        {}
        {\texttt{zero} = \texttt{zero}}
      \qquad\qquad\qquad
    \frac
        {a = b}
        {\texttt{succ}(a) = \texttt{succ}(b)}
    $$
    Our first rule is an axiom declaring that \texttt{zero} is equal to itself, and our second rule shows that abts of the form $\texttt{succ}$ are equal only if their arguments are. Observe that these are exhaustive rules in that they are necessary and sufficient for the formation of $=$.
\end{example}

\subsection{Derivations}

To show that an inductively defined judgement holds, we need to exhibit a \emph{derivation} of it.

\begin{defin}
    A \emph{derivation} of a judgement is a finite composition of rules, starting with axioms and ending with the judgement. It is a tree in which each node is a rule and whose children are derivations of its premises. We sometimes say that a derivation of $J$ is evidence for the validity of an inductively defined judgement $J$.

    Suppose we have a judgement $J$ and
    $$\frac
        {J_1\quad \cdots\quad J_k}
        {J}
    $$
    is an inference rule. Suppose $\triangledown_1, \dots, \triangledown_k$ are derivations of its premises, then
    $$\frac
        {\triangledown_1\quad \cdots\quad \triangledown_k}
        {J}
    $$
    is a derivation of its conclusion. Notice that if $k=0$ then the node has no children.
\end{defin}

Writing derivations as trees can be very enlightening to how the rules compose. Going back to our example with $\mathsf{nat}$ we can give an example of a derivation.

\begin{example}
    Here is a derivation of the judgement $\texttt{succ}(\texttt{succ}(\texttt{succ}(\texttt{zero})))\ \mathsf{nat}$:
    
    \begin{prooftree}
        \AxiomC{}
        \UnaryInfC{ $\texttt{zero}\ \mathsf{nat}$ }
        \UnaryInfC{ $\texttt{succ}(\texttt{zero})\ \mathsf{nat}$ }
        \UnaryInfC{ $\texttt{succ}(\texttt{succ}(\texttt{zero}))\ \mathsf{nat}$ }
        \UnaryInfC{ $\texttt{succ}(\texttt{succ}(\texttt{succ}(\texttt{zero})))\ \mathsf{nat}$ }
    \end{prooftree}
\end{example}

\begin{remark}
    To show that a judgement is \emph{derivable} we need only give a derivation for it. There are two main methods for finding derivations:
    \begin{itemize}
        \item \emph{Forward chaining} or \emph{bottom-up construction}
        \item \emph{Backward chaining} or \emph{top-down construction}
    \end{itemize}
    
    Forward chaining starts with the axioms and works forward towards the desired conclusion. Backward chaining starts with the desired conclusion and works backwards towards the axioms.
\end{remark}

It is easy to observe the \emph{algorithmic} nature of these two processes. In fact this is an important point to think about, since it may become relevent in the future.

\begin{lemma}
    Given a derivable judgement $J$, there is an algorithm giving a derivation for $J$ by forward chaining.
\end{lemma}

\begin{proof}
    This is not a difficult algorithm to descrive. We start with a set of rules $\mathcal{R} := \varnothing $ which we initially set to be empty. Now we consider all the rules that have premises in $\mathcal{R}$, initially this will be all the axioms. We add these rules to $\mathcal{R}$ and repeat this process until $J$ appears as a conclusion of one of the rules in $\mathcal{R}$. It is not difficult to see that this will necesserily give all derivations of all derivable judgements and since $J$ is derivable, it will eventually give a derivation for $J$.
\end{proof}

\begin{remark}
    Notice how we had to specify that our judgement is derivable. Since if were not, then our process would not terminate, hence would not be an algorithm. It is also worth noting that this algorithm is very inefficient since the size of $\mathcal{R}$ will grow rapidly, especially when we have more rules available. This is sort of a brute force approach. What we will need is more clever picking of the rules we wish to add. This is nontrivial problem and is basically what a mathematician does.
\end{remark}

Forward chaining does not take into account any of the information given by the judgement $J$. The algorithm is in a sense blind. 









% Statics and dynamics of programming langauges
\section{Statics and Dynamics}

How can we in general design programming languages to assertain certain behaviours. Static and dynamic typing for instance. Different constructs and data types such as products and sums. Later we will look at a deep correspondance between programming and logic which gives us an indication of what a programming language ought to have.

Statics: Type checking
Dynamics: Computation, equational rules, transition systems (reduction with betas and etas)

We will introduce typing and think carefully about another structural rule: The exchange rule, we will see that it is inadmissible and infact not necesserily needed. Infact later when we think about dependent types we will see that it is in general "complete nonsense". HOWEVER it is essential for some models of STLC.

We will end up with STLC. But we will also show how to add sum types.

We will also model the semantics of such programming languages (at least the statics of) using categories.

Later we will see that Curry-Howard is very suggestive about quantifiers, can we add these? YES!

Then we can introduce our favorite dependent types. Show how useful they are for programmers and mathematicans

We will now try to design programming languages that can have types, types allow us to restrict what terms we can apply functions to. Something take for granted very often in mathematics and to a lesser extent in programming. Programming langauges such as C don't really type check, which means functions that should be applied can be. There are different strengths to type checking, some check at compilation (which is arguably to most sensible) but others check during run time but this means a program cannot be garanteed to be safe.

The ideas of types are very deep, so when combined with a flexibly expressible type system (dependent types) it leads to a powerful correctness tool.




% Discuss product types
% Sum types
% basic data types

% TODO Move
% History and implications of curry howard
\section{Curry-Howard correspondence}

\subsection{Introduction}

In this section we outline a detailed history of what is known as the Curry-Howard correspondence. This is an important thing to consider since there are many powerful ideas that get uncovered in this story. They will shape the future of thought on the subject, so it is worthwhile to understand what was motivating mathematicians and computer scientists at the time.

Many of these ideas were developed before the comprehension of the modern idea of a computer! So in a way it is quite remarkable that these ideas were even considered. Hence we will discuss their original motivations.

This story will develop our ideas of what needs to be added to the simply typed lambda calculus going forward. How these ideas will behave with what we have studied, and finally what the future of the subject looks like.x


%%%
\subsection{Mathematical logic}\label{logic_chapter}

At the beginning of the 20th century, Whitehead and Russell published their \emph{Principia Mathematica} \cite{GlossarWiki:Whitehead_Russell:1910}, demonstrating to mathematicians of the time that formal logic could express much of mathematics. It served to popularise modern mathematical logic leading to many mathematicians taking a more serious look at topic such as the foundations of mathematics.

One of the most influential mathematicians of the time was David Hilbert. Inspired by Whitehead and Russell's vision, Hilbert and his colleagues at G\"ottingen became leading researchers in formal logic. Hilbert proposed the \emph{Entscheidungsproblem} (decision problem), that is, to develop an ``effectually calculable procedure'' to determine the truth or falsehood of any logical statement. At the 1930 Mathematical Congress in K\"onigsberg, Hilbert affirmed his belief in the conjecture, concluding with his famous words ``Wir m\"ussen wissen, wir werden wissen'' (``We must know, we will know''). At the very same conference, Kurt G\"odel announced his proof that arithmetic is incomplete \cite{GlossarWiki:Goedel:1931}, not every statement in arithmetic can be proven.

This however did not deter logicians, who were still interested in understanding why the \emph{Entscheidungsproblem} was not attainable.
For this, a formal definition of ``effectively calculable'' was required.
So along came three candidate definitions of what it meant to be ``effectively calculable'': \emph{$\lambda$-calculus}, published in 1936 by Alonzo Church \cite{church-unsolvableproblemof-1936}; \emph{recursive functions}, proposed by G\"odel in 1934 later published in 1936 by Stephen Kleene \cite{kleene1936}; and finally \emph{Turing machines} in 1937 by Alan Turing \cite{turing1936a}.

%%%
\subsection{\texorpdfstring{$\lambda$}{}-calculus}

$\lambda$-calculus was discovered by Church at Princeton in the 1930s, originally as a way to define notations for logical formulas.
It is a very compact and simple idea, with only three constructs: variables; $\lambda$-abstraction; and function application.
Curry developed the closely related idea of combinatory logic around the same time \cite{curry1930a, curry1930b}.

Interestingly, Curry had introduced the notion of \emph{Combinators} into logic for the very same reason reason we introduced abstract binding trees: to avoid mentioning named variables \cite{Sorensen}.

It was realised at the time by Church and others that ``There may, indeed, be other applications of the system than its use as a logic.'' \cite{church1932, church1933}.
This meant that $\lambda$-calculus was worth studying as a topic of interest in it's own right.
This became explicitly apparent when Church discovered a way of encoding numbers as terms of $\lambda$-calculus, known as the \emph{Church encoding} of the natural numbers. From this addition and multiplication could also be defined.

However the problem of defining a predecessor function alluded Church and his students, in fact Church later became convinced that it was impossible.
Fortunately Kleene later discovered, at his dentist's office, how to define the predecessor function \cite{kleene1935a, kleene1935b, 4392910}.
This led to Church to later propose that $\lambda$-definability ought to be the definition of ``effectively calculable'', culminating into what is now known as Church's Thesis. Church went on to demonstrate that the problem of determining whether or not a given $\lambda$-term  has a normal form is not $\lambda$-definable.
This is now known as the Halting Problem. Put differently this says that no program written in the $\lambda$-calculus can determine whether a program written in the $\lambda$-calculus halts or not.

%%%
\subsection{Recursive functions}

In 1933 G\"odel arrived in Princeton, unconvinced by Church's claim that every effectively calculable function was $\lambda$-definable. Church responded by offering that if Go\"odel would propose a different definition, then Church would ``undertake to prove it was included in $\lambda$-definability''. In a series of lectures at Princeton, G\"odel proposed what came to be known as ``general recursive functions'' as his candidate for effective calculability. Kleene later published the definition \cite{kleene1936}. Church later outlined a proof that it was equivalent to the $\lambda$-calculus \cite{church1936} and Kleene later published it in detail \cite{kleene1936b}. This however did not have the intended effect on G\"odel, whereby he then became convinced that his own definition was incorrect!

%%%
\subsection{Turing machines}

Alan Turing was at Cambridge when he independently formulated his own idea of what it meant to be ``effectively calculable'', what is now known today as a \emph{Turing machine}. He used it to show that the Entscheidungsproblem is undecidable, meaning that it cannot be proven to be true or false. Before publication, Turing's advisor Max Newman was worried since Church had already published a solution, but since Turing's approach was sufficiently novel it was published anyway \cite{turing1936a}. Turing had added an appendix sketching the equivalence of $\lambda$-definability to Turing machines. It was Turing's argument that later convinced G\"odel that this was the correct notion of ``effectively calculable''.

Of course today the argument for Turing machines as a candidate for computation seems obvious.
We are surrounded by computers in our daily lives, all based loosely on the idea of a Turing machine.
From this it is easy to see that Turing's ideas had a \emph{huge} influence on the notions of computation.

%%%
\subsection{The problem with \texorpdfstring{$\lambda$}{}-calculus as a logic}

Church's students Kleene and Rosser quickly discovered that $\lambda$-calculus was inconsistent as a logic \cite{kleene1935c}.
A logic is deemed \emph{inconsistent} if every statement can be proven.
For example assuming $1 = 2$ can lead to many bizarre consequences, such as all logical formulas becoming true, one way or another.
In that way, arithmetic with the assumption that $1 = 2$, is \emph{inconsistent as a logic}.
Curry later published a simplified version of Kleene and Rosser's result which became known as \emph{Curry's paradox} \cite{curry1942}.
Curry's paradox was related to Russell's paradox, in that a predicate was allowed to act on itself.

Russell's paradox is typically seen as a paradox of set theory, but can usually be phrased in a much more general manner. The basic idea is this: Let $A$ be the set of all sets that do not contain themselves. The question is, does $A$ belong to itself? Clearly, if it did then it would not be an element of the set. If it did not, then it would have to be an element. Either way there is a contradiction, hence we have a \emph{logical paradox}.

The issue arises with the definition of $A$. In it we defined it as something quantifying over a lot of things, but most importantly itself. This self reference is exactly the issue that leads to such a paradox. The idea of self-reference isn't that harmful if kept under control however, particularly if a relation is \emph{well-founded}.

But allowing all predicates (formulas quantifying over other formulas), leads to silly situations as above. Much of modern set theory has been developed in order to avoid being able to write down paradoxical statements as above. We will see many of these ideas in a type theoretic form later on. A good introduction to basic set theory is \cite{johnstone1987notes}.


What is nice about Church's STLC is that every term has a normal form, or in the language of Turing machines every computation halts \cite{turing1936a}. From this consistency of Church's STLC as a logic could be established, not every logical formula is true.

%%%
\subsection{Types to the rescue}

Types were originally introduced as a method to avoid paradoxes occurring in the type-free world.
However mathematicians had naturally stratified objects into different categories, without any consideration to types before \cite{GANDY1977173, kamareddine2002}.
Russell was one of the first mathematicians to introduce a formal theory of types \cite{GlossarWiki:Whitehead_Russell:1910}, precisely to avoid the paradox baring his name.
In order to solve this problem, Church adapted a solution similar to Russell's.
The first presentation of a simple theory of types was given in Church's influential paper \cite{church1940}, where he introduced the simply typed lambda calculus.

Being typed had some immediate consequences, especially on the ideas of $\lambda$-calculus as a notion of computation. We saw in Example \ref{y_comb}, certain computational properties of the untyped lambda calculus, such as recursion are lost. What we are left with is a strictly weaker programming language. But one that is at least consistent as a system of logic.

%%%
\subsection{Types in the design of programming languages}

[[TODO]]

%%%
\subsection{Propositions as types}

We have previously discussed a condensed form of the two judgements $a \Leftarrow A$ and $a \Rightarrow A$, which we will denote $a : A$. This is in fact the judgement considered without bidirectional type checking, and the one found in most literature on the subject. We had our own reasons for choosing this set up but now we want to discuss how types and propositions are related, hence we will only be mentioning $a : A$.

The basic idea of the propositions as types is to consider types as propositional formulas, and terms as proof of those propositions. Suppose I had a type $\mathrm{TheSkyIsBlue}$. Then terms of this type would correspond to evidence or proofs that the sky is indeed blue. Type formers can then be seen as logical connectives. Suppose $A$ is the type ``Is a Cow''. Terms of $A$ are proofs that what ever we are considering, it \emph{is a cow}. Now suppose there is a type $B$ called ``Goes moo''. Terms of this type are proofs that what ever we are considering, \emph{goes moo}. Say we want to prove the statement: ``If it is a cow, then it goes moo''.
Proving such a statement would go something like this: Suppose what we are considering is a cow, through this series of logical steps we arrive at the conclusion that it goes moo.
If $A$ and $B$ were \emph{propositions}, then it would be agreed that we have proven $A \implies B$. It's as if we have taken a proof of $A$ and turned it into a proof of $B$. We have types that do exactly that: Function types!
So our proof of $A \to B$ is really just a function $\lambda x . y$ that takes in a proof $x : A$ and gives a proof $y : B$.

Of course implication isn't what logic is all about, a natural question to ask is what conjunction (the fancy word for ``and'') corresponds to. Proving that ``\emph{the sky is blue}'' and ``\emph{the grass is green}'', in a way, requires \emph{two} proofs, one corresponding to each proposition. Let $A$ be the type denoting that the sky is blue and $B$ be the type denoting that the grass is green. Then $A \times B$, the product type, is the type that both these ``propositions'' hold. How do we construct a term of the product type? Well we need to give a pair $(a, b)$, which consists of a term $a : A$ and a term $b : B$. Or in other words, to give a proof of $A \times B$ we need to give a proof of $a$ and a proof of $B$. We will later look at other logical connectives. [[TODO reference other logical connectives]].

Curiously, Curry had noticed something similar in \cite[footnote 28]{10.2307/2266302}, though his motivations were far less bovinal: \say{Note the similarity of the postulates for $F$ and those for $P$. If in any of the former postulates we change $F$ to $P$ and drop the combinator we have the corresponding postulate for $P$.}
Here $P$ is the combinator for implication and $F$ is a ``functionality'' combinator, whereby $F A B f$ essentially means $f : A \to B$.
There is evidence to suggest that this idea wasn't new to Curry. Hindley \cite{hindley_1997} points out that a remark on page 588 of \cite{10.2307/86796} indicates this. Another hint is that the properties of implication (denoted $\supset$) are named PB, PC, PW and PK, after the combinators B, C, W and K.

The correspondence was made precise (in typed combinatory logic) by Curry and Feys in \cite[Chapter 9]{curry1958combinatory}. There are two theorems proved in this chapter, under the title of ``F-P transformation'' (the notation from earlier): Theorem 1 states that inhabited types are provable; Theorem 2 states that the converse.

We note that other authors had also independently noted a link between proofs and combinators \cite{meredith1963}. 

\subsection{Gentzen's cut-elimination}




\subsection{The correspondence is dynamic}






[Overview of the full nature of the observation, much deeper than a simple correspondence since logic is in some sense ``very correct'' and programming constructs corresponding to these must therefore also be ``very correct''.]

\subsection{First-order logic}


\begin{quotation}
    Figaro shaves all men of Seville who do not shave themselves.
    But he does not shave anyone who shaves himself.
    Therefore Figaro is not a man of Seville.
\end{quotation}

%%%
\subsection{Predicates [CHANGE] as types?}

[Talk about predicate quantifiers $\forall, \exists$ and what a ``dependent type ought to do'']

%%%
\subsection{Dependent types}


[Perhaps expand on the simply typed section]

[talk about pi and sigma types

[talk about ``dependent contexts'']




% Design simply typed lambda calculus with these implications in mind

% Model simply typed lambda calculus using syntax machinary
%
% Simply typed lambda calculus
%
\section{Simply typed lambda calculus} 


First develop the features needed. Discuss the arbitrary nature of such features, then use Curry-Howard as motivation for ``the language that ought to be''. Develop STLC, discuss in detail the implications, give categorical semantics. Discuss breifly the dynamics of simply typed lambda calculus. A big disadvantage of STLC over the untyped version (which we ought to discuss since we have the tools to) is that there is no recursion. There are many ways to fix this, see G\"odel for example. In order to fix this we will introduce dependent types.

We begin by discussing the syntax of our type theory. We have a set of types $\mathbf{T}::= $
and a set of terms $\mathbf{T}::=$

\subsection{Judgements}


[[TODO: Clean up this whole paragraph(s)]]
We begin with our basic judgements. Of which there will be 5. Our STLC will have bidirectional typechecking, in that we will distinguish between the direction of type checking. There are several advantages of this and historically the two main systems called STLC are Curry's and Church's which simply differ in the direction of type checking. By having both directions and a sort of ``mode-switching rule'' we have far greater control and ease when describing type checking properties. We will also need to have a notion of \emph{judgemental equality} since we wish to do some computation. There are variations of this theme discussed in the statics chapter that allow us to have transition systems instead but we will use an equational style since transition systems can be derived from this. This also has the advantage of STLC becomming what is known as an ``equational theory''. This will be a useful feature for when we want to derrive categorical semantics. 

A context is a list of basic judgements. Our basic judgements are $x : A$. [[No it is not fix this]]

There are 5 judgements that we have:

\begin{itemize}
    \item $\Gamma \vdash A\ \mathsf{type}$ - ``$A$ is a type in context $\Gamma$''.
    \item $\Gamma \vdash T \Leftarrow A$ - ``$T$ can be checked to have type $A$ in context $\Gamma$''.
    \item $\Gamma \vdash T \Rightarrow A$ - ``$T$ synthesises the type $A$ in context $\Gamma$''.
    \item $\Gamma \vdash A \equiv B\ \mathsf{type}$ - ``$A$ and $B$ are jdugementally equal types in context $\Gamma$''.
    \item $\Gamma \vdash S \equiv T : A$ - ``$S$ and $T$ are judgementally equal terms of type $A$ in context $\Gamma$''.
\end{itemize}

\subsection{Structural rules}

Structural rules will dictate how our judgements interact with eachother, how different contexts can be formed and how substitution works. This is all roughly what a ``type theory'' ought to provide.

We begin with the \emph{variable} rule, this says that if a term $x$ appears with a type $A$ as an element in a context $\Gamma$ then $x$ synthesises a type $A$ in context $\Gamma$. Or written more succiently as:

$$
    \frac{(x:A) \in \Gamma }{\Gamma \vdash x \Rightarrow A}
$$

Other structural rules: weakening, contraction and substitution are all admissible. [[What does it mean for a rule to be admissible? We have defined this previously but we need to carefully state these facts, and prove them too!]]

\subsection{Mode-switching}

One of the features of bidirectional type checking is that we can switch the mode we are in. This is expressed as the mode switching rule:

\begin{prooftree}
    \AxiomC{$\Gamma \vdash T \Rightarrow A$}
    \AxiomC{$\Gamma \vdash A \equiv B \ \mathsf{type}$}
    \BinaryInfC{$\Gamma \vdash T \Leftarrow B$}
\end{prooftree}

This rule has been specially set up in that it will be the \emph{only way} to derive $\Gamma \vdash T \Leftarrow B$. [[TODO: talk more about this]]

In a unidirectional type system, the judgements $\Gamma \vdash T \Rightarrow A$ and $\Gamma \vdash T \Leftarrow B$ are collapsed into one: $\Gamma \vdash T : A$. And now the mode-switching rule may have a more familiar form:

\begin{prooftree}
    \AxiomC{$\Gamma \vdash T : A$}
    \AxiomC{$\Gamma \vdash A \equiv B \ \mathsf{type}$}
    \BinaryInfC{$\Gamma \vdash T : B$}
\end{prooftree}

Which shows that it is actually a rule about substituting along a judgemental equality! But this is a problem since a type checking algorithm will have to decide when to stop doing this. This is one of the big advantages that bidirectional type checking has over unidirectional type checking. The type checking algorithm will be simpler! [[TODO: Clean up and discuss type checking in more detail]]

\subsection{Equality rules}
Finally we have some structural rules for our two judgemental equality judgements. We wish for these to be an equivalence relation and that they are compatible with eachother.

First we begin with the structural rules for the judgement form $- \equiv -\ \mathbf{type}$:

We wish for our judgemental equality of types to be reflexive:
\begin{prooftree}
    \AxiomC{}
    \RightLabel{($\equiv_{\mathbf{type}}$-reflexivity)}
    \UnaryInfC{$\Gamma \vdash A \equiv A\ \mathbf{type}$}
\end{prooftree}

We want our judgemental equality of types to be symmetric:
\begin{prooftree}
    \AxiomC{$\Gamma \vdash A \equiv B \ \mathbf{type}$}
    \RightLabel{($\equiv_{\mathsf{type}}$-symmetry)}
    \UnaryInfC{$\Gamma \vdash B \equiv A \ \mathbf{type}$}
\end{prooftree}

and our judgemental equality of types to be transitive:

\begin{prooftree}
    \AxiomC{$\Gamma \vdash B \ \mathsf{type}$}
    \AxiomC{$\Gamma \vdash A \equiv B\ \mathsf{type}$}
    \AxiomC{$\Gamma \vdash B \equiv C\ \mathsf{type}$}
    \RightLabel{($\equiv_\mathsf{type}$-transitivity)}
    \TrinaryInfC{$\Gamma \vdash A \equiv C\ \mathsf{type}$}
\end{prooftree}

Notice how the previous rule also checks that $B$ is a type. This is because if we did not do this, we could insert any symbol in. This is clearly undesirable. It also demonstrates how subtly sensitive rules are.

Now we list the rules making the judgement form $- \equiv - : A$ into an equivalence relation:

We wish for our judgemental equality of terms to be reflexive:
\begin{prooftree}
    \AxiomC{}
    \RightLabel{($\equiv_{\mathsf{term}}$-reflexivity)}
    \UnaryInfC{$\Gamma \vdash T \equiv T : A$}
\end{prooftree}

We want our judgemental equality of terms to be symmetric:
\begin{prooftree}
    \AxiomC{$\Gamma \vdash S \equiv T : A$}
    \RightLabel{($\equiv_{\mathsf{term}}$-symmetry)}
    \UnaryInfC{$\Gamma \vdash T \equiv S : A$}
\end{prooftree}

and our judgemental equality of terms to be transitive:
\begin{prooftree}
    \AxiomC{$\Gamma \vdash T \Leftarrow A $}
    \AxiomC{$\Gamma \vdash S \equiv T : A$}
    \AxiomC{$\Gamma \vdash T \equiv R : A$}
    \RightLabel{($\equiv_{\mathsf{term}}$-transitivity)}
    \TrinaryInfC{$\Gamma \vdash S \equiv R : A$}
\end{prooftree}

as we stated before for transitivity judgemental equality of types we need to also check that the middle term $T$ is actually a term.

Finally we need a rule that will make  that judgemental equality of types and judgemental equality of terms interact the way we expect them to:
\begin{prooftree}
    \AxiomC{$\Gamma \vdash A \ \mathsf{type}$}
    \AxiomC{$\Gamma \vdash S \equiv T : A$}
    \AxiomC{$\Gamma \vdash A \equiv B\ \mathsf{type}$}
    \RightLabel{($\equiv_{\mathsf{term}}$-$\equiv_{\mathsf{type}}$-compat)}
    \TrinaryInfC{$\Gamma \vdash S \equiv T : B$}
\end{prooftree}


\subsection{Type formers}
What we have constructed thusfar is essentially an ``empty type theory''. What we have included which other authors typcially gloss over is a clean way of constructing a typechecking algorithm: bidirectional typechecking and an account of judgemental equality. We now study what are known as type formers, typically when we wish to add a new type to a type theory we need to think about a collection of rules. These can roughly be sorted into 5 kinds of rules:

\begin{itemize}
    \item Formation rules - How can I construct my type?
    \item Introduction rules - Which terms synthesise this type?
    \item Elimination rules - How can terms of this type be used?
    \item Computation (or equality) rules - How do terms of this type compute? (Normalise, etc.)
    \item Congruence rules - How do all the previous rules interact with judgemental equality
\end{itemize}

We make a note that although we will be providing all the rules, the congruence rules can be typically derrived from the others. Although we do not know exactly how to do this so we will provide them explicitly. We also note that not every type need computation rules.

\subsection{Inversion lemmas}
Having listed all these rules we need some lemmas detailing how different terms can \emph{only} come from a set of specified rules. This is a crucial analysis if we wish to construct a type checking algorithm.

\subsection{$\to$-types}

Building on top of our ``empty type theory'' we introduce $\to$ the function type former:

\subsubsection{Formation rules}

We start with our formation rule which simply states that in order to derive $\Gamma \vdash A \to B \ \mathsf{type}$ it is required to derive that both $\Gamma \vdash A \ \mathsf{type}$ and $\Gamma \vdash B \ \mathsf{type}$:

\begin{prooftree}
    \AxiomC{$\Gamma \vdash A\ \mathsf{type}$}
    \AxiomC{$\Gamma \vdash B\ \mathsf{type}$}
    \RightLabel{}
    \BinaryInfC{$\Gamma \vdash A \to B \ \mathsf{type}$}
\end{prooftree}

\subsubsection{Introduction rules}
\subsubsection{Elimination rules}
\subsubsection{Computation rules}
\subsubsection{Congruence rules}

\begin{comment}
%\subsection{Lambda calculus}
%We recall that there are 3 kinds of expressions in lambda calculus: variables, abstractions and applications. These are defined inductively on themselves. A variable is simply a string of characters from an alphabet. A lambda abstraction looks like $\lambda x.y$ where $x$ is some variable and $y$ is some expression. There are alternate ways of writing this, allowing us to drop the need for naming $x$, for example de Brujin indices. Finally an application is simply the concatenation $ab$ of two expressions $a$ and $b$. We will assume that  This fully describes the syntax of this type theory. We will now introduce some rules that tell us which expressions we can derive from other expressions. Firstly we have $\beta$-reduction which tells us if we have an expression of the form $(\lambda x . y)z$ this can be reduced to an expression where all occurrences of $x$ in $y$ are replaced with the expression $z$. We also have $\alpha$-conversion which I would argue isn't really a rule as naming of variables can be completely avoided in the first place using de Brujin indices or even combinators. \cite{BarendregtHenk2013Lcwt, hottbook}

%\subsection{Contexts}
%In mathematics we work with contexts implicitly. That is there is always an ambient knowledge of what has been defined. Mostly due to the nature of how we read mathematical papers. We can make this explicit using contexts. We will not however, use contexts in our discussion of type theory but we will provide a formal exposition in the appendix.

\subsection{Judgements}
Our judgements:
\begin{center}
    \begin{tabular}{c | c}
        $\Gamma\ \mathrm{ctx}$ &  $\Gamma$ is a well-formed context. \\
        $\Gamma \vdash A\ \mathrm{Type}$ & $A$ is a type in context $\Gamma$. \\
        $\Gamma \vdash x : A$ & $x$ is a term of type $A$ in context $\Gamma$. \\
%        $\Gamma \vdash x \equiv y : A$ & the terms $x$ and $y$ of type $A$ are definitionally equal in context $\Gamma$
    \end{tabular}
\end{center}


Type theory ``will be about'' deriving judgements from other judgements. Which can be concisely summarised in the form of an inference rule

$$\frac{A_1\quad A_2 \quad \cdots \quad A_n}{B}$$

which says that given the judgements $A_1,\dots,A_n$ we can derive the judgement $B$.

\subsection{Structural rules}
We now look at the rules that govern contexts and the structure of our type system.

We begin with a rule stating that the empty context (which as contexts are sets or lists is well-defined) is well-formed. Which is another way of stating that the context was grown in a specified way and is not just an arbitrary list or set of variables.

\begin{prooftree}
    \AxiomC{}
    \RightLabel{empty-ctx}
    \UnaryInfC{$\varnothing$ ctx}
    \singleLine
\end{prooftree}

We also want the concatenation of two well-formed contexts to be well-formed.

\begin{prooftree}
    \AxiomC{$\Gamma$ ctx}
    \AxiomC{$\Delta$ ctx}
    \BinaryInfC{$\Gamma,\Delta$ ctx}
\end{prooftree}

We omit rules about repeating or removing repeated elements and ordering lists (think of them as finite sets).

A variable is a statement of the form $x : A$ where $x$ is known as the term and $A$ its type.

\subsection{Function types}

We introduce a formation rule for the function type.

\begin{prooftree}
    \RightLabel{$(\to)$-form}
    \AxiomC{$\Gamma \vdash A\ \mathrm{Type}$}
    \AxiomC{$\Gamma \vdash B\ \mathrm{Type}$}
    \BinaryInfC{$\Gamma \vdash A \to B\ \mathrm{Type}$}
\end{prooftree}

We now need a rule for producing terms of this new type. We introduce the introduction rule for the function type.

\begin{prooftree}
    \RightLabel{$(\to)$-intro}
    \AxiomC{$\Gamma, x : A \vdash y : B$}
    \UnaryInfC{$\Gamma \vdash (\lambda x . y) : A \to B$}
\end{prooftree}

We will sometimes call this lambda abstraction. We next introduce a way to apply these functions to terms in their domains. We introduce our elimination rule for the function type.

\begin{prooftree}
    \RightLabel{$(\to)$-elim}
    \AxiomC{$\Gamma \vdash f : A \to B$}
    \AxiomC{$\Gamma \vdash a : A$}
    \BinaryInfC{$\Gamma \vdash f(a) : B$}
\end{prooftree}

This is essentially useless unless we have a way to compute (or reduce) this expression. This is where our computation rule comes in. The computation rule will tell us how our elimination rule and introduction rule interact.
\begin{prooftree}
    \RightLabel{$(\to)$-comp}
    \AxiomC{$(\lambda x . y) : A \to B$}
    \AxiomC{$\Gamma, a : A \vdash (\lambda x.y)a : B$}
    \AxiomC{$\Gamma, x : A, y : B, (\lambda x . y) : A \to B, a : A \vdash (\lambda x . y) (a) : B$}
    \UnaryInfC{$\Gamma \vdash y[x / a] : B$}
\end{prooftree}

%%%%%%%%%%%%%%%%%%%%

We will describe what is known as a simply typed lambda calculus. There is a lot of literature on type theory, and it doesn't seem that there are many authors in agreement of ways to present it.

In \cite{BarendregtHenk2013Lcwt} a more type theoretic approach, analysing the type theory mostly in the syntactic world. This gives us a good starting point for how we want our type theory to be presented however it may not be so easy to keep an eye on how the categorical semantics (the ways we model types in mathematics) behave. In order to do this we will use references such as \cite{CroleRoyL1993Cft, JacobsCLTT, LambekJ1986Itho}. This will be from the more categorical logic school of thought, which will study type theory that is "generated" by certain categories in interest.

We start by describing a general class of simple type theories as outlined in \cite{JacobsCLTT}. Firstly we introduce the notion of a {\it signature}. Similar accounts can be found in \cite{CroleRoyL1993Cft}. This will essentially consist of "generating" a category from some signature (which can be thought of as a stripped down type theory syntax), and then studying the functors from that category into other categories. This allows nice properties from the second category to be "pulled back" onto our type theory giving it features we desire.

\begin{defin}
	A {\bf signature} is a pair $(\Typ, \mathcal{F})$ where $\Typ$ is a finite set of {\bf basic} (or {\bf atomic}) {\bf types}. And a functor $\mathcal{F} : \Typ^\star \times \Typ \to \Set$. Where $\Typ^\star$ is the Kleene-Star operation on a set (or the free monoid over $\Typ$), defined as $X^\star := \bigcup_{n\in \N} X^n$ whose elements are finite tuples of elements of $X$ for a set $X$. We have $\mathbf{Set}$ for the category of finite sets. Note that the sets in the domain of the functor are realised as discrete categories.
\end{defin}

We will usually write a signature as $\Sigma := (\Typ, \mathcal{F})$, denote $|\Sigma|:=\Typ$ and write $F: \sigma_1,\dots,\sigma_n\to\sigma_{n+1}$ when $F \in \mathcal{F}(( \sigma_1,\dots,\sigma_n ), \sigma_{n+1})$.

\begin{defin}
    Let $\Var$ be a countable set. Elements $x\in \Var$ are called {\bf variables}.
\end{defin}

Note this style of variables is essentially de Brujin indices. But allows us to have a set of names for our variables, which allows future annoyances like $\alpha$-equivalence to be sorted out easily due to the plentiful existence of bijections from $\Var \to \Var$.

\begin{defin}
	A {\bf variable declaration} is a pair $(x, \sigma) \in \Var \times \Typ$ usually written as $x : \sigma$. This can be read as "the variable $x$ has type $\sigma$. We will define $\Dec:=\Var \times \Typ$.
\end{defin}

\begin{defin}
    A {\bf context} $\Gamma$ is an element of $\Con:=\Dec^\star$. In other words, a context is a finite list of variable declarations. We will usually write a context $\Gamma$ as $v_1 : \sigma_1, \dots ,v_n : \sigma_n$. Note that the Kleene-Star has a monoid structure with operation $","$. We can thus give $\Con$ a monoid structure and write, for contexts $\Gamma$ and $\Delta$ another context $\Gamma,\Delta$ which is the concatenation of two contexts. The notation here allows the "expanded version" to coincide, as in $\Gamma,\Delta$ can be written as $v_1 : \sigma_1, \dots ,v_n : \sigma_n, w_1 : \tau_1, \dots, w_m, \tau_m$.
\end{defin}

We also note that there is a canonical inclusion $\Dec \hookrightarrow \Con$ given that $\Dec$ freely generates the monoid $\Con$. This will allow us to write $\Gamma, x:\tau$ for $v_1 : \sigma_1, \dots ,v_n : \sigma_n, x:\tau$.

We now denote the basic statements of our language. These statements are called {\bf judgements} and we will derive

%%%%%%%%%%%%%%%%%%%%
\end{comment}









% Notes on statics and dynamics



% Category theory
% Category theory


\section{Category theory}

\subsection{Introduction}

%Category theory has a pervasive influence throughout mathematics. It is used as an organisational tools, allowing thoughts and ideas about structure and preservation to be expressed in clear, familiar terms. But it is much more than that. Category theory 

%We wish to model dependent types using category theory. In order to do this we will introduce some important category theory, give examples and illustrate how one might go about modelling dependent types. A model is a loose term used to describe the process of finding a mathematical structure, studying how it acts and using this to reason about your desired thing to study. This process can be made much more rigorous than described here, and discussion of the process is really an escapade of mathematical philosophy which we will gloss over for the sake of clarity.

 

We will introduce basic category theory. Good references are: \cite{category, BarrWellsCTCS, MacLaneSaunders1998Cftw,rotman2008introduction}

Category theory will allow us to model the desired behaviour of dependent types. 

% Definition of category
\begin{defin}
	A {\bf category} $\mathcal{C}$ consists of:
	\begin{itemize}
		\item A class $\text{Ob}(\mathcal{C})$ (usually simply denoted $\mathcal{C}$ without ambiguity) of {\bf objects}.
		\item For each object $A,B \in \mathcal{C}$, a set $\mathcal{C}(A,B)$ of \textbf{morphisms} or \textbf{arrows} called a \textbf{homset}. When writing $f \in \mathcal{C}(A,B)$ we usually denote this $f : A \to B$.
		\item For each object $A \in \mathcal{C}$ a morphism $1_A : A \to A$ called the {\bf identity}.
		\item For each object $A,B,C \in \mathcal{C}$, and for each $f : A \to B$ and $g : B \to C$ there is a function (written infix or sometimes simply omitted ($gf \equiv g \circ f$)
		
		$$
			- \circ - : \mathcal{C}(B,C) \times \mathcal{C}(A,B) \to \mathcal{C}(A,C)
		$$
		
		called {\bf composition}.
	\end{itemize}
	
	Such that the following hold:
	
	\begin{itemize}
		\item (Identity) For each $A,B \in \mathcal{C}$ and $f : A \to B$ we have $f \circ 1_A = f$ and $1_B \circ f = f$.
		\item (Associativity) For all $A,B,C,D \in \mathcal{C}$ and $f : A \to B$, $g : B \to C$, $h : C \to D$. We have: $h \circ (g \circ f) = (h \circ g) \circ f$.
	\end{itemize}
\end{defin}

%% remark about collections of morphisms
\begin{remark}
    There are many similar and mostly equivalent definitions of category in mathematics. The mostly fall into two main camps: how they treat their collection of morphisms. The two definitions are equivalent in the usual foundations of mathematics but each has their own advantages. In books such as \cite{riehl2017category} a collection of morphisms is used. This approach lends itself more naturally to the notion of an \textit{internal category} which will be an important concept later on. The other definition uses a family of collections of morphisms which lends itself to easily generalise to the notion of an \textit{enriched category}, the definitive reference for which is \cite{kelly1982basic}.

    The reason it cannot be swept under the rug so easily is because the issue of size is fundamental in category theory. Depending on what definition we chose, it will effect how we can talk about it. For an introduction to category theory, these ideas would mostly confuse the reader, hence we will simply point to \cite{2008arXiv0810.1279S} for a survey on how size issues are treated in category theory. From here on 
\end{remark}

We now give some examples:

% Category of sets
\begin{example}
	The \textbf{category of sets} denoted $\Set$ is the category whose objects are small\footnote{due to Russellian paradoxes we must distinguish between "all sets" and "enough sets". See appendix for details. } sets and morphisms are functions between sets. Composition is given by composition of functions. This is a very important category in category theory for reasons we shall come across later.
\end{example}

Choosing the direction in which our arrows point was arbitrary, but it does also mean that if we had chosen the other way we would also get a category. So every category we make canonically comes with a "friend".

% Opposite category
\begin{example}
	For any category $\mathcal{C}$, there is another category called the {\bf opposite category} $\mathcal{C}^\op$ whose objects are the same as $\mathcal{C}$ however the homsets are defined as follows: $\mathcal{C}^\op(x,y):=\mathcal{C}(y,x)$. Composition is defined using the composition from the original category.
\end{example}

%% Reword this
[NEEDS REWORDING]
Size is a common issue in category theory with many similar ways of dealing with it. It can however cause much confusion and hoop-jumping to be correct. For our purposes we will safely ignore these issues. A formal treatment can be found in the appendix. [TODO: Add formal treatment of size].

\begin{defin}
	We call a category {\bf small} if its class of objects is really a set.
\end{defin}

\begin{defin}
	Let $\mathcal{C},\mathcal{D}$ be categories. A {\bf functor} $F$ from $\mathcal{C}$ to $\mathcal{D}$ (written $F : C \to D$) consists of:
	
	\begin{itemize}
		\item An object $F(A)\in \mathcal{D}$, for all $A \in \mathcal{C}$ (also denoted $FA$).
		\item For each $A,B \in \mathcal{C}$, a function $F_{A,B} : \mathcal{C}}(A,B) \to \mathcal{D}(FA,FB)$ (also denoted $F$).
		\item For each $A \in \mathcal{C}$, $F(1_A) = 1_{FA}$.
		\item For each $A,B,C \in \mathcal{C}$, $f : A \to B$, $g : B \to C$, we have $$F(g \circ f) = F(g)\circ F(f)$$
	\end{itemize}
\end{defin}

\begin{remark}
    Historically in category theory, one would define covariant, as defined above, and contravariant functors, as a result this terminology has crept into uses of category in certain fields [REFERNCE pretty much any homological algebra book before 80s]. Contravariant functors mean to swap the order of composition when the functor is applied. In modern category theory texts, this is completely dropped as a contravariant functor from $\mathcal{C}$ to $\mathcal{D}$ is simply a covaraint functor from $\mathcal{C}^\op$ to $\mathcal{D}$. Henceforth, we shall not mention co(tra)variance of functors and refer to them simply a functors.
\end{remark}

\begin{remark}
    Given two functors $F : \mathcal{C} \to \mathcal{D}$ and $G : \mathcal{D} \to \mathcal{E}$ we can make a new functor $G \circ F$ called its \textbf{composite}, by first applying $F$ then applying $G$ on objects or morphisms. It is simple to check that this is indeed a functor. 
\end{remark}

Now that we have 'morphisms' between categories we can define another category:

\begin{example}
	The category of small categories $\Cat$ has objects small categories and morphisms functors. Composition is given by composition of functors.
\end{example}

%% Talk about natural transformations
\begin{defin}
    [Definition of natural transformation]
\end{defin}

%% Talk about functor categories
\begin{example}
    Given two categories $\mathcal{C}$ and $\mathcal{D}$ we can from a category $[\mathcal{C}, \mathcal{D}]$ called the functor category between $\mathcal{C}$ and $\mathcal{D}$. It's objects are functors $\mathcal{C} \to \mathcal{D}$ and morphisms are natural transformations between functors.
\end{example}

Special cases of this example include:

\begin{example}
    A functor $\mathcal{C}^\op \to \Set$ is typically called a \textbf{presheaf} in geometry and logic. They live in the functor category $[\mathcal{C}^\op, \Set]$ which we will call the \textbf{category of presheaves}. This is an interesting construction as it acts like the category $\mathcal{C}$ in some ways with some nice properties from $\Set$.
\end{example}

% Co(ntra)variant hom-functors

%% Prove yoneda lemma
[CHECK THIS] One of the first theorems that is proven in category theory is the \textbf{Yoneda lemma}. It says if an object acts like a certain object in every possible way, then it must be isomorphic to that object. Akin to how particles are discovered in particle accerlators by observing how they interact when bombarded with differnt particles.

\begin{lemma}
    Let $\mathcal{C}$ be a category. There is an embedding $\mathbf{y} : \mathcal{C} \to [\mathcal{C}^\op, \Set]$. 
    Where $\mathbf{y}(A) := \mathcal{C}( A, - )$, maps each object to its contravariant hom functor. 
    Presheaves that arise this way are called \textbf{representable presheaves}.
\end{lemma}

\begin{remark}
    ![WHAT IS A FULL AND FAITHFUL FUNCTOR?]
    An embedding is a functor that is full and faithful. We haven't actually proven that the "yoneda embedding" is an embedding however this is a corollary of the yoneda lemma which will prove now.
\end{remark}

[PICTURES]

\begin{theorem}{Yoneda lemma}
    Let $\mathcal{C}$ be a category. For all $X \in [\mathcal{C}^\op, \Set]$, there is a natural isomorphism between the following functors:
        $$[\mathcal{C}^\op, \Set](\mathbf{y}(-), X) \cong X(-)$$
\end{theorem}

\begin{remark}
    The set of natural transformations between $\mathbf{y}(A)$ and a presheaf $X$ is bijective to the sections of $X$ at $A$.
\end{remark}



%   There are 3 main corollaries of the yoneda lemma
%   1. The yoneda embedding is an embedding
%   2. Representable objects are unique
%   3. Being a representable object is a universal construction




% Notes from mikes talk, should really be integrated into the rest of the thesis
\section{Theories and models}

[NOTE: This is a rough outline of what the document ought to look like, not even worthy of being a draft]

[TODO: Find references for these]
\begin{defin}
    A theory asserts data and axioms.
    A model is a particular example of a theory.
\end{defin}

For example a model of "the theory of groups" in the category of sets is simply a group. A model of "the theory of groups" in the category of topological spaces is a topological group. A model of "the theory of groups" in a the category of manifolds is a Lie group.

Categorical semantics is a general procedure to go from "a theory" to the notion of an internal object in some category.

The internal objects of interest is a model of the theory in a cateagory.

Then anything we prove formally about the theory is true for all models of the theory in any category.

For each kind of "type theory" there is a corresponding kind of "structured category" in which we consider models.

\begin{itemize}
\item Lawvere theories $\leftrightarrow$ Category with finite products
\item Simply typed lambda calculus $\leftrightarrow$ Cartesian closed category
\item Dependent type theory $\leftrightarrow$ Locally CC category
\end{itemize}

A doctrine specifies:
 - A collection of type constructors
 - A categorical structure realizing these constructors as operations.

Once we fix a doctrine $\mathbb{D}$, then a $\mathbb{D}$-theory specifies "generaing" or "axiomatic" types and terms.
A $\mathbb{D}$-category is one pocessing the specified structure.
A model of a $\mathcal{D}$-theory $T$ in a $\mathcal{D}$-category $C$ realizes the types and terms in $T$ as objects and morphisms of $C$.

A finite-product theory is a type theory with unit and cartesian product as te only type constructors. Plus any number of axioms.

Example:

The theory of magmas has one axiomatic type M, and axiomatic terms $\vdash e : M$ and $x : M, y : M \vdash xy : M$. For monoids and groups we will need equality axioms.

Let $T$ be a finite-product theory, $C$ a category with finite products

A mdoel of $T$ in $C$ assigns:

\begin{enumerate}
\item To each type $A$ in $T$, an object $\llbracket A \rrbracket$ in $C$
\item To each judgement derivable in $T$:
    $$x_1 : A_1, \dots, x_n : A_n \vdash b : B$$
    A morphism in $C$
    $$\llbracket A_1 \rrbracket \times \dots \times \llbracket A_n \rrbracket \xrightarrow{\llbracket b \rrbracket} \llbracket B \rrbracket$$
\item Such that $\llbracket A \times B \rrbracket = \llbracket A \rrbracket \times \llbracket B \rrbracket$ etc.
\end{enumerate}

To define a model of $T$ in $C$, it suffices to interpret the axioms, since they "freely generate" the model.

%% Talk about doctrines

Talk about doctrines

Talk about type theory categries adjunction via syntactic category and complet category. (Syntax-semantics adjunction) Possible to set it up to be an equivalence but may not be needed.

WHY Categorical semantics:
\begin{enumerate}
 \item Proving things in a D-theory means it is valid for models of that D-theory in all categories
 \item We can use type theory to prove things about a category by working in its complete theory (internal language)
 \item We can use category theory to prove things about a type theory by working in its syntactic category.
\end{enumerate}








\bibliographystyle{plain} 
\bibliography{uthesis}


\end{document}


% Demonstration of and usefulness of dependent types as a logic and programming language
%\section{Dependent types}

We have seen previously that the Curry-Howard correspondance is a deep parallel between logic and computation. We therefore will use it as a guiding principle for a type theory. This was originally sketched by Curry [[CITE]] and the proejct taken up by Per Martin-L\"of [[CITE]]. In order to begin modifying our rules for the STLC we need to introduce the notion of a \emph{universe}.

\subsection{Universes}

Originally Martin-L\"of had added a type of all types. But this, unsuprisingly, led to Russellian paradoxes. This is known as Giraurds paradox. [[CITE]]. There is a simple resolution to this, which is inspired to a similar technique in set theory known as \emph{Grothendieck universes}. Though the type theory counterpart is much simpler to state. [[CITE]] Category theorists actually use such universes but usually restricted to two, small and large sets. [[CITE]]

There are two approaches to univereses. Universes a la Russel and universes a la Tarski. The former is much simpler to state but loses unicity of typing. The latter keeps unicity of typing and corresponds closely with the semantic models, however unfortunately has many annotations and extra congruence rules. It is generally believed that the latter can be compressed into the former, and the former annotated to give the latter. [[CITE]]

Of course we don't actually \emph{need} universes to discuss dependent types, but we will soon see that there aren't many interesting dependnet types we can write down if we have no way of letting types vary over terms. In order to do this we need to be able to write down a \emph{type family}, which is a function $F : A \to \mathcal{U}$ from a type $A$ to some universe $U$, giving us each $F(a)$ as a type, i.e. $F(a)$ varies with $a:A$.


% Derive categorical semantics for stlc % talk about LCCC
%\section{Categorical semantics of stlc}

% Model dependently typed lambda calculus using syntax machinary
%\section{Rigourous treatment of dependent types}

% Now talk about the problems of coming up with a similar model for dependent types
%\section{Problems with categorical semantics of dependent type theories}

% Survey various ways of overcomming these obstacles
%\section{Look at the various models of depenent type theories, detail on how they work}







%   Second plan for writing this documentclass
%
%   1. Introduce type theories
%   2. Add different types comment on uses
%   3. Talk about what dependent types should be
%   4. Define dependent types
%   5. Introduce pi and sigma types 
%   6. Introduce uses and applications of dependent types
%   7. Detail categorical semantics of type theories
%   8. Survey categorical semantics for dependent types
%   9. Derive steve awodeys natural model of dependent types
%   10. Should interweave category theory
%
%
%
%



%
%   Plan for writing this document
%  
%   1. Complete bibliography and use
%   2. Introduce simply typed lambda calculus (similar in style to the way John wrote it for his course notes)
%   3. Lay out general outline of how new types are added
%   4. Talk about different ways to introduce dependent types
%   5. introduce universes
%   6. Prefer the type families/univereses approach
%   7. Talk about type families
%   8. Introduce Pi types
%   9. Examples of Pi types
%   10. Introduce Sigma types
%   11. Examples of Sigma types
%   12. How they interact together
%   13. Relationship with logic
%   14. Curry-Howard correspondance
%   15. Inductive types
%   16. Examples, uses, general syntax, implementation (W-types)
%   17. Identity types
%   18. Discuss the induction principle on the identity type and axiom K etc.
%   19. Consider no restriction of axiom K
%   20. Give Identity type as inductive definition
%   21. Say that what we have now is Martin-Lof intensional type theory, reference all these papers about its semantics
% ~ ~ ~ ~ ~ Experimental territory / Could be a masters? ~ ~ ~ ~ ~
%   22. Consider functional extensionality
%   23. Consider univalence with examples
%   24. Introduce higher inductive types
%   25. Examples of HITs with reference to semantics by Schulman-Lumsdaine
%   26. Discuss HoTT, what has been done (in algebraic topology etc.) its usefulness as a language for mathematics
%   27. Category theory in HoTT
%   28. Survey of synthetic homotopy theory
%   the list goes on...



