\documentclass{article}

\usepackage[english]{babel}

\usepackage[utf8]{inputenc}
\usepackage{bussproofs}
\usepackage{amsmath}
\usepackage{amssymb}
\usepackage{amsthm}

\usepackage{framed}

\usepackage{natbib}
\usepackage{url}
\usepackage{dirtytalk}
\usepackage{pst-node}
\usepackage{tikz-cd}
\usepackage{enumerate}

\usepackage{forest}
\usepackage{mwe}
\usepackage{verbatim}



%\bibliographystyle{bath.bst}

\theoremstyle{definition}
\newtheorem{defin}{Definition}[subsection]
\newtheorem{example}[defin]{Example}
\newtheorem{theorem}[defin]{Theorem}
\newtheorem{remark}[defin]{Remark}
\newtheorem{lemma}[defin]{Lemma}
\newtheorem{cor}[defin]{Corollary}

\newcommand{\N}{\mathbb{N}} %% Need to say NN has 0

\newcommand{\catname}[1]{{\normalfont\textbf{#1}}}
\newcommand{\Set}{\catname{Set}}
\newcommand{\Cat}{\catname{Cat}}

\newcommand{\el}{\mathrm{el}}
\newcommand{\op}{\mathrm{op}}
\newcommand{\Ty}{\mathbf{Ty}}

\newcommand{\of}{\mathbf{of}}

%%
%   Common variables
%%

\newcommand{\FV}{\mathrm{FV}}
\newcommand{\A}{\mathbb{A}}
\newcommand{\T}{\mathsf{T}}

\newcommand{\Var}{\mathbf{Var}}
\newcommand{\Dec}{\mathbf{Dec}}
\newcommand{\Con}{\mathbf{Con}}
\newcommand{\Typ}{\mathbf{Type}} %%
\newcommand{\Tm}{\mathbf{Term}} %%%
\newcommand{\Jud}{\mathbf{Jud}}


\newcommand{\Id}{\mathrm{Id}}

%%
%   Type theories
%%

\newcommand{\stcu}{${\boldsymbol \lambda}_\to^{\mathrm{Cu}}$} % simply typed curry 
\newcommand{\stch}{${\boldsymbol \lambda}_\to^{\mathrm{Ch}}$} % simply typed church
\newcommand{\stdb}{${\boldsymbol \lambda}_\to^{\mathrm{dB}}$} % simply typed de brujin
\newcommand{\utbe}{${\boldsymbol {\lambda \beta \eta}}$ } % untyped beta eta

%%
%   Turn styles
%%

\newcommand{\vdashstcu}{\vdash_{\boldsymbol \lambda_\to}^{\mathrm{Cu}}} % vdash for curry
\newcommand{\vdashstcu}{\vdash_{\boldsymbol \lambda_\to}^{\mathrm{Ch}}} % vdash for church
\newcommand{\vdashstcu}{\vdash_{\boldsymbol \lambda_\to}^{\mathrm{dB}}} % vdash for de Brujin

%%
%   Harper notations
%%

\newcommand{\Term}{\mathbf{Term}}
\newcommand{\App}{\mathrm{App}}


\title{Introduction to dependent type theory}
\author{Ali Caglayan}

\begin{document}

\maketitle

\tableofcontents

\newpage

%%Simply typed lambda calculus (STLC) has been well documented and studied by type theorists and mathematicians, and it's features have been used by many programming languages [NEED REFERENCE].

%In \cite{BarendregtHenk2013Lcwt} it is noted that \say{Research monographs on dependent and inductive types are lacking.} This will essentially be one of the goals of this thesis, to provide a guide for mathematicians and computer scientists about the use of dependent type theory. As this document is written there is no single account of all approaches to \i{dependent} type theory.

%Awodey \cite{2014arXiv1406.3219A} made an observation that Dybjer's \cite{dybjer1996} categories with families (CwF) is a presheaf category with a representable natural transformation (it's fibers are representable). He then proceeds to show conditions needed to model a dependent type theory with $\Pi$, $\Sigma$ and $\mathrm{Id}$ types.


%This thesis will have three main goals.

%\begin{enumitem}
%	\item To present a dependent type theory
%	\item To model the semantics of such a type theory using categorical methods
%	\item To discuss the applications to mathematics and computer science (proof assistants, programming languages and foundations)
%\end{enumitem}

%Finally we may also discuss recent developments of something called "Homotopy type theory" and how that fits into the general picture.

%Roughly a \textit{type system} is a set of loosely organised rules outlining how ``atomic sentences'' called \textit{judgements} can be derived from each other in a given context. A \textit{context} can simply be thought of as a list of terms. 

%The aim of this thesis is to present to two sorts of audience, the utility of dependent type theory. The audiences that I have in mind are computer scientists, roughly individuals who wish to write good code, and mathematicians, roughly individuals who wish to write good proofs.

%These will be our main aims however we do also wish to develop the machinery formally.

%\section{Propositions as types}

%There is a rich interplay between programming and logic known as the Curry-Howard correspondance or propositions as types. 





%\section{What is type theory}

%Type theory is the study of types systems. That is a system that orginizes data manipulated by programs into types. This has been a very useful concept in computer science. It has allowed the writing of programs taht a more 

%\subsection{Lambda calculus}
%\subsection{Modelling type theory}
%\section{What is dependent type theory?}
%\subsection{What are dependent types?}
%\subsection{Motivation for computer scientists}
%\subsection{Motivation for mathematicians}
%\subsection{Category theory}
%\subsection{Categorical logic}
%\subsection{Future directions}

\begin{comment}
\section{Introduction}

The goal of this thesis is to introduce dependent types to the undergraduate reader. We set out 

The goal of this dissertation is to learn how to mathematically design a programming language. 


The aim of this thesis is to introduce the notion of dependent types to an undergraduate reader. The main idea of dependent types is very simple, yet deceptively subtle however, since modelling such a formalism is quite tricky. This is evidenced by the fact that there is a lot of disagreement in type theory what has or hasn't been proven. This however is a familiar story in mathematics and is usually remedied by trying to understand what has been done better. Usually with the help of a new perspective. 

Dependent types however, are not only of interest to mathematicians but also programmers. Dependent type theory (much like simply typed lambda calculus) is very much a programming language allowing the expression of ideas previously too difficult to express. This is very much facilitated by its deep connection to predicate logic.
\end{comment}

\section{Introduction}

Dependent types have been around for a while. [[Introduction with citation]]. The fact that they haven't been used widely in programming and mathematics suggests that their exposition is in dire need of attention. This is one of the goals this dissertation aims to achieve. We also note that for type theorists, categorical semantics can be daunting and obscure. For mathematicians, computer scientific ideas seem out of reach. 


\begin{itemize}
\item a[Begin with history and implications of curry Howard]

\item a[outline the ``what they should do'' of dependent types]

\item a[start to rigoursly model syntax and talk about how bad a job most authors do]

\item a[small section about classical inductive definitions]

\item a[small section on why categorical semantics]

\item a[model simply typed lambda calculus with categorical semantics]

\item a[show natural extensions of the idea and why contexts break when dependnet]

\item a[outline different approches to solving these problems]

\item a[discuss Awodey's natural models]

\item a[finally talk about future directions for type theory]

\item a[maybe some mention on applications to programming (generalising various constructs, polymorphism, GA data types)]

\item a[equality, inductive types, [[[[[maybe a tinsy bit of homotopy type theory]]]]]]
\end{itemize}




%\newcommand{\tm}{\mathrm{tm}}
\newcommand{\ty}{\mathrm{ty}}
\newcommand{\fst}{\mathrm{fst}}
\newcommand{\snd}{\mathrm{snd}}

%
% Simply typed lambda calculus
%
\section{Simply typed lambda calculus} 


First develop the features needed. Discuss the arbitrary nature of such features, then use Curry-Howard as motivation for ``the language that ought to be''. Develop STLC, discuss in detail the implications, give categorical semantics. Discuss breifly the dynamics of simply typed lambda calculus. A big disadvantage of STLC over the untyped version (which we ought to discuss since we have the tools to) is that there is no recursion. There are many ways to fix this, see G\"odel for example. In order to fix this we will introduce dependent types.

We begin by discussing the syntax of our type theory. We will start by specifying the sorts $\mathcal{S}$ of our type theory.

\begin{defin}
    The sorts of simply typed lambda calculus are terms and types $\mathcal{S} := \{ \tm , \ty\}$.
\end{defin}

We now specify the operators (with generalized arities) that we defined in definition \ref{owga}. In remark \ref{opdata} we discussed the data needed to give an operator, therefore we will present all our operators in the following table.

\begin{defin}
    The operators in the syntax of simply typed lambda calculus are given by the following table:
    \begin{center}
        \begin{tabular}{ c|c|c|c|c|c|c }
        Op & Sort & Vars & Type args & Term args & Scoping & Syntax \\
        \hline
        $\to$           & \ty &  --- & $A,B$ &  ---  &  ---  & $A \to B$            \\
        $\times$        & \ty &  --- & $A,B$ &  ---  &  ---  & $A \times B$         \\
        $(-,-)$         & \tm &  --- &  ---  & $x,y$ &  ---  & $(x,y)$              \\
        $\lambda$       & \tm &  $x$ & $A,B$ &  ---  &  $M$  & $\lambda (x : A).M$  \\
        $\mathrm{App}$  & \tm &  --- & $A,B$ &  ---  & $M,N$ & $M N$
        \end{tabular}
    \end{center}
\end{defin}

\begin{remark}
    Note that some of the syntax loses information that was put in. The application is the main example of this. In practice if we know the type of $M$ and $N$ we can deduce the type of $M N$ just from the rules we will define later. The syntax is sugared or \emph{syntactic sugar} so we do not have to write so much. If done incorrectly it could be considered an abuse of notation. It should be possible to \emph{desugar} the syntax by adding an \emph{annotated} version of an operator. For example for application instead of $M N$ we could write $\mathrm{App}_{A,B}}(M;N)$. Having this information in the syntax will be useful when we want to induct over syntax, for example when proving an intiality theorem. But in practice we will save ourselves from having to write it out.
\end{remark}

\begin{defin}
    We can now construct our raw terms and types as the collection of abts (see definition \ref{abt}) over the previously defined data $\mathrm{Term} := \mathcal{B}[\varnothing]_{\tm}$ and $\mathrm{Type} := \mathcal{B}[\varnothing]_{\ty}$.
\end{defin}

\begin{remark}
    Note that we have no variables. This is because if we set the definition of abt up correctly we don't need any, but terms can have subterms (subtrees of the abt) which have variables. The sets $\mathrm{Term}$ and $\mathrm{Type}$ become \emph{all} the types and terms we ought to be able to write down from scratch.
\end{remark}

We now need to define judgements about our syntax and write down the rules to write them down. [[Make a note about substitution because afik we haven't defined it properly yet]]. 

\subsection{Judgements}


[[TODO: Clean up this whole paragraph(s)]]
We begin with our basic judgements. Of which there will be 5. Our STLC will have bidirectional typechecking, in that we will distinguish between the direction of type checking. There are several advantages of this and historically the two main systems called STLC are Curry's and Church's which simply differ in the direction of type checking. By having both directions and a sort of ``mode-switching rule'' we have far greater control and ease when describing type checking properties. We will also need to have a notion of \emph{judgemental equality} since we wish to do some computation. There are variations of this theme discussed in the statics chapter that allow us to have transition systems instead but we will use an equational style since transition systems can be derived from this. This also has the advantage of STLC becomming what is known as an ``equational theory''. This will be a useful feature for when we want to derrive categorical semantics. 

A context is a list of basic judgements. Our basic judgements are $x : A$. [[No it is not fix this]]

There are 5 judgements that we have:

\begin{itemize}
    \item $\Gamma \vdash A\ \mathsf{type}$ - ``$A$ is a type in context $\Gamma$''.
    \item $\Gamma \vdash T \Leftarrow A$ - ``$T$ can be checked to have type $A$ in context $\Gamma$''.
    \item $\Gamma \vdash T \Rightarrow A$ - ``$T$ synthesises the type $A$ in context $\Gamma$''.
    \item $\Gamma \vdash A \equiv B\ \mathsf{type}$ - ``$A$ and $B$ are jdugementally equal types in context $\Gamma$''.
    \item $\Gamma \vdash S \equiv T : A$ - ``$S$ and $T$ are judgementally equal terms of type $A$ in context $\Gamma$''.
\end{itemize}

\subsection{Structural rules}

Structural rules will dictate how our judgements interact with eachother, how different contexts can be formed and how substitution works. This is all roughly what a ``type theory'' ought to provide.

\begin{defin}
    We begin with the \emph{variable} rule, this says that if a term $x$ appears with a type $A$ as an element in a context $\Gamma$ then $x$ synthesises a type $A$ in context $\Gamma$. Or written more succiently as:

    % Variable rule
    \begin{prooftree}
        \AxiomC{$(x:A) \in \Gamma$}
        \RightLabel{(var)}
        \UnaryInfC{$\Gamma \vdash x \Rightarrow A$}
    \end{prooftree}
\end{defin}

Other structural rules: weakening, contraction and substitution are all admissible. [[What does it mean for a rule to be admissible? We have defined this previously but we need to carefully state these facts, and prove them too!]]

\end{defin}
    One of the features of bidirectional type checking is that we can switch the mode we are in. This is expressed as the mode switching rule:

    % Switch rule
    \begin{prooftree}
        \AxiomC{$\Gamma \vdash t \Rightarrow A$}
        \AxiomC{$\Gamma \vdash A \equiv B \ \mathsf{type}$}
        \RightLabel{(switch)}
        \BinaryInfC{$\Gamma \vdash t \Leftarrow B$}
    \end{prooftree}
\end{defin}

\begin{remark}
    This rule has been specially set up in that it will be the \emph{only way} to derive $\Gamma \vdash T \Leftarrow B$. These are the kinds of properties we would like our syntax to have. A careful analysis will be done under the name of \emph{inversion lemma}. [[Link to inversion lemma?]]

    In a unidirectional type system, the judgements $\Gamma \vdash T \Rightarrow A$ and $\Gamma \vdash T \Leftarrow B$ are collapsed into one: $\Gamma \vdash T : A$. And now the mode-switching rule may have a more familiar form:

    \begin{prooftree}
        \AxiomC{$\Gamma \vdash t : A$}
        \AxiomC{$\Gamma \vdash A \equiv B \ \mathsf{type}$}
        \BinaryInfC{$\Gamma \vdash t : B$}
    \end{prooftree}

    Which shows that it is actually a rule about substituting along a judgemental equality! But this is a problem since a type checking algorithm will have to decide when to stop doing this. This is one of the big advantages that bidirectional type checking has over unidirectional type checking. The type checking algorithm will be simpler! [[TODO: Clean up and discuss type checking in more detail]]
\end{remark}

\begin{remark}
    Occasionally, we will simply mode-switch using reflexivity $\Gamma \vdash A \equiv A \ \mathsf{type}$, in which case we will abbreviate the rule as follows:
    % compact switch
    \begin{prooftree}
        \AxiomC{$\Gamma \vdash t \Rightarrow A$}
        \RightLabel{(switch)}
        \UnaryInfC{$\Gamma \vdash t \Leftarrow A$}
    \end{prooftree}
\end{remark}

\subsection{Equality rules}
Finally we have some structural rules for our two judgemental equality judgements. We wish for these to be an equivalence relation and that they are compatible with eachother.

First we begin with the structural rules for the judgement form $- \equiv -\ \mathsf{type}$:

\begin{defin}

    % Reflexivity of judgemental equality of types
    We wish for our judgemental equality of types to be reflexive:
    \begin{prooftree}
        \AxiomC{\Gamma \vdash A \ \mathsf{type}}
        \RightLabel{($\equiv_{\mathsf{type}}$-reflexivity)}
        \UnaryInfC{$\Gamma \vdash A \equiv A\ \mathsf{type}$}
    \end{prooftree}

    % Symmetry of judgemental equality of types
    We want our judgemental equality of types to be symmetric:
    \begin{prooftree}
        \AxiomC{$\Gamma \vdash A \equiv B \ \mathsf{type}$}
        \RightLabel{($\equiv_{\mathsf{type}}$-symmetry)}
        \UnaryInfC{$\Gamma \vdash B \equiv A \ \mathsf{type}$}
    \end{prooftree}

    and our judgemental equality of types to be transitive:

    % Transitivity of judgemental equality of types
    \begin{prooftree}
        \AxiomC{$\Gamma \vdash B \ \mathsf{type}$}
        \AxiomC{$\Gamma \vdash A \equiv B\ \mathsf{type}$}
        \AxiomC{$\Gamma \vdash B \equiv C\ \mathsf{type}$}
        \RightLabel{($\equiv_\mathsf{type}$-transitivity)}
        \TrinaryInfC{$\Gamma \vdash A \equiv C\ \mathsf{type}$}
    \end{prooftree}

    Notice how the previous rule also checks that $B$ is a type. This is because if we did not do this, we could insert any symbol in. This is clearly undesirable. It also demonstrates how subtly sensitive rules are.

    Now we list the rules making the judgement form $- \equiv - : A$ into an equivalence relation:

    % Reflexivity of judgemental equality of terms
    We wish for our judgemental equality of terms to be reflexive:
    \begin{prooftree}
        \AxiomC{$\Gamma \vdash t \Rightarrow A$}
        \RightLabel{($\equiv_{\mathsf{term}}$-reflexivity)}
        \UnaryInfC{$\Gamma \vdash t \equiv t : A$}
    \end{prooftree}

    % Symmetry of judgemental equality of terms
    We want our judgemental equality of terms to be symmetric:
    \begin{prooftree}
        \AxiomC{$\Gamma \vdash s \equiv t : A$}
        \RightLabel{($\equiv_{\mathsf{term}}$-symmetry)}
        \UnaryInfC{$\Gamma \vdash t \equiv s : A$}
    \end{prooftree}

    % Transitivity of judgemental equality of terms
    and our judgemental equality of terms to be transitive:
    \begin{prooftree}
        \AxiomC{$\Gamma \vdash t \Leftarrow A $}
        \AxiomC{$\Gamma \vdash s \equiv t : A$}
        \AxiomC{$\Gamma \vdash t \equiv r : A$}
        \RightLabel{($\equiv_{\mathsf{term}}$-transitivity)}
        \TrinaryInfC{$\Gamma \vdash s \equiv r : A$}
    \end{prooftree}

    as we stated before for transitivity judgemental equality of types we need to also check that the middle term $T$ is actually a term.

    % judgemental equality of types - judgemental equality of terms - congruence
    Finally we need a rule that will make  that judgemental equality of types and judgemental equality of terms interact the way we expect them to:
    \begin{prooftree}
        \AxiomC{$\Gamma \vdash A \ \mathsf{type}$}
        \AxiomC{$\Gamma \vdash s \equiv t : A$}
        \AxiomC{$\Gamma \vdash A \equiv B\ \mathsf{type}$}
        \RightLabel{($\equiv_{\mathsf{term}}$-$\equiv_{\mathsf{type}}$-compat)}
        \TrinaryInfC{$\Gamma \vdash s \equiv t : B$}
    \end{prooftree}
\end{defin}

\subsection{Type formers}
What we have constructed thusfar is essentially an ``empty type theory''. What we have included which other authors typcially gloss over is a clean way of constructing a typechecking algorithm: bidirectional typechecking and an account of judgemental equality. We now study what are known as type formers, typically when we wish to add a new type to a type theory we need to think about a collection of rules. These can roughly be sorted into 5 kinds of rules:

\begin{itemize}
    \item Formation rules - How can I construct my type?
    \item Introduction rules - Which terms synthesise this type?
    \item Elimination rules - How can terms of this type be used?
    \item Computation (or equality) rules - How do terms of this type compute? (Normalise, etc.)
    \item Congruence rules - How do all the previous rules interact with judgemental equality
\end{itemize}

We make a note that although we will be providing all the rules, the congruence rules can be typically derrived from the others. Although we do not know exactly how to do this so we will provide them explicitly. We also note that not every type need computation rules.

Building on top of our ``empty type theory'' we introduce $\to$ the function type former:

\begin{defin}

    Our formation rules tell us how to construct arrow types from other types:
    
    % -> formation
    \begin{prooftree}
        \AxiomC{$\Gamma \vdash A\ \mathsf{type}$}
        \AxiomC{$\Gamma \vdash B\ \mathsf{type}$}
        \RightLabel{($\to$-form)}
        \BinaryInfC{$\Gamma \vdash A \to B \ \mathsf{type}$}
    \end{prooftree}

    Our introduction rule tells us how to construct terms of our type. This is also known as $\lambda$-abstraction:

    % -> introduction
    \begin{prooftree}
        \AxiomC{$\Gamma , x : A\vdash M \Leftarrow B$}
        \RightLabel{($\to$-intro)}
        \UnaryInfC{$\Gamma \vdash \lambda x . M \Rightarrow A \to B$}
    \end{prooftree}

    Our elimination rule tells us how to use terms of this type. For function types this corresponds to application:

    % -> elimination
    \begin{prooftree}
        \AxiomC{$\Gamma \vdash M \Leftarrow A \to B$}
        \AxiomC{$\Gamma \vdash N \Leftarrow A$}
        \RightLabel{($\to$-elim)}
        \BinaryInfC{$\Gamma \vdash M N \Rightarrow B$}
    \end{prooftree}

    And finally we have computation rules which tell us how to compute our terms. We will later prove results about normalisation of the lambda calculus. We start with $\beta$-reduction which tells us how applicated functions compute:

    % -> beta
    \begin{prooftree}
        \AxiomC{$\Gamma , x : A \vdash y \Leftarrow B$}
        \AxiomC{$\Gamma \vdash t \Leftarrow A$}
        \RightLabel{($\to$-$\beta$)}
        \BinaryInfC{$\Gamma \vdash (\lambda x . y) t \equiv y[t / x] : B$}
    \end{prooftree}

    Then we introduce $\eta$-conversion which tells us if two functions applied to the same term and are judgementally equal then the functions are judgementally equal. This is ``function extensionality'' for judgemental equality.

    % -> eta
    \begin{prooftree}
        \AxiomC{$\Gamma , y : A \vdash M y \equiv M' y : B$}
        \RightLabel{($\to$-$\eta$)}
        \UnaryInfC{$\Gamma \vdash M \equiv M' : A \to B$}
    \end{prooftree}

    Finally we have to make sure all our rules respect judgemental equality. This means showing that $\to$ respects judgemental equality of types and that $\lambda$-terms and applications respect judgemental equality of terms.

    % -> formation congruence
    \begin{prooftree}
        \AxiomC{$\Gamma \vdash A \equiv A' \ \mathsf{type}$}
        \AxiomC{$\Gamma \vdash B \equiv B' \ \mathsf{type}$}
        \RightLabel{($\to$-$\equiv_{\mathsf{type}}$-cong)}
        \BinaryInfC{$\Gamma \vdash A \to B \equiv A' \to B' \ \mathsf{type}$}
    \end{prooftree}

    % -> introduction congruence
    \begin{prooftree}
        \AxiomC{$\Gamma , x : A \vdash M \equiv M' : B$}
        \RightLabel{($\to$-$\equiv_{\mathsf{term}}$-cong)}
        \UnaryInfC{$\Gamma \vdash \lambda x . M \equiv \lambda x . M' : A \to B$}
    \end{prooftree}

    % -> elimination congruence
    \begin{prooftree}
        \AxiomC{$\Gamma \vdash M \equiv M' : A \to B$}
        \AxiomC{$\Gamma \vdash N \equiv N' : A$}
        \RightLabel{($\to$-elim-cong)}
        \BinaryInfC{$\Gamma \vdash M N \equiv M' N' : A \to B$}
    \end{prooftree}

\end{defin}

\begin{remark}
    Notice that we don't ensure that types compute the same way. This is because the computation rules will not be used in the type checking process and are therefore irrelevant to the inversion lemmas. Later we will prove that ``fully reduced'' computations are in fact equal. This is known as the Church-Rosser theorem.
\end{remark}

We define the product type as follows.

\begin{defin}[Product type]
    
    Given two types, we have their product type:
    
    % Product formation
    \begin{prooftree}
        \AxiomC{$\Gamma \vdash A \ \mathsf{type}$}
        \AxiomC{$\Gamma \vdash B \ \mathsf{type}$}
        \RightLabel{($\times$-form)}
        \BinaryInfC{$\Gamma \vdash A \times B \ \mathsf{type}$}
    \end{prooftree}
    
    We define ordered pairs as taking a term of each type:
    
    % Product introduction
    \begin{prooftree}
        \AxiomC{$\Gamma \vdash a \Leftarrow A$}
        \AxiomC{$\Gamma \vdash b \Leftarrow B$}
        \RightLabel{($\times$-intro)}
        \BinaryInfC{$\Gamma \vdash (a, b) \Rightarrow A \times B$}
    \end{prooftree}
    
    We give two eliminators for pairs, the first and second elements:
    
    % Product eliminators
    \begin{center}
        \AxiomC{$\Gamma \vdash t \Leftarrow A \times B$}
        \RightLabel{($\times$-elim${}_1$)}
        \UnaryInfC{$\Gamma \vdash \fst(t) \Rightarrow A$}        
        \DisplayProof
        \hskip 1.5em
        \AxiomC{$\Gamma \vdash t \Leftarrow A \times B$}
        \RightLabel{($\times$-elim${}_2$)}
        \UnaryInfC{$\Gamma \vdash \snd(t) \Rightarrow B$}
        \DisplayProof
    \end{center}
    
    And we finally need to dictate how this is computed:
    
    \begin{center}
        \AxiomC{$\Gamma \vdash x \Leftarrow A$}
        \AxiomC{$\Gamma \vdash y \Leftarrow B$}
        \RightLabel{($\times$-$\beta_1$)}
        \BinaryInfC{$\Gamma \vdash \fst(x,y)\equiv x : A$}
        \DisplayProof
        \hskip 1.5em
        \AxiomC{$\Gamma \vdash x \Leftarrow A$}
        \AxiomC{$\Gamma \vdash y \Leftarrow B$}
        \RightLabel{($\times$-$\beta_2$)}
        \BinaryInfC{$\Gamma \vdash \snd(x,y)\equiv y : B$}
        \DisplayProof
    \end{center}
    
    However we need to be careful since there is a nontrivial equality we must also add as a rule:
    
    \begin{prooftree}
        \AxiomC{$\Gamma \vdash t \Rightarrow A \times B$}
        \RightLabel{($\times$-$\eta$)}
        \UnaryInfC{$\Gamma \vdash (\fst(t),\snd(t))\equiv t : A \times B$}
    \end{prooftree}
    
\end{defin}

We will also need to add a unit type. This will be the simplest type, with only one term.

\begin{defin}[Unit type]
    We begin with the formation rules, essentially saying that the unit type exists.

    % Unit formation
    \begin{prooftree}
        \AxiomC{}
        \RightLabel{($\mathbf{1}$-form)}
        \UnaryInfC{$\mathbf{1}\ \mathsf{type}$}
    \end{prooftree}

    We then say that the unit type has a term:

    % Unit introduction
    \begin{prooftree}
        \AxiomC{}
        \RightLabel{($\mathbf{1}$-intro)}
        \UnaryInfC{$\Gamma \vdash * \Rightarrow \mathbf{1}$}
    \end{prooftree}
\end{defin}

\begin{remark}
    We don't need to give any more rules since the unit type has all the properties we need. Our rules for $\to$ allow us to build constant functions anyway. And we note that all functions $\mathbf{1} \to A$ are constant functions!
\end{remark}

[[TODO: Clear up wording maybe?]]
\begin{remark}
    We make an important note that this is not the simplest presentation of the STLC of which there are many variations thereof. We chose judgemental equality and bidirectional type checking because these are features we will need if we are to enrich our type system with dependent types.
\end{remark}

\subsection{Inversion lemmas}
Having listed all these rules we need some lemmas detailing how different terms can \emph{only} come from a set of specified rules. This is a crucial analysis if we wish to construct a type checking algorithm. An inversion lemma for a type theory is typically very difficult to state, and extremely tedious to prove. But nontheless is essential if we want to induct over terms.

Luckily we set up syntax in such a way that we only need induct over the syntax. So we pick a syntactic form and the inversion lemma will tell us exactly how we can arrive at that conclusion. Let us list all term syntax we can create in STLC:

\begin{itemize}
    \item $x$ where $x$ is a variable.
    \item $\lambda x . M$ where $M$ is a term.
    \item $(x, y)$ where $x$ and $y$ are terms.
    \item $\fst, \snd$ the eliminators of $\times$
    \item $*$ the element of $\mathbf{1}$
    \item $\mathrm{ind}_{\mathbf{1}}$ the inductor of $\mathbf{1}$
\end{itemize}

%Probably not true if we set things up correctly
%We note that these inversion lemmas will only be applicable to judgement forms such as $\Gamma \vdash x \Rightarrow A$ and $\Gamma \vdash x \Leftrightarrow A$ as judgemental equality will be far too complicated (and perhaps even impossible) to decidably derive. In times where our judgemental equality 

\begin{lemma}
    
\end{lemma}

[[TODO: State this beast]]
\begin{lemma}
    In the STLC the following term forms are generated by certain rules...
\end{lemma}

\subsection{Normalisation and Canonicity}

[[These two concepts are very related, we should find some way to talk about it, including Church-Rosser]]


\begin{comment}
%\subsection{Lambda calculus}
%We recall that there are 3 kinds of expressions in lambda calculus: variables, abstractions and applications. These are defined inductively on themselves. A variable is simply a string of characters from an alphabet. A lambda abstraction looks like $\lambda x.y$ where $x$ is some variable and $y$ is some expression. There are alternate ways of writing this, allowing us to drop the need for naming $x$, for example de Brujin indices. Finally an application is simply the concatenation $ab$ of two expressions $a$ and $b$. We will assume that  This fully describes the syntax of this type theory. We will now introduce some rules that tell us which expressions we can derive from other expressions. Firstly we have $\beta$-reduction which tells us if we have an expression of the form $(\lambda x . y)z$ this can be reduced to an expression where all occurrences of $x$ in $y$ are replaced with the expression $z$. We also have $\alpha$-conversion which I would argue isn't really a rule as naming of variables can be completely avoided in the first place using de Brujin indices or even combinators. \cite{BarendregtHenk2013Lcwt, hottbook}

%\subsection{Contexts}
%In mathematics we work with contexts implicitly. That is there is always an ambient knowledge of what has been defined. Mostly due to the nature of how we read mathematical papers. We can make this explicit using contexts. We will not however, use contexts in our discussion of type theory but we will provide a formal exposition in the appendix.

\subsection{Judgements}
Our judgements:
\begin{center}
    \begin{tabular}{c | c}
        $\Gamma\ \mathrm{ctx}$ &  $\Gamma$ is a well-formed context. \\
        $\Gamma \vdash A\ \mathrm{Type}$ & $A$ is a type in context $\Gamma$. \\
        $\Gamma \vdash x : A$ & $x$ is a term of type $A$ in context $\Gamma$. \\
%        $\Gamma \vdash x \equiv y : A$ & the terms $x$ and $y$ of type $A$ are definitionally equal in context $\Gamma$
    \end{tabular}
\end{center}


Type theory ``will be about'' deriving judgements from other judgements. Which can be concisely summarised in the form of an inference rule

$$\frac{A_1\quad A_2 \quad \cdots \quad A_n}{B}$$

which says that given the judgements $A_1,\dots,A_n$ we can derive the judgement $B$.

\subsection{Structural rules}
We now look at the rules that govern contexts and the structure of our type system.

We begin with a rule stating that the empty context (which as contexts are sets or lists is well-defined) is well-formed. Which is another way of stating that the context was grown in a specified way and is not just an arbitrary list or set of variables.

\begin{prooftree}
    \AxiomC{}
    \RightLabel{empty-ctx}
    \UnaryInfC{$\varnothing$ ctx}
    \singleLine
\end{prooftree}

We also want the concatenation of two well-formed contexts to be well-formed.

\begin{prooftree}
    \AxiomC{$\Gamma$ ctx}
    \AxiomC{$\Delta$ ctx}
    \BinaryInfC{$\Gamma,\Delta$ ctx}
\end{prooftree}

We omit rules about repeating or removing repeated elements and ordering lists (think of them as finite sets).

A variable is a statement of the form $x : A$ where $x$ is known as the term and $A$ its type.

\subsection{Function types}

We introduce a formation rule for the function type.

\begin{prooftree}
    \RightLabel{$(\to)$-form}
    \AxiomC{$\Gamma \vdash A\ \mathrm{Type}$}
    \AxiomC{$\Gamma \vdash B\ \mathrm{Type}$}
    \BinaryInfC{$\Gamma \vdash A \to B\ \mathrm{Type}$}
\end{prooftree}

We now need a rule for producing terms of this new type. We introduce the introduction rule for the function type.

\begin{prooftree}
    \RightLabel{$(\to)$-intro}
    \AxiomC{$\Gamma, x : A \vdash y : B$}
    \UnaryInfC{$\Gamma \vdash (\lambda x . y) : A \to B$}
\end{prooftree}

We will sometimes call this lambda abstraction. We next introduce a way to apply these functions to terms in their domains. We introduce our elimination rule for the function type.

\begin{prooftree}
    \RightLabel{$(\to)$-elim}
    \AxiomC{$\Gamma \vdash f : A \to B$}
    \AxiomC{$\Gamma \vdash a : A$}
    \BinaryInfC{$\Gamma \vdash f(a) : B$}
\end{prooftree}

This is essentially useless unless we have a way to compute (or reduce) this expression. This is where our computation rule comes in. The computation rule will tell us how our elimination rule and introduction rule interact.
\begin{prooftree}
    \RightLabel{$(\to)$-comp}
    \AxiomC{$(\lambda x . y) : A \to B$}
    \AxiomC{$\Gamma, a : A \vdash (\lambda x.y)a : B$}
    \AxiomC{$\Gamma, x : A, y : B, (\lambda x . y) : A \to B, a : A \vdash (\lambda x . y) (a) : B$}
    \UnaryInfC{$\Gamma \vdash y[x / a] : B$}
\end{prooftree}

%%%%%%%%%%%%%%%%%%%%

We will describe what is known as a simply typed lambda calculus. There is a lot of literature on type theory, and it doesn't seem that there are many authors in agreement of ways to present it.

In \cite{BarendregtHenk2013Lcwt} a more type theoretic approach, analysing the type theory mostly in the syntactic world. This gives us a good starting point for how we want our type theory to be presented however it may not be so easy to keep an eye on how the categorical semantics (the ways we model types in mathematics) behave. In order to do this we will use references such as \cite{CroleRoyL1993Cft, JacobsCLTT, LambekJ1986Itho}. This will be from the more categorical logic school of thought, which will study type theory that is "generated" by certain categories in interest.

We start by describing a general class of simple type theories as outlined in \cite{JacobsCLTT}. Firstly we introduce the notion of a {\it signature}. Similar accounts can be found in \cite{CroleRoyL1993Cft}. This will essentially consist of "generating" a category from some signature (which can be thought of as a stripped down type theory syntax), and then studying the functors from that category into other categories. This allows nice properties from the second category to be "pulled back" onto our type theory giving it features we desire.

\begin{defin}
	A {\bf signature} is a pair $(\Typ, \mathcal{F})$ where $\Typ$ is a finite set of {\bf basic} (or {\bf atomic}) {\bf types}. And a functor $\mathcal{F} : \Typ^\star \times \Typ \to \Set$. Where $\Typ^\star$ is the Kleene-Star operation on a set (or the free monoid over $\Typ$), defined as $X^\star := \bigcup_{n\in \N} X^n$ whose elements are finite tuples of elements of $X$ for a set $X$. We have $\mathbf{Set}$ for the category of finite sets. Note that the sets in the domain of the functor are realised as discrete categories.
\end{defin}

We will usually write a signature as $\Sigma := (\Typ, \mathcal{F})$, denote $|\Sigma|:=\Typ$ and write $F: \sigma_1,\dots,\sigma_n\to\sigma_{n+1}$ when $F \in \mathcal{F}(( \sigma_1,\dots,\sigma_n ), \sigma_{n+1})$.

\begin{defin}
    Let $\Var$ be a countable set. Elements $x\in \Var$ are called {\bf variables}.
\end{defin}

Note this style of variables is essentially de Brujin indices. But allows us to have a set of names for our variables, which allows future annoyances like $\alpha$-equivalence to be sorted out easily due to the plentiful existence of bijections from $\Var \to \Var$.

\begin{defin}
	A {\bf variable declaration} is a pair $(x, \sigma) \in \Var \times \Typ$ usually written as $x : \sigma$. This can be read as "the variable $x$ has type $\sigma$. We will define $\Dec:=\Var \times \Typ$.
\end{defin}

\begin{defin}
    A {\bf context} $\Gamma$ is an element of $\Con:=\Dec^\star$. In other words, a context is a finite list of variable declarations. We will usually write a context $\Gamma$ as $v_1 : \sigma_1, \dots ,v_n : \sigma_n$. Note that the Kleene-Star has a monoid structure with operation $","$. We can thus give $\Con$ a monoid structure and write, for contexts $\Gamma$ and $\Delta$ another context $\Gamma,\Delta$ which is the concatenation of two contexts. The notation here allows the "expanded version" to coincide, as in $\Gamma,\Delta$ can be written as $v_1 : \sigma_1, \dots ,v_n : \sigma_n, w_1 : \tau_1, \dots, w_m, \tau_m$.
\end{defin}

We also note that there is a canonical inclusion $\Dec \hookrightarrow \Con$ given that $\Dec$ freely generates the monoid $\Con$. This will allow us to write $\Gamma, x:\tau$ for $v_1 : \sigma_1, \dots ,v_n : \sigma_n, x:\tau$.

We now denote the basic statements of our language. These statements are called {\bf judgements} and we will derive

%%%%%%%%%%%%%%%%%%%%
\end{comment}









%% General type theory stuff

\section{Syntax}

\section{Rules}

\section{Semantics}

\begin{defin}
	Let $C$ be a small category. The {\bf category of elements} $\el(F)$ of a functor 
	$F : C \to \Set$ is the following pullback in \Cat:
	
	\begin{equation}
		\begin{tikzcd}
			%	\el(F) \arrow[rr, "\rho_F"] & & \Set
			\el(F) \arrow[dd, "\pi_F"'] \arrow[rr, "\rho_F"] &  & \Set_* \arrow[dd, "U"] \\
			&  &  \\
			C \arrow[rr, "F"] &  & \Set
		\end{tikzcd}
	\end{equation}

	where $U$ is the forgetful functor from the category of pointed sets 
	$\Set_*$ to the category of sets $\Set$.
\end{defin}
Thus $\el : [C, \Set] \to \Cat$ is a functor. TODO: Prove this.


\begin{defin}
	A {\bf category with families (CwF)} consists of:
	\begin{itemize}
		\item A small category $C$
		\item A terminal object $1 \in C$
		\item Two presheaves $\Tm, \Ty \in [C^\op, \Set]$
		\item A morphism of presheaves $\of : \Tm \to \Ty$
		\item An algebraic representation of the map $\of$ or in other words
		a right adjoint to $\el(\of) : \el(\Tm) \to \el(\Ty)$ (TODO: Don't link 
		this definition so much need rephrasing).
	\end{itemize}
\end{defin}




%% Category theory


\section{Category theory}

\subsection{Introduction}

%Category theory has a pervasive influence throughout mathematics. It is used as an organisational tools, allowing thoughts and ideas about structure and preservation to be expressed in clear, familiar terms. But it is much more than that. Category theory 

%We wish to model dependent types using category theory. In order to do this we will introduce some important category theory, give examples and illustrate how one might go about modelling dependent types. A model is a loose term used to describe the process of finding a mathematical structure, studying how it acts and using this to reason about your desired thing to study. This process can be made much more rigorous than described here, and discussion of the process is really an escapade of mathematical philosophy which we will gloss over for the sake of clarity.

 

We will introduce basic category theory. Good references are: \cite{category, BarrWellsCTCS, MacLaneSaunders1998Cftw,rotman2008introduction}

Category theory will allow us to model the desired behaviour of dependent types. 

% Definition of category
\begin{defin}
	A {\bf category} $\mathcal{C}$ consists of:
	\begin{itemize}
		\item A class $\mathrm{Ob}(\mathcal{C})$ (usually simply denoted $\mathcal{C}$ without ambiguity) of {\bf objects}.
		\item For each object $A,B \in \mathcal{C}$ a set $\hom_{\mathcal{C}}(A,B)$ of \textbf{morphisms} or \textbf{arrows} called a \textbf{hom-set} (written $\hom$ when the context is clear). When writing $f \in \hom(A,B)$ we usually denote this $f : A \to B$.
		\item For each object $A \in \mathcal{C}$ a morphism $1_A : A \to A$ called the {\bf identity}.
		\item For each object $A,B,C \in \mathcal{C}$, and for each $f : A \to B$ and $g : B \to C$ there is a function (written infix or sometimes simply omitted ($fg \equiv f \circ g$)
		
		$$
			- \circ - : \hom(B,C) \times \hom(A,B) \to \hom(A,C)
		$$
		
		called {\bf composition}.
	\end{itemize}
	
	Such that the following hold:
	
	\begin{itemize}
		\item (Identity) For each $A,B \in \mathcal{C}$ and $f : A \to B$ we have $f \circ 1_A = f$ and $1_B \circ f = f$.
		\item (Associativity) For all $A,B,C,D \in \mathcal{C}$ and $f : A \to B$, $g : B \to C$, $h : C \to D$. We have: $h \circ (g \circ f) = (h \circ g) \circ f$.
	\end{itemize}
\end{defin}

\begin{defin}
    A \textbf{category} $\mathcal{C}$ consists of:
    \begin{itemize}
        \item A class of objects $\mathrm{Ob}(\mathcal{C})$
        \item For each $x, y \in \mathrm{Ob}(\mathcal{C})$ a set of \textbf{morphisms} $\hom_{\mathcal{C}}(x,y)$ called the \textbf{hom-set} or set of \textbf{arrows}. 
    \end{itemize}
\end{defin}


\begin{remark}
    There are many similar and mostly equivalent definitions of category in mathematics. The mostly fall into two main camps: how they treat their collection of morphisms. The two definitions are equivalent in the usual foundations of mathematics but each has their own advantages. In books such as \cite{riehl2017category} a collection of morphisms is used. This approach lends itself more naturally to the notion of an \textit{internal category} which will be an important concept later on. The other definition uses a family of collections of morphisms which lends itself to easily generalise to the notion of an \textit{enriched category}, the definitive reference for which is \cite{kelly1982basic}.

    The reason it cannot be swept under the rug so easily is because the issue of size is fundamental in category theory. Depending on what definition we chose, it will effect how we can talk about it. For an introduction to category theory, these ideas would mostly confuse the reader, hence we will simply point to \cite{2008arXiv0810.1279S} for a survey on how size issues are treated in category theory. From here on 
\end{remark}

We now give some examples:

% Category of sets
\begin{example}
	The \textbf{category of sets} denoted $\Set$ is the category whose objects are small\footnote{due to Russellian paradoxes we must distinguish between "all sets" and "enough sets". See appendix for details. } sets and morphisms are functions between sets. Composition is given by composition of functions. This is a very important category in category theory for reasons we shall come across later.
\end{example}

Choosing the direction in which our arrows point was arbitrary, but it does also mean that if we had chosen the other way we would also get a category. So every category we make canonically comes with a friend.

% Opposite category
\begin{example}
	For any category $\mathcal{C}$, there is another category called the {\bf opposite category} $\mathcal{C}^\op$ whose objects are the same as $\mathcal{C}$ however the homsets are defined as follows: $\hom_{\mathcal{C}^\op}(x,y):=\hom_{\mathcal{C}}(y,x)$. Composition is defined using the composition from the original category.
\end{example}

Size is a common issue in category theory with many similar ways of dealing with it. It can however cause much confusion and hoop-jumping to be correct. This is the nature of how things are done in set theory however. Later we will investigate other logical frameworks and 

\begin{defin}
	We call a category {\bf small} if its class of objects is really a set.
\end{defin}

\begin{defin}
	Let $\mathcal{C},\mathcal{D}$ be categories. A {\bf functor} $F$ from $\mathcal{C}$ to $\mathcal{D}$ (written $F : C \to D$) consists of:
	
	\begin{itemize}
		\item An object $F(A)\in \mathcal{D}$, for all $A \in \mathcal{C}$ (also denoted $FA$).
		\item For each $A,B \in \mathcal{C}$, a function $F_{A,B} : \hom_{\mathcal{C}}(A,B) \to \hom_{\mathcal{D}}(FA,FB)$ (also denoted $F$).
		\item For each $A \in \mathcal{C}$, $F(1_A) = 1_{FA}$.
		\item For each $A,B,C \in \mathcal{C}$, $f : A \to B$, $g : B \to C$, we have $$F(g \circ f) = F(g)\circ F(f)$$
	\end{itemize}
\end{defin}

\begin{remark}
    Historically in category theory, one would define covariant, as defined above, and contravariant functors. Contravariant functors mean to swap the order of composition when the functor is applied. In modern category theory texts, this is completely dropped as a contravariant functor from $\mathcal{C}$ to $\mathcal{D}$ is simply a covaraint functor from $\mathcal{C}^\op$ to $\mathcal{D}$. Henceforth, we shall not mention co(tra)variance of functors and refer to them simply a functors.
\end{remark}

\begin{remark}
    Given two functors $F : \mathcal{C} \to \mathcal{D}$ and $G : \mathcal{D} \to \mathcal{E}$ we can make a new functor $G \circ F$ by first applying $F$ then applying $G$. It is simple to check that this is indeed a functor. 
\end{remark}

Now that we have 'morphisms' between categories we can define another category:

\begin{example}
	The category of small categories $\Cat$ has objects small categories and morphisms functors. Composition is given by composition of functors.
\end{example}



%% Talk about natural transformations

%% Talk about functor categories

%% Define presheaves

%% Prove yoneda lemma










%Let $\Sigma$ be a signature with $\Typ=|\Sigma|$ as its underlying set of types.

Let $\Var$ be a countable set, elements of which will be called (term) variables.

A variable declaration is a pair $(x, t) \in \Var \times \Typ$ usually written $x:t$.

Given that $\Var$ is countable there is a function $v : \N \to \Var$. Therefore when we write $v_n$ we are specifying the $n$th element of $\Var$. A context is then a sequence of types $\Typ^*$, whose elements $\Gamma = (t_1,\dots, t_n)$, we write like this $\Gamma = v_1:t_1, \dots, v_n:t_n$. Now if we have another context $\Delta =v_1:u_1, \dots, v_m:u_m$, we can concatenate them like this: $\Gamma,\Delta = v_1:t_1,\dots, v_n:t_n, v_{n+1}:u_1,\dots,v_{n+m}: u_m$. So a context is really just a sequence of types, but the index of the sequence also refers to the variable. The set of contexts is called $\Con$.

Judgements are the basic statements or assertions of our theory. We will have starting judgements (perhaps called axioms) whereby we derive other judgements according to rules.

One judgement in this simply typed lambda calculus can be defined as a triple $(\Gamma, t, T) \in \Con \times \Var \times \Typ$. 

Another is a well-formed context, which 

We may also add other kinds of judgements so we will accumulate those in a set called $\Jud$.

An inference rule is a function $\Jud^* \to \Jud$. We will pick these carefully as they will essentially "generate" our type theory.

For simply typed lambda calculus


% induction in set theory

\section{Induction}


\begin{comment}

We begin by preparing the tools we will use later on. One of these tools will be structural induction. We will instead prove[state?] a vastly more general theorem[principle?] of set theory called the \textit{well-founded induction principal}.



This is a standard theorem in set theory [cite set theory books]. For definitions of well-foundedness that have been treated carefully we follow \cite[\S 8]{2018arXiv180805204S}.

\begin{defin}\ 
    \begin{enumerate}[(i)]
        \item A \textbf{graph} is a set $X$, whose elements are called \textbf{nodes}, equipped with a binary relation $\prec$.
        \item If $x \prec y$ then we say $x$ is a \textbf{child} of $y$.
        \item We call a graph \textbf{pointed} if it has a distinguished node $\star$ called the \textbf{root}.
        \item A pointed graph is \textbf{accessible}, if for every $x \in X$, there exists a path $x = x_n \prec \cdots \prec x_0 = \star$.
        \item For any node $x$ of $X$ we write $X/x$ to denote the graph whose nodes are nodes of $X$ that admit a path to $x$. This is naturally pointed by $x$. The relation is the same, simply restricted to the subset $X/x$ of $X$.
    \end{enumerate}
\end{defin}

\begin{remark}
    Note that the definition of accessible relies on a definition of natural numbers. If one was to carry out this construction in a setting more general than sets we would need what is called a \textit{Natural Numbers Object (NNO)}.
\end{remark}

We now make sure that subsets of graphs bring the nodes' parents.

\begin{defin}
    A subset $S$ of a graph $X$ is \textbf{inductive} if for any node $x$ in $X$, when all the children of $x$ are in $S$, $x$ is also in $S$.
\end{defin}

\begin{defin}
    A graph $X$ is \textbf{well-founded} if any inductive subset of $X$ is equal to all of $X$.
\end{defin}

\begin{remark}
    This is slightly weaker than classical versions of well-foundedness in logic, in which a graph would be well-founded if every inhabited subset has a $\prec$-minimal element. In fact such a definition would imply the law of excluded middle.
\end{remark}

\begin{lemma}\label{sswfg}
    Any subset of a well-founded graph is a well-founded graph with the induced relation.
\end{lemma}

We will follow the proof outlined in \cite[Chapter 3]{winskel1993formal}. This has been the simplest proof to understand. Other proofs can be found in for example \cite{johnstone1987notes} and \cite[Chapter 7]{barwise1982handbook}. However we found these proof too technical and verbose for simple use.

\begin{theorem}[Principle of well-founded induction]
    Let $X$ be a well-founded graph and $P$ be a property of nodes of $X$.
    We have that $$\forall x \in X, P(x) \Leftrightarrow \forall x \in X,((\forall y \prec x, P(y)) \Rightarrow P(x)).$$
\end{theorem}

\begin{proof}
    The proof in the forward direction is an easy tautology. For the converse, we assume  $ \forall x \in X,((\forall y \prec x, P(y)) \Rightarrow P(x))$. Clearly, the set of all such $y$ as we have described $\{y \prec x | P(y)\}$ is a subset of $X$, and hence a well-founded graph by \ref{sswfg} but not inductive as a subset of $X$. [FINISH PROOF, just a matter of applying the definitions]
\end{proof}

[Then here we will talk about structural induction as a special case of well-founded induction. Inductively defined sets and whatnot. This should cover all uses later on.]

\end{comment}

\subsection{Well-founded induction}

The notion of well-founded induction is a standard theorem of set theory. The classical proof of which usually uses the law of excluded middle \cite[p. 62]{johnstone1987notes}, \cite[Ch. 7]{barwise1982handbook}. It's use in the formal semantics of programming languages is not much different either \cite[Ch. 3]{winskel1993formal}. There are however more constructive notions of well-foundedness \cite[\S 8]{2018arXiv180805204S} with more careful use of excluded middle. We will follow \cite{10.2307/2275781}, as this is the simplest to understand, and we won't be using this material much other than an initial justification for induction in classical mathematics.

\begin{defin}
    Let $X$ be a set and $\prec$ a binary relation on $X$. A subset $Y \subseteq X$ is called \textbf{$\prec$-inductive} if
    $$
        \forall x \in X, \quad (\forall y \prec x,\ y \in Y) \Rightarrow x \in Y.
    $$
\end{defin}

\begin{defin}\label{wf}
    The relation $\prec$ is \textbf{well-founded} if the only $\prec$-inductive subset of $X$ is $X$ itself. A set $X$ equipped with a well-founded relation is called a \textit{well-founded set}.
\end{defin}

\begin{theorem}[Well-founded induction principle]
    Let $X$ be a well-founded set and $P$ a property of the elements of $X$ (a proposition). Then
    $$
        \forall x \in X, P(x) \Leftrightarrow \forall x \in X,\quad (\forall y \prec x, P(y)) \Rightarrow P(x).
    $$
\end{theorem}
\begin{proof}
    The forward direction is clearly true. For the converse, assume $\forall x \in X,((\forall y \prec x, P(y)) \Rightarrow P(x))$. Note that $P(y) \Leftrightarrow x \in Y := \{ x \in X \mid P(x)\} $ which means our assumption is equivalent to $\forall x \in X,\ (\forall y \prec x,\ y \in Y) \Rightarrow x \in Y$ which means $Y$ is $\prec$-inductive by definition. Hence by \ref{wf} $Y=X$ giving us $ \forall x \in X, P(x)$.
\end{proof}


% harper

\section{Type theory}

\subsection{Introduction}

We will follow the structure of syntax outlined in Harper \cite{harper_2016}. There are several reasons for this. Firstly, for example in Barendregt \cite{BarendregtHenk2013Lcwt} we have notions of substitution left to the reader under the assumption that they can be fixed. Generally Barendregt's style is like this and even when there is much formalism, it is done in a way that we find peculiar. We will follow Harper as an up to date, reputable and modern source.

In other type theory books such as \cite{CroleRoyL1993Cft, LambekJ1986Itho} not much attention is given to the syntax. For example notions like substitution are defined directly on terms of a lambda calculus, which we disagree with wholeheartedly. We think it is important to have a well-defined and well-understood notion of syntax before defining any type theories for a few reasons, firstly the structures we use are well understood and studied by other computer scientists in areas such as compilers [REF], and secondly, it will give us good notions of variables, bound variables and substituion. Also, our chosen structure will allow us to derive the semantics of our type theories in a much easier way, whereas if we hadn't paid attention to syntax we would have trouble keeping things well defined.


\subsection{Syntax}

We begin by outlining what exactly syntax is, and how to work with it. This will be important later on if we want to prove things about our syntax as we will essentially have good data structures to work with.

%We will begin with the notion of an {\it abstract syntax tree}. Which can be what is informally known as syntax, thus formal statements about the syntax are referring to its manifestation as an abstract syntax tree.

%% Sort
\begin{defin}[Sorts]
    Let $\mathcal{S}$ be a finite set, which we will call \textbf{sorts}. An element of $\mathcal{S}$ is called a \textbf {sort}.
\end{defin}

A sort could be a term, a type, a kind or even an expression. It should be thought of an abstract notion of the kind of syntactic element we have. Examples will follow making this clear. Let us fix a set $\mathcal S$.

%% Arity
\begin{defin}[Arities]
    An \textbf{arity} is an element $((s_1,\dots,s_n),s)$ of the set of \textbf{arities} $\mathcal{Q}:=\mathcal{S}^\star \times \mathcal{S}$ where $\mathcal{S}^\star$ is the Kleene-star operation on the set $\mathcal{S}$ (a.k.a the free monoid on $\mathcal{S}$ or set of finite tuples of elements of $\mathcal{S}$). An arity is typically written as $(s_1,\dots,s_n)s$. 
\end{defin}

Similarly to before, we fix a set $\mathcal{Q}$ of arities.

%% Operator
\begin{defin}[Operators]
    Let $\mathcal{O} :=\{ \mathcal{O}_\alpha \}_{\alpha \in \mathcal{Q}}$ be an $\mathcal{Q}$-indexed (arity-indexed) family of disjoint sets of \textbf{operators} for each arity. An element $o \in \mathcal{O}_\alpha$ is called an \textbf{operator} of arity $\alpha$. If $o$ is an operator of arity $(s_1,\dots,s_n)s$ then we say $o$ has \textbf{sort} $s$ and that $o$ has $n$ \textbf{arguments} of sorts $s_1,\dots,s_n$ respectively.
\end{defin}

Fix an $\mathcal{Q}$-indexed family $\mathcal{O}$ of sets of operators of each arity in $\mathcal{Q}$.

%% Variables
\begin{defin}[Variables]
    Let $\mathcal{X}:= \{ \mathcal{X}_s\}_{s \in \mathcal{S}}$ be an $\mathcal{S}$-indexed (sort-indexed) family of disjoint (finite?) sets $\mathcal{X}_s$ of \textbf{variables} of sort $s$. An element $x\in\mathcal{X}_s$ is called a \textbf{variable} $x$ of \textbf{sort} $s$. 
\end{defin}

%% Fresh variables
\begin{defin}[Fresh variables]
    We say that $x$ is \textbf{fresh} for $\mathcal{X}$ if $x \not\in \mathcal{X}_s$ for any sort $s\in \mathcal{S}$. Given an $x$ and a sort $s\in \mathcal{S}$ we can form the family $\mathcal{X},x$ of variables by adding $x$ to $\mathcal{X}_s$. 
\end{defin}

%% Remark about notation for adding variables
\begin{remark}
    The notation $\mathcal{X},x$ is ambiguous because the sort $s$ associated to $x$ is not written. But this can be remedied by being clear from the context what the sort of $x$ should be.
\end{remark}

\begin{defin}[Abstract syntax tree]
    The family $\mathcal{A}[\mathcal{X}]=\{ \mathcal{A}[\mathcal{X}]_s \}_{s \in \mathcal{S}}$ of \textbf{abstract syntax trees} (or asts), of \textbf{sort} $s$, is the smallest family satisfying the following properties:
    
    \begin{enumerate}
        \item A variable $x$ of sort $s$ is an ast of sort $s$: if $x \in X_s$, then $x \in \mathcal{A}[\mathcal{X}]_s$.
        
        \item Operators combine asts: If $o$ is an operator of arity $(s_1, \dots, s_n)s$, and if $a_1 \in \mathcal{A}[\mathcal{X}]_{s_1}, \dots, a_n \in \mathcal{A}[\mathcal{X}]_{s_n}$, then the ast $o(a_1;\dots; a_n) \in \mathcal{A}[\mathcal{X}]_s$.
    \end{enumerate}
\end{defin}

[Draw a picture here]

\begin{remark}
    The idea of a smallest family satisfying certain properties is that of structural induction. So another way to say this would be a family of sets inductively generated by the following constructors.
\end{remark}

\begin{remark}
    An ast can be thought of as a tree whose leaf nodes are variables and brach nodes are operators. 
\end{remark}

\begin{remark}
    When we prove properties $\mathcal{P}(a)$ of an ast $a$ we can do so by structural induction on the cases above.
\end{remark}

\begin{lemma}
    We have $\mathcal{A}[\mathcal{X}] \subseteq \mathcal{A}[\mathcal{Y}]$ if and only if $\mathcal{X} \subseteq \mathcal{Y}$.
\end{lemma}




%% Barendregt Simply typed lambda calculus

Barendregt \cite{BarendregtHenk2013Lcwt} (or B for short) introduces {\it simply typed lambda calculus} by introducing three versions \stcu, \stch, \stdb.

\section{Type theory}

\subsection{Untyped lambda calculus}

\begin{defin}

    Let $\Var$ (what B calls $\mathsf{V}$) be a set of variables perhaps defined as $\Var := \{x, x', x'', \dots\}$. We will use B's inductive notation and write this as $$\Var::= x \mid \Var '$$
    which is read: elements of $\Var$ are of the form $x$ or an element of $\Var$ with a $'$.

\end{defin}

We then define a set $\Tm$ (what B calls $\Lambda$) of terms (what B calls lambda terms). 

\begin{defin}

    Elements of $\Tm$ are defined as follows $$\Tm ::= \Var \mid \lambda\ \Var\ \Tm \mid \Tm\ \Tm$$
    where a {\bf term} is either a {\bf variable}, a {\bf lambda term} (usually of the form $\lambda x.t$) or an {\bf application} of two terms.

\end{defin}

B goes ahead and eases the notation slightly, which we also do. This is for readability mostly.


\begin{remark}
   
    We introduce the following notation:

    \begin{enumerate}[(i)]
        \item Letting $x,y,z, \dots, x_0,y_0,z_0, \dots, x_1,y_1,z_1,\dots$ denote arbitrary variables.
        \item $M,N,L,\dots$ denote arbitrary lambda terms (elements of $\Tm$).
        \item Application of terms is left-associative i.e. $A(B(C\cdots)) \equiv ABC\cdots$
        \item Lambda terms are right-associative i.e. $\lambda x_1.(\cdots (\lambda x_n.M) \cdots ) \equiv \lambda x_1 \cdots x_n.M$
    \end{enumerate}

\end{remark}

If we were to choose not to introduce these notational simplifications, it would be very tedious to write all the brackets and not very helpful to the reader.

We will now introduce the notion of a {\bf free variable}.

\begin{defin}

    Let $M \in \Tm$.
    
    \begin{enumerate}[(i)]
        \item The set of {\bf free variables} of $M$, written $\FV(M)$. Variables that are not free are called {\bf bound}.
        \item If $\FV(M) = \varnothing$, then $M$ is called {\bf closed} or a {\bf combinator}. The set of combinators can be written as $$\Tm^\varnothing = \{ M \in \Tm \mid \FV(M) = \varnothing\}$$
    \end{enumeate}
    We can define $\FV : \Tm \to P(\Var)$ by induction on $M$ which we can do due to the inductive definition of $\Tm$. So we have three cases:
    $$
        \begin{aligned}    
                &M \equiv x ,  &\FV(M) &:= \{x\} \\
                &M \equiv \lambda x . N,  &\FV(M)&:= \FV(N) - \{ x \} \\
                &M \equiv N L,  &\FV(M) &:= \FV(N) \cup \FV(L)
        \end{aligned}
    $$
\end{defin}

\begin{example}
Some well known combinators are $\mathbf{I} :\equiv \lambda x . x$, $\mathbf{K} :\equiv \lambda x y .y$ and $\mathbf{S}:\equiv \lambda x y z . xz(yz)$. These are well studied however we will not discuss them much here. For a comprehensive study of various combinators and their uses see \cite{smullyan2012mock}.
\end{example}

We now define (untyped) lambda calculus. B does this by defining what they call an equational theory on $\Tm$. This is where the calculus has a notion of equality. We will simply say that this equality is an equality from the metatheory (the logic used to define the calculus). For all intents and purposes our logic is first order logic with ZFC. Although it is very unlikely we will use choice anywhere.

\begin{defin}
    The symbol $\equiv$ denotes the equality in the metatheory. This will have all the usual properties of an equivalence relation and also play nicely with our terms. For example $M \equiv N \implies \lambda x . M \equiv \lambda x . N$.
\end{defin}

This means that we will not have to define properties like reflexivity and transitivity as they essentially come for free from our metatheory. This also has the advantage that we can comfotably add equalities (forcing two things to be equal) without having to define it in our calculus.

We go onto define \utbe  as the terms $\Tm$ modulo the equivalence relation of the equality in our metatheory. To which we will add the following equalities:

\[(\lambda x . M) N &\equiv M[x := N]\tag{$\boldsymbol \beta$-rule}\]
\[\lambda x . M x &\equiv M\tag{$\boldsymbol \eta$-rule}\]

Note that when we write terms from now on we are really talking about the representative of the equivalence class of terms in the set of terms modulo our definitional equality. B talks about reductive theories where we have essentially inference rules giving 

\begin{prooftree}
    \RightLabel{($\boldsymbol \beta$)}
    \AxiomC{$(\lambda x . M) N$}
    \UnaryInfC{$ M[x := N]$}
\end{prooftree}

\begin{prooftree}
    \RightLabel{($\boldsymbol \eta$)}
    \AxiomC{$\lambda x . M x$}
    \UnaryInfC{$ M$}
\end{prooftree}

\begin{remark}
It is here that B talks about $\alpha$-equivalence. We will go ahead and do the same by adding in equalities for $\alpha$-conversion of terms. Thus our terms modulo definitional equality will be up to $\alpha$-equivalence too.
\end{remark}

\begin{remark}
B also talks about properties of the reduction defined such as satisfaction of the Church-Rosser theorem. This is not entirely relevent here but may be important that it holds.
\end{remark}

\subsection{Simple types}

So far we have been working in untyped lambda calculus, which in itself has been the basis of many functional programming languages. However for our purposes we could argue it is uninteresting.

We will now try to classify our terms in such a way that we assign a type to them. Then we will restrict our lambda terms' applicability by checking the type. This may seem restrictive but it is a very useful notion that will be prevelent in the theory to come.

\begin{defin}


\end{defin}




\bibliographystyle{plain} 
\bibliography{uthesis}


\end{document}

%
%   Plan for writing this document
%  
%   1. Complete bibliography and use
%   2. Introduce simply typed lambda calculus (similar in style to the way John wrote it for his course notes)
%   3. Lay out general outline of how new types are added
%   4. Talk about different ways to introduce dependent types
%   5. introduce universes
%   6. Prefer the type families/univereses approach
%   7. Talk about type families
%   8. Introduce Pi types
%   9. Examples of Pi types
%   10. Introduce Sigma types
%   11. Examples of Sigma types
%   12. How they interact together
%   13. Relationship with logic
%   14. Curry-Howard correspondance
%   15. Inductive types
%   16. Examples, uses, general syntax, implementation (W-types)
%   17. Identity types
%   18. Discuss the induction principle on the identity type and axiom K etc.
%   19. Consider no restriction of axiom K
%   20. Give Identity type as inductive definition
%   21. Say that what we have now is Martin-Lof intensional type theory, reference all these papers about its semantics
% ~ ~ ~ ~ ~ Experimental territory / Could be a masters? ~ ~ ~ ~ ~
%   22. Consider functional extensionality
%   23. Consider univalence with examples
%   24. Introduce higher inductive types
%   25. Examples of HITs with reference to semantics by Schulman-Lumsdaine
%   26. Discuss HoTT, what has been done (in algebraic topology etc.) its usefulness as a language for mathematics
%   27. Category theory in HoTT
%   28. Survey of synthetic homotopy theory
%   the list goes on...



