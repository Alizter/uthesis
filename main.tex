\documentclass{article}

\include{DissertationDefs}    %% These are the includes required for the doc 

\usepackage[english]{babel}
\usepackage[utf8]{inputenc}
\usepackage{bussproofs}
\usepackage{framed}

% Here we squish our bibliography up
\usepackage{natbib}
\setlength{\bibsep}{1pt}
\renewcommand{\bibfont}{\small} 

% Todo purge useless packages
\usepackage{url}
\usepackage{dirtytalk}
\usepackage{pst-node}
\usepackage{tikz-cd}

\usepackage[shortlabels]{enumitem}
%\usepackage{enumerate}

\usepackage{forest}
\usepackage{mwe}
\usepackage{verbatim}
\usepackage{amsmath}
\usepackage{amssymb}
\usepackage{amsthm}
\usepackage{stmaryrd}
\usepackage[hidelinks]{hyperref}
\usepackage{pdflscape}

\usepackage[framemethod=tikz]{mdframed}
%\usepackage{tcolorbox}

\usepackage{silence}
\WarningFilter{mdframed}{You got a bad break}
\makeatletter
\mdf@PackageWarning{You got a bad break\MessageBreak
  because the last split box is empty\MessageBreak
  You have to change the settings}
\makeatother


\usepackage[toc,page]{appendix}
\usepackage[nottoc,numbib]{tocbibind}
% Ommiting this because it looks terrible
%\bibliographystyle{bath.bst}



% Theorems, definitions etc.
\theoremstyle{definition}
\newtheorem{defi}{Definition}[subsection]
\newtheorem{example}[defi]{Example}
\newtheorem{theorem}[defi]{Theorem}
\newtheorem{remark}[defi]{Remark}
\newtheorem{lemma}[defi]{Lemma}
\newtheorem{cor}[defi]{Corollary}

\newenvironment{defin}{\begin{mdframed}\begin{defi}}{\end{defi}\end{mdframed}}

\newcommand{\N}{\mathbb{N}} %% Need to say NN has 0


%%
%   Common variables
%%

\newcommand{\Var}{\mathbf{Var}}

\newcommand{\Id}{\text{Id}}

%%
%   Harper notations
%%

\newcommand{\Term}{\mathbf{Term}}
\newcommand{\App}{\text{App}}
\newcommand{\scope}{\triangleleft}
\newcommand{\soa}{\text{soa}}
\newcommand{\sov}{\text{sov}}

% stlc
\newcommand{\tm}{\mathrm{tm}}
\newcommand{\ty}{\mathrm{ty}}
\newcommand{\fst}{\mathrm{fst}}
\newcommand{\snd}{\mathrm{snd}}
\newcommand{\inl}{\mathrm{inl}}
\newcommand{\inr}{\mathrm{inr}}
\newcommand{\indz}{\mathbf{ind}_{\mathbf{0}}}
\newcommand{\indp}{\mathbf{ind}_{+}}
\newcommand{\indn}{\mathbf{ind}_\N}

% Type theories
\newcommand{\lc}{\lambda_{\to}}
\newcommand{\stlc}{$\lambda_{\to \times}$}
\newcommand{\stlct}{\lambda_{\to \times \N}}
\newcommand{\dtt}{\lambda\Pi_{\times}}

%% Multicol bib
\begin{comment}
\usepackage{multicol}

\makeatletter
\renewenvironment{thebibliography}[1]
     {\begin{multicols}{2}[\section*{\refname}]%
      \@mkboth{\MakeUppercase\refname}{\MakeUppercase\refname}%
      \list{\@biblabel{\@arabic\c@enumiv}}%
           {\settowidth\labelwidth{\@biblabel{#1}}%
            \leftmargin\labelwidth
            \advance\leftmargin\labelsep
            \@openbib@code
            \usecounter{enumiv}%
            \let\p@enumiv\@empty
            \renewcommand\theenumiv{\@arabic\c@enumiv}}%
      \sloppy
      \clubpenalty4000
      \@clubpenalty \clubpenalty
      \widowpenalty4000%
      \sfcode`\.\@m}
     {\def\@noitemerr
       {\@latex@warning{Empty `thebibliography' environment}}%
      \endlist\end{multicols}}
\makeatother
\end{comment}

\title{Simply Typed Lambda Calculus and the Curry-Howard Correspondence}
\author{Ali Caglayan}
\date{Bachelor of Science in Computer Science and Mathematics with Honours\\The University of Bath\\May 2019}


%\showoutput
\begin{document}

%\maketitle
%\input{title.tex}
\setcounter{page}{0}
\pagenumbering{roman}


\maketitle
\newpage


% Set this to the number of years consultation prohibition, or 0 if no limit
\consultation{0}
\newpage

\declaration{Simply Typed Lambda Calculus and the Curry-Howard Correspondence}{Ali Caglayan}
\newpage

\begin{abstract}
    The goal of this dissertation is to give an introduction to the formal study of lambda calculus and type theory. We begin by analysing the intuitive notion of \emph{syntax}, highlighting the many subtleties associated with it. We discuss possible solutions to these issues, but ultimately remark that it is very difficult to be certain of correctness. We will however give a notion of syntax which is ``correct enough'' for our purposes.

The next section is to discuss the formality of \emph{judgements}. This is a concept oft overlooked in the study of type theory. We will give a careful and detailed account of derivability and admissibility. We will also remark on inconsistencies of the treatment of certain concepts.

This will lead us into studying the \emph{simply type lambda calculus} (STLC), in some ways one of the simplest (functional) programming languages. We will give syntax, judgements and rules governing its semantics. After which, we will prove meta properties about our type theory and discuss the notion of \emph{type checking}.

We will then analyse the dynamics of the STLC. There is a long history of normalisation results we wish to briefly sketch. We will set up some machinery to prove some of these results. Finally we will discuss notions of canonicity and what these results mean for the design of programming languages.

Next there will be several examples of terms to be type checked. This will show the intricacies that go into designing a type checker. We will see that typing makes lambda calculus much weaker, in that many terms from the untyped lambda calculus cannot be typed. It is precisely these terms which gave the computational power of the untyped lambda calculus to begin with.

We will sketch some modifications to the simply typed lambda calculus that will give us certain desired features. We will show how these can be designed and discuss their normalisation results too.

Finally we will give a detailed account of the ideas that went in to, what is now known as the \emph{Curry-Howard} correspondence. This is a very deep package of ideas with far reaching consequences, of which we will try to make account of.

Our closing remarks will be about future directions in type theory, questions that need to be answered and future of programming language design.
\end{abstract}
\newpage

\tableofcontents
\newpage

\section*{Acknowledgements}
I would like to thank my advisor John Power for his guidance and support. For teaching me how to be a mathematician, how to think carefully and how to do category theory correctly. I think my parents for being very patient and supportive of me whilst writing this dissertation.

\newpage

\setcounter{page}{1}
\pagenumbering{arabic}

% Introduction and direction of thesis
%Simply typed lambda calculus (STLC) has been well documented and studied by type theorists and mathematicians, and it's features have been used by many programming languages [NEED REFERENCE].

%In \cite{BarendregtHenk2013Lcwt} it is noted that \say{Research monographs on dependent and inductive types are lacking.} This will essentially be one of the goals of this thesis, to provide a guide for mathematicians and computer scientists about the use of dependent type theory. As this document is written there is no single account of all approaches to \i{dependent} type theory.

%Awodey \cite{2014arXiv1406.3219A} made an observation that Dybjer's \cite{dybjer1996} categories with families (CwF) is a presheaf category with a representable natural transformation (it's fibers are representable). He then proceeds to show conditions needed to model a dependent type theory with $\Pi$, $\Sigma$ and $\mathrm{Id}$ types.


%This thesis will have three main goals.

%\begin{enumitem}
%	\item To present a dependent type theory
%	\item To model the semantics of such a type theory using categorical methods
%	\item To discuss the applications to mathematics and computer science (proof assistants, programming languages and foundations)
%\end{enumitem}

%Finally we may also discuss recent developments of something called "Homotopy type theory" and how that fits into the general picture.

%Roughly a \textit{type system} is a set of loosely organised rules outlining how ``atomic sentences'' called \textit{judgements} can be derived from each other in a given context. A \textit{context} can simply be thought of as a list of terms. 

%The aim of this thesis is to present to two sorts of audience, the utility of dependent type theory. The audiences that I have in mind are computer scientists, roughly individuals who wish to write good code, and mathematicians, roughly individuals who wish to write good proofs.

%These will be our main aims however we do also wish to develop the machinery formally.

%\section{Propositions as types}

%There is a rich interplay between programming and logic known as the Curry-Howard correspondance or propositions as types. 





%\section{What is type theory}

%Type theory is the study of types systems. That is a system that orginizes data manipulated by programs into types. This has been a very useful concept in computer science. It has allowed the writing of programs taht a more 

%\subsection{Lambda calculus}
%\subsection{Modelling type theory}
%\section{What is dependent type theory?}
%\subsection{What are dependent types?}
%\subsection{Motivation for computer scientists}
%\subsection{Motivation for mathematicians}
%\subsection{Category theory}
%\subsection{Categorical logic}
%\subsection{Future directions}

\begin{comment}
\section{Introduction}

The goal of this thesis is to introduce dependent types to the undergraduate reader. We set out 

The goal of this dissertation is to learn how to mathematically design a programming language. 


The aim of this thesis is to introduce the notion of dependent types to an undergraduate reader. The main idea of dependent types is very simple, yet deceptively subtle however, since modelling such a formalism is quite tricky. This is evidenced by the fact that there is a lot of disagreement in type theory what has or hasn't been proven. This however is a familiar story in mathematics and is usually remedied by trying to understand what has been done better. Usually with the help of a new perspective. 

Dependent types however, are not only of interest to mathematicians but also programmers. Dependent type theory (much like simply typed lambda calculus) is very much a programming language allowing the expression of ideas previously too difficult to express. This is very much facilitated by its deep connection to predicate logic.
\end{comment}

\section{Introduction}

Dependent types have been around for a while. [[Introduction with citation]]. The fact that they haven't been used widely in programming and mathematics suggests that their exposition is in dire need of attention. This is one of the goals this dissertation aims to achieve. We also note that for type theorists, categorical semantics can be daunting and obscure. For mathematicians, computer scientific ideas seem out of reach. 


\begin{itemize}
\item a[Begin with history and implications of curry Howard]

\item a[outline the ``what they should do'' of dependent types]

\item a[start to rigoursly model syntax and talk about how bad a job most authors do]

\item a[small section about classical inductive definitions]

\item a[small section on why categorical semantics]

\item a[model simply typed lambda calculus with categorical semantics]

\item a[show natural extensions of the idea and why contexts break when dependnet]

\item a[outline different approches to solving these problems]

\item a[discuss Awodey's natural models]

\item a[finally talk about future directions for type theory]

\item a[maybe some mention on applications to programming (generalising various constructs, polymorphism, GA data types)]

\item a[equality, inductive types, [[[[[maybe a tinsy bit of homotopy type theory]]]]]]
\end{itemize}




% Rigourous treatement and analysis of syntax
\section{Syntax}

\subsection{Introduction}

In order to design and describe programming languages we will need some formalism. We need a way of formally talking about the manipulation of symbols and variables. We also need to take into account variable binding. [CITE some stuff here]

In order to do this we define \emph{abstract binding trees} as described in \cite{harper_2016}, though we will be using an ever so slightly different definition which is outlined in \cite{


% Judgements, inference rules and general notions of logical
\section{Judgements}

We will now describe simply typed lambda calculus using our developed way of working with syntax. We will first describe judgements and how to specify a type system. Then our first example will be the simply typed lambda calculus. We use the ideas developed in \cite{harper_2016} though these ideas are much older. [Probably tracable back to Gentzen].

\begin{defin}
    The notion of a \emph{judgement} or \emph{assertion} is a logical statement about an abt. The property or relation itself is called a \emph{judgement form}. The judgement that an object or objects have that property or stand in relation is said to be an \emph{instance} of that judgement form. A judgment form has also historically been called a \emph{predicate} and its instances called \emph{subjects}.
\end{defin}

\begin{remark}
    Typically a judgement is denoted $\mathsf{J}$. We can write $a\ \mathsf{J}$, $\mathsf{J}\ a$ to denote the judgment asserting that the judgement form $\mathsf{J}$ holds for the abt $a$. For more abts this can also be written prefix, infix, etc. This will be done for readability. Typically for an unspecified judgement, that is an instance of some judgement form, we will write $J$.
\end{remark}

    $$\frac
        {}
        {}
    $$


\begin{defin}
    An \emph{inductive definition} of a judgement form consists of a collection of rules of the form
    
    $$\frac
        {J_1 \quad \cdots \quad J_k}
        {J}
    $$
    
    in which $J$ and $J_1, \dots , J_k$ are all judgements of the form being defined. THe judgements above the horizontal line are called the \emph{preimises} of the rules, and the judgement below the line is called its \emph{conclusion}. A rule with no premises is called an \emph{axiom}.
\end{defin}

\begin{remark}
    An inference rule is read as starting that the premises are \emph{sufficient} for the conclusion: to show $J$, it is enough to show each of $J_1, \dots J_k$. Axioms hold unconditionally. If the conclusion of a rule holds it is not necesserily the case that the premises held, in that the conclusion could have been derived by another rule.
\end{remark}

\begin{example}
    Consider the following judgement from $-\ \mathsf{nat}$, where $a\ \mathsf{nat}$ is read as ``$a$ is a natural number''. The following rules form an inductive definition of the judgement form $-\ \mathsf{nat}$:

    $$\frac
        {}
        {\texttt{zero}\ \mathsf{nat}}
      \qquad\qquad\qquad
      \frac
        {a\ \mathsf{nat}}
        {\texttt{succ}(a)\ \mathsf{nat}}
    $$

    We can see that an abt $a$ is zero or is of the form $\texttt{succ}(a)$. We see this by induction on the abt, the set of such abts has an operator $\texttt{succ}$. Taking these rules to be exhaustive, it follows that $\textt{succ}(a)$ is a natural number if and only if $a$ is.
\end{example}

\begin{remark}
    We used the word \emph{exhaustive} without really defining it. By this we mean necessary and sufficient. Which we will define now.
\end{remark}

\begin{defin}
    A collection of rules is considered to define the \emph{strongest} judgement form that \emph{closed under} (or \emph{respects}) those rules. To be closed under the rules means that the rules are \emph{sufficient} to show the validity of a judgement: $J$ holds if there is a way to obtain it using the given rules. To be the \emph{strongest} judgement form closed under the rules means that 
\end{defin}



% Model simply typed lambda calculus using syntax machinary
\newcommand{\tm}{\mathrm{tm}}
\newcommand{\ty}{\mathrm{ty}}
\newcommand{\fst}{\mathrm{fst}}
\newcommand{\snd}{\mathrm{snd}}

%
% Simply typed lambda calculus
%
\section{Simply typed lambda calculus} 


First develop the features needed. Discuss the arbitrary nature of such features, then use Curry-Howard as motivation for ``the language that ought to be''. Develop STLC, discuss in detail the implications, give categorical semantics. Discuss breifly the dynamics of simply typed lambda calculus. A big disadvantage of STLC over the untyped version (which we ought to discuss since we have the tools to) is that there is no recursion. There are many ways to fix this, see G\"odel for example. In order to fix this we will introduce dependent types.

We begin by discussing the syntax of our type theory. We will start by specifying the sorts $\mathcal{S}$ of our type theory.

\begin{defin}
    The sorts of simply typed lambda calculus are terms and types $\mathcal{S} := \{ \tm , \ty\}$.
\end{defin}

We now specify the operators (with generalized arities) that we defined in definition \ref{owga}. In remark \ref{opdata} we discussed the data needed to give an operator, therefore we will present all our operators in the following table.

\begin{defin}
    The operators in the syntax of simply typed lambda calculus are given by the following table:
    \begin{center}
        \begin{tabular}{ c|c|c|c|c|c|c }
        Op & Sort & Vars & Type args & Term args & Scoping & Syntax \\
        \hline
        $\to$           & \ty &  --- & $A,B$ &  ---  &  ---  & $A \to B$            \\
        $\times$        & \ty &  --- & $A,B$ &  ---  &  ---  & $A \times B$         \\
        $(-,-)$         & \tm &  --- &  ---  & $x,y$ &  ---  & $(x,y)$              \\
        $\lambda$       & \tm &  $x$ & $A,B$ &  ---  &  $M$  & $\lambda (x : A).M$  \\
        $\mathrm{App}$  & \tm &  --- & $A,B$ &  ---  & $M,N$ & $M N$
        \end{tabular}
    \end{center}
\end{defin}

\begin{remark}
    Note that some of the syntax loses information that was put in. The application is the main example of this. In practice if we know the type of $M$ and $N$ we can deduce the type of $M N$ just from the rules we will define later. The syntax is sugared or \emph{syntactic sugar} so we do not have to write so much. If done incorrectly it could be considered an abuse of notation. It should be possible to \emph{desugar} the syntax by adding an \emph{annotated} version of an operator. For example for application instead of $M N$ we could write $\mathrm{App}_{A,B}}(M;N)$. Having this information in the syntax will be useful when we want to induct over syntax, for example when proving an intiality theorem. But in practice we will save ourselves from having to write it out.
\end{remark}

\begin{defin}
    We can now construct our raw terms and types as the collection of abts (see definition \ref{abt}) over the previously defined data $\mathrm{Term} := \mathcal{B}[\varnothing]_{\tm}$ and $\mathrm{Type} := \mathcal{B}[\varnothing]_{\ty}$.
\end{defin}

\begin{remark}
    Note that we have no variables. This is because if we set the definition of abt up correctly we don't need any, but terms can have subterms (subtrees of the abt) which have variables. The sets $\mathrm{Term}$ and $\mathrm{Type}$ become \emph{all} the types and terms we ought to be able to write down from scratch.
\end{remark}

We now need to define judgements about our syntax and write down the rules to write them down. [[Make a note about substitution because afik we haven't defined it properly yet]]. 

\subsection{Judgements}


[[TODO: Clean up this whole paragraph(s)]]
We begin with our basic judgements. Of which there will be 5. Our STLC will have bidirectional typechecking, in that we will distinguish between the direction of type checking. There are several advantages of this and historically the two main systems called STLC are Curry's and Church's which simply differ in the direction of type checking. By having both directions and a sort of ``mode-switching rule'' we have far greater control and ease when describing type checking properties. We will also need to have a notion of \emph{judgemental equality} since we wish to do some computation. There are variations of this theme discussed in the statics chapter that allow us to have transition systems instead but we will use an equational style since transition systems can be derived from this. This also has the advantage of STLC becomming what is known as an ``equational theory''. This will be a useful feature for when we want to derrive categorical semantics. 

A context is a list of basic judgements. Our basic judgements are $x : A$. [[No it is not fix this]]

There are 5 judgements that we have:

\begin{itemize}
    \item $\Gamma \vdash A\ \mathsf{type}$ - ``$A$ is a type in context $\Gamma$''.
    \item $\Gamma \vdash T \Leftarrow A$ - ``$T$ can be checked to have type $A$ in context $\Gamma$''.
    \item $\Gamma \vdash T \Rightarrow A$ - ``$T$ synthesises the type $A$ in context $\Gamma$''.
    \item $\Gamma \vdash A \equiv B\ \mathsf{type}$ - ``$A$ and $B$ are jdugementally equal types in context $\Gamma$''.
    \item $\Gamma \vdash S \equiv T : A$ - ``$S$ and $T$ are judgementally equal terms of type $A$ in context $\Gamma$''.
\end{itemize}

\subsection{Structural rules}

Structural rules will dictate how our judgements interact with eachother, how different contexts can be formed and how substitution works. This is all roughly what a ``type theory'' ought to provide.

\begin{defin}
    We begin with the \emph{variable} rule, this says that if a term $x$ appears with a type $A$ as an element in a context $\Gamma$ then $x$ synthesises a type $A$ in context $\Gamma$. Or written more succiently as:

    % Variable rule
    \begin{prooftree}
        \AxiomC{$(x:A) \in \Gamma$}
        \RightLabel{(var)}
        \UnaryInfC{$\Gamma \vdash x \Rightarrow A$}
    \end{prooftree}
\end{defin}

Other structural rules: weakening, contraction and substitution are all admissible. [[What does it mean for a rule to be admissible? We have defined this previously but we need to carefully state these facts, and prove them too!]]

\end{defin}
    One of the features of bidirectional type checking is that we can switch the mode we are in. This is expressed as the mode switching rule:

    % Switch rule
    \begin{prooftree}
        \AxiomC{$\Gamma \vdash t \Rightarrow A$}
        \AxiomC{$\Gamma \vdash A \equiv B \ \mathsf{type}$}
        \RightLabel{(switch)}
        \BinaryInfC{$\Gamma \vdash t \Leftarrow B$}
    \end{prooftree}
\end{defin}

\begin{remark}
    This rule has been specially set up in that it will be the \emph{only way} to derive $\Gamma \vdash T \Leftarrow B$. These are the kinds of properties we would like our syntax to have. A careful analysis will be done under the name of \emph{inversion lemma}. [[Link to inversion lemma?]]

    In a unidirectional type system, the judgements $\Gamma \vdash T \Rightarrow A$ and $\Gamma \vdash T \Leftarrow B$ are collapsed into one: $\Gamma \vdash T : A$. And now the mode-switching rule may have a more familiar form:

    \begin{prooftree}
        \AxiomC{$\Gamma \vdash t : A$}
        \AxiomC{$\Gamma \vdash A \equiv B \ \mathsf{type}$}
        \BinaryInfC{$\Gamma \vdash t : B$}
    \end{prooftree}

    Which shows that it is actually a rule about substituting along a judgemental equality! But this is a problem since a type checking algorithm will have to decide when to stop doing this. This is one of the big advantages that bidirectional type checking has over unidirectional type checking. The type checking algorithm will be simpler! [[TODO: Clean up and discuss type checking in more detail]]
\end{remark}

\begin{remark}
    Occasionally, we will simply mode-switch using reflexivity $\Gamma \vdash A \equiv A \ \mathsf{type}$, in which case we will abbreviate the rule as follows:
    % compact switch
    \begin{prooftree}
        \AxiomC{$\Gamma \vdash t \Rightarrow A$}
        \RightLabel{(switch)}
        \UnaryInfC{$\Gamma \vdash t \Leftarrow A$}
    \end{prooftree}
\end{remark}

\subsection{Equality rules}
Finally we have some structural rules for our two judgemental equality judgements. We wish for these to be an equivalence relation and that they are compatible with eachother.

First we begin with the structural rules for the judgement form $- \equiv -\ \mathsf{type}$:

\begin{defin}

    % Reflexivity of judgemental equality of types
    We wish for our judgemental equality of types to be reflexive:
    \begin{prooftree}
        \AxiomC{\Gamma \vdash A \ \mathsf{type}}
        \RightLabel{($\equiv_{\mathsf{type}}$-reflexivity)}
        \UnaryInfC{$\Gamma \vdash A \equiv A\ \mathsf{type}$}
    \end{prooftree}

    % Symmetry of judgemental equality of types
    We want our judgemental equality of types to be symmetric:
    \begin{prooftree}
        \AxiomC{$\Gamma \vdash A \equiv B \ \mathsf{type}$}
        \RightLabel{($\equiv_{\mathsf{type}}$-symmetry)}
        \UnaryInfC{$\Gamma \vdash B \equiv A \ \mathsf{type}$}
    \end{prooftree}

    and our judgemental equality of types to be transitive:

    % Transitivity of judgemental equality of types
    \begin{prooftree}
        \AxiomC{$\Gamma \vdash B \ \mathsf{type}$}
        \AxiomC{$\Gamma \vdash A \equiv B\ \mathsf{type}$}
        \AxiomC{$\Gamma \vdash B \equiv C\ \mathsf{type}$}
        \RightLabel{($\equiv_\mathsf{type}$-transitivity)}
        \TrinaryInfC{$\Gamma \vdash A \equiv C\ \mathsf{type}$}
    \end{prooftree}

    Notice how the previous rule also checks that $B$ is a type. This is because if we did not do this, we could insert any symbol in. This is clearly undesirable. It also demonstrates how subtly sensitive rules are.

    Now we list the rules making the judgement form $- \equiv - : A$ into an equivalence relation:

    % Reflexivity of judgemental equality of terms
    We wish for our judgemental equality of terms to be reflexive:
    \begin{prooftree}
        \AxiomC{$\Gamma \vdash t \Rightarrow A$}
        \RightLabel{($\equiv_{\mathsf{term}}$-reflexivity)}
        \UnaryInfC{$\Gamma \vdash t \equiv t : A$}
    \end{prooftree}

    % Symmetry of judgemental equality of terms
    We want our judgemental equality of terms to be symmetric:
    \begin{prooftree}
        \AxiomC{$\Gamma \vdash s \equiv t : A$}
        \RightLabel{($\equiv_{\mathsf{term}}$-symmetry)}
        \UnaryInfC{$\Gamma \vdash t \equiv s : A$}
    \end{prooftree}

    % Transitivity of judgemental equality of terms
    and our judgemental equality of terms to be transitive:
    \begin{prooftree}
        \AxiomC{$\Gamma \vdash t \Leftarrow A $}
        \AxiomC{$\Gamma \vdash s \equiv t : A$}
        \AxiomC{$\Gamma \vdash t \equiv r : A$}
        \RightLabel{($\equiv_{\mathsf{term}}$-transitivity)}
        \TrinaryInfC{$\Gamma \vdash s \equiv r : A$}
    \end{prooftree}

    as we stated before for transitivity judgemental equality of types we need to also check that the middle term $T$ is actually a term.

    % judgemental equality of types - judgemental equality of terms - congruence
    Finally we need a rule that will make  that judgemental equality of types and judgemental equality of terms interact the way we expect them to:
    \begin{prooftree}
        \AxiomC{$\Gamma \vdash A \ \mathsf{type}$}
        \AxiomC{$\Gamma \vdash s \equiv t : A$}
        \AxiomC{$\Gamma \vdash A \equiv B\ \mathsf{type}$}
        \RightLabel{($\equiv_{\mathsf{term}}$-$\equiv_{\mathsf{type}}$-compat)}
        \TrinaryInfC{$\Gamma \vdash s \equiv t : B$}
    \end{prooftree}
\end{defin}

\subsection{Type formers}
What we have constructed thusfar is essentially an ``empty type theory''. What we have included which other authors typcially gloss over is a clean way of constructing a typechecking algorithm: bidirectional typechecking and an account of judgemental equality. We now study what are known as type formers, typically when we wish to add a new type to a type theory we need to think about a collection of rules. These can roughly be sorted into 5 kinds of rules:

\begin{itemize}
    \item Formation rules - How can I construct my type?
    \item Introduction rules - Which terms synthesise this type?
    \item Elimination rules - How can terms of this type be used?
    \item Computation (or equality) rules - How do terms of this type compute? (Normalise, etc.)
    \item Congruence rules - How do all the previous rules interact with judgemental equality
\end{itemize}

We make a note that although we will be providing all the rules, the congruence rules can be typically derrived from the others. Although we do not know exactly how to do this so we will provide them explicitly. We also note that not every type need computation rules.

Building on top of our ``empty type theory'' we introduce $\to$ the function type former:

\begin{defin}

    Our formation rules tell us how to construct arrow types from other types:
    
    % -> formation
    \begin{prooftree}
        \AxiomC{$\Gamma \vdash A\ \mathsf{type}$}
        \AxiomC{$\Gamma \vdash B\ \mathsf{type}$}
        \RightLabel{($\to$-form)}
        \BinaryInfC{$\Gamma \vdash A \to B \ \mathsf{type}$}
    \end{prooftree}

    Our introduction rule tells us how to construct terms of our type. This is also known as $\lambda$-abstraction:

    % -> introduction
    \begin{prooftree}
        \AxiomC{$\Gamma , x : A\vdash M \Leftarrow B$}
        \RightLabel{($\to$-intro)}
        \UnaryInfC{$\Gamma \vdash \lambda x . M \Rightarrow A \to B$}
    \end{prooftree}

    Our elimination rule tells us how to use terms of this type. For function types this corresponds to application:

    % -> elimination
    \begin{prooftree}
        \AxiomC{$\Gamma \vdash M \Leftarrow A \to B$}
        \AxiomC{$\Gamma \vdash N \Leftarrow A$}
        \RightLabel{($\to$-elim)}
        \BinaryInfC{$\Gamma \vdash M N \Rightarrow B$}
    \end{prooftree}

    And finally we have computation rules which tell us how to compute our terms. We will later prove results about normalisation of the lambda calculus. We start with $\beta$-reduction which tells us how applicated functions compute:

    % -> beta
    \begin{prooftree}
        \AxiomC{$\Gamma , x : A \vdash y \Leftarrow B$}
        \AxiomC{$\Gamma \vdash t \Leftarrow A$}
        \RightLabel{($\to$-$\beta$)}
        \BinaryInfC{$\Gamma \vdash (\lambda x . y) t \equiv y[t / x] : B$}
    \end{prooftree}

    Then we introduce $\eta$-conversion which tells us if two functions applied to the same term and are judgementally equal then the functions are judgementally equal. This is ``function extensionality'' for judgemental equality.

    % -> eta
    \begin{prooftree}
        \AxiomC{$\Gamma , y : A \vdash M y \equiv M' y : B$}
        \RightLabel{($\to$-$\eta$)}
        \UnaryInfC{$\Gamma \vdash M \equiv M' : A \to B$}
    \end{prooftree}

    Finally we have to make sure all our rules respect judgemental equality. This means showing that $\to$ respects judgemental equality of types and that $\lambda$-terms and applications respect judgemental equality of terms.

    % -> formation congruence
    \begin{prooftree}
        \AxiomC{$\Gamma \vdash A \equiv A' \ \mathsf{type}$}
        \AxiomC{$\Gamma \vdash B \equiv B' \ \mathsf{type}$}
        \RightLabel{($\to$-$\equiv_{\mathsf{type}}$-cong)}
        \BinaryInfC{$\Gamma \vdash A \to B \equiv A' \to B' \ \mathsf{type}$}
    \end{prooftree}

    % -> introduction congruence
    \begin{prooftree}
        \AxiomC{$\Gamma , x : A \vdash M \equiv M' : B$}
        \RightLabel{($\to$-$\equiv_{\mathsf{term}}$-cong)}
        \UnaryInfC{$\Gamma \vdash \lambda x . M \equiv \lambda x . M' : A \to B$}
    \end{prooftree}

    % -> elimination congruence
    \begin{prooftree}
        \AxiomC{$\Gamma \vdash M \equiv M' : A \to B$}
        \AxiomC{$\Gamma \vdash N \equiv N' : A$}
        \RightLabel{($\to$-elim-cong)}
        \BinaryInfC{$\Gamma \vdash M N \equiv M' N' : A \to B$}
    \end{prooftree}

\end{defin}

\begin{remark}
    Notice that we don't ensure that types compute the same way. This is because the computation rules will not be used in the type checking process and are therefore irrelevant to the inversion lemmas. Later we will prove that ``fully reduced'' computations are in fact equal. This is known as the Church-Rosser theorem.
\end{remark}

We define the product type as follows.

\begin{defin}[Product type]
    
    Given two types, we have their product type:
    
    % Product formation
    \begin{prooftree}
        \AxiomC{$\Gamma \vdash A \ \mathsf{type}$}
        \AxiomC{$\Gamma \vdash B \ \mathsf{type}$}
        \RightLabel{($\times$-form)}
        \BinaryInfC{$\Gamma \vdash A \times B \ \mathsf{type}$}
    \end{prooftree}
    
    We define ordered pairs as taking a term of each type:
    
    % Product introduction
    \begin{prooftree}
        \AxiomC{$\Gamma \vdash a \Leftarrow A$}
        \AxiomC{$\Gamma \vdash b \Leftarrow B$}
        \RightLabel{($\times$-intro)}
        \BinaryInfC{$\Gamma \vdash (a, b) \Rightarrow A \times B$}
    \end{prooftree}
    
    We give two eliminators for pairs, the first and second elements:
    
    % Product eliminators
    \begin{center}
        \AxiomC{$\Gamma \vdash t \Leftarrow A \times B$}
        \RightLabel{($\times$-elim${}_1$)}
        \UnaryInfC{$\Gamma \vdash \fst(t) \Rightarrow A$}        
        \DisplayProof
        \hskip 1.5em
        \AxiomC{$\Gamma \vdash t \Leftarrow A \times B$}
        \RightLabel{($\times$-elim${}_2$)}
        \UnaryInfC{$\Gamma \vdash \snd(t) \Rightarrow B$}
        \DisplayProof
    \end{center}
    
    And we finally need to dictate how this is computed:
    
    \begin{center}
        \AxiomC{$\Gamma \vdash x \Leftarrow A$}
        \AxiomC{$\Gamma \vdash y \Leftarrow B$}
        \RightLabel{($\times$-$\beta_1$)}
        \BinaryInfC{$\Gamma \vdash \fst(x,y)\equiv x : A$}
        \DisplayProof
        \hskip 1.5em
        \AxiomC{$\Gamma \vdash x \Leftarrow A$}
        \AxiomC{$\Gamma \vdash y \Leftarrow B$}
        \RightLabel{($\times$-$\beta_2$)}
        \BinaryInfC{$\Gamma \vdash \snd(x,y)\equiv y : B$}
        \DisplayProof
    \end{center}
    
    However we need to be careful since there is a nontrivial equality we must also add as a rule:
    
    \begin{prooftree}
        \AxiomC{$\Gamma \vdash t \Rightarrow A \times B$}
        \RightLabel{($\times$-$\eta$)}
        \UnaryInfC{$\Gamma \vdash (\fst(t),\snd(t))\equiv t : A \times B$}
    \end{prooftree}
    
\end{defin}

We will also need to add a unit type. This will be the simplest type, with only one term.

\begin{defin}[Unit type]
    We begin with the formation rules, essentially saying that the unit type exists.

    % Unit formation
    \begin{prooftree}
        \AxiomC{}
        \RightLabel{($\mathbf{1}$-form)}
        \UnaryInfC{$\mathbf{1}\ \mathsf{type}$}
    \end{prooftree}

    We then say that the unit type has a term:

    % Unit introduction
    \begin{prooftree}
        \AxiomC{}
        \RightLabel{($\mathbf{1}$-intro)}
        \UnaryInfC{$\Gamma \vdash * \Rightarrow \mathbf{1}$}
    \end{prooftree}
\end{defin}

\begin{remark}
    We don't need to give any more rules since the unit type has all the properties we need. Our rules for $\to$ allow us to build constant functions anyway. And we note that all functions $\mathbf{1} \to A$ are constant functions!
\end{remark}

[[TODO: Clear up wording maybe?]]
\begin{remark}
    We make an important note that this is not the simplest presentation of the STLC of which there are many variations thereof. We chose judgemental equality and bidirectional type checking because these are features we will need if we are to enrich our type system with dependent types.
\end{remark}

\subsection{Inversion lemmas}
Having listed all these rules we need some lemmas detailing how different terms can \emph{only} come from a set of specified rules. This is a crucial analysis if we wish to construct a type checking algorithm. An inversion lemma for a type theory is typically very difficult to state, and extremely tedious to prove. But nontheless is essential if we want to induct over terms.

Luckily we set up syntax in such a way that we only need induct over the syntax. So we pick a syntactic form and the inversion lemma will tell us exactly how we can arrive at that conclusion. Let us list all term syntax we can create in STLC:

\begin{itemize}
    \item $x$ where $x$ is a variable.
    \item $\lambda x . M$ where $M$ is a term.
    \item $(x, y)$ where $x$ and $y$ are terms.
    \item $\fst, \snd$ the eliminators of $\times$
    \item $*$ the element of $\mathbf{1}$
    \item $\mathrm{ind}_{\mathbf{1}}$ the inductor of $\mathbf{1}$
\end{itemize}

%Probably not true if we set things up correctly
%We note that these inversion lemmas will only be applicable to judgement forms such as $\Gamma \vdash x \Rightarrow A$ and $\Gamma \vdash x \Leftrightarrow A$ as judgemental equality will be far too complicated (and perhaps even impossible) to decidably derive. In times where our judgemental equality 

\begin{lemma}
    
\end{lemma}

[[TODO: State this beast]]
\begin{lemma}
    In the STLC the following term forms are generated by certain rules...
\end{lemma}

\subsection{Normalisation and Canonicity}

[[These two concepts are very related, we should find some way to talk about it, including Church-Rosser]]


\begin{comment}
%\subsection{Lambda calculus}
%We recall that there are 3 kinds of expressions in lambda calculus: variables, abstractions and applications. These are defined inductively on themselves. A variable is simply a string of characters from an alphabet. A lambda abstraction looks like $\lambda x.y$ where $x$ is some variable and $y$ is some expression. There are alternate ways of writing this, allowing us to drop the need for naming $x$, for example de Brujin indices. Finally an application is simply the concatenation $ab$ of two expressions $a$ and $b$. We will assume that  This fully describes the syntax of this type theory. We will now introduce some rules that tell us which expressions we can derive from other expressions. Firstly we have $\beta$-reduction which tells us if we have an expression of the form $(\lambda x . y)z$ this can be reduced to an expression where all occurrences of $x$ in $y$ are replaced with the expression $z$. We also have $\alpha$-conversion which I would argue isn't really a rule as naming of variables can be completely avoided in the first place using de Brujin indices or even combinators. \cite{BarendregtHenk2013Lcwt, hottbook}

%\subsection{Contexts}
%In mathematics we work with contexts implicitly. That is there is always an ambient knowledge of what has been defined. Mostly due to the nature of how we read mathematical papers. We can make this explicit using contexts. We will not however, use contexts in our discussion of type theory but we will provide a formal exposition in the appendix.

\subsection{Judgements}
Our judgements:
\begin{center}
    \begin{tabular}{c | c}
        $\Gamma\ \mathrm{ctx}$ &  $\Gamma$ is a well-formed context. \\
        $\Gamma \vdash A\ \mathrm{Type}$ & $A$ is a type in context $\Gamma$. \\
        $\Gamma \vdash x : A$ & $x$ is a term of type $A$ in context $\Gamma$. \\
%        $\Gamma \vdash x \equiv y : A$ & the terms $x$ and $y$ of type $A$ are definitionally equal in context $\Gamma$
    \end{tabular}
\end{center}


Type theory ``will be about'' deriving judgements from other judgements. Which can be concisely summarised in the form of an inference rule

$$\frac{A_1\quad A_2 \quad \cdots \quad A_n}{B}$$

which says that given the judgements $A_1,\dots,A_n$ we can derive the judgement $B$.

\subsection{Structural rules}
We now look at the rules that govern contexts and the structure of our type system.

We begin with a rule stating that the empty context (which as contexts are sets or lists is well-defined) is well-formed. Which is another way of stating that the context was grown in a specified way and is not just an arbitrary list or set of variables.

\begin{prooftree}
    \AxiomC{}
    \RightLabel{empty-ctx}
    \UnaryInfC{$\varnothing$ ctx}
    \singleLine
\end{prooftree}

We also want the concatenation of two well-formed contexts to be well-formed.

\begin{prooftree}
    \AxiomC{$\Gamma$ ctx}
    \AxiomC{$\Delta$ ctx}
    \BinaryInfC{$\Gamma,\Delta$ ctx}
\end{prooftree}

We omit rules about repeating or removing repeated elements and ordering lists (think of them as finite sets).

A variable is a statement of the form $x : A$ where $x$ is known as the term and $A$ its type.

\subsection{Function types}

We introduce a formation rule for the function type.

\begin{prooftree}
    \RightLabel{$(\to)$-form}
    \AxiomC{$\Gamma \vdash A\ \mathrm{Type}$}
    \AxiomC{$\Gamma \vdash B\ \mathrm{Type}$}
    \BinaryInfC{$\Gamma \vdash A \to B\ \mathrm{Type}$}
\end{prooftree}

We now need a rule for producing terms of this new type. We introduce the introduction rule for the function type.

\begin{prooftree}
    \RightLabel{$(\to)$-intro}
    \AxiomC{$\Gamma, x : A \vdash y : B$}
    \UnaryInfC{$\Gamma \vdash (\lambda x . y) : A \to B$}
\end{prooftree}

We will sometimes call this lambda abstraction. We next introduce a way to apply these functions to terms in their domains. We introduce our elimination rule for the function type.

\begin{prooftree}
    \RightLabel{$(\to)$-elim}
    \AxiomC{$\Gamma \vdash f : A \to B$}
    \AxiomC{$\Gamma \vdash a : A$}
    \BinaryInfC{$\Gamma \vdash f(a) : B$}
\end{prooftree}

This is essentially useless unless we have a way to compute (or reduce) this expression. This is where our computation rule comes in. The computation rule will tell us how our elimination rule and introduction rule interact.
\begin{prooftree}
    \RightLabel{$(\to)$-comp}
    \AxiomC{$(\lambda x . y) : A \to B$}
    \AxiomC{$\Gamma, a : A \vdash (\lambda x.y)a : B$}
    \AxiomC{$\Gamma, x : A, y : B, (\lambda x . y) : A \to B, a : A \vdash (\lambda x . y) (a) : B$}
    \UnaryInfC{$\Gamma \vdash y[x / a] : B$}
\end{prooftree}

%%%%%%%%%%%%%%%%%%%%

We will describe what is known as a simply typed lambda calculus. There is a lot of literature on type theory, and it doesn't seem that there are many authors in agreement of ways to present it.

In \cite{BarendregtHenk2013Lcwt} a more type theoretic approach, analysing the type theory mostly in the syntactic world. This gives us a good starting point for how we want our type theory to be presented however it may not be so easy to keep an eye on how the categorical semantics (the ways we model types in mathematics) behave. In order to do this we will use references such as \cite{CroleRoyL1993Cft, JacobsCLTT, LambekJ1986Itho}. This will be from the more categorical logic school of thought, which will study type theory that is "generated" by certain categories in interest.

We start by describing a general class of simple type theories as outlined in \cite{JacobsCLTT}. Firstly we introduce the notion of a {\it signature}. Similar accounts can be found in \cite{CroleRoyL1993Cft}. This will essentially consist of "generating" a category from some signature (which can be thought of as a stripped down type theory syntax), and then studying the functors from that category into other categories. This allows nice properties from the second category to be "pulled back" onto our type theory giving it features we desire.

\begin{defin}
	A {\bf signature} is a pair $(\Typ, \mathcal{F})$ where $\Typ$ is a finite set of {\bf basic} (or {\bf atomic}) {\bf types}. And a functor $\mathcal{F} : \Typ^\star \times \Typ \to \Set$. Where $\Typ^\star$ is the Kleene-Star operation on a set (or the free monoid over $\Typ$), defined as $X^\star := \bigcup_{n\in \N} X^n$ whose elements are finite tuples of elements of $X$ for a set $X$. We have $\mathbf{Set}$ for the category of finite sets. Note that the sets in the domain of the functor are realised as discrete categories.
\end{defin}

We will usually write a signature as $\Sigma := (\Typ, \mathcal{F})$, denote $|\Sigma|:=\Typ$ and write $F: \sigma_1,\dots,\sigma_n\to\sigma_{n+1}$ when $F \in \mathcal{F}(( \sigma_1,\dots,\sigma_n ), \sigma_{n+1})$.

\begin{defin}
    Let $\Var$ be a countable set. Elements $x\in \Var$ are called {\bf variables}.
\end{defin}

Note this style of variables is essentially de Brujin indices. But allows us to have a set of names for our variables, which allows future annoyances like $\alpha$-equivalence to be sorted out easily due to the plentiful existence of bijections from $\Var \to \Var$.

\begin{defin}
	A {\bf variable declaration} is a pair $(x, \sigma) \in \Var \times \Typ$ usually written as $x : \sigma$. This can be read as "the variable $x$ has type $\sigma$. We will define $\Dec:=\Var \times \Typ$.
\end{defin}

\begin{defin}
    A {\bf context} $\Gamma$ is an element of $\Con:=\Dec^\star$. In other words, a context is a finite list of variable declarations. We will usually write a context $\Gamma$ as $v_1 : \sigma_1, \dots ,v_n : \sigma_n$. Note that the Kleene-Star has a monoid structure with operation $","$. We can thus give $\Con$ a monoid structure and write, for contexts $\Gamma$ and $\Delta$ another context $\Gamma,\Delta$ which is the concatenation of two contexts. The notation here allows the "expanded version" to coincide, as in $\Gamma,\Delta$ can be written as $v_1 : \sigma_1, \dots ,v_n : \sigma_n, w_1 : \tau_1, \dots, w_m, \tau_m$.
\end{defin}

We also note that there is a canonical inclusion $\Dec \hookrightarrow \Con$ given that $\Dec$ freely generates the monoid $\Con$. This will allow us to write $\Gamma, x:\tau$ for $v_1 : \sigma_1, \dots ,v_n : \sigma_n, x:\tau$.

We now denote the basic statements of our language. These statements are called {\bf judgements} and we will derive

%%%%%%%%%%%%%%%%%%%%
\end{comment}









% Normalisation of STLC
\section{Normalisation of STLC}

%%
%%  Introduction
%%

\subsection{Introduction}
We now wish to analyse the computational power of our type theory.
When designing the type checking algorithm we made a point not to invoke any computational rules, since we want to be able to spot a correct program without running it. This is an issue later on where we see \emph{untyped} terms may not \emph{normalise}.

Computational rules, are our basic steps of computation. In the late 1930s, Alan Turing had a model of what it meant to be \emph{effectively calculable}, now known as a Turing Machine \cite{turing1936a}. Critically he included an appendix which outlined how Church's untyped lambda calculus was equivalent to his notion of effective calculability. We will discuss the historical nature of this further in chapter \ref{logic_chapter}.

One important question in computer science is knowing whether or not a computation will halt, known as the \emph{Halting problem}. It can in fact be shown that Turing machines, and by extension untyped lambda calculus, cannot decide their own halting problem. This stems from the fact that these gadgets are ``\emph{too good}'' at computation. It is closely related to G\"odel's analysis of the power of arithmetic. \cite{}

The main goal of this chapter will be to show that the STLC, defined in the previous chapter, has computations that \emph{always} terminate. This is known as \emph{strong normalisation}. This immediately highlights some of the weaknesses of the STLC, however we will later discuss what is missing and how this can be fixed. The technical term for this is that STLC is \emph{not} Turing-complete. \cite{} We will also see that there are \emph{untyped} lambda terms that cannot be \emph{typed}.

It can be observed that there is not a unique way to compute something, i.e. run the program. What is important is that the result is unique. Such a property is known as the Church-Rosser property. We will show that Church-Rosser property holds for $\beta$-, $\eta$- and $\beta\eta$-reductions. 

Finally we will discuss the issue of canonicity and modifications to the STLC, and what they mean with respect to these computational properties.

Guiding references for this chapter are \cite{Sorensen} and \cite{BarendregtHenk2013Lcwt}.

\subsection{Well-founded relations}

The notion of well-founded induction is a standard theorem of set theory. The classical proof of which usually uses the law of excluded middle \cite[p. 62]{johnstone1987notes}, \cite[Ch. 7]{barwise1982handbook}. It's use in the formal semantics of programming languages is not much different either \cite[Ch. 3]{winskel1993formal}. There are however more constructive notions of well-foundedness \cite[\S 8]{2018arXiv180805204S} with more careful use of excluded middle. We will follow \cite{10.2307/2275781}, as this is the simplest to understand, and we won't be using this material much other than an initial justification for induction in classical mathematics.

\begin{defin}
    Let $X$ be a set and $\prec$ a binary relation on $X$. A subset $Y \subseteq X$ is called \textbf{$\prec$-inductive} if
    $$
        \forall x \in X, \quad (\forall y \prec x,\ y \in Y) \Rightarrow x \in Y.
    $$
\end{defin}

\begin{defin}\label{wf}
    The relation $\prec$ is \textbf{well-founded} if the only $\prec$-inductive subset of $X$ is $X$ itself. A set $X$ equipped with a well-founded relation is called a \textit{well-founded set}.
\end{defin}

\begin{theorem}[Well-founded induction]
    Let $X$ be a well-founded set and $P$ a property of the elements of $X$ (a proposition). Then
    $$
        \forall x \in X, P(x) \quad \iff \quad  \forall x \in X,\ \ (\forall y \prec x, P(y)) \Rightarrow P(x).
    $$
\end{theorem}

\begin{proof}
    The forward direction is clearly true. For the converse, assume $\forall x \in X,((\forall y \prec x, P(y)) \Rightarrow P(x))$. Note that $P(y) \Leftrightarrow x \in Y := \{ x \in X \mid P(x)\} $ which means our assumption is equivalent to $\forall x \in X,\ (\forall y \prec x,\ y \in Y) \Rightarrow x \in Y$ which means $Y$ is $\prec$-inductive by definition. Hence by \ref{wf} $Y=X$ giving us $ \forall x \in X, P(x)$.
\end{proof}

% Compatible relation
First we define what we mean by a binary relation being \emph{compatible} with the syntax of the STLC.
\begin{defin}
    A binary relation $\succ$ on $\mathrm{Term}$ the set of all terms, is said to be \emph{compatible with the syntax of STLC} (or just simply \emph{compatible}) if the following conditions hold:
    \begin{enumerate}
        \item If $M \succ N$ then $\lambda x . M \succ \lambda x . N$.
        \item If $M \succ N$ then $M Z \succ N Z$.
        \item If $M \succ N$ then $Z M \succ Z N$.
        \item If $M \succ N$ then $(Z,M) \succ (Z,N)$.
        \item If $M \succ N$ then $(M, Z) \succ (N, Z)$.
    \end{enumerate}
\end{defin}

\begin{remark}
    The notion of compatibility allows us to make sure a relation also considers sub-terms. This is a tricky thing to get right but due to our focus on the correct structure of syntax we are fine.
\end{remark}

\begin{remark}
[[CLEAN THIS UP]]
    The reader may ask what relations have to do with normalisation, but it is a formalism that we have chosen. This is definitely not the only way to prove properties like Church-Rosser. The main reason we have chosen this method is for its simplicity. In fact earlier we discussed the dynamics of languages, this is exactly that. There are many ways to go about dynamics including transition systems and equational dynamics. Our approach corresponds to the more classical and simple transition systems approach. It can be shown that this is equivalent to equational dynamics in that a reduction step will be justified by application of rules from STLC.
\end{remark}

We will demonstrate our last remark by considering the following relation:

% same type relation
\begin{defin}
    Let $\sim_{\ty}$ denote the relation among terms of \emph{having the same type}. Suppose $\Gamma \vdash s \Leftarrow S$ and $\Gamma \vdash t \Leftarrow T$, then:
    $$
        s \sim_{\ty} t \iff \Gamma \vdash S \equiv T \ \mathsf{type}
    $$
\end{defin}

% having the same type is a compatible relation
\begin{lemma}
    The relation $\sim_{\ty}$ is a compatible relation.
\end{lemma}

\begin{proof}
    Suppose $M \sim_{\ty} N$, then we have $\Gamma \vdash M \Leftarrow S$, $\Gamma \vdash N \Leftarrow T$ and $\Gamma \vdash S \equiv T \ \mathsf{type}$.
    \begin{enumerate}
        \item [[TODO FINISH]]
    \end{enumerate}
\end{proof}

% Transitive and reflexive closure
\begin{defin}
    Given a relation $\succ$ on a set $X$, we denote by $\succ^+$ the \emph{transitive closure} of $\succ$. This is the smallest relation which coincides with $\succ$ and is transitive. We also consider the \emph{reflexive-transitive closure} $\succ^*$ of $\succ$ which is simply the relation $\Delta(X)\cup \succ^+ $ where $\Delta(X)$ is the image of the diagonal function $x \mapsto (x,x)$. (We've simply added that $x \succ^* x$)
\end{defin}

\begin{remark}
    Transitive closures correspond to chains of the relation, and reflexive-transitive closures allow for chains of length $0$. It should also be noted that we took the \emph{union} of a relation. This is a well-defined notion and can easily be seen to be a relation.
\end{remark}

Let $\to$ be a binary relation on a set $A$, $\twoheadrightarrow^+$ be its transitive closure and $\twoheadrightarrow$ be its reflexive-transitive closure.

\subsection{Normalising relations}

Now we define (very generally) what it means for an element of a set to be in \emph{normal form} and \emph{normalising} with respect to some relation.

% normal form with respect to a relation
\begin{defin}
    An element $a \in A$ is said to be of \emph{normal form} if $\forall b \in A$, $a {\not \to} b$.
\end{defin}

% weak normalisation with respect to a relation
\begin{defin}
    An element $a \in A$ is said to be \emph{normalising} (or \emph{weakly normalising}) if there is a reduction sequence $a \to a_1 \to a_2 \to \cdots \to a_n$ where $a_n$ is in normal form, for some $n$. We call $a_n$ a \emph{normal form} or \emph{reduct} of $a$.
\end{defin}

% 
\begin{remark}
    Note that not every reduction sequence is guaranteed to be finite. We also note that if $\to$ a relation is Church-Rosser (to be defined below) then $a_n$ is \emph{the} normal form or reduct.
\end{remark}

We discuss what it means for a relation to be Church-Rosser:

% Church-Rosser
\begin{defin}
    A relation $\to$ has the \emph{Church-Rosser} (CR) property if and only if for all $a,b,c \in A$ such that $a \twoheadrightarrow b$ and $a \twoheadrightarrow c$, there exists $d \in A$ with $b \twoheadrightarrow d$ and $c \twoheadrightarrow d$.
\end{defin}

\begin{remark}
    This says no matter what path we take along a relation, there will always be elements at which the paths cross.
\end{remark}

We will also need a slightly weaker version called weak Church-Rosser, for reasons we will see later:

% weak Church-Rosser
\begin{defin}
    A relation $\to$ has the \emph{weak Church-Rosser} (WCR) property if and only if for all $a, b, c \in A$ such that $a \to b$ and $a \to c$, there exists $d \in A$ with $b \twoheadrightarrow d$ and $c \twoheadrightarrow d$.
\end{defin}

We now state the obvious:

\begin{cor}\label{cr_is_wcr}
    If $\to$ is CR then $\to$ is WCR.
\end{cor}

\begin{proof}
    Observe that WCR is a special case of CR.
\end{proof}

The converse to this is in general \emph{false} but it is true when another condition holds, namely that $\to$ is \emph{strongly normalising}.

\begin{defin}
    A binary relation $\to$ is \emph{strongly normalising} (SN) if and only if there is no infinite sequence $a_0 \to a_1 \to a_2 \to  \cdots$.
\end{defin}

\begin{remark}
    In other words, a relation $\to$ is strongly normalising if and only if \emph{every} sequence $a_0 \to a_1 \to a_2 \to  \cdots$ terminates after a finite number of steps.
\end{remark}

\begin{remark}
    We typically also say an element is strongly normalising if the condition holds for that element. This allows us to state SN in a different (and perhaps more correct) way: A relation $\to$ is strongly normalising if each element is strongly normalising with respect to $\to$. Then we can define an element to be strongly normalising if all of it's reducts are strongly normalising. The nice thing about this definition is that we have seen it before, this is precisely what it means to be a \emph{well-founded relation} from Definition \ref{wf}. So $\to$ is strongly normalising if and only if it is well-founded. This is good because we can induct over it!
\end{remark}

\begin{cor}
    If a relation $\to$ is strongly normalising then every element is normalising.
\end{cor}

\begin{proof}
    By induction on $\to$ we see that either an element is in normal form, or it reduces to normal form. This is precisely what it means to be normalising.
\end{proof}

\subsection{Newman's lemma}

We now state a lemma which will be very useful. It is a sufficient condition for the converse of Corollary \ref{cr_is_wcr} to hold.

\begin{lemma}[Newman's Lemma]\label{newman}
    If $\to$ is strongly normalising and WCR then it is CR.
\end{lemma}

\begin{proof}
    Since $\to$ is strongly normalising, any $a \in A$ has a normal form. Call an element \emph{ambiguous} if $a$ reduces to two distinct normal forms. Clearly $\to$ is CR if there are no ambiguous elements of $A$.
    Assume, for contradiction, that there is an ambiguous $a$. We will show that there is another ambiguous $a'$ where $a \to a'$.
    Suppose we have $a \twoheadrightarrow b_1$ and $a \twoheadrightarrow b_2$ where $b_1$ and $b_2$ are two different normal forms. Both reductions must make at least one step, thus both reductions can be written as $a \to a_1 \twoheadrightarrow b_1$ and $a \to a_2 \twoheadrightarrow b_2$.
    Suppose $a_1 = a_2$ then we can choose $a' = a_1 = a_2$. Now suppose $a_1 \neq a_2$, we know by WCR that $a_1 \twoheadrightarrow b_3$ and $a_2 \twoheadrightarrow b_3$ for some $b_3$. We can assume that $b_3$ is a normal form. Since $b_1$ and $b_2$ are distinct, $b_3$ is different from $b_1$ or $b_2$ so we can choose $a' = a_1$ or $a'=a_2$.
    Since we can always choose an $a'$, we can repeat this process and get an infinite chain of ambiguous elements. It is clear that this contradicts strongly normalising, hence $A$ has no ambiguous elements.
\end{proof}


[[TODO add note about the knowledge of Newman's lemma, should be footnote in Sorensen]]


%%
%%  Beta reduction
%%

\subsection{ \texorpdfstring{$\beta$}{}-reduction}

Now we define what we mean by $\beta$-reduction and $\beta$-normal form.

% Define beta-reduction
\begin{defin}\label{beta_reduction}
    We define \emph{$\beta$-reduction} to be the least compatible relation $\to_{\beta}$ on $\mathrm{Term}$ satisfying the following conditions:
    \begin{enumerate}
        \item $(\lambda x . y)t \to_{\beta} y [t / x]$
        \item $\fst(x,y) \to_{\beta} x$
        \item $\snd(x,y) \to_{\beta} y$
    \end{enumerate}
    A term on the left hand side of any of the above is called a \emph{$\beta$-redex} (reducible expression) and the right hand sides are said to \emph{arise by contracting the redex}.
\end{defin}

\begin{remark}
    Observe that these are very similar to our $\beta$ rules, in fact they are exactly those. So the question may arise: why haven't we defined $\beta$-reduction using the rules that we already have? The answer is that we could but we would have a much harder time, the rules also take into account typing information but we are explicitly not worried about that since we will show later $\beta$-reduction doesn't change a typed terms type. It is somewhat simpler and clearer to focus purely on terms. We will later justify calling this $\beta$-reduction.
    Such dynamics falls under what is known as \emph{equational dynamics}. This would require us to have a suitable way of dealing with judgemental equality, which we feel would obstruct the inner workings of the result.[[DROP REFERENCE TO EARLIER SECTION]]
\end{remark}

% Define beta normal form
\begin{defin}
    A term $M$ is said to be in \emph{$\beta$-normal form} if it is in normal form with respect to $\to_\beta$.
\end{defin}

\begin{remark}
    That is to say a term is in $\beta$-normal form if there is no $\beta$-reduction to any other term. Or better yet, $M$ does not contain a $\beta$-redex.
\end{remark}

% Define multi-step beta reductions
\begin{defin}
    Let $\twoheadrightarrow_{\beta}$ be the transitive, reflexive closure of $\to_{\beta}$ called a \emph{multi-step $\beta$-reduction}.
\end{defin}

% Non normalising terms
\begin{remark}\label{beta_non_normalising_remark}
    Not every term is normalising. Take for example the term $\Omega=(\lambda x . x x)(\lambda x . x x)$ which cannot be typed as we will see later. There is an infinite reduction sequence:
    $$
        \Omega \to_{\beta} \Omega \to_{\beta} \Omega \to_{\beta} \Omega \to_{\beta} \cdots
    $$
    Since $\Omega$ cannot be given a type, it is deemed \emph{ill-typed}.
\end{remark}

This means we have to be careful which terms we are talking about. When talking about terms of the STLC we should add that we expect them to be well-typed (derivable). We will see later there are many syntactically valid terms that are ill-typed.

We want to now prove that every derivable term is $\beta$-normalising. In order to do this we need to keep track of available redexes and bound them. We will then show there is a reduction strategy that decreases this bound yielding our result.

This proof is usually attributed to an unpublished note of Turing but it has been rediscovered by various authors. We will follow the proof in Girard's book \cite{Girard1989}.

% Degree of a type
\begin{defin}
    The \emph{degree $\partial(T)$ of a type $T$} is defined by:
    \begin{itemize}
        \item $\partial(T) := 1$ if $T$ is atomic.
        \item $\partial(U \times V), \partial(U \to V) := \max(\partial(U), \partial(V))+1$.
    \end{itemize}
\end{defin}

% Degree of a redex
\begin{defin}
    The \emph{($\beta$-)degree $\partial_{\beta}(t)$ of a redex} is defined by:
    \begin{itemize}
        \item $\partial_{\beta}(\fst(u,v)), \partial_{\beta}(\snd(u,v)) := \partial(U\times V)$ where $\Gamma \vdash (u, v) \Leftarrow U \times V$.
        \item $\partial_{\beta}((\lambda x . v) u) := \partial(U \to V)$ where $\Gamma \vdash \lambda x . v \Leftarrow U \to V$.
    \end{itemize}
\end{defin}

% Degree of a term
\begin{defin}
    The \emph{($\beta$-)degree $d_{\beta}(t)$ of a term} is the maximum of the degrees of its redexes:
    $$
        d_{\beta}(t) := \max \{\partial_{\beta} (s) \mid s \text{ is a redex in } t\}
    $$
\end{defin}

\begin{remark}
    A redex is associated to two degrees, one as a redex and another as a term. Since a redex $r$ may contain other redexes we have that $\partial (r) \le d(r)$. It should be noted we have defined degree to mean 3 different things here, but as long as we are careful we should not get confused.
\end{remark}

% partial T < partial r
\begin{lemma}\label{beta_redex_ineq}
    If $r$ is a redex of type $T$ then $\partial(T) < \partial_{\beta}(r)$. 
\end{lemma}

\begin{proof}
    Checking the cases for $r$:
    \begin{itemize}
        \item $\partial (T) < \partial_{\beta}(\fst(t, u)) = \max(\partial(T), \partial(U)) + 1$.
        \item $\partial (T) < \partial_{\beta}(\snd(u, t)) = \max(\partial(U), \partial(T)) + 1$.
        \item $\partial (T) < \partial_{\beta}((\lambda x . t)u) = \max(\partial(U), \partial(T)) + 1$.
    \end{itemize}
\end{proof}

% substitution inequality
\begin{lemma}\label{beta_sub_ineq}
    If $\Gamma , x : T \vdash t \Leftarrow U$ then $d_{\beta}(t[u/x]) \leq \max(d_{\beta}(t), d_{\beta}(u), \partial(T))$.
\end{lemma}

\begin{proof}
    Analysing the redexes of $t[u/x]$ we find that they fall into the following cases:
    \begin{itemize}
        \item They are redexes of $t$ (in which $u$ has become $x$).
        \item They are redexes of $u$, proliferating due to each occurrence of $x$ in $t$.
        \item They are formed when $t$ is of the form $\fst(x)$, $\snd(x)$, or $x v$ for $u$ of the form $(u', u'')$, $(u', u'')$, or $\lambda y . u'$ respectively. These new redexes have degree $\partial(T)$.
    \end{itemize}
\end{proof}

% reduction inequality
\begin{lemma}\label{beta_reduct_ineq}
    If $t \to_{\beta} u$ then $d_{\beta}(u) \le d_{\beta}(t)$.    
\end{lemma}

\begin{proof}
    Consider the reduction where $u$ is obtained from $t$ by replacing the redex $r$ in $u$ by $c$. Now we consider all the redexes of $u$ where we find:
    \begin{itemize}
        \item redexes which were originally in $t$, but not in $r$, and have been modified by the replacement of $r$ by $c$. Observe that their degree does not change.
        \item redexes which were originally in $c$. But $c$ is obtained by reducing $r$, or in other words a substitution in $r$. Notice $(\lambda x . s)s'$ becomes $s[s'/x]$ and Lemma \ref{beta_sub_ineq} tells us that $d_{\beta}(c) \le \max(d_{\beta}(s), d_{\beta}(s'), \partial(T))$, where $T$ is the type of $x$. But by Lemma \ref{beta_redex_ineq} we have $\partial (T) \le \partial (r)$. Applying $\max$ gives us $\max(d(s), d(s'), \partial(T)) \le \max(d_{\beta}(s), d_{\beta}(s'), \partial_{\beta}(r))$ and hence $d_{\beta}(c) \le \max(d_{\beta}(s), d_{\beta}(s'), \partial(r))=d(r)$.
        \item redexes which come from replacing $r$ by $c$. These redexes have degree equal to $\partial(T)$ where $T$ is the type of $r$. By Lemma \ref{beta_redex_ineq} we have $\partial(T) \le \partial (r)$.
    \end{itemize}
\end{proof}

Next we will prove a lemma bounding the number of redexes of a certain degree.

% number of redexes inequality
\begin{lemma}\label{beta_redex_number_ineq}
    Let $r$ be a redex of maximal degree $n$ in $t$, and suppose that all redexes strictly contained in $r$ have degree less than $n$. If $u$ is obtained from $t$ by reducing $r$ to $c$. Then $u$ has strictly fewer redexes of degree $n$.
\end{lemma}

\begin{proof}
    When the reduction happens we make the following observations:
    \begin{itemize}
        \item The redexes outside $r$ in $t$ remain $u$.
        \item The redexes strictly inside $r$ are in general conserved but sometimes become more prolific. Take for example $(\lambda x . (x, x)) s \to_{\beta} (s, s)$. The number of redexes in the reduct are double that of redex on the left. However the degree of the proliferated redexes must be strictly less than $n$.
        \item The redex $r$ is destroyed and possibly replaced by redexes of strictly smaller degree.
    \end{itemize}
\end{proof}

\begin{remark}
    Although not defined, we take the meaning of \emph{a redex strictly inside} to be a redex that is not the whole redex.
\end{remark}

We now have all the machinery needed to prove that typed terms in the STLC are strongly $\beta$-normalising.

% Every term is beta normalising
\begin{theorem}\label{beta_SN}
    Every derivable term $\Gamma \vdash t \Leftarrow A$ in the STLC is strongly $\beta$-normalising.
\end{theorem}

\begin{proof}
    Consider the function $\mu : \Term \to \N \times \N$ which takes $t \mapsto (n, m)$ where $n = d_{\beta}(t)$ and $m$ is the number of redexes in $t$ of degree $n$. By Lemma \ref{beta_redex_number_ineq} it is possible to choose a redex $r$ of $t$ in such a way that, after reduction of $r$ to $c$, the reduct $t'$ satisfies $\mu(t') < \mu(t)$. Thus by double induction on $n$ and $m$ it is possible to see that $\mu(t)$ can always be decreased until $t$ is normal.
\end{proof}

\begin{remark}
    The ordering in $\mu(t') < \mu(t)$ on $\N \times \N $ is the lexicographic ordering. Meaning $(n', m') < (n, m)$ if and only if $n' < n$ or $n'=n$ and $m' < m$. (Think Alphabetical order).
\end{remark}

\begin{remark}
    Since we have decreasing sequences of natural numbers we must have a finite number of reductions in \emph{any} reduction sequence. Of course we have weakly normalising too.
\end{remark}

% Coherence lemma
\begin{lemma}\label{beta_coh}
    Suppose $\Gamma \vdash M \Leftarrow T$ and $M \twoheadrightarrow_{\beta} N$, then $\Gamma \vdash M \equiv N : T$.
\end{lemma}

\begin{proof}
    We will only sketch the proof here. It will require inducting over syntax and the definition of $\twoheadrightarrow_\beta$. The main part to notice is that as $\to_\beta$ is a compatible relation, we can destruct the syntax down and isolate a redex. Then using ($\to$-$\beta$), ($\times$-$\beta_1$) and ($\times$-$\beta_2$) we can combine their results with congruence rules to build the term back up. It will be a technically finicky proof.
\end{proof}

\begin{remark}
    Although we haven't checked Lemma \ref{beta_coh}, it would be extremely surprising if it were false. So we have a strong feeling that it ought to be true. 
\end{remark}

We now wish to prove that $\to_\beta$ is weakly Church-Rosser. First we take some results from Takahashi \cite{TakahashiM1989PRIL, Takahashi:1995:PR9:207177.207191}, who considers \emph{parallel reductions}. This is a stronger relation than $\to_{\beta}$ and weaker than $\twoheadrightarrow_{\beta}$. This might not seem like much but, parallel ($\beta$-)reduction, denoted $\Rightarrow_{\beta}$ (not to be confused with our typing judgements), satisfies the diamond in Church-Rosser. And since $\twoheadrightarrow_{\beta}$ is the transitive closure of $\Rightarrow_{\beta}$, it too satisfies the diamond in Church-Rosser hence $\to_{\beta}$ has the Church-Rosser property. We will formalise this argument as follows and consider $\beta$, $\eta$ and $\beta \eta$ reductions along the way.

% Parallel beta reduction
\begin{defin}
    \emph{Parallel $\beta$-reduction}, $\Rightarrow_{\beta}$, is defined inductively on terms by the following rules:
    \begin{enumerate}
        \item $x \Rightarrow_{\beta} x$ for a variable or constant $x$.
        \item $\lambda x . M \Rightarrow_{\beta} \lambda x . M'$ if $M \Rightarrow_{\beta} M'$.
        \item $MN \Rightarrow_{\beta} M' N'$ if $M \Rightarrow_{\beta} M'$ and $N \Rightarrow_{\beta} N'$.
        \item $(M, N) \Rightarrow_{\beta} (M', N')$ if $M \Rightarrow_{\beta} M'$ and $N \Rightarrow_{\beta} N'$.
        \item $(\lambda x . M)N \Rightarrow_{\beta} M'[N' / x]$ if $M \Rightarrow_{\beta} M'$ and $N \Rightarrow_{\beta} N'$.
        \item $\fst(M, N) \Rightarrow_{\beta} M'$ if $M \Rightarrow_{\beta} M'$.
        \item $\snd(M, N) \Rightarrow_{\beta} N'$ if $N \Rightarrow_{\beta} N'$.
    \end{enumerate}
\end{defin}

\begin{remark}
    If we expand the definition of compatible in the definition of $\to_{\beta}$ it may appear to be identical to the definition of $\Rightarrow_{\beta}$. The key difference is the direction in which we are building up the terms. In the above definition we are breaking down the syntax and making sure that \emph{all} components also satisfy the relation. We will see later the relation with $\to_{\beta}$.
\end{remark}

\begin{remark}
    The name comes from the fact that parallel reduction can reduce many $\beta$-redexes at once, unlike usual reduction.
\end{remark}

% => beta is reflexive
\begin{cor}\label{beta_par_refl}
    The relation $\Rightarrow_\beta$ is reflexive.
\end{cor}

\begin{proof}
    Observe that ignoring the last three rules in the definition of $\Rightarrow_\beta$ we still cover all the syntax.
\end{proof}

The following lemma shows the strengths of our notions of $\beta$-reduction. It will be very useful later on.

% M -> M'  ==>  M => M'  ==>  M ->> M'
\begin{lemma}\label{beta_par_imp}
    We have the following implications:
    $$
        M \to_\beta M' \quad \implies \quad M \Rightarrow_\beta M' \quad \implies \quad M \twoheadrightarrow_\beta M'
    $$
\end{lemma}

\begin{proof}
    For the first implication, observe that a redex in $M$ is being contracted to such that $M \to_\beta M'$. We can also contract the redex in the definition of $M \Rightarrow_\beta M'$ by choosing the correct rule. For the second implication, proceed by induction on $M$:
    \begin{itemize}
        \item If $M = x \Rightarrow_\beta M'$ then clearly $M'=x$ hence $x \twoheadrightarrow x$.
        \item If $M = \lambda x . N$ then $M' = \lambda x . N'$ where $N \Rightarrow_\beta N'$, which by the induction hypothesis gives us $N \twoheadrightarrow_\beta N'$, and since $\twoheadrightarrow_\beta$ is the transitive, reflexive closure of a compatible relation, we have $M \twoheadrightarrow_\beta M'$.
        \item If $M = (a, b)$, then by induction hypothesis and $\twoheadrightarrow_\beta$ being compatible we have $M \twoheadrightarrow_\beta M'$.
        \item Finally for the case that $M$ is a $\beta$-redex, observe that $\twoheadrightarrow_\beta$ can reduce this redex, and by the induction hypothesis and subredexes of that.
    \end{itemize}
\end{proof}

\begin{remark}
    The previous proof may be considered as a sketch since we didn't explicitly check every case. This can definitely be done but it is not so interesting.
\end{remark}

We can now discuss \emph{the complete ($\beta$-)development of a term}. This is a way of completely reducing down \emph{all} $\beta$-redexes at once.

\begin{defin}
    The \emph{complete ($\beta$-)development} of a term $\Gamma \vdash t \Leftarrow A$, written $t^*$ is defined by induction on syntax:
    \begin{itemize}
        \item For a variable or constant $x^* = x$.
        \item $(\lambda x . M)^* = \lambda x . M^*$.
        \item $(M N)^* = M^* N^*$ if $MN$ is not a $\beta$-redex.
        \item $(a, b)^* = (a^*, b^*)$.
        \item $((\lambda x . M)N)^* = M^* [N^* / x]$.
        \item $(\fst (a, b))^* = a^*$.
        \item $(\snd (a, b))^* = b^*$.
    \end{itemize}
\end{defin}

\begin{lemma}
    Given a term $\Gamma \vdash t \Leftarrow A$, $t^*$ is in $\beta$-normal form.
\end{lemma}

\begin{proof}
    Observe by induction that the complete development rids a term of \emph{all} $\beta$-redexes.
\end{proof}


Here is a technical lemma that is the driving force behind our proof of being Church-Rosser. It says that the complete development is always the most reduced form of a term.

\begin{lemma}\label{cd_lemma_beta}
    Suppose $M \Rightarrow_\beta N$ then $N \Rightarrow_\beta M^*$.
\end{lemma}

\begin{proof}
    We proceed by induction on $M \Rightarrow_\beta N$:
    \begin{itemize}
        \item Suppose $M = x$ then $M = x \Rightarrow_\beta x = N$. Hence $N = x \Rightarrow_\beta x^* = M^*$.
        \item Suppose $M = \lambda x . t \Rightarrow_\beta \lambda x . t'$. Then $t \Rightarrow_\beta t'$ and $t' \Rightarrow_\beta t^*$ by the induction hypothesis, hence $N = \lambda x . t' \Rightarrow_\beta \lambda x . t^* = (\lambda x . t)^* = M^*$.
        \item Suppose $M = a b$ and $M$ is not a $\beta$-redex, then $M = a b \Rightarrow_\beta a ' b ' = N$, with $a \Rightarrow_\beta a'$ and $b \Rightarrow_\beta b'$. Then by the induction hypotheses, we have $a' \Rightarrow_\beta a^*$ and $b' \Rightarrow_\beta b^*$, yielding $a' b' \Rightarrow_\beta a^* b^* = (a b)^*=M^*$, since $M$ is not a $\beta$-redex.
        \item Suppose $M = (a, b) \Rightarrow_\beta (a' , b')$ where $a \Rightarrow_\beta a'$ and $b \Rightarrow_\beta b'$. Then by the induction hypotheses we have $a' \Rightarrow_\beta a^*$ and $b' \Rightarrow_\beta b^*$. Hence $N = (a', b') \Rightarrow_\beta (a^*, b^*) = (a, b)^* = M^*$.
        \item Suppose $M = (\lambda x . y) t \Rightarrow_\beta y'[t'/x]$ with $y \Rightarrow_\beta y'$ and $t \Rightarrow_\beta t'$. By our induction hypotheses: $y' \Rightarrow_\beta y^*$ and $t' \Rightarrow_\beta t^*$. It can be shown by induction on the syntax of $y'$ that $y'[t'/x] \Rightarrow_\beta y^*[t^* / x]$ however the proof would get to long. Assuming this we have $y' [t'/x] \Rightarrow_\beta y^*[t^* / x] = (y[t/x])^*$, again by induction on $y$ and $t$, and again we omit since it would lengthen the proof substantially, finally giving us $N \Rightarrow_\beta M^*$.
        \item Finally suppose $M = \fst(a, b) \Rightarrow_\beta a'$ where $a \Rightarrow_\beta a'$, by our induction hypothesis, $a' \Rightarrow_\beta a^*$ so $N = a' \Rightarrow_\beta a^* = (\fst(a, b))^* = M^*$.
        \item The case for $\snd$ is very similar to the case for $\fst$.
    \end{itemize}
\end{proof}

Now we show that $\Rightarrow_\beta$ satisfies a diamond property.

\begin{cor}\label{diamond_par_beta}
    Given $M \Rightarrow_\beta N_1$ and $M \Rightarrow_\beta N_2$ then $N_1 \Rightarrow_\beta M'$ and $N_2 \Rightarrow_\beta M'$ for some $M'$.
\end{cor}

\begin{proof}
    By Lemma \ref{cd_lemma_beta} we observe that $M' = M^*$ gives us the desired result.
\end{proof}

We now have all we need to prove our desired result.

\begin{lemma}\label{beta_WCR}
    $\beta$-reduction is weakly Church-Rosser.
\end{lemma}

\begin{proof}
    Given $a \to_\beta b$ and $a \to_\beta b'$ we see that by Lemma \ref{beta_par_imp}, we have $a \Rightarrow_\beta b$ and $a \Rightarrow_\beta b'$. Hence by the diamond property of $\Rightarrow_\beta$ (Corollary \ref{diamond_par_beta}), we have $b \Rightarrow_\beta a'$ and $b' \Rightarrow_\beta a'$ for some $a'$. Which by Lemma \ref{beta_par_imp} again, gives us $b \twoheadrightarrow_\beta a'$ and $b' \twoheadrightarrow_\beta a'$.
\end{proof}

\begin{theorem}
    $\beta$-reduction is Church-Rosser (on well-typed terms).
\end{theorem}

\begin{proof}
    $\beta$-reduction is strongly normalising by Lemma \ref{beta_SN} and weakly Church-Rosser by Lemma \ref{beta_WCR}. Hence by Newman's Lemma (\ref{newman}) we have that $\beta$-reduction is Church-Rosser. 
\end{proof}

\begin{remark}
    In \cite{Takahashi:1995:PR9:207177.207191, barendregt1984lambda} it is stated that the diamond property of $\Rightarrow_\beta$ (Lemma \ref{diamond_par_beta}) directly implies that $\to_\beta$ is Church-Rosser. We could not understand how this implication has come about, so we instead use Newman's lemma.
\end{remark}


%%
%%  Eta reduction
%%

\subsection{ \texorpdfstring{$\eta$}{}-reduction}

The proof of Church-Rosser for $\eta$-reduction will be simpler than that of $\beta$-reduction. This is because reduced $\eta$-redexes have comparably tame behaviour and don't product any new redexes.

% Define eta reduction
\begin{defin}
    We define \emph{$\eta$-reduction} to be the least compatible relation $\to_{\eta}$ on $\mathrm{Term}$ satisfying the following conditions:
    \begin{enumerate}
        \item $\lambda x . f x \to_{\eta} f$
        \item $(\fst(t), \snd(t)) \to_{\eta} t$
    \end{enumerate}
    Just like for $\beta$-reduction we have the notions of \emph{$\eta$-redex} and terms that \emph{arise by contracting the redex}.
\end{defin}

% Define eta normal form
\begin{defin}
    A term is said to be in \emph{$\eta$-normal form} if it is in normal form with respect to $\to_{\eta}$.
\end{defin}

% Define multi-step eta reductions
\begin{defin}
    Let $\twoheadrightarrow_{\eta}$ be the transitive, reflexive closure of $\to_{\eta}$ called a \emph{multi-step $\eta$-reduction}.
\end{defin}

We will now show that $\to_\eta$ is strongly normalising.

\begin{remark}
    Originally we had thought to modify the proof of $\beta$-normal\-isa\-tion, and make it work for $\eta$. However, this is where the difference between the two is key.
    $\beta$-normal\-isa\-tion has the power to create new $\beta$-redexes whereas $\eta$-normalisation never does. In fact $\eta$-normalisation is strongly normalising even in the untyped lambda calculus. This suggests that talking about degrees is not the correct approach and there ought to be some other metric for which can be used to bound $\eta$-reducible terms. Based off of work in \cite{Fortune1983}, the authors of \cite[Ex. 3.21]{Sorensen} define a \emph{depth} function for terms. We believe this to be the actual depth of the underlying tree of the abstract binding tree of the syntax of the term. But that is not a relevant result for now.
\end{remark}

\begin{defin}
    Given a term $t$ we define the \emph{depth $\mathsf{\delta(t)}$ of $t$} by induction on terms:
    \begin{itemize}
        \item $\delta (x):=0$ for $x$ a variable or constant.
        \item $\delta (a b) := 1+ \max(\delta(a), \delta(b))$.
        \item $\delta (\lambda x . y):= 1 + \delta(y)$.
        \item $\delta ((a, b)) := 1 + \max(\delta(a), \delta(b))$.  
    \end{itemize}
\end{defin}

% eta bounds
\begin{lemma}\label{eta_red_bound}
    If $t \to_{\eta} u$ then $\delta(u) < \delta(t)$.
\end{lemma}

\begin{proof}
    Observe that since $\to_{\eta}$ is a compatible relation, we need only prove the statement for a redex. We do this by cases:
    \begin{itemize}
        \item
        $$
            \begin{aligned}
                \delta ((\fst(s), \snd(s))) &= 1+ \max(\delta(\fst(s)), \delta(\snd(s))) \\
                &= 1 + \max(1+ \delta(s), 1+ \delta(s)) \\
                &= \delta(s)+ 2
            \end{aligned}
        $$
        \item
        $$
            \begin{aligned}
                \delta (\lambda x . s x) &= 1 + \delta(s x) \\
                &= 2 + \max(\delta(s), \delta(x)) \\
                &= \delta(s) + 2
            \end{aligned}
        $$
    \end{itemize}
    Observe that in both cases we have that the depth of a redex $s$ is $\delta(s) = \delta(r) + 2$ where $r$ is the reduct of $s$. However at the level of terms we cannot guarantee equality due to the nature of depth and compatibility. 
\end{proof}

\begin{lemma}\label{eta_SN}
    $\eta$-reduction is strongly normalising.
\end{lemma}

\begin{proof}
    By Lemma \ref{eta_red_bound} we have that the depth of any $\eta$-reduction sequence is strictly decreasing. Hence there may only be finitely many steps in any given $\eta$-reduction sequence.
\end{proof}

% Coherence lemma
\begin{lemma}\label{eta_coh}
    Suppose $\Gamma \vdash M \Leftarrow T$ and $M \twoheadrightarrow_{\eta} N$, then $\Gamma \vdash M \equiv N : T$.
\end{lemma}

\begin{proof}
    The proof ought to be similar to the sketch outlined in Lemma \ref{beta_coh}.
\end{proof}

Now we define a notion of \emph{parallel}

% parallel eta reductions
\begin{defin}\label{eta_par}
    \emph{Parallel $\eta$-reduction}, $\Rightarrow_\eta$, is defined inductively on terms by the following rules:
    \begin{enumerate}
        \item $x \Rightarrow_\eta x$ for a variable or constant $x$.
        \item $\lambda x . M \Rightarrow_\eta \lambda x . M'$ if $M \Rightarrow_\eta M'$.
        \item $M N \Rightarrow_\eta M' N'$ if $M \Rightarrow_\eta M'$ and $N \Rightarrow_\eta N'$.
        \item $(M, N) \Rightarrow_\eta (M', N')$ if $M \Rightarrow_\eta M'$ and $N \Rightarrow_\eta N'$.
        \item $\lambda x . M x \Rightarrow_\eta M'$ if $M \Rightarrow_\eta M'$ and $x \not\in \mathrm{FV}(M)$.
        \item $(\fst(t), \snd(t)) \Rightarrow_\eta t'$ if $t \Rightarrow_\eta t'$.
    \end{enumerate}
\end{defin}

\begin{remark}
    Notice the condition on $\lambda x . M x$ we have to check that $M$ is specifically not ``in scope'' of $x$. Since $M$ on it's own does not make sense otherwise.
\end{remark}

\begin{cor}
    The relation $\Rightarrow_\eta$ is reflexive.
\end{cor}

\begin{proof}
    Observe that ignoring the last two rules, still qualifies any term $t$ to satisfy $t \Rightarrow_\eta t$.
\end{proof}

Now we show the strength of $\Rightarrow_\eta$ relative to the other two $\eta$-reduction relations.

\begin{lemma}\label{eta_par_imp}
    We have the following implications:
    $$
        M \to_\eta M' \quad \implies \quad M \Rightarrow_\eta M' \quad \implies \quad M \twoheadrightarrow_\eta M'
    $$
\end{lemma}

\begin{proof}
    This first implication is trivial. The second implication can be done by induction on $M$ and the definition of $M \Rightarrow_\eta M'$.
\end{proof}

Now we define a way of completely reducing all $\eta$-redexes of a term.

\begin{defin}
    The \emph{complete ($\eta$-)development} of a term $t$, written $t^*$ is defined by induction on syntax:
    \begin{itemize}
        \item For a variable or constant $x^* = x$.
        \item $(\lambda x . M)^* = \lambda x . M^*$ if $\lambda x . M$ is not an $\eta$-redex.
        \item $(M N)^* = M^* N^*$.
        \item $(a, b)^* = (a^*, b^*)$ if $(a, b)$ is not an $\eta$-redex.
        \item $(\lambda x . M x)^* = M^*$ if $x \not\in \mathrm{FV}(M)$.
        \item $(\fst(t), \snd(t))^* = t^*$.
    \end{itemize}
\end{defin}

\begin{remark}
    We have overloaded the notation $t^*$ but since this section is only concerned with $\eta$-reduction this is fine.
\end{remark}

\begin{remark}
    Notice we have not mentioned any typing information about $t$. $\eta$-reduction is quite strong even without the presence of types.
\end{remark}

\begin{lemma}\label{cd_eta_normal}
    Given a term $t$, $t^*$ is in $\eta$-normal form.
\end{lemma}

\begin{proof}
    Observe by induction that the complete development rids a term of \emph{all} $\eta$-redexes.
\end{proof}

Now for the technical lemma that will give us Church-Rosser.

\begin{lemma}\label{cd_lemma_eta}
    Suppose $M \Rightarrow_\eta N$ then $N \Rightarrow_\eta M^*$.
\end{lemma}

\begin{proof}
    Begin by induction on $M \Rightarrow_\eta N$:
    \begin{itemize}
        \item If $M = x$, then $x \Rightarrow_\eta N$ so $N  = x \Rightarrow_\eta x = M^*$.
        \item If $M = \lambda x . t$ and $M$ is not an $\eta$-redex, then $M - \lambda x . t \Rightarrow_\eta \lambda x . t' = N$. By definition $t \Rightarrow_\eta t'$, and by the induction hypothesis $t' \Rightarrow_\eta t^*$. Hence $N = \lambda x . t' \Rightarrow_\eta \lambda x . t^* = (\lambda x. t)^*$ since $(\lambda x . t)$ is not an $\eta$-redex.
        \item If $M = a b \Rightarrow_\eta a' b' = N$. By definition $a \Rightarrow_\eta a'$ and $b \Rightarrow_\eta b'$. By the induction hypotheses $a' \Rightarrow_\eta a^*$ and $b' \Rightarrow_\eta b^*$. Hence $N = a'b' \Rightarrow_\eta a^* b^* = (a b)^* = M^*$.
        \item If $M = (a, b)$ and $(a, b)$ is not an $\eta$-redex, then $M = (a, b) \Rightarrow_\eta (a', b') = N$. By definition $a \Rightarrow_\eta a'$ and $b \Rightarrow_\eta b'$. By the induction hypotheses $a' \Rightarrow_\eta a^*$ and $b' \Rightarrow_\eta b^*$. Hence $N = (a',b') \Rightarrow_\eta (a^*, b^*) = (a, b)^* = M^*$ since $(a, b)$ is not an $\eta$-redex.
        \item If $M = \lambda x . t x$ for $x \not\in \mathrm{FV}(t)$, then $M = \lambda x . t x \Rightarrow_\eta N$. There are two cases for $N$:
        \begin{itemize}
            \item If $N = t'$ for some $t \Rightarrow_\eta t'$ then $M^* = (t')^*$ and hence $N = t' \Rightarrow_\eta t^* = M^*$.
            \item If $N = \lambda x . t' x$ for some $t x\Rightarrow_\eta t' x$ then our induction hypothesis gives $t' x \Rightarrow_\eta (t x)^* = t^* x$. Hence $t' \Rightarrow_\eta t^*$. Thus by definition we have $\lambda x . t' x \Rightarrow_\eta t^*$ since $t' \Rightarrow_\eta t^*$. The induction here is a bit tricky and would probably benefit with some clearer notation. 
        \end{itemize}
        \item Finally if $M =(\fst(t), \snd(t)) \Rightarrow_\eta N$ then we have two cases for $N$:
        \begin{itemize}
            \item If $N = t'$ for some $t \Rightarrow_\eta t'$, then by the induction hypothesis $t' \Rightarrow_\eta t^* = (\fst(t), \snd(t))^* = M^*$.
            \item If $N = ((\fst(t'), \snd(t'))$ for some $t \Rightarrow_\eta t'$, then by the induction hypothesis $t' \Rightarrow_\eta t^*$.
            Hence by definition we have $((\fst(t'), \snd(t')) \Rightarrow_\eta t^*$ since $t' \Rightarrow_\eta t^*$.
        \end{itemize}
    \end{itemize}
\end{proof}


Now we show that $\Rightarrow_\eta$ satisfies a diamond property.

\begin{cor}\label{diamond_par_eta}
    Given $M \Rightarrow_\eta N_1$ and $M \Rightarrow_\eta N_2$ then $N_1 \Rightarrow_\eta M'$ and $N_2 \Rightarrow_\eta M'$ for some $M'$.
\end{cor}

\begin{proof}
    By Lemma \ref{cd_lemma_eta} we observe that $M' = M^*$ gives us the desired result.
\end{proof}

We now have all we need to prove our desired result.

\begin{lemma}\label{eta_WCR}
    $\eta$-reduction is weakly Church-Rosser.
\end{lemma}

\begin{proof}
    Given $a \to_\eta b$ and $a \to_\eta b'$ we see that by Lemma \ref{eta_par_imp}, we have $a \Rightarrow_\eta b$ and $a \Rightarrow_\eta b'$. Hence by the diamond property of $\Rightarrow_\eta$ (Corollary \ref{diamond_par_eta}), we have $b \Rightarrow_\eta a'$ and $b' \Rightarrow_\eta a'$ for some $a'$. Which by Lemma \ref{eta_par_imp} again, gives us $b \twoheadrightarrow_\eta a'$ and $b' \twoheadrightarrow_\eta a'$.
\end{proof}

\begin{theorem}
    $\eta$-reduction is Church-Rosser.
\end{theorem}

\begin{proof}
    $\eta$-reduction is strongly normalising by Lemma \ref{eta_SN} and weakly Church-Rosser by Lemma \ref{eta_WCR}. Hence by Newman's Lemma (\ref{newman}) we have that $\eta$-reduction is Church-Rosser. 
\end{proof}

\begin{remark}
    Notice yet again how we at no point used the typing of terms for $\eta$.
\end{remark}

\begin{remark}
    $\eta$-reduction is typically seen as an easy case and it is not so common to see proofs written out explicitly for it. Lemma \ref{cd_lemma_eta} for example is not such an easy proof to write or read. This is due to the iterated uses of induction. One way to make this easier to check is to use a proof assistant. This however would take some time to set up properly, and may risk diverging our attention.
\end{remark}


%%
%%  Beta eta reduction
%%

\subsection{\texorpdfstring{$\beta \eta$}-reduction}

We now begin the intricate business of mixing the two reductions, keeping note of how they interact, and finally showing that, used together, they satisfy Church-Rosser.

\begin{defin}
    We define \emph{$\beta\eta$-reduction}, $\to_{\beta\eta}$ to be the union of the relations $\to_\beta$ and $\to_\eta$.
\end{defin}

\begin{remark}
    Observe that $\to_{\beta\eta}$ is also a compatible relation.
\end{remark}

\begin{defin}
    We define $\twoheadrightarrow_{\beta\eta}$ as the transitive, reflexive closure of $\to_{\beta\eta}$.
\end{defin}

\begin{lemma}\label{beta_SN_iff_beta_eta_SN}
    A term $\Gamma \vdash t \Leftarrow A$ has a $\beta$-normal form if and only if it has a $\beta \eta$-normal form.
\end{lemma}

\begin{proof}
    A similar proof can be found in \cite[Corollary 15.1.5]{barendregt1984lambda}, this would of course have to be modified to accommodate for product types. The idea of the proof is to show that $M \twoheadrightarrow_{\beta\eta} N$ implies $M \twoheadrightarrow_{\beta} P \twoheadrightarrow_{\eta} N$ for some $P$. Then since $\eta$-reduction is strongly normalising, it must be the case that $\beta$-reduction is strongly normalising if and only if $\beta \eta$-reduction is.
\end{proof}

\begin{cor}\label{beta_eta_SN}
    $\beta \eta$-reduction is strongly normalising on typed terms.
\end{cor}

\begin{proof}
    By Lemma \ref{beta_SN} and Lemma \ref{beta_SN_iff_beta_eta_SN}.
\end{proof}

Next we introduce parallel $\beta \eta$-reduction.

\begin{defin}
    \emph{Parallel $\beta\eta$-reduction}, $\Rightarrow_{\beta\eta}$, is defined inductively on terms by the following rules:
    \begin{enumerate}
        \item $x \Rightarrow_{\beta\eta} x$ for a variable or constant $x$.
        \item $\lambda x . M \Rightarrow_{\beta\eta} \lambda x . M'$ if $M \Rightarrow_{\beta\eta} M'$.
        \item $M N \Rightarrow_{\beta\eta} M' N'$ if $M \Rightarrow_{\beta\eta} M'$ and $N \Rightarrow_{\beta\eta} N'$.
        \item $(M, N) \Rightarrow_{\beta\eta} (M', N')$ if $M \Rightarrow_{\beta\eta} M'$ and $N \Rightarrow_{\beta\eta} N'$.
        \item $(\lambda x . M) N \Rightarrow_{\beta\eta} M '[N'/x]$ if $M \Rightarrow_{\beta\eta} M'$ and $N \Rightarrow_{\beta\eta} N'$.
        \item $\fst(M, N) \Rightarrow_{\beta\eta} M'$ if $M \Rightarrow_{\beta\eta} M'$.
        \item $\snd(M, N) \Rightarrow_{\beta\eta} N'$ if $N \Rightarrow_{\beta\eta} N'$.
        \item $\lambda x . M x \Rightarrow_{\beta\eta}M'$ if $M \Rightarrow_{\beta\eta} M'$ and $x \not \in \mathrm{FV}(M)$.
        \item $(\fst (t), \snd(t)) \Rightarrow_{\beta\eta} t'$ if $t \Rightarrow_{\beta\eta} t'$.
    \end{enumerate}
\end{defin}

\begin{cor}\label{beta_eta_par_refl}
    $\Rightarrow_{\beta\eta}$ is reflexive.
\end{cor}

\begin{proof}
    Observe that any term can be put through the definition of $\Rightarrow_{\beta\eta}$ even by ignoring the last five cases.
\end{proof}

Next we give the technical lemma that will let us prove Church-Rosser.

\begin{lemma}\label{beta_eta_par_imp}
    We have the following implications:
    $$
        M \to_{\beta\eta} M' \quad \implies \quad M \Rightarrow_{\beta\eta} M' \quad \implies \quad M \twoheadrightarrow_{\beta\eta} M'
    $$
\end{lemma}

\begin{proof}
    This first implication is trivial. The second implication can be done by induction on $M$ and the definition of $M \Rightarrow_{\beta\eta} M'$.
\end{proof}

Now we need a way of fully $\beta \eta$-reducing a term.

\begin{defin}
    The \emph{complete ($\beta\eta$-)development} of a term $t$, written $t^*$ is defined by induction on syntax:
    \begin{itemize}
        \item For a variable or constant $x^*= x$.
        \item $(\lambda x . M)^* = \lambda x . M^*$ if $\lambda x . M$ is not an $\eta$-redex.
        \item $(M N)^* = M^* N^*$ if $MN$ is not a $\beta$-redex.
        \item $(a, b)^* = (a^*, b^*)$ if $(a, b)$ is not an $\eta$-redex.
        \item $((\lambda x . M) N)^* = M^* [N^* / x]$.
        \item $(\fst(a, b))^* = a^*$.
        \item $(\snd(a, b))^* = b^*$.
        \item $(\lambda x . M x)^* = M^*$ if $x \not \in \mathrm{FV}(M)$.
        \item $(\fst(t), \snd(t))^* = t^*$.
    \end{itemize}
\end{defin}

\begin{remark}
    We have yet again overridden the notation $t^*$. As before we keep the notion contained within the section to avoid any confusion.
\end{remark}

\begin{lemma}
    Given a term $\Gamma \vdash t \Leftarrow A$, $t^*$ is in $\beta \eta$-normal form.
\end{lemma}

\begin{proof}
    Observe that any $\beta \eta$-redexes will be reduced. This proof can be done by induction on syntax. The induction may have to go a bit deeper than just over term forms since we need to single out the cases for $\beta \eta$-redexes.
\end{proof}

Now we can prove our technical lemma that will give us Church-Rosser for $\beta\eta$.

\begin{lemma}\label{cd_lemma_beta_eta}
    Suppose $M \Rightarrow_{\beta\eta} N$, then $N \Rightarrow_{\beta\eta} M^*$.    
\end{lemma}

\begin{proof}
    We begin by induction on $M \Rightarrow_{\beta\eta} N$:
    \begin{itemize}
        \item If $M = x \Rightarrow_{\beta\eta} x = N$, then $N = x \Rightarrow_{\beta\eta} x = x^* = M^*$.
        \item If $M = \lambda x . t$ is not an $\eta$-redex, then $M = \lambda x . t \Rightarrow_{\beta\eta} \lambda x . t'$ where $t \Rightarrow_{\beta\eta} t'$, then by the induction hypothesis, $t' \Rightarrow_{\beta\eta} t^*$, hence $\lambda x . t' \Rightarrow_{\beta\eta} \lambda x . t^*=(\lambda x . t)* = M^*$ since $t$ is not an $\eta$-redex.
        \item If $M = a b$ is not a $\beta$-redex, then $M = a b \Rightarrow_{\beta\eta} a' b'$ where $a \Rightarrow_{\beta\eta} a'$ and $b \Rightarrow_{\beta\eta} b'$. By our induction hypotheses we have $a' \Rightarrow_{\beta\eta} a^*$ and $b' \Rightarrow_{\beta\eta} b^*$, hence $a' b' \Rightarrow_{\beta\eta} a^* b^* = (ab)^* = M^*$ since $ab$ is not a $\beta$-redex.
        \item If $M = (a, b)$ is not an $\eta$-redex, then $M = (a, b) \Rightarrow_{\beta\eta} (a', b')$ where $a \Rightarrow_{\beta\eta} a'$ and $b \Rightarrow_{\beta\eta} b'$. By our induction hypotheses we have $a' \Rightarrow_{\beta\eta} a^*$ and $b' \Rightarrow_{\beta\eta} b^*$, hence $(a', b') \Rightarrow_{\beta\eta} (a^*, b^*) = (a, b)^* = M^*$ since $(a, b)$ is not an $\eta$-redex.
        \item If $M = (\lambda x . y)t \Rightarrow_{\beta\eta} N$, by induction on $N$:
        \begin{itemize}
            \item If $N = y'[t'/x]$ where $y \Rightarrow_{\beta\eta} y'$ and $t \Rightarrow_{\beta\eta} t'$. By our induction hypotheses, we have $y' \Rightarrow_{\beta\eta} y^*$ and $t' \Rightarrow_{\beta\eta}t^*$. By induction on $y'$ and $t'$ it can be shown that $N = y'[t'/x] \Rightarrow_{\beta\eta} y^*[t^*/x]=((\lambda x . y)t)^* = M^*$.
            \item If $N = (\lambda x . y')t'$ where $y \Rightarrow_{\beta\eta} y'$ and $y \Rightarrow_{\beta\eta} t'$. By the our induction hypotheses, we have $y' \Rightarrow_{\beta\eta} y^*$ and $t' \Rightarrow_{\beta\eta} t^*$. Hence $N = (\lambda x . y')t' \Rightarrow_{\beta\eta} y^*[t^*/x] = ((\lambda x . M)t)^* = M^*$.
        \end{itemize}
        \item If $M = \fst(a, b) \Rightarrow_{\beta\eta} N$, by induction on $N$:
        \begin{itemize}
            \item If $N= a'$ where $a \Rightarrow_{\beta\eta} a'$, then by our induction hypothesis, $a' \Rightarrow_{\beta\eta} a^*$, hence $N = a' \Rightarrow_{\beta\eta} a^* = (\fst(a, b))^* = M^*$.
            \item If $N = \fst(a', b')$ where $a \Rightarrow_{\beta\eta} a'$ and $b \Rightarrow_{\beta\eta} b'$. Then $\fst(a', b') \Rightarrow_{\beta\eta} a^* = (\fst(a, b))^* = M^*$ since $a' \Rightarrow_{\beta\eta} a^*$ by our induction hypothesis.
        \end{itemize}
        \item If $M = \snd(a, b) \Rightarrow_{\beta\eta} N$, by induction on $N$:
        \begin{itemize}
            \item If $N= b'$ where $b \Rightarrow_{\beta\eta} b'$, then by our induction hypothesis, $b' \Rightarrow_{\beta\eta} b^*$, hence $N = b' \Rightarrow_{\beta\eta} b^* = (\snd(a, b))^* = M^*$.
            \item If $N = \snd(a', b')$ where $a \Rightarrow_{\beta\eta} a'$ and $b \Rightarrow_{\beta\eta} b'$. Then $N = \snd(a', b') \Rightarrow_{\beta\eta} b^* = (\snd(a, b))^* = M^*$ since $b' \Rightarrow_{\beta\eta} b^*$ by our induction hypothesis.
        \end{itemize}
        \item If $M = \lambda x. t x$ where $x \not \in \mathrm{FV}(t)$ then $M \Rightarrow_{\beta\eta} N$. Induction over $N$:
        \begin{itemize}
            \item If $N = t'$ where $t \Rightarrow_{\beta\eta} t'$, then by our induction hypothesis $t' \Rightarrow_{\beta\eta} t^* = (\lambda x . t x)^* = M^*$.
            \item If $N = \lambda x . t' x$ where $t \Rightarrow_{\beta\eta} t'$ and $x \not \in \mathrm{FV}(t')$. By our induction hypothesis $t' \Rightarrow_{\beta\eta} t^*$, hence $N = \lambda x . t' x \Rightarrow_{\beta\eta} t^* = (\lambda x . t x)^* = M^*$.
        \end{itemize}
        \item If $M = (\fst(t), \snd(t))$ and $M \Rightarrow_{\beta\eta} N$. By induction on $N$:
        \begin{itemize}
            \item If $N = t'$ where $t \Rightarrow_{\beta\eta} t'$, then by our induction hypothesis $t' \Rightarrow_{\beta\eta} t^*$, hence $N = t' \Rightarrow_{\beta\eta} t^* = (\fst(t), \snd(t))^* = M^*$.
            \item If $N = (\fst(t'), \snd(t'))$ where $t \Rightarrow_{\beta\eta} t'$ then by our induction hypothesis, $t' \Rightarrow_{\beta\eta} t^*$ hence $(\fst(t'), \snd(t')) \Rightarrow_{\beta\eta} t^* = (\fst (t), \snd(t))^* = M^*$.
        \end{itemize}
    \end{itemize}
\end{proof}

\begin{cor}\label{diamond_par_beta_eta}
    Given $M \Rightarrow_{\beta\eta} N_1$ and $M \Rightarrow_{\beta\eta} N_2$ then $N_1 \Rightarrow_{\beta\eta} M'$ and $N_2 \Rightarrow_{\beta\eta} M'$ for some $M'$.
\end{cor}

\begin{proof}
    By Lemma \ref{cd_lemma_beta_eta} we observe that $M' = M^*$ gives us the desired result.
\end{proof}

Now we can show that $\beta\eta$-reduction is weakly Church-Rosser.

\begin{lemma}\label{beta_eta_WCR}
    $\beta\eta$-reduction is weakly Church-Rosser.
\end{lemma}

\begin{proof}
    Given $a \to_{\beta\eta} b$ and $a \to_{\beta\eta} b'$ we see that by Lemma \ref{beta_eta_par_imp}, we have $a \Rightarrow_{\beta\eta} b$ and $a \Rightarrow_{\beta\eta} b'$. Hence by the diamond property of $\Rightarrow_{\beta\eta}$ (Corollary \ref{diamond_par_beta_eta}), we have $b \Rightarrow_{\beta\eta} a'$ and $b' \Rightarrow_{\beta\eta} a'$ for some $a'$. Which by Lemma \ref{beta_eta_par_imp} again, gives us $b \twoheadrightarrow_{\beta\eta} a'$ and $b' \twoheadrightarrow_{\beta\eta} a'$.
\end{proof}

\begin{theorem}
    $\beta\eta$-reduction is Church-Rosser (for typed terms).
\end{theorem}

\begin{proof}
    $\beta\eta$-reduction is strongly normalising by Lemma \ref{beta_eta_SN} and weakly Church-Rosser by Lemma \ref{beta_eta_WCR}. Hence by Newman's Lemma (\ref{newman}) we have that $\beta\eta$-reduction is Church-Rosser for typed terms. 
\end{proof}

\begin{comment}

%%
%% Canonicity
%%

\subsection{Canonicity}

\begin{defin}
    A term $\Gamma \vdash t \Leftarrow A$ is said to be in \emph{canonical form} if syntactically it is only built from variables in $\Gamma$ and constructors of the type $A$.
\end{defin}

\begin{remark}
    A more precise way to say this is perhaps that the derivation tree of $\Gamma \vdash t \Leftarrow A$ only consists of structural rules and introduction rules corresponding to $A$.
\end{remark}

From this we see some immediate consequences.

\begin{lemma}
    If a term $\Gamma \vdash t \Leftarrow A$ is in canonical form then it is necessarily in $\beta \eta$-normal form.
\end{lemma}

\begin{proof}
    If it was not in $\beta \eta$-normal form then it would contain some redex which would mean that the derivation of $\Gamma \vdash t \Leftarrow A$ uses ($\to$-intro). Hence $t$ cannot be in canonical form.
\end{proof}

However it is not so obvious that the converse is true:

\begin{lemma}
    If a term $\Gamma \vdash t \Leftarrow A$ is in $\beta \eta$-normal form, then it is in canonical form.
\end{lemma}

\begin{proof}
    [[TODO]]
\end{proof}



\end{comment}



\begin{comment}
%% beta eta reduction 
We next wish to show that $\beta$-normal forms occur if and only if $\beta \eta$-normal forms occur.

\begin{lemma}\label{beta_SN_beta_eta_SN}
    A term $t$ have a $\beta$-normal form if and only if it has a $\beta \eta$-normal form.
\end{lemma}

\begin{proof}
    A similar proof can be found in \cite[Corollary 15.1.5]{barendregt1984lambda}, this would of course have to be modified to accommodate for product types.
\end{proof}

\begin{remark}
    In particular this means that $\to_{\beta}$ being strongly normalising implies that $\to_{\beta \eta}$ is strongly normalising.
\end{remark}

% beta eta reduction is strongly normalising
\begin{cor}
    $\beta \eta$-reduction is strongly normalising.
\end{cor}

\begin{proof}
    By Lemma \ref{beta_SN} and Lemma \ref{beta_SN_beta_eta_SN}.
\end{proof}


\subsection{Closing remarks}

The best reference for these arguments is \cite{barendregt1984lambda}. There have been a few papers after which many of the arguments have been vastly simplified notably \cite{TakahashiM1989PRIL}. However to our knowledge there is no single account of many of these results. A possible future direction is via the use of \emph{categorical semantics}, which associate each ``type theory'' to an appropriately structured category. Typically these only model the \emph{statics} of the type theory, but there are attempts in the literature to realise the reductions as 2-cells in a \emph{bicategory}. Then a property such as Church-Rosser would be about the existence of pushouts in the Hom-category. 

\subsection{Canonicity}

\begin{defin}
    A term $\Gamma \vdash t \Leftarrow A$ is said to be in \emph{canonical form} if syntactically it is only built from variables in $\Gamma$ and constructors of the type $A$.
\end{defin}

\begin{remark}
    A more precise way to say this is perhaps that the derivation tree of $\Gamma \vdash t \Leftarrow A$ only consists of structural rules and introduction rules corresponding to $A$.
\end{remark}

From this we see some immediate consequences.

\begin{lemma}
    If a term $\Gamma \vdash t \Leftarrow A$ is in canonical form then it is necessarily in $\beta \eta$-normal form.
\end{lemma}

\begin{proof}
    If it was not in $\beta \eta$-normal form then it would contain some redex which would mean that the derivation of $\Gamma \vdash t \Leftarrow A$ uses $($\to$-intro)$. Hence $t$ cannot be in canonical form.
\end{proof}

However it is not so obvious that the converse is true:

\begin{lemma}
    If a term $\Gamma \vdash t \Leftarrow A$ is in $\beta \eta$-normal form, then it is in canonical form.
\end{lemma}

\begin{proof}
    
\end{proof}










Canonicity is a property of a type theory that ensures that a given term of a type $\Gamma \vdash t \Leftarrow T$ has a \emph{canonical form}, in that $t$ is only made up of variables from $\Gamma$ and constructors of $T$. Such a property would guarantee that 

[[These two concepts are very related, we should find some way to talk about it, including Church-Rosser]]
\end{comment}


% Examples of STLC
\section{STLC Examples}

%% Here are examples I would like to add
%
%    i) \x.x                (Identity)
%   ii) \x.\y.xy            (Application function)
%  iii) ((\x.\y.(x+y))3)5   (Obviously with N and + added)
%   iv) (\x.x)(\x.x)        (Doesn't type-check)
%    v) \x.xx               (Doesn't type-check (M combinator (Mockingbird)))
%   vi) \x.\y.((xy)(xy))    (Doesn't type-check)
%  vii) \x.\y.\z.x(yz)      (B combinator (Bluebird))
% viii) \x.\y.\z.xyzz       (W* combinator (Warbler once removed))
%   ix) \x.\y.y(xy)         (O combinator (Owl))
%    x) \x.\y.\z.x(y,z)     (Curry)


Untyped lambda calculus, as we mentioned, is in fact \emph{stronger} than the typed lambda calculus. This we will see by looking at some examples of type checking. Many of these are combinators from untyped lambda calculus in combinatory logic. \ref{} [[Need reference of Mockingbird combinator thing]]

Note we don't have very much choice on types, so it may be useful to enrich our type theory with $+$-types or even the natural numbers. But we will see soon that these both are special cases of dependent types.

\subsection{Identity function $\lambda x . x$}

\begin{example}[Identity function]
    Let's consider the following lambda term $\lambda x . x$. We wish to find a type $T$ such that given some context $\Gamma$ we have $\Gamma \vdash \lambda x . x \Leftarrow T$. Our inversion lemma will tell us exactly which rules let us get to this point. So we will essentially be performing a tree search. Firstly we need to switch modes to get $\lambda x . x \Rightarrow T$. But mode switching also lets us change our 
    \begin{prooftree}
        \AxiomC{$\Gamma \vdash \lambda x . x \Rightarrow T$}
    \end{prooftree}
\end{example}

\subsection{Function application $\lambda x . \lambda y . x y$}

\begin{example}
    Here is another example of a term that type checks. Unfortunately we see the disadvantage with type-setting derivation trees: they are very difficult to write down, and get really wide very quickly.  We want to find a type $T$ such that $\Gamma \vdash \lambda x . \lambda y . x y \Leftarrow T$ is true. Here is a derivation tree: 
        \begin{landscape}
            \centering
            \vspace*{\fill}
            \begin{prooftree}
                %\rootAtTop
                \def\ScoreOverhang{1pt}
                %%%
                \AxiomC{$x : A \in \Gamma , x : A, y : C$}
                \LeftLabel{(var)}
                \UnaryInfC{$\Gamma , x : A, y : C \vdash x \Rightarrow A$}
                \AxiomC{}
                \RightLabel{$(***)$}
                \UnaryInfC{$\Gamma , x : A, y : C \vdash C \to D \equiv A \ \mathsf{type}$}
                    %\insertBetweenHyps{\hskip -5pt}
                \BinaryInfC{$\Gamma , x : A, y : C \vdash x \Leftarrow C \to D$}
                \AxiomC{}
                \RightLabel{$(\dagger)$}
                \UnaryInfC{$\Gamma , x : A, y : C \vdash y \Leftarrow C$}
                \LeftLabel{($\to$-elim)}                
                    \insertBetweenHyps{\hskip -20pt}
                \BinaryInfC{$\Gamma , x : A, y : C \vdash x y \Rightarrow D$}
                \AxiomC{}
                \RightLabel{($\equiv_{\mathsf{type}}$-refl)}
                \UnaryInfC{$\Gamma , x : A, y : C \vdash D \equiv D\ \mathsf{type}$}
                \LeftLabel{(switch)}
                    %\insertBetweenHyps{\hskip -10pt}
                \BinaryInfC{$\Gamma , x : A , y : C \vdash xy \Leftarrow D$}
                \LeftLabel{($\to$-intro)}
                \UnaryInfC{$\Gamma , x : A \vdash \lambda y . x y \Rightarrow C \to D$}
                \AxiomC{}
                \RightLabel{$(**)$}
                \UnaryInfC{$\Gamma , x : A \vdash B \equiv C \to D \ \mathsf{type}$}
                \LeftLabel{(switch)}
                    \insertBetweenHyps{\hskip -150pt}
                \BinaryInfC{$ \Gamma , x : A \vdash \lambda y . xy \Leftarrow B$}
                \LeftLabel{($\to$-intro)}
                \UnaryInfC{$\Gamma \vdash \lambda x . \lambda y . x y \Rightarrow A \to B$}
                \AxiomC{}
                \RightLabel{$(*)$}
                \UnaryInfC{$\Gamma \vdash T \equiv A \to B \ \mathsf{type}$}
                \LeftLabel{(switch)}
                    \insertBetweenHyps{\hskip -90pt}
                \BinaryInfC{$\Gamma \vdash \lambda x . \lambda y . x y \Leftarrow T$}
            \end{prooftree}
            \vfill
        \end{landscape}
        
        \begin{proof}
        We begin with the judgement $\Gamma \vdash \lambda x . \lambda y . x y \Leftarrow T$, now the only way to arrive at this judgement is via the mode-switching rule. Whilst doing this we add type variables $A$ and $B$ which can easily be seen to form into $A \to B$ and let $T \equiv A \to B$. We can come back later and validate this judgement. The mode-switching should have given us $\Gamma \vdash \lambda x . \lambda y . x y \Rightarrow A \to B$ which we can only arrive at by applying the ($\to$-intro) rule. This gives us $\Gamma , x : A \vdash \lambda y . xy \Leftarrow B$. Which we have to mode-switch, and as before we take this chance to introduce type variables $C$ and $D$ in order to arrive at the judgement $\Gamma , x : A \vdash \lambda y . x y \Rightarrow C \to D$. This allows us to apply ($\to$-intro) giving us $\Gamma , x : A , y : C \vdash xy \Leftarrow D$. Now we apply the ($\to$-elim) rule since we have an application. For this we need $\Gamma , x : A, y : C \vdash y \Leftarrow C$, which is marked as $(\dagger)$, and observe that a simple application of mode-switching and the variable rule allows us to derive this judgement. The other hypothesis we need is $\Gamma , x : A, y : C \vdash x \Leftarrow C \to D$. Again by mode-switching and setting $C \to D \equiv A$ we get $\Gamma , x : A, y : C \vdash x \Rightarrow A$ which is clearly derivable by the variable rule.
        
        Now we have 3 type equations $(*)$, $(**)$ and $(***)$, substituting back in we get $\Gamma \vdash T \equiv (C \to D) \to C \to D$ for some types $C$ and $D$. So $\Gamma \vdash \lambda x . \labmda y . x y \Leftarrow T$ if we have types $C$ and $D$.
        \end{proof}
\end{example}

\begin{remark}
    There is a lot going on the the previous example, but crucially it should be observed that it is in fact the \emph{inversion lemmas} that allow us to make choices of which rules to use. So a type-checking algorithm would have to make choices based on what the inversion lemmas say. We also introduced equalities of types which was brushed over. In general, type equalities are only generated by reflexivity so in a way our equations were lifted to equality of syntax. This gave us a classical equality problem. Since all our syntax are trees, we can easily decide their equality. [[CAN YOU???!!]]
\end{remark}

%% Here are examples I would like to add
%
%    i) \x.x                (Identity)
%   ii) \x.\y.xy            (Application function)
%   iv) (\x.x)(\x.x)        (Doesn't type-check)
%    v) \x.xx               (Doesn't type-check (M combinator (Mockingbird)))
%   vi) \x.\y.((xy)(xy))    (Doesn't type-check)
%  vii) \x.\y.\z.x(yz)      (B combinator (Bluebird) function composition!)
%   ix) \x.\y.y(xy)         (O combinator (Owl))
%    x) \x.\y.\z.x(y,z)     (Curry!)

\subsection{Mockingbird $\lambda x . x x$} % Definitely does not type check M
\subsection{$(\lambda x . x)(\lambda x . x)$} %  MI
\subsection{$\lambda x . \lambda y . (xy)(xy)$} % Does not type check BMB
\subsection{Y-combinator $\lambda x . (\lambda y . x (y y)) (\lambda y . x (y y))$} % Does not type check :O oh no!! no recursion!
\subsection{Function composition $\lambda x . \lambda y . \lambda z . x ( y z)$} % Function comp
\subsection{Owl combinator $\lambda x . \lambda y . \lambda z . y (x y)$} % Type checks ((A->B)->A)->(A->B)->A
\subsection{Currying $\lambda x . \lambda y . \lambda z . x (y, z)$} % Curry
\subsection{Swap $\lambda t . (\snd(t), \fst(t))$}








% Extra STLC
\section{Simply typed lambda calculus with products, sums and natural numbers}

The Curry-Howard correspondance suggests that a programming language ought to have several features corresponding to logic. We will add some features to the STLC to make it as powerful as a propositional logic.

\subsection{Natural numbers}

We add natural numbers. This will be our first example of an \emph{inductive type}. We will call the corresponding type theory $\lambda_{\to \times \NN}$ and note that it enjoys \emph{canonicity}. Meaning that not only do all terms \emph{normalise} but they normalise to a canonical form. This means if we have a function that computes a natural number, we are garanteed to get a numeral (an iterated number of sucessors to zero). If we had some rules in our type theory that broke canonicity, we may get a term that type checks as a natural number but isn't jdugementally equal to one.

\subsection{Sum types}

[[Sum types go by the name of unions in C, whereas product types correspond to structs.]]

They are like disjoint unions of sets.

Their induction principle is very simple, to build a function out of $A + B$ it suffices to give a function out of $A$ and another out of $B$.




% History and implications of curry howard
\section{Curry-Howard correspondance}

\subsection{Mathematical logic}

At the beginning of the 20th century, Whitehead and Russell pubished their \emph{Principia Mathematica} \cite{GlossarWiki:Whitehead_Russell:1910}, demonstrating to mathematicians of the time that formal logic could express much of mathematics. It served to popularise modern mathematical logic leading to many mathematicians taking a more serious look at topic such as the foundations of mathematics.

One of the most influencial mathematicians of the time was David Hilbert. Inspired by Whitehead and Russell's vision, Hilbert and his coleagues at G\"ottingen became leading researchers in formal logic. Hilbert proposed the \emph{Entscheidungsproblem} (decision problem), that is, to develop an ``effectually calculable procedure'' to determine the truth or falsehood of any logical statement. At the 1930 Mathematical Congress in K\"onigsberg, Hilbert affirmed his belief in the conjecture, concluding with his famous words ``Wir m\"ussen wissen, wir werden wissen'' (``We must know, we will know''). At the very same conference, Kurt G\"odel announced his proof that arithmetic is incomplete \cite{GlossarWiki:Goedel:1931}, not every statement in arithmetic can be proven.

This however did not deter logicians, who were still interested in understanding why the \emph{Entscheidungsproblem} was undecidable, for this a formal efinition of ``effectively calculable'' was required. So along came three proposed definitions of what it meant to be ``effectively calculable'': \emph{lambda calculus}, pusblished in 1936 by Alonzo Church \cite{church-unsolvableproblemof-1936}; \emph{recursive functions}, proposed by G\"odel in 1934 later published in 1936 by Stephen Kleene \cite{Kleene1936}; and finally \emph{Turing machines} in 1937 by Alan Turing \cite{turing1936a}.

\subsection{Lambda calculus}

(Untyped) lambda calculus was discovered by Church at princeton, originally as a way to define notations for logical formulas. It is a remarkaly compact idea, with only three constructs: variables; lambda abstraction; and function application. It was realised at the time by Church and others that ``There may, indeed, be other applications of the system than its use as a logic.'' [CITATION NEEDED]\cite{}. Church discovered a way of encoding numbers as terms of lambda calculus. From this addition and multiplication could be defined. Kleene later discovered how to define the predecessor function. [CITATION NEEDED] \cite{}. Church later rpoposed $\lambda$-definability as the definition of ``effectively calculable'', what is now known as Church's Thesis, and demonstrated that the problem of determining whether or not a given $\lambda$-term  has a normal form is not $\lambda$-definable. This is now known as the Halting Problem. 

\subsection{Recursive functions}

In 1933 G\"odel arrived in Princeton, unconvinced by Church's claim that every effectively calculable function was $\lambda$-definable. Church responded by offering that if Go\"odel would propose a different definition, then Church would ``undertake to prove it was included in $\lambda$-definability''. In a series of lectures at Princeton, G\"odel proposed what came to be known as ``general recursive functions'' as his candidate for effective calculability. Kleene later published the definition [CITATION NEEDED]\cite{}. Church later outlined a proof [CITATION NEEDED]\cite{} and Kleene later published it in detail. This however did not have the intended effect on G\"odel, whereby he then became convinced that his own definition was incorrect.

\subsection{Turing machines}

Alan Turing was at Camrbdige when he independently formulated his own idea of what it means to be "effectively calculable", now known today as Turing machines. He used it to show that the Entscheidungsproblem is undecidable, that is it cannot be proven to be true or false. Before publication, Turing's advisor Max Newman was worried since Church had published a solution, but since Turing's approach was sufficiently novel it was published anyway. Turing had added an appendix sketching the equivalence of $\lambda$-definability to Turing machines. It was Turings argument that later convinced G\"odel that this was the correct notion of ``effectively calculable''.

\subsection{Russells paradox}

[Talk about the origin of types and stuff]

\subsection{The problem with lambda calculus as a logic}

Church's lambda calculus turned out to be inconsistent. \cite{}[CITATION NEEDED]. The reason was related to russels paradox, in that a predicate was allowed to act on itself. This led to an abandoning of the use of lambda calculus as a logic for a short time. In order to solve this Church adapted a solution similar to Russell's: use types. What was discovered is now known today as \emph{simply-typed lambda calculus}. \cite{} [CITATION NEEDED, 10 ?]. What is nice about Church's STLC is that every term has a normal form, or in the language of Turing machines every computation halts. \cite{} [CITATION NEEDED] From this consistency of Church's STLC as a logic could be established.

\subsection{Types to the rescue}

[Talk in detail why typing is good for mathematicians, programmers and logicians]

\subsection{The theory of proof a la Gentzen}

[Go into the history of the theory of proof e.g. Gentzen's work; take notice of natural deduction]

\subsection{Curry and Howard}

[Curry makes an observation that Gentzens natural deduction corresponds to simply typed lambda calculus, Howard takes this further and defines it formally, eventually predicting a notion of dependent type.

\subsection{Propositions as types}

[Overview of the full nature of the observation, much deeper than a simple correspondance since logic is in some sense ``very correct'' and programming constructs corresponding to these must therefore also be ``very correct''.]

\subsection{Predicates [CHANGE] as types?}

[Talk about predicate quantifiers $\forall, \exists$ and what a ``dependent type ought to do'']


\subsection{Dependent types}




[Perhaps expand on the simply typed section]

[talk about pi and sigma types

[talk about ``dependent contexts'']




% Conclusion
\section{Conclusion and Future directions}

The natural direction to consider after ariving at the Curry-Howard correspondence are dependent types. We discussed mostly propositional logic in this dissertation, but there was nothing stopping us from considering first-order logic. First-order since predicates can \emph{quantify over} propositions. These correspond to the familar $\forall$ and $\exists$ that mathematicians are acustomed to using.

Dependent types generalise their non-dependent counterparts by allowing certain types that form the overall type to \emph{depend} on the value of the terms coming from a type elsewhere. So for example one could write $\mathbf{Months} \times \mathbf{Days}$, to americanly have, a type of dates. Clearly this is complete nonsense since we can have a pair $(\mathrm{Feb}, 31)$. Ideally we would like such terms to not be well-typed. The solution is to let the type of $\mathbf{Days}$ \emph{depend} on the type of $\mathbf{Months}$. In a dependent type theory one would write $$\sum_{m : \mathbf{Months}} \mathbf{Days}(m)$$, where $\mathbf{Days}(m)$ is the type of days of the month $m$. Terms of this \emph{Sigma type} are called dependent pairs. Now the term $(\mathrm{Feb}, 28)$ type checks as before, however $(\mathrm{Feb}, 29)$ doesn't. What we have written is complete nonsense. Clearly the type checker is upset because $\mathbf{Days}(\mathrm{Feb})$ does not have a term $29$.

It turns out dependent type theories have similar normalisation properties. But as a programming language, they are still not understood as well as some of our other programming langauges \cite{Sorensen, DEBRUIJN1994141, DEBRUIJN1972381}. In the future, mainstream functional programming languages such as Haskell, will slowly gain support for dependent types. Generalising a vast array of previous programming features such as Generalized Algrbraic Data Types, Parametricity, Polymorphism and so on \cite{2016arXiv161007978E}. This will allow programmers to reason in rich ways about the correctness of their programs and allow mathematicans to write proofs in a programming language, due to the Curry-Howard correspondence. This is already done at a mass scale today with Formal verification software.

And finally back to basics, we hope that in the future, syntax and its subtleties can be sorted out for good, so that computer scientists won't need to spread white lies when discussing type theories. There is some recent work (formalised too!) in these directions \cite{Binding_Syntax_Theory}. From what we have read, this is essentially a formalised version of Harper's abts, noting that the idea is not unique to him.

Our original goal was to study the categorical semantics of dependent type theories, but along the journey we learnt that even the simple things are not so well understood yet. This means there is oppurtunity to learn, teach and grow.


\newpage
\bibliographystyle{plain} 
\bibliography{uthesis}

%Fixes toc numbering for appendix
\addtocontents{toc}{\protect\setcounter{tocdepth}{1}}
\newpage
\begin{appendices}
	\section{Simply typed lambda calculus $\lambda_{\to \times}$}\label{stlc_rule}

This is the full-presentation of the simply typed lambda calculus $\lambda_{\to \times}$. It has function types, product types and a unit type.

\subsection{Syntax}\label{stlc_syntax}

Written in BNF:

$$
    \mathrm{Term} ::= x \mid \lambda x . a \mid (a, b) \mid a b \mid c
$$

$$
    \mathrm{Type} ::= \mathbf{1} \mid A \times B \mid A \to B
$$

Or listed as operators:

\begin{center}
        \begin{tabular}{|c|c|c|c|c|c|c|}
        \hline Op & Sort & Vars & Type args & Term args & Scoping & Syntax \\
        \hline $\to$           & \ty &  --- & $A,B$ &  ---  &  ---  & $A \to B$            \\
        \hline $\times$        & \ty &  --- & $A,B$ &  ---  &  ---  & $A \times B$         \\
        \hline $(-,-)$         & \tm &  --- &  ---  & $x,y$ &  ---  & $(x,y)$              \\
        \hline $\lambda$       & \tm &  $x$ & $A,B$ &  ---  &  $M$  & $\lambda (x : A).M$  \\
        \hline $\mathrm{App}$  & \tm &  --- & $A,B$ &  ---  & $M,N$ & $M N$ \\
        \hline
    \end{tabular}
\end{center}

\subsection{Judgements}\label{stlc_judgements}

\begin{center}
    \begin{tabular}{|l|l|}
        \hline Judgement & Meaning \\
        \hline $\Gamma \vdash A\ \mathsf{type}$          & $A$ is a type in context $\Gamma$. \\
        \hline $\Gamma \vdash T \Leftarrow A$            & $T$ can be checked to have type $A$ in context $\Gamma$. \\
        \hline $\Gamma \vdash T \Rightarrow A$           & $T$ synthesises the type $A$ in context $\Gamma$. \\
        \hline $\Gamma \vdash A \equiv B\ \mathsf{type}$ & $A$ and $B$ are judgmentally equal types in context $\Gamma$. \\
        \hline $\Gamma \vdash S \equiv T : A$            & $S$ and $T$ are judgmentally equal terms of type $A$ in context $\Gamma$. \\
        \hline
    \end{tabular}
\end{center}

\subsection{Structural rules}\label{stlc_structural}

% Variable rule
\begin{center}\label{stlc_rule_var}\label{stlc_rule_switch}
    \AxiomC{$(x:A) \in \Gamma$}
    \RightLabel{(var)}
    \UnaryInfC{$\Gamma \vdash x \Rightarrow A$}
    \DisplayProof
        \hskip 1.5em
% Switch rule
    \AxiomC{$\Gamma \vdash t \Rightarrow A$}
    \AxiomC{$\Gamma \vdash A \equiv B \ \mathsf{type}$}
    \RightLabel{(switch)}
    \BinaryInfC{$\Gamma \vdash t \Leftarrow B$}
    \DisplayProof
\end{center}

% Compact switch
\begin{prooftree}\label{stlc_rule_cswitch}
    \AxiomC{$\Gamma \vdash t \Rightarrow A$}
    \RightLabel{(cswitch)}
    \UnaryInfC{$\Gamma \vdash t \Leftarrow A$}
\end{prooftree}

[[TODO: Include admissible rules?]]

\subsection{Equality rules}\label{stlc_eq}

% Reflexivity of judgemental equality of types
\begin{prooftree}\label{stlc_rule_type_refl}\label{stlc_type_symm}
    \AxiomC{$\Gamma \vdash A \ \mathsf{type}$}
    \RightLabel{($\equiv_{\mathsf{type}}$-refl)}
    \UnaryInfC{$\Gamma \vdash A \equiv A\ \mathsf{type}$}
    \DisplayProof
        \hskip 1.5em
% Symmetry of judgemental equality of types
    \AxiomC{$\Gamma \vdash A \equiv B \ \mathsf{type}$}
    \RightLabel{($\equiv_{\mathsf{type}}$-symm)}
    \UnaryInfC{$\Gamma \vdash B \equiv A \ \mathsf{type}$}
    \DisplayProof
\end{prooftree}

% Transitivity of judgemental equality of types
\begin{prooftree}\label{stlc_rule_type_tran}
    \AxiomC{$\Gamma \vdash B \ \mathsf{type}$}
    \AxiomC{$\Gamma \vdash A \equiv B\ \mathsf{type}$}
    \AxiomC{$\Gamma \vdash B \equiv C\ \mathsf{type}$}
    \RightLabel{($\equiv_\mathsf{type}$-tran)}
    \TrinaryInfC{$\Gamma \vdash A \equiv C\ \mathsf{type}$}
\end{prooftree}

% Reflexivity of judgemental equality of terms
\begin{prooftree}\label{stlc_rule_term_refl}\label{stlc_rule_term_symm}
    \AxiomC{$\Gamma \vdash t \Leftarrow A$}
    \RightLabel{($\equiv_{\mathsf{term}}$-refl)}
    \UnaryInfC{$\Gamma \vdash t \equiv t : A$}
    \DisplayProof
        \hskip 1.5em
% Symmetry of judgemental equality of terms
    \AxiomC{$\Gamma \vdash s \equiv t : A$}
    \RightLabel{($\equiv_{\mathsf{term}}$-symm)}
    \UnaryInfC{$\Gamma \vdash t \equiv s : A$}
    \DisplayProof
\end{prooftree}

% Transitivity of judgemental equality of terms
\begin{prooftree}\label{stlc_rule_term_tran}
    \AxiomC{$\Gamma \vdash t \Leftarrow A $}
    \AxiomC{$\Gamma \vdash s \equiv t : A$}
    \AxiomC{$\Gamma \vdash t \equiv r : A$}
    \RightLabel{($\equiv_{\mathsf{term}}$-tran)}
    \TrinaryInfC{$\Gamma \vdash s \equiv r : A$}
\end{prooftree}

% judgemental equality of types - judgemental equality of terms - congruence
\begin{prooftree}\label{stlc_rule_term_type_cong}
    \AxiomC{$\Gamma \vdash A \ \mathsf{type}$}
    \AxiomC{$\Gamma \vdash s \equiv t : A$}
    \AxiomC{$\Gamma \vdash A \equiv B\ \mathsf{type}$}
    \RightLabel{($\equiv_{\mathsf{term}}$-$\equiv_{\mathsf{type}}$-cong)}
    \TrinaryInfC{$\Gamma \vdash s \equiv t : B$}
\end{prooftree}

\subsection{Function type}\label{stlc_rule_arrow}

% -> formation
\begin{center}\label{stlc_rule_arrow_form}\label{stlc_rule_arrow_intro}
    \AxiomC{$\Gamma \vdash A\ \mathsf{type}$}
    \AxiomC{$\Gamma \vdash B\ \mathsf{type}$}
    \RightLabel{($\to$-form)}
    \BinaryInfC{$\Gamma \vdash A \to B \ \mathsf{type}$}
    \DisplayProof
        \hskip 1.5em
% -> introduction
    \AxiomC{$\Gamma , x : A\vdash M \Leftarrow B$}
    \RightLabel{($\to$-intro)}
    \UnaryInfC{$\Gamma \vdash \lambda x . M \Rightarrow A \to B$}
    \DisplayProof
\end{center}

% -> elimination
\begin{prooftree}\label{stlc_rule_arrow_elim}
    \AxiomC{$\Gamma \vdash M \Leftarrow A \to B$}
    \AxiomC{$\Gamma \vdash N \Leftarrow A$}
    \RightLabel{($\to$-elim)}
    \BinaryInfC{$\Gamma \vdash M N \Rightarrow B$}
\end{prooftree}

% -> beta
\begin{center}\label{stlc_rule_arrow_beta}\label{stlc_rule_arrow_eta}
    \AxiomC{$\Gamma , x : A \vdash y \Leftarrow B$}
    \AxiomC{$\Gamma \vdash t \Leftarrow A$}
    \RightLabel{($\to$-$\beta$)}
    \BinaryInfC{$\Gamma \vdash (\lambda x . y) t \equiv y[t / x] : B$}
    \DisplayProof
        \hskip 1.5em
% -> eta
    \AxiomC{$\Gamma , y : A \vdash M y \equiv M' y : B$}
    \RightLabel{($\to$-$\eta$)}
    \UnaryInfC{$\Gamma \vdash M \equiv M' : A \to B$}
    \DisplayProof
\end{center}

% -> formation congruence
\begin{prooftree}\label{stlc_rule_arrow_form_cong}
    \AxiomC{$\Gamma \vdash A \equiv A' \ \mathsf{type}$}
    \AxiomC{$\Gamma \vdash B \equiv B' \ \mathsf{type}$}
    \RightLabel{($\to$-$\equiv_{\mathsf{type}}$-cong)}
    \BinaryInfC{$\Gamma \vdash A \to B \equiv A' \to B' \ \mathsf{type}$}
\end{prooftree}

% -> introduction congruence
\begin{prooftree}\label{stlc_rule_arrow_intro_cong}
    \AxiomC{$\Gamma , x : A \vdash M \equiv M' : B$}
    \RightLabel{($\to$-$\equiv_{\mathsf{term}}$-cong)}
    \UnaryInfC{$\Gamma \vdash \lambda x . M \equiv \lambda x . M' : A \to B$}
\end{prooftree}

% -> elimination congruence
\begin{prooftree}\label{stlc_rule_arrow_elim_cong}
    \AxiomC{$\Gamma \vdash M \equiv M' : A \to B$}
    \AxiomC{$\Gamma \vdash N \equiv N' : A$}
    \RightLabel{($\to$-elim-cong)}
    \BinaryInfC{$\Gamma \vdash M N \equiv M' N' : A \to B$}
\end{prooftree}

\subsection{Product type}\label{stlc_rule_prod}

% x introduction
\begin{prooftree}\label{stlc_rule_prod_form}\label{stlc_rule_prod_intro}
    \AxiomC{$\Gamma \vdash A \ \mathsf{type}$}
    \AxiomC{$\Gamma \vdash B \ \mathsf{type}$}
    \RightLabel{($\times$-form)}
    \BinaryInfC{$\Gamma \vdash A \times B \ \mathsf{type}$}
    \DisplayProof
% x formation
    \AxiomC{$\Gamma \vdash a \Leftarrow A$}
    \AxiomC{$\Gamma \vdash b \Leftarrow B$}
    \RightLabel{($\times$-intro)}
    \BinaryInfC{$\Gamma \vdash (a, b) \Rightarrow A \times B$}
    \DisplayProof
\end{prooftree}



% x eliminators
\begin{center}\label{stlc_rule_prod_elim}
    \AxiomC{$\Gamma \vdash t \Leftarrow A \times B$}
    \RightLabel{($\times$-elim${}_1$)}
    \UnaryInfC{$\Gamma \vdash \fst(t) \Rightarrow A$}        
    \DisplayProof
        \hskip 1.5em
    \AxiomC{$\Gamma \vdash t \Leftarrow A \times B$}
    \RightLabel{($\times$-elim${}_2$)}
    \UnaryInfC{$\Gamma \vdash \snd(t) \Rightarrow B$}
    \DisplayProof
\end{center}

% x betas
\begin{center}\label{stlc_rule_prod_beta}
    \AxiomC{$\Gamma \vdash x \Leftarrow A$}
    \AxiomC{$\Gamma \vdash y \Leftarrow B$}
    \RightLabel{($\times$-$\beta_1$)}
    \BinaryInfC{$\Gamma \vdash \fst(x,y)\equiv x : A$}
    \DisplayProof
        \hskip 1.5em
    \AxiomC{$\Gamma \vdash x \Leftarrow A$}
    \AxiomC{$\Gamma \vdash y \Leftarrow B$}
    \RightLabel{($\times$-$\beta_2$)}
    \BinaryInfC{$\Gamma \vdash \snd(x,y)\equiv y : B$}
    \DisplayProof
\end{center}

% x eta
\begin{prooftree}\label{stlc_rule_prod_eta}
    \AxiomC{$\Gamma \vdash \fst(t) \equiv \fst(t') : A$}
    \AxiomC{$\Gamma \vdash \snd(t) \equiv \snd(t') : B$}
    \RightLabel{($\times$-$\eta$)}
    \BinaryInfC{$\Gamma \vdash t \equiv t' : A \times B$}
\end{prooftree}

% x form cong
\begin{prooftree}\label{stlc_rule_prod_form_cong}
    \AxiomC{$\Gamma \vdash A \equiv A' \ \mathsf{type}$}
    \AxiomC{$\Gamma \vdash B \equiv B' \ \mathsf{type}$}
    \RightLabel{($\times$-$\equiv_{\mathsf{type}}$-cong)}
    \BinaryInfC{$\Gamma \vdash A \times B \equiv A' \times B' \ \mathsf{type}$}
\end{prooftree}

% x intro cong
\begin{prooftree}\label{stlc_rule_prod_intro_cong}
    \AxiomC{$\Gamma \vdash a \equiv a' : A$}
    \AxiomC{$\Gamma \vdash b \equiv b' : B$}
    \RightLabel{($\times$-$\equiv_{\mathsf{term}}$-cong)}
    \BinaryInfC{$\Gamma \vdash (a,b) \equiv (a',b') : A \times B$}
\end{prooftree}

% x elim1 cong
\begin{prooftree}\label{stlc_rule_prod_elim1_cong}
    \AxiomC{$\Gamma \vdash t \equiv t' : A \times B$}
    \RightLabel{($\times$-elim${}_1$-cong)}
    \UnaryInfC{$\Gamma \vdash \fst(t) \equiv \fst(t') : A$}
\end{prooftree}

% x elim2 cong
\begin{prooftree}\label{stlc_rule_prod_elim2_cong}
    \AxiomC{$\Gamma \vdash t \equiv t' : A \times B$}
    \RightLabel{($\times$-elim${}_2$-cong)}
    \UnaryInfC{$\Gamma \vdash \snd(t) \equiv \snd(t') : B$}
\end{prooftree}

\subsection{Unit type}\label{stlc_rule_unit}

% Unit formation
\begin{center}\label{stlc_rule_unit_form}\label{stlc_rule_unit_intro}
    \AxiomC{}
    \RightLabel{($\mathbf{1}$-form)}
    \UnaryInfC{$\mathbf{1}\ \mathsf{type}$}
    \DisplayProof
        \hskip 1.5em
% Unit introduction
    \AxiomC{}
    \RightLabel{($\mathbf{1}$-intro)}
    \UnaryInfC{$\Gamma \vdash * \Rightarrow \mathbf{1}$}
    \DisplayProof
\end{center}

	
    \begin{landscape}
    	\section{Examples}
        \subsection{\texorpdfstring{$\eta$}{}-rule} \begin{prooftree}
        \AxiomC{$\Gamma \vdash t \Leftarrow A \times B$}
        \LeftLabel{($\times$-elim${}_1$)}
        \UnaryInfC{$\Gamma \vdash \fst(t) \Rightarrow A$}
        \LeftLabel{(cswitch)}
        \UnaryInfC{$\Gamma \vdash \fst(t) \Leftarrow A$}
        
        \AxiomC{$\Gamma \vdash t \Leftarrow A \times B$}
        \RightLabel{($\times$-elim${}_2$)}
        \UnaryInfC{$\Gamma \vdash \snd(t) \Rightarrow B$}
        \RightLabel{(cswitch)}
        \UnaryInfC{$\Gamma \vdash \snd(t) \Leftarrow B$}
        
        \LeftLabel{($\times$-$\beta_1$)}
        \BinaryInfC{$\Gamma \vdash \fst(\fst(t), \snd(t)) \equiv \fst(t) : A \times B$}
        
        \AxiomC{$\Gamma \vdash t \Leftarrow A \times B$}
        \LeftLabel{($\times$-elim${}_1$)}
        \UnaryInfC{$\Gamma \vdash \fst(t) \Rightarrow A$}
        \LeftLabel{(cswitch)}
        \UnaryInfC{$\Gamma \vdash \fst(t) \Leftarrow A$}
        
        \AxiomC{$\Gamma \vdash t \Leftarrow A \times B$}
        \RightLabel{($\times$-elim${}_2$)}
        \UnaryInfC{$\Gamma \vdash \snd(t) \Rightarrow B$}
        \RightLabel{(cswitch)}
        \UnaryInfC{$\Gamma \vdash \snd(t) \Leftarrow B$}
        
        \RightLabel{($\times$-$\beta_2$)}
        \BinaryInfC{$\Gamma \vdash \snd(\fst(t), \snd(t)) \equiv \snd(t) : A \times B$}
        
        \LeftLabel{($\times$-$\eta$)}
        \BinaryInfC{$\Gamma \vdash (\fst(t), \snd(t))\equiv t : A \times B$}
        
    \end{prooftree}
 \label{ex3}
        \subsection{Function application \texorpdfstring{$\lambda x . \lambda y . x y$}{}} \begin{prooftree}
    %\rootAtTop
    \def\ScoreOverhang{1pt}
    %%%
    \AxiomC{$x : A \in \Gamma , x : A, y : C$}
    \LeftLabel{(var)}
    \UnaryInfC{$\Gamma , x : A, y : C \vdash x \Rightarrow A$}
    \AxiomC{}
    \RightLabel{$(***)$}
    \UnaryInfC{$\Gamma , x : A, y : C \vdash C \to D \equiv A \ \mathsf{type}$}
        %\insertBetweenHyps{\hskip -5pt}
    \BinaryInfC{$\Gamma , x : A, y : C \vdash x \Leftarrow C \to D$}
    \AxiomC{}
    \RightLabel{$(\dagger)$}
    \UnaryInfC{$\Gamma , x : A, y : C \vdash y \Leftarrow C$}
    \noLine
    \UnaryInfC{$\vdots$}
    \noLine
    \UnaryInfC{$\vdots$}
        \insertBetweenHyps{\hskip -40pt}
    \LeftLabel{($\to$-elim)}                
    \BinaryInfC{$\Gamma , x : A, y : C \vdash x y \Rightarrow D$}
    \AxiomC{}
    \RightLabel{($\equiv_{\mathsf{type}}$-refl)}
    \UnaryInfC{$\Gamma , x : A, y : C \vdash D \equiv D\ \mathsf{type}$}
    \LeftLabel{(switch)}
        \insertBetweenHyps{\hskip -50pt}
    \BinaryInfC{$\Gamma , x : A , y : C \vdash xy \Leftarrow D$}
    \LeftLabel{($\to$-intro)}
    \UnaryInfC{$\Gamma , x : A \vdash \lambda y . x y \Rightarrow C \to D$}
    \AxiomC{}
    \RightLabel{$(**)$}
    \UnaryInfC{$\Gamma , x : A \vdash B \equiv C \to D \ \mathsf{type}$}
    \LeftLabel{(switch)}
        \insertBetweenHyps{\hskip -130pt}
    \BinaryInfC{$ \Gamma , x : A \vdash \lambda y . xy \Leftarrow B$}
    \LeftLabel{($\to$-intro)}
    \UnaryInfC{$\Gamma \vdash \lambda x . \lambda y . x y \Rightarrow A \to B$}
    \AxiomC{}
    \RightLabel{$(*)$}
    \UnaryInfC{$\Gamma \vdash T \equiv A \to B \ \mathsf{type}$}
    \LeftLabel{(switch)}
        \insertBetweenHyps{\hskip -90pt}
    \BinaryInfC{$\Gamma \vdash \lambda x . \lambda y . x y \Leftarrow T$}
\end{prooftree}
 \label{ex2}
        \newpage\subsection{Function composition \texorpdfstring{$\lambda x . \lambda y . \lambda z . x ( y z)$}{}} \begin{prooftree}
    \def\ScoreOverhang{1pt}
    \AxiomC{$(x :B \to C) \in \Gamma , x : B \to C, y : A \to B , z : A$}
    \LeftLabel{(var)}
    \UnaryInfC{$\Gamma , x : B \to C, y : A \to B , z : A \vdash x \Rightarrow B \to C$}
    \LeftLabel{(cswitch)}
    \UnaryInfC{$\Gamma , x : B \to C, y : A \to B , z : A \vdash x \Leftarrow B \to C$}
    
        \AxiomC{$(z : A) \in \Gamma , x : B \to C, y : A \to B , z : A$}
        \LeftLabel{(var)}
        \UnaryInfC{$\Gamma , x : B \to C, y : A \to B , z : A \vdash z \Rightarrow A$}
        \LeftLabel{(cswitch)}
        \UnaryInfC{$\Gamma , x : B \to C, y : A \to B , z : A \vdash z \Leftarrow A$}
        
        \AxiomC{$(y : A \to B) \in \Gamma , x : B \to C, y : A \to B , z : A$}
        \RightLabel{(var)}
        \UnaryInfC{$\Gamma , x : B \to C, y : A \to B , z : A \vdash y \Rightarrow A \to B$}
        \RightLabel{(cswitch)}
        \UnaryInfC{$\Gamma , x : B \to C, y : A \to B , z : A \vdash y \Leftarrow A \to B$}
    
    \insertBetweenHyps{\hskip 10pt}
    \RightLabel{($\to$-elim)}
    \BinaryInfC{$\Gamma , x : B \to C, y : A \to B , z : A \vdash yz \Rightarrow B$}
    \RightLabel{(cswitch)}
    \UnaryInfC{$\Gamma , x : B \to C, y : A \to B , z : A \vdash yz \Leftarrow B$}
    \noLine
    \UnaryInfC{$\vdots$}
    \noLine
    \UnaryInfC{$\vdots$}
    
    \insertBetweenHyps{\hskip -160pt}
    \LeftLabel{($\to$-elim)}
    \BinaryInfC{$\Gamma , x : B \to C, y : A \to B , z : A \vdash x(yz) \Rightarrow C$}
    \LeftLabel{(cswitch)}
    \UnaryInfC{$\Gamma , x : B \to C, y : A \to B , z : A \vdash x(yz) \Leftarrow C$}
    \LeftLabel{($\to$-intro)}
    \UnaryInfC{$\Gamma , x : B \to C, y : A \to B \vdash \lambda z.x(yz) \Rightarrow A \to C$}
    \LeftLabel{(cswitch)}
    \UnaryInfC{$\Gamma , x : B \to C, y : A \to B \vdash \lambda z.x(yz) \Leftarrow A \to C$}
    \LeftLabel{($\to$-intro)}
    \UnaryInfC{$\Gamma , x : B \to C \vdash\lambda y.\lambda z.x(yz) \Rightarrow (A \to B) \to (A \to C)$}
    \LeftLabel{(cswitch)}
    \UnaryInfC{$\Gamma , x : B \to C \vdash\lambda y.\lambda z.x(yz) \Leftarrow (A \to B) \to (A \to C)$}
    \LeftLabel{($\to$-intro)}
    \UnaryInfC{$\Gamma \vdash \lambda x.\lambda y.\lambda z.x(yz) \Rightarrow (B \to C) \to (A \to B) \to (A \to C)$}
    \LeftLabel{(cswitch)}
    \UnaryInfC{$\Gamma \vdash \lambda x.\lambda y.\lambda z.x(yz) \Leftarrow (B \to C) \to (A \to B) \to (A \to C)$}
\end{prooftree}
 \label{ex7}        
        \newpage\subsection{Currying \texorpdfstring{$\lambda x . \lambda y . \lambda z . x (y, z)$}{}} \begin{prooftree}
    \def\ScoreOverhang{1pt}
    \AxiomC{$(x : A \times B \to C) \in \Gamma, x : A \times B \to C , y : A, z : B$}
    \LeftLabel{(var)}
    \UnaryInfC{$\Gamma, x : A \times B \to C , y : A, z : B \vdash x \Rightarrow A \times B \to C$}
    \LeftLabel{(cswitch)}
    \UnaryInfC{$\Gamma, x : A \times B \to C , y : A, z : B \vdash x \Leftarrow A \times B \to C$}
    
    \AxiomC{$(y : A) \in \Gamma, x : A \times B \to C , y : A, z : B$}
    \LeftLabel{(var)}
    \UnaryInfC{$\Gamma , x : A \times B \to C, y : A, z : B \vdash y \Rightarrow A$}
    \LeftLabel{(cswitch)}
    \UnaryInfC{$\Gamma , x : A \times B \to C, y : A, z : B \vdash y \Leftarrow A$}
    
    \AxiomC{$(z : B) \in \Gamma , x : A \times B \to C, y : A, z : B$}
    \RightLabel{(var)}
    \UnaryInfC{$\Gamma , x : A \times B \to C, y : A, z : B \vdash z \Rightarrow B$}
    \RightLabel{(cswitch)}
    \UnaryInfC{$\Gamma , x : A \times B \to C, y : A, z : B \vdash z \Leftarrow B$}
    
    \LeftLabel{($\times$-intro)}
    \BinaryInfC{$\Gamma , x : A \times B \to C, y : A, z : B \vdash (y, z) \Rightarrow A \times B$}
    \LeftLabel{(cswitch)}
    \UnaryInfC{$\Gamma , x : A \times B \to C, y : A, z : B \vdash (y, z) \Leftarrow A \times B$}
    
    \LeftLabel{($\to$-elim)}
    \UnaryInfC{$\Gamma, x : A \times B \to C , y : A, z : B \vdash x (y , z) \Rightarrow C$}
    \LeftLabel{(cswitch)}
    \UnaryInfC{$\Gamma, x : A \times B \to C , y : A, z : B \vdash x (y , z) \Leftarrow C$}
    \LeftLabel{($\to$-intro)}
    \UnaryInfC{$\Gamma, x : A \times B \to C , y : A \vdash \lambda z . x (y , z) \Rightarrow B \to C$}
    \LeftLabel{(cswitch)}
    \UnaryInfC{$\Gamma, x : A \times B \to C , y : A \vdash \lambda z . x (y , z) \Leftarrow B \to C$}
    \LeftLabel{($\to$-intro)}
    \UnaryInfC{$\Gamma, x : A \times B \to C \vdash \lambda y . \lambda z . x (y , z) \Rightarrow A \to B \to C$}
    \LeftLabel{(cswitch)}
    \UnaryInfC{$\Gamma, x : A \times B \to C \vdash \lambda y . \lambda z . x (y , z) \Leftarrow A \to B \to C$}
    \LeftLabel{($\to$-intro)}
    \UnaryInfC{$\Gamma \vdash \lambda x . \lambda y . \lambda z . x (y , z) \Rightarrow (A \times B \to C) \to A \to B \to C$}
    \LeftLabel{(cswitch)}
    \UnaryInfC{$\Gamma \vdash \lambda x . \lambda y . \lambda z . x (y , z) \Leftarrow (A \times B \to C) \to A \to B \to C$}
    
\end{prooftree}
 \label{ex8}
        \newpage\subsection{Uncurry} \begin{prooftree}
    \def\ScoreOverhang{1pt}
    \AxiomC{$(x : A \to B \to C) \in \Gamma , x : A \to B \to C, y : A \times B$}
    \LeftLabel{(var)}
    \UnaryInfC{$\Gamma , x : A \to B \to C, y : A \times B \vdash x \Rightarrow A \to B \to C$}
    \LeftLabel{(cswitch)}
    \UnaryInfC{$\Gamma , x : A \to B \to C, y : A \times B \vdash x \Leftarrow A \to B \to C$}
    
    \AxiomC{$(y : A \times B) \in \Gamma , x : A \to B \to C, y : A \times B$}
    \RightLabel{(var)}
    \UnaryInfC{$\Gamma , x : A \to B \to C, y : A \times B \vdash y \Rightarrow A \times B$}
    \RightLabel{(cswitch)}
    \UnaryInfC{$\Gamma , x : A \to B \to C, y : A \times B \vdash y \Leftarrow A \times B$}
    \RightLabel{($\times$-elim${}_1$}
    \UnaryInfC{$\Gamma , x : A \to B \to C, y : A \times B \vdash \fst(y) \Rightarrow A$}
    \RightLabel{(cswitch)}
    \UnaryInfC{$\Gamma , x : A \to B \to C, y : A \times B \vdash \fst(y) \Leftarrow A$}
    
    \LeftLabel{($\to$-elim)}
    \BinaryInfC{$\Gamma , x : A \to B \to C, y : A \times B \vdash x(\fst(y)) \Rightarrow A$}
    \LeftLabel{(cswitch)}
    \UnaryInfC{$\Gamma , x : A \to B \to C, y : A \times B \vdash x(\fst(y)) \Leftarrow A$}
    \noLine
    \UnaryInfC{$\vdots$}
    \noLine
    \UnaryInfC{$\vdots$}
    \noLine
    \UnaryInfC{$\vdots$}
    \noLine
    \UnaryInfC{$\vdots$}
    
    \AxiomC{$(y : A \times B) \in \Gamma , x : A \to B \to C, y : A \times B$}
    \RightLabel{(var)}
    \UnaryInfC{$\Gamma , x : A \to B \to C, y : A \times B \vdash y \Rightarrow A \times B$}
    \RightLabel{(cswitch)}
    \UnaryInfC{$\Gamma , x : A \to B \to C, y : A \times B \vdash \snd(y) \Leftarrow B$}
    \RightLabel{($\times$-elim${}_2$}
    \UnaryInfC{$\Gamma , x : A \to B \to C, y : A \times B \vdash \snd(y) \Rightarrow B$}
    \RightLabel{(cswitch)}
    \UnaryInfC{$\Gamma , x : A \to B \to C, y : A \times B \vdash \snd(y) \Leftarrow B$}
    
    
    \insertBetweenHyps{\hskip -200pt}
    \LeftLabel{($\to$-elim)}
    \BinaryInfC{$\Gamma , x : A \to B \to C, y : A \times B \vdash x(\fst(y))(\snd(y)) \Rightarrow C$}
    \LeftLabel{(cswitch)}
    \UnaryInfC{$\Gamma , x : A \to B \to C, y : A \times B \vdash x(\fst(y))(\snd(y)) \Leftarrow C$}
    \LeftLabel{($\to$-intro)}
    \UnaryInfC{$\Gamma , x : A \to B \to C \vdash \lambda y . \vdash x(\fst(y))(\snd(y)) \Rightarrow A \times B \to C$}
    \LeftLabel{(cswitch)}
    \UnaryInfC{$\Gamma , x : A \to B \to C \vdash \lambda y . \vdash x(\fst(y))(\snd(y)) \Leftarrow A \times B \to C$}
    \LeftLabel{($\to$-intro)}
    \UnaryInfC{$\Gamma \vdash \lambda x . \lambda y .  \vdash x(\fst(y))(\snd(y)) \Rightarrow (A \to B \to C) \to A \times B \to C$}
    \LeftLabel{(cswitch)}
    \UnaryInfC{$\Gamma \vdash \lambda x . \lambda y .  \vdash x(\fst(y))(\snd(y)) \Leftarrow (A \to B \to C) \to A \times B \to C$}
\end{prooftree}
 \label{ex9}
        
    \end{landscape}
\end{appendices}



\end{document}
