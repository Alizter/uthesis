Let $\Sigma$ be a signature with $\Typ=|\Sigma|$ as its underlying set of types.

Let $\Var$ be a countable set, elements of which will be called (term) variables.

A variable declaration is a pair $(x, t) \in \Var \times \Typ$ usually written $x:t$.

Given that $\Var$ is countable there is a function $v : \N \to \Var$. Therefore when we write $v_n$ we are specifying the $n$th element of $\Var$. A context is then a sequence of types $\Typ^*$, whose elements $\Gamma = (t_1,\dots, t_n)$, we write like this $\Gamma = v_1:t_1, \dots, v_n:t_n$. Now if we have another context $\Delta =v_1:u_1, \dots, v_m:u_m$, we can concatenate them like this: $\Gamma,\Delta = v_1:t_1,\dots, v_n:t_n, v_{n+1}:u_1,\dots,v_{n+m}: u_m$. So a context is really just a sequence of types, but the index of the sequence also refers to the variable. The set of contexts is called $\Con$.

Judgements are the basic statements or assertions of our theory. We will have starting judgements (perhaps called axioms) whereby we derive other judgements according to rules.

One judgement in this simply typed lambda calculus can be defined as a triple $(\Gamma, t, T) \in \Con \times \Var \times \Typ$. 

Another is a well-formed context, which 

We may also add other kinds of judgements so we will accumulate those in a set called $\Jud$.

An inference rule is a function $\Jud^* \to \Jud$. We will pick these carefully as they will essentially "generate" our type theory.

For simply typed lambda calculus
