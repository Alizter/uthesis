\section{Curry-Howard correspondence}

\subsection{Introduction}

In this section we outline a detailed history of what is known as the Curry-Howard correspondence. This is an important thing to consider since there are many powerful ideas that get uncovered in this story. They will shape the future of thought on the subject, so it is worthwhile to understand what was motivating mathematicians and computer scientists at the time.

Many of these ideas were developed before the comprehension of the modern idea of a computer! So in a way it is quite remarkable that these ideas were even considered. Hence we will discuss their original motivations.

This story will develop our ideas of what needs to be added to the simply typed lambda calculus going forward. How these ideas will behave with what we have studied, and finally what the future of the subject looks like.x


%%%
\subsection{Mathematical logic}\label{logic_chapter}

At the beginning of the 20th century, Whitehead and Russell published their \emph{Principia Mathematica} \cite{GlossarWiki:Whitehead_Russell:1910}, demonstrating to mathematicians of the time that formal logic could express much of mathematics. It served to popularise modern mathematical logic leading to many mathematicians taking a more serious look at topic such as the foundations of mathematics.

One of the most influential mathematicians of the time was David Hilbert. Inspired by Whitehead and Russell's vision, Hilbert and his colleagues at G\"ottingen became leading researchers in formal logic. Hilbert proposed the \emph{Entscheidungsproblem} (decision problem), that is, to develop an ``effectually calculable procedure'' to determine the truth or falsehood of any logical statement. At the 1930 Mathematical Congress in K\"onigsberg, Hilbert affirmed his belief in the conjecture, concluding with his famous words ``Wir m\"ussen wissen, wir werden wissen'' (``We must know, we will know''). At the very same conference, Kurt G\"odel announced his proof that arithmetic is incomplete \cite{GlossarWiki:Goedel:1931}, not every statement in arithmetic can be proven.

This however did not deter logicians, who were still interested in understanding why the \emph{Entscheidungsproblem} was not attainable.
For this, a formal definition of ``effectively calculable'' was required.
So along came three candidate definitions of what it meant to be ``effectively calculable'': \emph{$\lambda$-calculus}, published in 1936 by Alonzo Church \cite{church-unsolvableproblemof-1936}; \emph{recursive functions}, proposed by G\"odel in 1934 later published in 1936 by Stephen Kleene \cite{kleene1936}; and finally \emph{Turing machines} in 1937 by Alan Turing \cite{turing1936a}.

%%%
\subsection{\texorpdfstring{$\lambda$}{}-calculus}

$\lambda$-calculus was discovered by Church at Princeton in the 1930s, originally as a way to define notations for logical formulas.
It is a very compact and simple idea, with only three constructs: variables; $\lambda$-abstraction; and function application.
Curry developed the closely related idea of combinatory logic around the same time \cite{curry1930a, curry1930b}.

Interestingly, Curry had introduced the notion of \emph{Combinators} into logic for the very same reason reason we introduced abstract binding trees: to avoid mentioning named variables \cite{Sorensen}.

It was realised at the time by Church and others that ``There may, indeed, be other applications of the system than its use as a logic.'' \cite{church1932, church1933}.
This meant that $\lambda$-calculus was worth studying as a topic of interest in it's own right.
This became explicitly apparent when Church discovered a way of encoding numbers as terms of $\lambda$-calculus, known as the \emph{Church encoding} of the natural numbers. From this addition and multiplication could also be defined.

However the problem of defining a predecessor function alluded Church and his students, in fact Church later became convinced that it was impossible.
Fortunately Kleene later discovered, at his dentist's office, how to define the predecessor function \cite{kleene1935a, kleene1935b, 4392910}.
This led to Church to later propose that $\lambda$-definability ought to be the definition of ``effectively calculable'', culminating into what is now known as Church's Thesis. Church went on to demonstrate that the problem of determining whether or not a given $\lambda$-term  has a normal form is not $\lambda$-definable.
This is now known as the Halting Problem. Put differently this says that no program written in the $\lambda$-calculus can determine whether a program written in the $\lambda$-calculus halts or not.

%%%
\subsection{Recursive functions}

In 1933 G\"odel arrived in Princeton, unconvinced by Church's claim that every effectively calculable function was $\lambda$-definable. Church responded by offering that if Go\"odel would propose a different definition, then Church would ``undertake to prove it was included in $\lambda$-definability''. In a series of lectures at Princeton, G\"odel proposed what came to be known as ``general recursive functions'' as his candidate for effective calculability. Kleene later published the definition \cite{kleene1936}. Church later outlined a proof that it was equivalent to the $\lambda$-calculus \cite{church1936} and Kleene later published it in detail \cite{kleene1936b}. This however did not have the intended effect on G\"odel, whereby he then became convinced that his own definition was incorrect!

%%%
\subsection{Turing machines}

Alan Turing was at Cambridge when he independently formulated his own idea of what it meant to be ``effectively calculable'', what is now known today as a \emph{Turing machine}. He used it to show that the Entscheidungsproblem is undecidable, meaning that it cannot be proven to be true or false. Before publication, Turing's advisor Max Newman was worried since Church had already published a solution, but since Turing's approach was sufficiently novel it was published anyway \cite{turing1936a}. Turing had added an appendix sketching the equivalence of $\lambda$-definability to Turing machines. It was Turing's argument that later convinced G\"odel that this was the correct notion of ``effectively calculable''.

Of course today the argument for Turing machines as a candidate for computation seems obvious.
We are surrounded by computers in our daily lives, all based loosely on the idea of a Turing machine.
From this it is easy to see that Turing's ideas had a \emph{huge} influence on the notions of computation.

%%%
\subsection{The problem with \texorpdfstring{$\lambda$}{}-calculus as a logic}

Church's students Kleene and Rosser quickly discovered that $\lambda$-calculus was inconsistent as a logic \cite{kleene1935c}.
A logic is deemed \emph{inconsistent} if every statement can be proven.
For example assuming $1 = 2$ can lead to many bizarre consequences, such as all logical formulas becoming true, one way or another.
In that way, arithmetic with the assumption that $1 = 2$, is \emph{inconsistent as a logic}.
Curry later published a simplified version of Kleene and Rosser's result which became known as \emph{Curry's paradox} \cite{curry1942}.
Curry's paradox was related to Russell's paradox, in that a predicate was allowed to act on itself.

Russell's paradox is typically seen as a paradox of set theory, but can usually be phrased in a much more general manner. The basic idea is this: Let $A$ be the set of all sets that do not contain themselves. The question is, does $A$ belong to itself? Clearly, if it did then it would not be an element of the set. If it did not, then it would have to be an element. Either way there is a contradiction, hence we have a \emph{logical paradox}.

The issue arises with the definition of $A$. In it we defined it as something quantifying over a lot of things, but most importantly itself. This self reference is exactly the issue that leads to such a paradox. The idea of self-reference isn't that harmful if kept under control however, particularly if a relation is \emph{well-founded}.

But allowing all predicates (formulas quantifying over other formulas), leads to silly situations as above. Much of modern set theory has been developed in order to avoid being able to write down paradoxical statements as above. We will see many of these ideas in a type theoretic form later on. A good introduction to basic set theory is \cite{johnstone1987notes}.


What is nice about Church's STLC is that every term has a normal form, or in the language of Turing machines every computation halts \cite{turing1936a}. From this consistency of Church's STLC as a logic could be established, not every logical formula is true.

%%%
\subsection{Types to the rescue}

Types were originally introduced as a method to avoid paradoxes occurring in the type-free world.
However mathematicians had naturally stratified objects into different categories, without any consideration to types before \cite{GANDY1977173, kamareddine2002}.
Russell was one of the first mathematicians to introduce a formal theory of types \cite{GlossarWiki:Whitehead_Russell:1910}, precisely to avoid the paradox baring his name.
In order to solve this problem, Church adapted a solution similar to Russell's.
The first presentation of a simple theory of types was given in Church's influential paper \cite{church1940}, where he introduced the simply typed lambda calculus.

Being typed had some immediate consequences, especially on the ideas of $\lambda$-calculus as a notion of computation. We saw in Example \ref{y_comb}, certain computational properties of the untyped lambda calculus, such as recursion are lost. What we are left with is a strictly weaker programming language. But one that is at least consistent as a system of logic.

%%%
\subsection{Types in the design of programming languages}

[[TODO]]

%%%
\subsection{Propositions as types}

We have previously discussed a condensed form of the two judgements $a \Leftarrow A$ and $a \Rightarrow A$, which we will denote $a : A$. This is in fact the judgement considered without bidirectional type checking, and the one found in most literature on the subject. We had our own reasons for choosing this set up but now we want to discuss how types and propositions are related, hence we will only be mentioning $a : A$.

The basic idea of the propositions as types is to consider types as propositional formulas, and terms as proof of those propositions. Suppose I had a type $\mathrm{TheSkyIsBlue}$. Then terms of this type would correspond to evidence or proofs that the sky is indeed blue. Type formers can then be seen as logical connectives. Suppose $A$ is the type ``Is a Cow''. Terms of $A$ are proofs that what ever we are considering, it \emph{is a cow}. Now suppose there is a type $B$ called ``Goes moo''. Terms of this type are proofs that what ever we are considering, \emph{goes moo}. Say we want to prove the statement: ``If it is a cow, then it goes moo''.
Proving such a statement would go something like this: Suppose what we are considering is a cow, through this series of logical steps we arrive at the conclusion that it goes moo.
If $A$ and $B$ were \emph{propositions}, then it would be agreed that we have proven $A \implies B$. It's as if we have taken a proof of $A$ and turned it into a proof of $B$. We have types that do exactly that: Function types!
So our proof of $A \to B$ is really just a function $\lambda x . y$ that takes in a proof $x : A$ and gives a proof $y : B$.

Of course implication isn't what logic is all about, a natural question to ask is what conjunction (the fancy word for ``and'') corresponds to. Proving that ``\emph{the sky is blue}'' and ``\emph{the grass is green}'', in a way, requires \emph{two} proofs, one corresponding to each proposition. Let $A$ be the type denoting that the sky is blue and $B$ be the type denoting that the grass is green. Then $A \times B$, the product type, is the type that both these ``propositions'' hold. How do we construct a term of the product type? Well we need to give a pair $(a, b)$, which consists of a term $a : A$ and a term $b : B$. Or in other words, to give a proof of $A \times B$ we need to give a proof of $a$ and a proof of $B$. For a longer list see Table \ref{tab1}.

Curiously, Curry had noticed something similar in \cite[footnote 28]{10.2307/2266302}, though his motivations were far less bovinal: \say{Note the similarity of the postulates for $F$ and those for $P$. If in any of the former postulates we change $F$ to $P$ and drop the combinator we have the corresponding postulate for $P$.}
Here $P$ is the combinator for implication and $F$ is a ``functionality'' combinator, whereby $F A B f$ essentially means $f : A \to B$.
There is evidence to suggest that this idea wasn't new to Curry. Hindley \cite{hindley_1997} points out that a remark on page 588 of \cite{10.2307/86796} indicates this. Another hint is that the properties of implication (denoted $\supset$) are named PB, PC, PW and PK, after the combinators B, C, W and K.

The correspondence was made precise (in typed combinatory logic) by Curry and Feys in \cite[Chapter 9]{curry1958combinatory}. There are two theorems proved in this chapter, under the title of ``F-P transformation'' (the notation from earlier): Theorem 1 states that inhabited types are provable; Theorem 2 states that the converse. It was a famous ``privately circulated'' paper of Howard \cite{howard1980formulae} that made the links between cut-elimination and ($\beta$-)normalisation.

Such a correspondence was great for type theory since now normalisation of terms could be proven in the world of logic.

It should be noted that Curry and Howard share most of the credit for this work, yet De Bruijn, independently formulated a statement about a correspondence between proofs and objects. He was, furthermore, one of the first to make practical use of it. He develped the ``mathematical language'' Automath \cite{DEBRUIJN1994141}. 

\begin{remark}
    We note that other authors had also independently noted a link between proofs and combinators \cite{meredith1963}. 
\end{remark} 

\begin{table}
  \centering
    \begin{tabular}{r|r}
        Logic & $\lambda$-calculus \\
        \hline  
        Logical formula & Type (Def. \ref{terms_types}) \\
        Propositional variable & Type variable (Def. \ref{terms_types}) \\
        Logical connective & Type former (Def. \ref{typing_rules}) \\
        Implication & Function type (Def. \ref{function_type}) \\
        Conjunction (and) & Product type (Def. \ref{product_type}) \\
        Disjunction (or) & Sum type (Def. \ref{sum_type}) \\
        Absurdity (false) & Empty type (Def. \ref{empty_def}) \\
        Tautology (truth) & Unit type (Def. \ref{unit_type}) \\
        Proof & Term (Def. \ref{terms_types}) \\
        Assumption & Term variable (Def. \ref{terms_types}) \\
        Introduction & Constructor (Def. \ref{typing_rules}) \\
        Elimination & Destructor (Def. \ref{typing_rules}) \\
        Proof detour & Redex (Def. \ref{beta_eta_reduction}) \\
        Cut elimination & Reduction (Def. \ref{beta_eta_reduction}) \\
        Normal proof & Normal form (Def. \ref{normal_form}) \\
        Provability & Inhabitation   
    \end{tabular}
    \caption{Here is a table sumarising the depth of the correspondence. The table explains the terminology for eliminators/destructors, introductors/constructors.}
    \label{tab1}
\end{table}

\begin{remark}
    We note that some authors like to call this an \emph{isomorphsim} but it is not entire clear what a \emph{morphism} is between a logic and a type theory.
\end{remark}











