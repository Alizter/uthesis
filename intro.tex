%Simply typed lambda calculus (STLC) has been well documented and studied by type theorists and mathematicians, and it's features have been used by many programming languages [NEED REFERENCE].

%In \cite{BarendregtHenk2013Lcwt} it is noted that \say{Research monographs on dependent and inductive types are lacking.} This will essentially be one of the goals of this thesis, to provide a guide for mathematicians and computer scientists about the use of dependent type theory. As this document is written there is no single account of all approaches to \i{dependent} type theory.

%Awodey \cite{2014arXiv1406.3219A} made an observation that Dybjer's \cite{dybjer1996} categories with families (CwF) is a presheaf category with a representable natural transformation (it's fibers are representable). He then proceeds to show conditions needed to model a dependent type theory with $\Pi$, $\Sigma$ and $\mathrm{Id}$ types.


%This thesis will have three main goals.

%\begin{enumitem}
%	\item To present a dependent type theory
%	\item To model the semantics of such a type theory using categorical methods
%	\item To discuss the applications to mathematics and computer science (proof assistants, programming languages and foundations)
%\end{enumitem}

%Finally we may also discuss recent developments of something called "Homotopy type theory" and how that fits into the general picture.

%Roughly a \textit{type system} is a set of loosely organised rules outlining how ``atomic sentences'' called \textit{judgements} can be derived from each other in a given context. A \textit{context} can simply be thought of as a list of terms. 

%The aim of this thesis is to present to two sorts of audience, the utility of dependent type theory. The audiences that I have in mind are computer scientists, roughly individuals who wish to write good code, and mathematicians, roughly individuals who wish to write good proofs.

%These will be our main aims however we do also wish to develop the machinery formally.

%\section{Propositions as types}

%There is a rich interplay between programming and logic known as the Curry-Howard correspondance or propositions as types. 





%\section{What is type theory}

%Type theory is the study of types systems. That is a system that orginizes data manipulated by programs into types. This has been a very useful concept in computer science. It has allowed the writing of programs taht a more 

%\subsection{Lambda calculus}
%\subsection{Modelling type theory}
%\section{What is dependent type theory?}
%\subsection{What are dependent types?}
%\subsection{Motivation for computer scientists}
%\subsection{Motivation for mathematicians}
%\subsection{Category theory}
%\subsection{Categorical logic}
%\subsection{Future directions}

\section{Introduction}

The goal of this thesis is to introduce dependent types to the undergraduate reader. We set out 

The goal of this dissertation is to learn how to mathematically design a programming language. 


The aim of this thesis is to introduce the notion of dependent types to an undergraduate reader. The main idea of dependent types is very simple, yet deceptively subtle however, since modelling such a formalism is quite tricky. This is evidenced by the fact that there is a lot of disagreement in type theory what has or hasn't been proven. This however is a familar story in mathematics and is usually remidied by trying to understand what has been done better. Usually with the help of a new persepctive. 

Dependent types however, are not only of interest to mathematicians but also programmers. Dependent type theory (much like simply typed lambda calculus) is very much a programming langauge allowing the expression of ideas previously too difficult to express. This is very much facilaiated by its deep connection to predicate logic.


\begin{itemize}
\item a[Begin with history and implications of curry howard]

\item a[outline the ``what they should do'' of dependent types]

\item a[start to rigoursly model syntax and talk about how bad a job most authors do]

\item a[small section about classical inductive definitions]

\item a[small section on why categorical semantics]

\item a[model simply typed lambda calculus with categorical semantics]

\item a[show natural extensions of the idea and why contexts break when dependnet]

\item a[outline different approches to solving these problems]

\item a[discuss Awodey's natural models]

\item a[finally talk about future directions for type theory]

\item a[maybe some mention on applications to programming (generalising various constructs, polymorphism, GA data types)]

\item a[equality, inductive types, [[[[[maybe a tinsy bit of homotopy type theory]]]]]]
\end{itemize}


