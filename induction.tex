% induction in set theory

\section{Induction}

We begin by preparing the tools we will use later on. One of these tools will be structural induction. We will instead prove[state?] a vastly more general theorem[principle?] of set theory called the \textit{well-founded induction principal}.

\subsection{Well-founded induction}

This is a standard theorem in set theory [cite set theory books]. For definitions of well-foundedness that have been treated carefully we follow \cite[\S 8]{2018arXiv180805204S}.

\begin{defin}\ 
    \begin{enumerate}[(i)]
        \item A \textbf{graph} is a set $X$, whose elements are called \textbf{nodes}, equipped with a binary relation $\prec$.
        \item If $x \prec y$ then we say $x$ is a \textbf{child} of $y$.
        \item We call a graph \textbf{pointed} if it has a distinguished node $\star$ called the \textbf{root}.
        \item A pointed graph is \textbf{accessible}, if for every $x \in X$, there exists a path $x = x_n \prec \cdots \prec x_0 = \star$.
        \item For any node $x$ of $X$ we write $X/x$ to denote the graph whose nodes are nodes of $X$ that admit a path to $x$. This is naturally pointed by $x$. The relation is the same, simply restricted to the subset $X/x$ of $X$.
    \end{enumerate}
\end{defin}

\begin{remark}
    Note that the definition of accessible relies on a definition of natural numbers. If one was to carry out this construction in a setting more general than sets we would need what is called a \textit{Natural Numbers Object (NNO)}.
\end{remark}

We now make sure that subsets of graphs bring the nodes' parents.

\begin{defin}
    A subset $S$ of a graph $X$ is \textbf{inductive} if for any node $x$ in $X$, when all the children of $x$ are in $S$, $x$ is also in $S$.
\end{defin}

\begin{defin}
    A graph $X$ is \textbf{well-founded} if any inductive subset of $X$ is equal to all of $X$.
\end{defin}

\begin{remark}
    This is slightly weaker than classical versions of well-foundedness in logic, in which a graph would be well-founded if every inhabited subset has a $\prec$-minimal element. In fact such a definition would imply the law of excluded middle.
\end{remark}

\begin{lemma}\label{sswfg}
    Any subset of a well-founded graph is a well-founded graph with the induced relation.
\end{lemma}

We will follow the proof outlined in \cite{winskel1993formal}. This has been the simplest proof to understand. Other proofs can be found in for example \cite{johnstone1987notes} and \cite[Chapter 7]{barwise1982handbook}. However we found these proof too technical and verbose for simple use.

\begin{theorem}[Principle of well-founded induction]
    Let $X$ be a well-founded graph and $P$ be a property of nodes of $X$.
    We have that $$\forall x \in X, P(x) \Leftrightarrow \forall x \in X,((\forall y \prec x, P(y)) \Rightarrow P(x)).$$
\end{theorem}

\begin{proof}
    The proof in the forward direction is an easy tautology. For the converse, we assume  $ \forall x \in X,((\forall y \prec x, P(y)) \Rightarrow P(x))$ [FINISH PROOF, just a matter of applying the definitions]
\end{proof}

[Then here we will talk about structural induction as a special case of well-founded induction. Inductively defined sets and whatnot. This should cover all uses later on.]
