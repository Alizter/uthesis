\section{Conclusion and Future directions}

The natural direction to consider after ariving at the Curry-Howard correspondence are dependent types. We discussed mostly propositional logic in this dissertation, but there was nothing stopping us from considering first-order logic. First-order since predicates can \emph{quantify over} propositions. These correspond to the familar $\forall$ and $\exists$ that mathematicians are acustomed to using.

Dependent types generalise their non-dependent counterparts by allowing certain types that form the overall type to \emph{depend} on the value of the terms coming from a type elsewhere. So for example one could write $\mathbf{Months} \times \mathbf{Days}$, to americanly have, a type of dates. Clearly this is complete nonsense since we can have a pair $(\mathrm{Feb}, 31)$. Ideally we would like such terms to not be well-typed. The solution is to let the type of $\mathbf{Days}$ \emph{depend} on the type of $\mathbf{Months}$. In a dependent type theory one would write $$\sum_{m : \mathbf{Months}} \mathbf{Days}(m)$$, where $\mathbf{Days}(m)$ is the type of days of the month $m$. Terms of this \emph{Sigma type} are called dependent pairs. Now the term $(\mathrm{Feb}, 28)$ type checks as before, however $(\mathrm{Feb}, 29)$ doesn't. What we have written is complete nonsense. Clearly the type checker is upset because $\mathbf{Days}(\mathrm{Feb})$ does not have a term $29$.

It turns out dependent type theories have similar normalisation properties. But as a programming language, they are still not understood as well as some of our other programming langauges \cite{Sorensen, DEBRUIJN1994141, DEBRUIJN1972381}. In the future, mainstream functional programming languages such as Haskell, will slowly gain support for dependent types. Generalising a vast array of previous programming features such as Generalized Algrbraic Data Types, Parametricity, Polymorphism and so on \cite{2016arXiv161007978E}. This will allow programmers to reason in rich ways about the correctness of their programs and allow mathematicans to write proofs in a programming language, due to the Curry-Howard correspondence. This is already done at a mass scale today with Formal verification software.

And finally back to basics, we hope that in the future, syntax and its subtleties can be sorted out for good, so that computer scientists won't need to spread white lies when discussing type theories. There is some recent work (formalised too!) in these directions \cite{Binding_Syntax_Theory}. From what we have read, this is essentially a formalised version of Harper's abts, noting that the idea is not unique to him.

Our original goal was to study the categorical semantics of dependent type theories, but along the journey we learnt that even the simple things are not so well understood yet. This means there is oppurtunity to learn, teach and grow.
